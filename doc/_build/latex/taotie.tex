%% Generated by Sphinx.
\def\sphinxdocclass{report}
\documentclass[letterpaper,10pt,english]{sphinxmanual}
\ifdefined\pdfpxdimen
   \let\sphinxpxdimen\pdfpxdimen\else\newdimen\sphinxpxdimen
\fi \sphinxpxdimen=.75bp\relax

\PassOptionsToPackage{warn}{textcomp}
\usepackage[utf8]{inputenc}
\ifdefined\DeclareUnicodeCharacter
% support both utf8 and utf8x syntaxes
  \ifdefined\DeclareUnicodeCharacterAsOptional
    \def\sphinxDUC#1{\DeclareUnicodeCharacter{"#1}}
  \else
    \let\sphinxDUC\DeclareUnicodeCharacter
  \fi
  \sphinxDUC{00A0}{\nobreakspace}
  \sphinxDUC{2500}{\sphinxunichar{2500}}
  \sphinxDUC{2502}{\sphinxunichar{2502}}
  \sphinxDUC{2514}{\sphinxunichar{2514}}
  \sphinxDUC{251C}{\sphinxunichar{251C}}
  \sphinxDUC{2572}{\textbackslash}
\fi
\usepackage{cmap}
\usepackage[T1]{fontenc}
\usepackage{amsmath,amssymb,amstext}
\usepackage{babel}



\usepackage{times}
\expandafter\ifx\csname T@LGR\endcsname\relax
\else
% LGR was declared as font encoding
  \substitutefont{LGR}{\rmdefault}{cmr}
  \substitutefont{LGR}{\sfdefault}{cmss}
  \substitutefont{LGR}{\ttdefault}{cmtt}
\fi
\expandafter\ifx\csname T@X2\endcsname\relax
  \expandafter\ifx\csname T@T2A\endcsname\relax
  \else
  % T2A was declared as font encoding
    \substitutefont{T2A}{\rmdefault}{cmr}
    \substitutefont{T2A}{\sfdefault}{cmss}
    \substitutefont{T2A}{\ttdefault}{cmtt}
  \fi
\else
% X2 was declared as font encoding
  \substitutefont{X2}{\rmdefault}{cmr}
  \substitutefont{X2}{\sfdefault}{cmss}
  \substitutefont{X2}{\ttdefault}{cmtt}
\fi


\usepackage[Bjarne]{fncychap}
\usepackage{sphinx}

\fvset{fontsize=\small}
\usepackage{geometry}

% Include hyperref last.
\usepackage{hyperref}
% Fix anchor placement for figures with captions.
\usepackage{hypcap}% it must be loaded after hyperref.
% Set up styles of URL: it should be placed after hyperref.
\urlstyle{same}

\usepackage{sphinxmessages}
\setcounter{tocdepth}{0}



\title{Taotie}
\date{Jul 14, 2019}
\release{0.1}
\author{Song Huang}
\newcommand{\sphinxlogo}{\vbox{}}
\renewcommand{\releasename}{Release}
\makeindex
\begin{document}

\pagestyle{empty}
\sphinxmaketitle
\pagestyle{plain}
\sphinxtableofcontents
\pagestyle{normal}
\phantomsection\label{\detokenize{index::doc}}

\begin{quote}

“The taotie is a motif commonly found on Chinese ritual bronze vessels from the Shang and Zhou
dynasty…In ancient Chinese mythology like “Classic of Mountains and Seas”, the “taotie” (饕餮)
is one of the “four evil creatures of the world”…a greedy and gluttonous son of the Jinyun
clan, who lived during the time of the
legendary Yellow Emperor.” - \sphinxhref{https://en.wikipedia.org/wiki/Taotie}{Wikipedia}

“缙云氏有不才子,贪于饮食,冒于货贿,天下谓之饕餮”
- \sphinxhref{https://zh.wikipedia.org/wiki/\%E9\%A5\%95\%E9\%A4\%AE}{《吕氏春秋·先识》}
\end{quote}

\noindent\sphinxincludegraphics[width=600\sphinxpxdimen]{{taotie_logo}.png}


\chapter{Starting Your Career as a Researcher}
\label{\detokenize{index:starting-your-career-as-a-researcher}}

\section{Getting Started as a Researcher}
\label{\detokenize{resource/research/getting_started:getting-started-as-a-researcher}}\label{\detokenize{resource/research/getting_started::doc}}
\begin{figure}[htbp]
\centering
\capstart

\noindent\sphinxincludegraphics{{/Users/songhuang/Dropbox/work/project/taotie/doc/_build/doctrees/images/507d4a74bba11789c49c00786c01746085935377/phd051017s}.gif}
\caption{Ah..research life}\label{\detokenize{resource/research/getting_started:id1}}\end{figure}


\subsection{Motivation and Mission Statement}
\label{\detokenize{resource/research/getting_started:motivation-and-mission-statement}}\begin{itemize}
\item {} 
To become a scientist is to dive into the unknown on behalf of the
human-kind, it is never an easy job but can be very rewarding as
well. It is good idea to always check your motivation, think about
your mission statement, and confront your career and personal choices
sincerely.

\item {} 
\sphinxhref{https://arxiv.org/abs/1805.09963}{Choose Your Own Adventure: Developing A Values-Oriented Framework
for Your Career by Lucianne
Walkowicz}
\begin{itemize}
\item {} 
I think you should read this, every word of it, before you read
anything else. This is probably the best career advice I have ever
read. I worked on my \sphinxstylestrong{mission statement} right after I read
this.

\end{itemize}

\end{itemize}


\subsection{Inclusiveness and Community Building}
\label{\detokenize{resource/research/getting_started:inclusiveness-and-community-building}}\begin{itemize}
\item {} 
Read through the \sphinxhref{https://aas.org/ethics}{AAS Code of Ethics} or
the \sphinxhref{http://asa.astronomy.org.au/code\_of\_ethics.php}{ASA Code of
Ethics}
\begin{itemize}
\item {} 
Also the \sphinxhref{https://aas.org/policies/anti-harassment-policy-aas-division-meetings-activities}{Anti-Harassment Policy for AAS \& Division Meetings \&
Activities}

\end{itemize}

\item {} 
\sphinxhref{https://astro-outlist.github.io/}{Astronomy and Astrophysics
Outlist}
\begin{itemize}
\item {} 
“As professionals in astronomy and astrophysics, whether we are
students, faculty, staff, librarians or are working in other
positions, we all like to believe that our work environment is
determined only by our capabilities as students and researchers,
and that it is free from personal bias. A productive professional
atmosphere depends on open and accepting interactions of
individuals free from discrimination and harassment.”

\end{itemize}

\item {} 
\sphinxhref{http://www.astronomyallies.com/Astronomy\_Allies/Welcome.html}{AstronomyAllies - Anti-harassment in
Astronomy}

\item {} 
\sphinxhref{http://www.astrobetter.com/wiki/Diversity}{Equal Opportunity Astronomy: Articles \&
Resources} collected by
AstroBetter.

\end{itemize}


\subsection{ORCID and Google Scholar}
\label{\detokenize{resource/research/getting_started:orcid-and-google-scholar}}\begin{itemize}
\item {} 
\sphinxhref{https://orcid.org/}{ORCID} provides a persistent digital
identifier that distinguishes you from every other researcher and,
through integration in key research workflows such as manuscript and
grant submission, supports automated linkages between you and your
professional activities ensuring that your work is recognized.
\begin{itemize}
\item {} 
It is a very good idea to register an ORCID and maintaine it from
time to time.

\end{itemize}

\item {} 
It is also a good idead to start a \sphinxhref{https://scholar.google.com}{Google
scholar} page

\end{itemize}


\subsection{Choosing a Programing Language}
\label{\detokenize{resource/research/getting_started:choosing-a-programing-language}}\begin{itemize}
\item {} 
It has become clear that \sphinxhref{https://www.python.org/}{Python} has
become the new Lingua franca in astrophysics and cosmology research
and it is probably the language you want to learn first. We also
prepare a series of lists to collect \sphinxstylestrong{Python} related resources and
tools, including \sphinxhref{https://github.com/dr-guangtou/taotie/blob/master/programing/python\_basic.md}{basic learning materials for
Python}
,\sphinxhref{https://github.com/dr-guangtou/taotie/blob/master/programing/python\_performance.md}{resources on improving the performace of
Python},
\sphinxhref{https://github.com/dr-guangtou/taotie/blob/master/programing/python\_optimazaton.md}{on model fitting or optimization in
Python},
\sphinxhref{https://github.com/dr-guangtou/taotie/blob/master/programing/python\_statistics.md}{on statistical analysis and model in
Python},
\sphinxhref{https://github.com/dr-guangtou/taotie/blob/master/programing/python\_visualization.md}{on data visualization in
Python},
\sphinxhref{https://github.com/dr-guangtou/taotie/blob/master/programing/python\_write\_yourown\_project.md}{on getting started with your own Python
project}.

\item {} 
\sphinxhref{https://en.wikipedia.org/wiki/C\_(programming\_language)}{C} and
\sphinxhref{https://en.wikipedia.org/wiki/C\%2B\%2B}{C++} are still at the core
of many important astrophysical applications, e.g. numerical
simulations, data reduction or analysis that requires high efficiency
or good performance. Many important astrophysical softwares use \sphinxstylestrong{C}
or \sphinxstylestrong{C++} as the core and then wrap it up using \sphinxstylestrong{Python}. Here we
also provide some basic resources for using
\sphinxhref{https://github.com/dr-guangtou/taotie/blob/master/programing/clang\_basic.md}{C}
and
\sphinxhref{https://github.com/dr-guangtou/taotie/blob/master/programing/cpp\_basic.md}{C++}
in research.
\sphinxhref{http://fortranwiki.org/fortran/show/HomePage}{Fortran} is another
historically important language known for its excellent numerical
performance. Its influence in astronomy is decreasing right now, but
you will still see it in scenarios that rely on high-performance
computing.

\item {} 
\sphinxhref{https://julialang.org/}{Julia} is another intersting high-level
programming language on the rising. There are \sphinxhref{https://discourse.julialang.org/t/julia-motivation-why-werent-numpy-scipy-numba-good-enough/2236}{several key advantages
over
Python}
but it is still a young language. Right now, \sphinxstylestrong{Julia} tools for
astronomy and astrophysics are still limited, but we decide to start
to \sphinxhref{https://github.com/dr-guangtou/taotie/blob/master/programing/julia\_basic.md}{collect relevant resoures and
tools}

\item {} 
At the same time, \sphinxhref{https://www.r-project.org/about.html}{R}
statistical language also has some interesting applications in
astronomy;
\sphinxhref{https://en.wikipedia.org/wiki/IDL\_(programming\_language)}{IDL}
was yesterday’s language for astronomical data reduction, but for
historical reasons, many projects/instruments are still using it. If
you just start your research in astronomy, we no longer recommend you
to learn it. But in case you have to face it in research, it is \sphinxhref{http://mathesaurus.sourceforge.net/idl-numpy.html}{not
difficult if you have a background in
Python}

\item {} 
Although their applications in astrophysics are limited,
\sphinxhref{https://www.mathworks.com/products/matlab.html}{MATLAB} is widely
used for data reduction and numerical simulation, and
\sphinxhref{http://www.wolfram.com/mathematica/}{Mathematica} is a very
popular tool for symbolic analysis and theoretical research. Neither
of them is free, so please make sure you have the correct license to
use them.

\item {} 
And it never hurts if you can learn some basic programming skills
related to webpage making, e.g.
\sphinxhref{https://www.w3schools.com/html/}{HTML},
\sphinxhref{https://www.w3schools.com/css/}{CSS}, and
\sphinxhref{https://www.javascript.com/}{Javascript}.

\item {} 
\sphinxstylestrong{It is also important to remember: never tool-shaming others!}.
There are excellent scientists who still rely on \sphinxstylestrong{IDL},
\sphinxhref{https://en.wikipedia.org/wiki/IRAF}{IRAF}, \sphinxstylestrong{Fortran} for
research, and using
\sphinxhref{https://www.astro.princeton.edu/~rhl/sm/}{SuperMongo} or
\sphinxhref{http://www.gnuplot.info/}{gnuplot} for data visualization. There
are plenty of personal and practical reasons to do so and they can
still do great science. Use your energy for something more positive
and productive.

\end{itemize}


\subsubsection{Misc}
\label{\detokenize{resource/research/getting_started:misc}}\begin{itemize}
\item {} 
\sphinxhref{http://people.duke.edu/~ccc14/sta-663-2019/}{The STA663 courses at Duke University compiled a nice list of useful
resources for computer skills that are useful for
research}

\item {} 
\sphinxhref{https://stackoverflow.com/}{StackOverflow} is a community
knowledge database for programming. It can be your best friends in so
many situations, and you should consider helping others too.

\end{itemize}


\subsection{Organizing Your Research Project}
\label{\detokenize{resource/research/getting_started:organizing-your-research-project}}\begin{itemize}
\item {} 
\sphinxhref{https://github.com/}{GitHub} or any other on line code
repositories (e.g. \sphinxhref{https://about.gitlab.com/}{GitLab},
\sphinxhref{https://bitbucket.org/}{bitbucket},
\sphinxhref{https://coding.net/git}{coding}) can help organize your
scientific project. It can help you do version control, back up
research results, and also share results and code with the community.
\begin{itemize}
\item {} 
\sphinxhref{https://github.com/github/hub}{hub - A command-line tool that makes git easier to use with
GitHub}

\item {} 
You can also keep your project synced across multiple platforms.
Please see \sphinxhref{https://moox.io/blog/keep-in-sync-git-repos-on-github-gitlab-bitbucket/}{this
article}.
Notice that \sphinxstylestrong{gitlab} is using \sphinxstylestrong{v4} API now.

\item {} 
It is easy to \sphinxhref{https://moox.io/blog/keep-in-sync-git-repos-on-github-gitlab-bitbucket/}{keeping in sync your Git repos on Github, Gitlab,
and
Bitbucket}.
All you need to do is to make sure the repos share the same name,
and add multiple remotes to the same local data.

\end{itemize}

\item {} 
And \sphinxstylestrong{GitHub} allows you to \sphinxhref{https://github.blog/2019-06-06-generate-new-repositories-with-repository-templates}{create new repository based on a
template}
\begin{itemize}
\item {} 
There are template available that can help you get started:
\begin{itemize}
\item {} 
\sphinxhref{https://github.com/uwescience/shablona}{shablona - A template for small scientific python
projects}

\end{itemize}

\end{itemize}

\item {} 
If you want to start a package as your project, you can try
\sphinxhref{https://github.com/audreyr/cookiecutter}{cookiecutter} - A
command-line utility that creates projects from templates for Python,
Javascript, Ruby, Markdown, CSS, HTML etc.
\begin{itemize}
\item {} 
If you are interested in using \sphinxstylestrong{astropy} as a good template for
Python project, the \sphinxhref{https://github.com/astropy/package-template}{astropy package
template} is
available too.

\end{itemize}

\end{itemize}


\subsection{Organizing a Programming Environment}
\label{\detokenize{resource/research/getting_started:organizing-a-programming-environment}}\begin{itemize}
\item {} 
Before starting some serious projects, you should be a little more
patient on the learning curve and try to cultivate some good habbits.
Good examples are everywhere!

\item {} 
Don’t waste too much time choosing editors or IDEs, just pick the
first one you like, learn how to use it; if it does grow on you,
change to another one. Both \sphinxhref{https://www.vim.org/}{vim} and
\sphinxhref{https://www.gnu.org/software/emacs/}{emacs} are excellent tools;
\sphinxhref{https://atom.io/}{atom},
\sphinxhref{https://code.visualstudio.com/}{VScode}, and
\sphinxhref{https://www.sublimetext.com/}{sublime} are all very good IDEs.
They \sphinxstylestrong{all} have amazing capabilities and can help you become a
great coder and scientist.

\item {} 
There are some useful resources that will save your time setting up
the environment:
\begin{itemize}
\item {} 
\sphinxhref{https://github.com/syl20bnr/spacemacs}{spacemacs - A community-driven Emacs
distribution}

\item {} 
\sphinxhref{https://github.com/SpaceVim/SpaceVim}{spacevim - A community-driven modular vim
distribution}

\item {} 
\sphinxhref{https://github.com/emacs-tw/awesome-emacs}{Awesome Emacs - A community driven list of useful Emacs packages,
libraries and
others}

\item {} 
\sphinxhref{https://github.com/neovim/neovim}{neovim - Vim-fork focused on extensibility and
usability}

\item {} 
\sphinxhref{https://github.com/viatsko/awesome-vscode}{awesome-vscode - A curated list of delightful VS Code packages
and resources}

\item {} 
\sphinxhref{https://github.com/mehcode/awesome-atom}{awesome-atom - A curated list of delightful Atom packages and
resources}

\end{itemize}

\end{itemize}


\subsection{Backing-up Your Research}
\label{\detokenize{resource/research/getting_started:backing-up-your-research}}\begin{itemize}
\item {} 
This is as important as you can possibly imagine.

\item {} 
Off-line Backup:
\begin{itemize}
\item {} 
You should constantly back-up your harddrive using external
harddrive. Both MacOSX (e.g.
\sphinxhref{https://support.apple.com/en-us/HT201250}{TimeMachine}) and
Linux (\sphinxhref{https://wiki.ubuntu.com/TimeVault}{TimeVault} and
\sphinxhref{http://duplicity.nongnu.org/}{Duplicity}) have systems that
help you backup data.

\item {} 
You can also easily backup your entire system or certain directory
using a command line tool
\sphinxhref{https://linux.die.net/man/1/rsync}{rsync}: \sphinxstylestrong{rsync -av \textendash{}delete
/Directory1/ /Directory2/}
\begin{itemize}
\item {} 
On Linux, you can also use
\sphinxhref{https://opensource.com/article/17/11/how-use-cron-linux}{Cron}
to automatically backup files at any given time. For example,
you can follow the instruction
\sphinxhref{https://nickjanetakis.com/blog/automatic-offline-file-backups-with-bash-and-rsync}{here}

\end{itemize}

\end{itemize}

\item {} 
Online Backup:
\begin{itemize}
\item {} 
It is encouraged to use service like the
\sphinxhref{https://www.dropbox.com}{Dropbox} to constantly backup
important research-related files (e.g. draft, code, and figures).
In mainland China, \sphinxhref{https://www.jianguoyun.com/}{jianguoyun
(坚果云)} is an alternative.

\end{itemize}

\end{itemize}


\subsection{Keeping Research Notes and Documents}
\label{\detokenize{resource/research/getting_started:keeping-research-notes-and-documents}}\begin{itemize}
\item {} 
\sphinxhref{https://en.wikipedia.org/wiki/Markdown}{Markdown} is a
lightweight markup language with plain text formatting syntax. It is
very easy to learn and can help you make well-organize notes and
documents that can be easily converted into other format (\sphinxstylestrong{HTML} or
\sphinxstylestrong{LaTeX}).
\begin{itemize}
\item {} 
\sphinxhref{https://guides.github.com/features/mastering-markdown/}{Mastering Markdown by GitHub
Guides}
is a very good start.

\item {} 
If you want to learn more details, use \sphinxhref{https://www.markdownguide.org/}{the Markdown
Guide}.

\item {} 
Most of the editors and IDEs support the \sphinxstylestrong{.md} or \sphinxstylestrong{.markdown}
format documents through extensions. They can help you check the
syntax. There are also a lot of markdown editors on all platforms.

\end{itemize}

\item {} 
Whatever notes or documents you are keeping for your research, make
sure it can be backed-up and is searchable. Using software like the
\sphinxhref{https://www.onenote.com/signin?wdorigin=ondc}{OneNote} from
Microsoft, or on-line service like
\sphinxhref{https://evernote.com}{evernote} would be a good idea. If your
project is already on \sphinxstylestrong{Github}, you can just use \sphinxstylestrong{git} to version
control and back-up your documents. \sphinxhref{https://guides.github.com/features/wikis/}{GitHub wiki
pages} are another
great way to keep notes.

\end{itemize}


\subsection{Publishing Your Science}
\label{\detokenize{resource/research/getting_started:publishing-your-science}}\begin{itemize}
\item {} 
\sphinxhref{https://www.scimagojr.com/journalrank.php?category=3103}{A list of journals in Astronomy and
Astrophysics}
\begin{itemize}
\item {} 
Don’t pay too much attention to the impact factor or H-index.

\end{itemize}

\item {} 
Writing a paper can be painful, but it is one of the most important
step in your research life. We have \sphinxhref{https://github.com/dr-guangtou/taotie/blob/master/research/writing\_paper.md}{a separate document talking
about writing
papers}.

\end{itemize}


\subsection{Sharing Your Science}
\label{\detokenize{resource/research/getting_started:sharing-your-science}}\begin{itemize}
\item {} 
\sphinxhref{https://en.wikipedia.org/wiki/Open\_science}{Open Science} is good
for everybody!

\item {} 
You can share your results using \sphinxstylestrong{Github}: you can share codes,
notebooks, and draft together. But it is not very good if you have
large amount of data to share.

\item {} 
\sphinxhref{https://zenodo.org/}{zenodo - a general-purpose open-access repository developed under
the European OpenAIRE program and operated by
CERN}

\item {} 
\sphinxhref{https://dataverse.org/}{Dataverse - open source research data repository
software}

\item {} 
\sphinxhref{https://ascl.net/}{The Astrophysics Source Code Library (ASCL)}

\item {} 
\sphinxhref{https://osf.io/}{OSF - Open Science Framework}

\end{itemize}


\subsubsection{Talking about Your Science}
\label{\detokenize{resource/research/getting_started:talking-about-your-science}}\begin{itemize}
\item {} 
It takes a lot of practice to know how to give a good talk, but there
could be some useful tips to follow:
\begin{itemize}
\item {} 
\sphinxhref{https://www.nature.com/articles/d41586-018-07780-5}{How to give a great scientific talk by
Nature}

\item {} 
\sphinxhref{https://www.sciencemag.org/careers/2019/04/three-tips-giving-great-research-talk}{Three tips for giving a great research talk by
Science}

\item {} 
\sphinxhref{https://astrobites.org/2018/02/10/speak-your-science-part-1/}{Speak your science by Astrobites (three
parts)}

\item {} 
\sphinxhref{https://arxiv.org/abs/1712.08088}{How to Give a Great Talk by Chat
Hull}

\end{itemize}

\end{itemize}


\subsubsection{Making a Scientific Poster}
\label{\detokenize{resource/research/getting_started:making-a-scientific-poster}}\begin{itemize}
\item {} 
\sphinxhref{https://www.makesigns.com/tutorials/}{We’re Here To Help You Make The Best Scientific
Poster}

\item {} 
\sphinxhref{https://osf.io/ef53g/}{Better Scientific Poster}
\begin{itemize}
\item {} 
By Mike Morrison. A new, faster approach to designing research
posters. Includes templates

\item {} 
There is a \sphinxhref{https://www.youtube.com/watch?v=1RwJbhkCA58\&feature=youtu.be}{Youtube video that describes the motivation and
design}

\item {} 
\sphinxhref{https://github.com/rafaelbailo/betterposter-latex-template}{The LaTeX
template}

\item {} 
\sphinxhref{https://github.com/GerkeLab/betterposter}{The R Markdown
template}

\end{itemize}

\end{itemize}


\subsection{Reading Paper}
\label{\detokenize{resource/research/getting_started:reading-paper}}\begin{itemize}
\item {} 
It is important to read as much as you can. It is important to follow
\sphinxstylestrong{arXiv} regularly.
\begin{itemize}
\item {} 
You can check if your institute is using
\sphinxhref{https://www.voxcharta.org}{voxCharta}, a on-line platform to
vote on papers and organize \sphinxstylestrong{arXiv} discussion.

\item {} 
It is good idea to have a routine that keeps record of interesting
papers. Here is an examply by
\sphinxhref{https://github.com/dr-guangtou/daily\_astroph}{me}

\end{itemize}

\item {} 
\sphinxhref{https://astrobites.org}{Astrobites} is a very good website to
follow recent interesting papers from the perspective of a graduate
student.
\begin{itemize}
\item {} 
They also provide some good advices on reading papers: \sphinxhref{https://astrobites.org/2017/12/19/tools-for-reading-papers-part-1/}{Part
I},
\sphinxhref{https://astrobites.org/2018/03/09/tools-for-reading-papers-part-2/}{Part
II},
\sphinxhref{https://astrobites.org/2018/09/06/tools-for-reading-papers-part-3/}{Part
III}

\end{itemize}

\end{itemize}


\subsubsection{On Using arXiv and SAO/NASA ADS}
\label{\detokenize{resource/research/getting_started:on-using-arxiv-and-sao-nasa-ads}}

\paragraph{arXiv}
\label{\detokenize{resource/research/getting_started:arxiv}}\begin{itemize}
\item {} 
\sphinxhref{https://arxiv.org/help/submit}{To submit an article to arXiv}
\begin{itemize}
\item {} 
Please read this webpage first…submitting paper to arXiv sometimes
can be annoying.

\end{itemize}

\item {} 
\sphinxhref{https://arxiv.org/localtime}{Local time at arxiv.org}
\begin{itemize}
\item {} 
To remind you the deadline for submitting paper to arXiv

\end{itemize}

\item {} 
\sphinxhref{https://github.com/arXiv}{The official arXiv github repositories}

\item {} 
\sphinxhref{https://github.com/lukasschwab/arxiv.py}{arxiv.py - Python wrapper for the arXiv
API}

\item {} 
\sphinxhref{https://github.com/google-research/arxiv-latex-cleaner}{arXiv LaTeX Cleaner: Easily clean the LaTeX code of your paper to
submit to
arXiv}

\end{itemize}


\paragraph{SAO/NASA ADS}
\label{\detokenize{resource/research/getting_started:sao-nasa-ads}}\begin{itemize}
\item {} 
\sphinxhref{http://adsabs.github.io/help/search/}{Tutorial for using the new ADS
search}

\item {} 
\sphinxhref{https://github.com/adsabs}{Official SAO/NASA ADS github
repositories}

\item {} 
\sphinxhref{https://github.com/andycasey/ads}{ads - A Python Module to Interact with NASA’s ADS that Doesn’t
Suck}

\end{itemize}


\subsection{Communicating with Others}
\label{\detokenize{resource/research/getting_started:communicating-with-others}}\begin{itemize}
\item {} 
\sphinxhref{https://slack.com/}{Slack} has become the most common way to
organize a small collaboration. Even the free version can be very
useful.

\item {} 
Telecon becomes more and more frequently used to communicate among
collaborators in different institutes and timezones. Commonly used
telecon tools including \sphinxhref{https://www.skype.com/en/}{Skype},
\sphinxhref{https://zoom.us/}{zoom},
\sphinxhref{https://www.gotomeeting.com/}{GoToMeetings}
\begin{itemize}
\item {} 
All of these tools are free and cross-platform, and easy to use.
You can share screen using them for remote presentation too.

\end{itemize}

\item {} 
\sphinxhref{https://doodle.com/make-a-poll}{Doodle} is the most commonly used
tool to create a poll to decide the time slot for a meeting or
telecon.

\end{itemize}


\subsection{Personal Website}
\label{\detokenize{resource/research/getting_started:personal-website}}\begin{itemize}
\item {} 
It is actually pretty important to have a visible personal website
that links your CV and contact information. Make sure that it can be
found by search engine.

\item {} 
This is especially important if you try to find job in another
country (e.g. get a PhD in China, want a post-doc job in Europe) or
when you know the hiring committee is not familiar with you.

\item {} 
\sphinxhref{https://pages.github.com/}{GitHub Pages} is pretty good choice to
make a nice-looking personal website. And there are some \sphinxhref{https://pages.github.com/themes/}{easy-to-use
templates available}, and there
are \sphinxhref{https://jekyllthemes.io/github-pages-templates}{more fancy ones
available}
\begin{itemize}
\item {} 
\sphinxhref{https://marisacarlos.com/pages/create-simple-academic-website}{How to Create a Simple Academic
Website}

\end{itemize}

\item {} 
\sphinxhref{https://github.com/alshedivat/al-folio}{al-folio - A beautiful Jekyll theme for
academics}
\begin{itemize}
\item {} 
This is a pretty good template for academic personal wesbsite

\end{itemize}

\item {} 
Good examples (personal choice: clean and informative)
\begin{itemize}
\item {} 
\sphinxhref{http://adrian.pw/}{Adrian Price-Whelan}; the code can be found
\sphinxhref{https://github.com/adrn/adrn.github.io}{here}

\item {} 
\sphinxhref{https://dfm.io/}{Dan Foreman-Mackey}; the code can be found
\sphinxhref{https://github.com/dfm/dfm.io}{here}

\end{itemize}

\end{itemize}


\subsection{Conference and Talks}
\label{\detokenize{resource/research/getting_started:conference-and-talks}}

\subsubsection{Scientific Conference}
\label{\detokenize{resource/research/getting_started:scientific-conference}}\begin{itemize}
\item {} 
Behave yourself professionally during conference or workshop. Please
pay attention to the code of conduct. As an example, you can read the
\sphinxhref{https://www.eso.org/sci/meetings/2018/tcl2018/code.html}{Code of Conduct for ESO Workshops \&
Conferences}

\item {} 
\sphinxhref{http://www.cadc-ccda.hia-iha.nrc-cnrc.gc.ca/en/meetings/}{CADA International Astronomy
Meetings}
is a very good place to check if there is anything conference that
interests you in the future. There is a RSS Feed and a \sphinxstylestrong{iCal}
subscription.

\item {} 
{[}@astromeetings Twitter
account{]}(\sphinxurl{https://twitter.com/astromeetings?lang=en}) is also a good
way to follow the on-going conferences in your field.
\begin{itemize}
\item {} 
It has become routine for a conferece to have a designated hashtag
on Twitter for people to twit about the talk. We cannot go to all
conferences (and it is \sphinxhref{https://onlinelibrary.wiley.com/doi/pdf/10.1111/1746-692X.12106}{bad for the mother
earth})

\end{itemize}

\item {} 
\sphinxhref{https://www.iau.org/science/meetings/future/}{Future IAU
Meetings}

\end{itemize}


\subsubsection{On-line Colloquium}
\label{\detokenize{resource/research/getting_started:on-line-colloquium}}\begin{itemize}
\item {} 
With Youtube, it is pretty easy to enjoy great astrophysical
colloquium in universities and institutes all over the world. Here
are a few good channels to get started:
\begin{itemize}
\item {} 
\sphinxhref{https://www.youtube.com/channel/UCApHNlZLkxmiV95A0ChueYg}{CfA
Colloquium}
and \sphinxhref{https://www.youtube.com/channel/UCTuACIrLKPTlp6XMZbeipig/featured}{ITC
Video}
from Harvard/CfA

\item {} 
\sphinxhref{https://www.youtube.com/user/AstronomyHeidelberg}{Heidelberg
Astronomy}

\item {} 
\sphinxhref{https://www.youtube.com/user/SimonsFoundation/playlists}{CCA
Seminars}.
Some of them are about astronomy and cosmology.

\item {} 
\sphinxhref{https://www.youtube.com/user/UofUPhysAstro/featured}{Dept of Physics \& Astronomy at the University of
Utah}

\end{itemize}

\end{itemize}


\section{Setting Up a Computer Environment for Research}
\label{\detokenize{resource/research/computer_basics:setting-up-a-computer-environment-for-research}}\label{\detokenize{resource/research/computer_basics::doc}}

\subsection{Terminal Tools}
\label{\detokenize{resource/research/computer_basics:terminal-tools}}

\subsubsection{Terminal Emulator}
\label{\detokenize{resource/research/computer_basics:terminal-emulator}}\begin{itemize}
\item {} 
\sphinxhref{https://www.iterm2.com/}{iTerm2 - macOS Terminal Replacement}
\begin{itemize}
\item {} 
A nice free choice for \sphinxstylestrong{MacOSX}.
\sphinxhref{https://github.com/gnachman/iTerm2}{iTerm2} brings the
terminal into the modern age with features you never knew you
always wanted.

\item {} 
A \sphinxhref{https://gist.github.com/squarism/ae3613daf5c01a98ba3a}{cheat sheet of iTerm2
shortcuts}

\end{itemize}

\item {} 
It becomes easier and easier to do research under Windows system:
\begin{itemize}
\item {} 
\sphinxhref{https://github.com/microsoft/terminal}{Microsoft’s terminal - The new Windows Terminal, and the original
Windows console host}

\item {} 
\sphinxhref{https://cmder.net/}{cmder - Portable console emulator for
Windows}

\end{itemize}

\item {} 
Different Linux releases have there own default terminal emulator and
normally they are good enough for research works. Meanwhile, there
are some nice choices of drop-down style emulators:
\begin{itemize}
\item {} 
\sphinxhref{https://github.com/lanoxx/tilda}{tilda - A Gtk based drop down terminal for Linux and
Unix}

\item {} 
\sphinxhref{https://github.com/Guake/guake}{guake - Drop-down terminal for
GNOME}
\begin{itemize}
\item {} 
\sphinxstylestrong{Guake} is a dropdown terminal made for the \sphinxstylestrong{GNOME} desktop
environment. \sphinxstylestrong{Guake}’s style of window is based on an FPS
game, and one of its goals is to be easy to reach.

\end{itemize}

\end{itemize}

\end{itemize}


\subsubsection{Terminal Multiplexer}
\label{\detokenize{resource/research/computer_basics:terminal-multiplexer}}\begin{itemize}
\item {} 
\sphinxhref{http://git.savannah.gnu.org/cgit/screen.git}{GNU Screen}
\begin{itemize}
\item {} 
\sphinxstylestrong{Screen} is available in most Unix/Linux system release.

\item {} 
\sphinxhref{https://linuxize.com/post/how-to-use-linux-screen/}{How to use Linux
Screen}

\end{itemize}

\item {} 
\sphinxhref{https://github.com/tmux/tmux}{tmux - a terminal multiplexer for Unix-like operating
systems}
\begin{itemize}
\item {} 
Similar to \sphinxstylestrong{Screen}. There are a lot of discussions about the
pros and cons of \sphinxstylestrong{Screen} and \sphinxstylestrong{tmux}, but as a beginner,
either is fine.

\item {} 
Read \sphinxhref{https://thoughtbot.com/blog/a-tmux-crash-course}{A tmux crash
course} and
\sphinxhref{https://gist.github.com/MohamedAlaa/2961058}{tmux shortcuts and cheat
sheet} before you
use it.

\end{itemize}

\end{itemize}


\subsection{Shell Environment}
\label{\detokenize{resource/research/computer_basics:shell-environment}}\begin{quote}

Fluency on the command line is a skill often neglected or considered
arcane, but it improves your flexibility and productivity as an
engineer in both obvious and subtle ways. \textendash{} The Art of Command Line
\end{quote}
\begin{itemize}
\item {} 
\sphinxstylestrong{The same also applies to student of science and researcher.}

\item {} 
\sphinxhref{https://github.com/jlevy/the-art-of-command-line}{The Art of Command
Line}
\begin{itemize}
\item {} 
In multiple languages, including Chinese

\end{itemize}

\item {} 
\sphinxhref{https://www.howtogeek.com/412055/37-important-linux-commands-you-should-know/}{37 Important Linux Commands You Should
Know}

\item {} 
\sphinxhref{https://github.com/herrbischoff/awesome-command-line-apps}{awesome-command-line-apps - Use your terminal shell to do awesome
things}
\begin{itemize}
\item {} 
A curated list of useful command line apps, in celebration of the
TUI.

\end{itemize}

\item {} 
\sphinxhref{https://github.com/herrbischoff/awesome-macos-command-line}{awesome-macos-command-line - Use your macOS terminal shell to do
awesome
things}
\begin{itemize}
\item {} 
A curated list of shell commands and tools specific to OS X.

\item {} 
A lot of these actually work for Linux too.

\end{itemize}

\end{itemize}


\subsubsection{\sphinxstylestrong{Bash} related}
\label{\detokenize{resource/research/computer_basics:bash-related}}\begin{itemize}
\item {} 
\sphinxhref{https://github.com/dylanaraps/pure-bash-bible}{Pure Bash Bible}
\begin{itemize}
\item {} 
A collection of pure bash alternatives to external processes.

\end{itemize}

\item {} 
\sphinxhref{https://github.com/denysdovhan/bash-handbook}{Bash Notebook}
\begin{itemize}
\item {} 
In depth Bash notebook in multiple languages

\end{itemize}

\item {} 
\sphinxhref{https://github.com/Bash-it/bash-it}{Bash-it - a collection of community Bash commands and scripts for
Bash 3.2+}
\begin{itemize}
\item {} 
Achieve some of \sphinxstylestrong{zsh}’s functions in \sphinxstylestrong{bash}.

\end{itemize}

\end{itemize}


\subsubsection{Other than \sphinxstylestrong{bash}}
\label{\detokenize{resource/research/computer_basics:other-than-bash}}\begin{itemize}
\item {} 
\sphinxhref{https://ohmyz.sh/}{Oh My Zsh}
\begin{itemize}
\item {} 
Code on Github is
\sphinxhref{https://github.com/robbyrussell/oh-my-zsh/}{here}

\end{itemize}

\item {} 
\sphinxhref{https://github.com/jorgebucaran/fish-cookbook}{The Fish Cookbook}
\begin{itemize}
\item {} 
Tips and recipes for fish, from shell to plate

\end{itemize}

\end{itemize}


\subsection{System Tools}
\label{\detokenize{resource/research/computer_basics:system-tools}}\begin{itemize}
\item {} 
\sphinxhref{https://github.com/cjbassi/gotop}{gotop - system monitor}
\begin{itemize}
\item {} 
A terminal based graphical activity monitor inspired by gtop and
vtop

\end{itemize}

\item {} 
\sphinxhref{https://github.com/tldr-pages/tldr}{tldr - command document for
human}
\begin{itemize}
\item {} 
Simplified and community-driven man pages

\end{itemize}

\item {} 
\sphinxhref{https://github.com/nvbn/thefuck}{thefuck - Magnificent app which corrects your previous console
command}
\begin{itemize}
\item {} 
Well….it is actually kind of useful…

\end{itemize}

\end{itemize}


\bigskip\hrule\bigskip



\subsection{MacOS}
\label{\detokenize{resource/research/computer_basics:macos}}\begin{itemize}
\item {} 
\sphinxhref{https://github.com/sb2nov/mac-setup}{mac-setup: Installing Development environment on
macOS}
\begin{itemize}
\item {} 
The more readable website is
\sphinxhref{http://sourabhbajaj.com/mac-setup/iTerm/tree.html}{here}

\end{itemize}

\end{itemize}


\section{Job Search and Career Related}
\label{\detokenize{resource/research/job_and_career:job-search-and-career-related}}\label{\detokenize{resource/research/job_and_career::doc}}
\begin{figure}[htbp]
\centering
\capstart

\noindent\sphinxincludegraphics{{/Users/songhuang/Dropbox/work/project/taotie/doc/_build/doctrees/images/3ffe6b93acfd5dd556c25458dc372f4316f4c36b/phd091007s}.gif}
\caption{Can I have a job}\label{\detokenize{resource/research/job_and_career:id1}}\end{figure}


\subsection{Find Astronomy Related Jobs}
\label{\detokenize{resource/research/job_and_career:find-astronomy-related-jobs}}\begin{itemize}
\item {} 
\sphinxhref{https://jobregister.aas.org/}{AAS Job Register}
\begin{itemize}
\item {} 
Probably the most important website if you want to find a post-doc
or faculty job.

\end{itemize}

\item {} 
Astrophysics Jobs Rumor Mill
\begin{itemize}
\item {} 
For \sphinxhref{http://www.astrobetter.com/wiki/Rumor+Mill+Faculty-Staff}{faculty and
staff}

\item {} 
For \sphinxhref{http://www.astrobetter.com/wiki/Rumor+Mill}{postdoc and term
job}

\item {} 
Make sure that you don’t spend too much time here during job
season.

\end{itemize}

\end{itemize}


\subsection{Can I haz a job?}
\label{\detokenize{resource/research/job_and_career:can-i-haz-a-job}}\begin{itemize}
\item {} 
Getting an acedemic job is never easy, but at least you can prepare
these things:

\end{itemize}


\subsubsection{General Advice}
\label{\detokenize{resource/research/job_and_career:general-advice}}\begin{itemize}
\item {} 
\sphinxhref{https://arxiv.org/abs/1805.09963}{Choose Your Own Adventure: Developing A Values-Oriented Framework
for Your Career by Lucianne
Walkowicz}
\begin{itemize}
\item {} 
I think you should read this, every word of it, before you read
anything else. This is probably the best career advice I have ever
read. I worked on my \sphinxstylestrong{mission statement} right after I read
this.

\end{itemize}

\item {} 
\sphinxhref{https://gradschool.cornell.edu/academic-progress/pathways-to-success/}{Pathways to Success - A Holistic Approach to Professional
Development from Cornell
University}
\begin{itemize}
\item {} 
The name sounds like a cliché, but some of the advice here is
pretty good.

\end{itemize}

\item {} 
\sphinxhref{https://www.astrobetter.com/blog/2017/01/19/advice-from-a-faculty-hiring-committee-chair/}{Advice from a Faculty Hiring Committee
Chair}

\item {} 
\sphinxhref{http://www.astrokatie.com/solicited-advice}{Advice for Aspiring Astrophysicists by Katie
Mack}

\item {} 
\sphinxhref{https://ixkael.github.io/advice/job-season-tips/}{Post-doc / fellowship applications
tips}

\item {} 
\sphinxhref{http://physics.uq.edu.au/ap/ecrmentoring/wp-content/uploads/2012/07/Gaensler-Maddison\_ASA\_ECR\_workshop\_GettingAJob-Notes.pdf}{ASA Early Career Researcher Mentoring Workshop \textendash{} Getting a Job Note
by Bryan Gaensler \& Sarah
Maddison}
\begin{itemize}
\item {} 
These are some very good tips.

\end{itemize}

\end{itemize}


\subsubsection{CV (Resume) and Publication List}
\label{\detokenize{resource/research/job_and_career:cv-resume-and-publication-list}}\begin{itemize}
\item {} 
Guide to use \sphinxhref{https://gradschool.cornell.edu/academic-progress/pathways-to-success/prepare-for-your-career/take-action/resumes-and-cvs/}{resumes and CVs from Cornell
University}

\item {} 
You can make a CV using whatever software you want (e.g. Word,
Pages), but LaTeX makes very nice and professional looking CV.

\item {} 
There are a good selection of \sphinxhref{https://www.overleaf.com/gallery/tagged/cv}{LaTeX CV templates on
Overleaf}. Some of
these are a little too fancy for acedemic, but they all look pretty
good.

\end{itemize}


\subsubsection{Personal Website}
\label{\detokenize{resource/research/job_and_career:personal-website}}\begin{itemize}
\item {} 
It is actually pretty important to have a visible personal website
that links your CV and contact information. Make sure that it can be
found by search engine.

\item {} 
This is especially important if you try to find job in another
country (e.g. get a PhD in China, want a post-doc job in Europe) or
when you know the hiring committee is not familiar with you.

\item {} 
\sphinxhref{https://pages.github.com/}{GitHub Pages} is pretty good choice to
make a nice-looking personal website. And there are some \sphinxhref{https://pages.github.com/themes/}{easy-to-use
templates available}, and there
are \sphinxhref{https://jekyllthemes.io/github-pages-templates}{more fancy ones
available}
\begin{itemize}
\item {} 
\sphinxhref{https://marisacarlos.com/pages/create-simple-academic-website}{How to Create a Simple Academic
Website}

\end{itemize}

\item {} 
Good examples (personal choice: clean and informative)
\begin{itemize}
\item {} 
\sphinxhref{http://adrian.pw/}{Adrian Price-Whelan}; the code can be found
\sphinxhref{https://github.com/adrn/adrn.github.io}{here}

\item {} 
\sphinxhref{https://dfm.io/}{Dan Foreman-Mackey}; the code can be found
\sphinxhref{https://github.com/dfm/dfm.io}{here}

\end{itemize}

\end{itemize}


\subsubsection{Recommendation Letters}
\label{\detokenize{resource/research/job_and_career:recommendation-letters}}

\subsubsection{Research Plan and/or Research Proposal}
\label{\detokenize{resource/research/job_and_career:research-plan-and-or-research-proposal}}\begin{itemize}
\item {} 
\sphinxhref{https://gradschool.cornell.edu/academic-progress/pathways-to-success/prepare-for-your-career/take-action/research-statement/}{Guide to use research statement from Cornell
University}

\item {} 
\sphinxhref{https://aas.org/grants-and-prizes/hints-preparing-research-proposals}{Hints on Preparing Research
Proposals}

\end{itemize}


\subsection{General Career Advice}
\label{\detokenize{resource/research/job_and_career:general-career-advice}}\begin{itemize}
\item {} 
Astrobites also have some nice career inside and outside astronomy
(use \sphinxstylestrong{Career Navigation} as keyword):
\begin{itemize}
\item {} 
\sphinxhref{https://astrobites.org/2016/06/01/alternative-careers-leveraging-your-astronomy-degree-for-data-science/}{Alternative Careers: Leveraging your Astronomy Degree for Data
Science}

\item {} 
\sphinxhref{https://astrobites.org/2017/01/20/astronomers-beyond-astronomy-using-your-degree-outside-of-academia/}{Astronomers Beyond Astronomy: Using Your Degree Outside of
Academia}

\end{itemize}

\item {} 
AstroBetter has a pretty nice series of career profiles:
\begin{itemize}
\item {} 
\sphinxhref{https://www.astrobetter.com/blog/2014/09/25/career-profiles-astronomer-to-defense-researcher/}{Astronomer to Defense
Researcher}

\item {} 
\sphinxhref{https://www.astrobetter.com/blog/2014/10/02/career-profiles-astronomer-to-data-visualization-specialist-and-adjunct-associate-professor/}{Astronomer to Data Visualization Specialist and Adjunct Associate
Professor}

\item {} 
\sphinxhref{https://www.astrobetter.com/blog/2014/09/18/career-profiles-astronomer-to-chief-of-the-nautical-almanac-office-at-the-us-naval-observatory/}{Astronomer to Chief of the Nautical Almanac Office at the US
Naval
Observatory}

\item {} 
\sphinxhref{https://www.astrobetter.com/blog/2014/09/11/career-profiles-astronomer-to-associate-director-of-the-nasa-lunar-science-institute/}{Astronomer to Associate Director of the NASA Lunar Science
Institute}

\item {} 
\sphinxhref{https://www.astrobetter.com/blog/2014/08/28/career-profiles-astronomer-to-university-administrator-in-a-center-for-teaching-learning/}{Astronomer to University Administrator in a Center for Teaching \&
Learning}

\item {} 
\sphinxhref{https://www.astrobetter.com/blog/2014/09/04/career-profiles-astronomer-to-senior-editor-for-nature/}{Astronomer to Senior Editor for
Nature}

\item {} 
\sphinxhref{https://www.astrobetter.com/blog/2014/08/14/career-profiles-astronomer-to-president-of-a-defense-industry-company/}{Astronomer to President of a Defense Industry
Company}

\item {} 
\sphinxhref{https://www.astrobetter.com/blog/2014/07/31/career-profiles-astronomer-to-image-processor-for-stsci/}{Astronomer to Image Processor for
STScI}

\item {} 
\sphinxhref{https://www.astrobetter.com/blog/2014/07/24/career-profiles-astronomer-to-financial-analyst/}{Astronomer to Financial
Analyst}

\item {} 
\sphinxhref{https://www.astrobetter.com/blog/2014/07/17/career-profiles-astronomer-to-full-professor-of-physics-at-a-small-liberal-arts-college/}{Astronomer to Full Professor of Physics at a Small Liberal Arts
College}

\item {} 
\sphinxhref{https://www.astrobetter.com/blog/2014/07/10/career-profiles-astronomer-to-head-of-bioinformatics/}{Astronomer to Head of
Bioinformatics}

\item {} 
\sphinxhref{https://www.astrobetter.com/blog/2014/06/12/career-profiles-astronomer-to-data-scientist/}{Astronomer to Data
Scientist}

\item {} 
\sphinxhref{https://www.astrobetter.com/blog/2014/06/05/career-profiles-astronomer-to-tenure-track-faculty-at-a-california-community-college/}{Astronomer to Tenure Track Faculty at a California Community
College}

\item {} 
\sphinxhref{https://www.astrobetter.com/blog/2014/05/22/career-profiles-astronomer-to-tenure-track-faculty-at-a-teaching-focused-institution/}{Astronomer to Tenure Track Faculty at a Teaching-Focused
Institution}

\item {} 
\sphinxhref{https://www.astrobetter.com/blog/2014/05/01/career-profiles-astronomer-to-faculty-and-entrepreneur/}{Astronomer to Faculty and
Entrepreneur}

\item {} 
\sphinxhref{https://www.astrobetter.com/blog/2014/03/20/career-profiles-astronomer-to-senior-staff-scientist-in-industry/}{Astronomer to Senior Staff Scientist in
Industry}

\item {} 
\sphinxhref{https://www.astrobetter.com/blog/2014/03/13/career-profiles-astronomer-to-data-scientist-at-fidelity-investments/}{Astronomer to Data Scientist at Fidelity
Investments}

\item {} 
\sphinxhref{https://www.astrobetter.com/blog/2014/03/06/career-profiles-astronomer-to-director-for-the-centre-of-excellence-for-all-sky-astrophysics/}{Astronomer to Director for the Centre of Excellence for All-Sky
Astrophysics}

\end{itemize}

\end{itemize}


\section{About Writing Academic Paper}
\label{\detokenize{resource/research/writing_paper:about-writing-academic-paper}}\label{\detokenize{resource/research/writing_paper::doc}}
\begin{figure}[htbp]
\centering
\capstart

\noindent\sphinxincludegraphics{{keep_calm_write_on}.png}
\caption{Keep calm and write on}\label{\detokenize{resource/research/writing_paper:id1}}\end{figure}


\subsection{Basics}
\label{\detokenize{resource/research/writing_paper:basics}}\begin{itemize}
\item {} 
\sphinxhref{https://github.com/alexieleauthaud/RedWoodTools/wiki/Writing-Papers}{Notes on writing papers by Prof. Alexie
Leauthaud}

\end{itemize}


\subsubsection{Examples}
\label{\detokenize{resource/research/writing_paper:examples}}\begin{itemize}
\item {} 
Good example of papers with good reproducibility:
\begin{itemize}
\item {} 
\sphinxhref{https://ui.adsabs.harvard.edu/\#abs/arXiv:1810.06559}{STARRY: Analytic Occultation Light
Curves}

\item {} 
Format, equation, figure, open source code, and awesomefont links.

\end{itemize}

\end{itemize}


\subsection{English}
\label{\detokenize{resource/research/writing_paper:english}}\begin{itemize}
\item {} 
\sphinxhref{https://www.grammarly.com}{Grammarly}

\item {} 
\sphinxhref{https://en.linguee.com/english-chinese}{Linguee - Good place to search for example usage of a
word}

\item {} 
\sphinxhref{https://linggle.com}{Linggle 10\textasciicircum{}12 - Another place to search for real world usage of a
word}

\end{itemize}


\subsection{Useful Tools}
\label{\detokenize{resource/research/writing_paper:useful-tools}}\begin{itemize}
\item {} 
\sphinxhref{https://github.com/yymao/adstex}{adstex - manage your bibtex
entries}
\begin{itemize}
\item {} 
By Yao-Yuan Mao; Automated generation of NASA ADS bibtex entries
directly from citation keys in your TeX source files

\end{itemize}

\item {} 
\sphinxhref{http://astrofrog.github.io/acknowledgment-generator/}{The Astronomy Acknowledgement
Generator}
\begin{itemize}
\item {} 
This is very useful! The data are kept
\sphinxhref{https://github.com/astrofrog/acknowledgment-generator}{here}

\end{itemize}

\item {} 
\sphinxhref{https://pdfcompressor.com}{PDF Compressor}
\begin{itemize}
\item {} 
This is useful when you want to compress the quality of a figure.

\end{itemize}

\item {} 
\sphinxhref{https://tabula.technology}{Tabula - Tool for liberating data tables locked inside PDF
files}

\end{itemize}


\subsection{LaTeX}
\label{\detokenize{resource/research/writing_paper:latex}}

\subsubsection{Online LaTeX Editor}
\label{\detokenize{resource/research/writing_paper:online-latex-editor}}\begin{itemize}
\item {} 
\sphinxhref{https://www.overleaf.com}{Overleaf}
\begin{itemize}
\item {} 
The most commonly used one. The free service is already very good.

\end{itemize}

\item {} 
\sphinxhref{https://www.authorea.com/}{Authorea}
\begin{itemize}
\item {} 
A powerful publishing platform for articles, data, figures,
preprints. More than just a LaTeX editor.

\end{itemize}

\end{itemize}


\subsubsection{LaTeX resources}
\label{\detokenize{resource/research/writing_paper:latex-resources}}\begin{itemize}
\item {} 
\sphinxhref{https://math.uoregon.edu/wp-content/uploads/2014/12/compsymb-1qyb3zd.pdf}{Complete list of LaTeX symbols
(PDF)}

\item {} 
\sphinxhref{https://oeis.org/wiki/List\_of\_LaTeX\_mathematical\_symbols}{List of LaTeX mathematical
symbols}

\item {} 
\sphinxhref{https://en.wikibooks.org/wiki/LaTeX}{LaTeX open book on
WikiBooks}
\begin{itemize}
\item {} 
Very good place to search for basic usage.

\end{itemize}

\item {} 
\sphinxhref{https://en.wikibooks.org/wiki/LaTeX/Colors}{List of color names in LaTeX along with other useful tips about
color}

\end{itemize}


\subsubsection{Useful LaTeX tips}
\label{\detokenize{resource/research/writing_paper:useful-latex-tips}}\begin{itemize}
\item {} 
\sphinxhref{https://tex.stackexchange.com/questions/280462/link-to-arbitrary-part-of-text}{Link to aribitary part of
text}
\begin{itemize}
\item {} 
Using \sphinxstylestrong{\textbackslash{}hypertarget} and \sphinxstylestrong{\textbackslash{}hyperlink}

\end{itemize}

\end{itemize}


\subsection{Figure}
\label{\detokenize{resource/research/writing_paper:figure}}

\subsubsection{Flowchart}
\label{\detokenize{resource/research/writing_paper:flowchart}}\begin{itemize}
\item {} 
\sphinxhref{https://github.com/knsv/mermaid}{mermaid - Generation of diagram and flowchart from text in a similar
manner as markdown}
\begin{itemize}
\item {} 
\sphinxstylestrong{mermaid}, a simple markdown-like script language for generating
charts from text via javascript.

\end{itemize}

\end{itemize}


\section{Statistics, Data Science, \& Machine Learning}
\label{\detokenize{stats_data_ml:statistics-data-science-machine-learning}}\label{\detokenize{stats_data_ml::doc}}\begin{itemize}
\item {} 
As astronomy becomes more and more data intensive, as cosmology becomes more and more accurate,
statistical modeling and even machine learning are becoming the most promising and perhaps the
only way for astronomers to tackle increasingly difficult questions.

\item {} 
Not all scientisits with a astronomy background are literate in the language of statistics, or
(real) data science, or machine learning, it is important to acknowledge this and keep learning.

\end{itemize}


\subsection{Contents}
\label{\detokenize{stats_data_ml:contents}}

\subsubsection{Learning Material for Basic Statistical Methods}
\label{\detokenize{resource/research/stats_basic:learning-material-for-basic-statistical-methods}}\label{\detokenize{resource/research/stats_basic::doc}}

\paragraph{Interactive Courses}
\label{\detokenize{resource/research/stats_basic:interactive-courses}}\begin{itemize}
\item {} 
\sphinxhref{https://github.com/KIPAC/StatisticalMethods}{Special Topics in Astrophysics: Statistical Methods @
Stanford}
\begin{itemize}
\item {} 
\sphinxhref{https://github.com/KIPAC/StatisticalMethods/tree/master/chunks}{There are a lot of useful notebooks using astrophysical problems
as
examples}

\end{itemize}

\end{itemize}


\subsubsection{Basic Data Science Related Topics}
\label{\detokenize{resource/research/data_science:basic-data-science-related-topics}}\label{\detokenize{resource/research/data_science::doc}}

\paragraph{Baisc Tutorials}
\label{\detokenize{resource/research/data_science:baisc-tutorials}}\begin{itemize}
\item {} 
\sphinxhref{https://github.com/jakevdp/PythonDataScienceHandbook}{Python Data Science
Handbook}
\begin{itemize}
\item {} 
Free book on using Python for data science. Can read
\sphinxhref{https://jakevdp.github.io/PythonDataScienceHandbook/}{here}

\end{itemize}

\item {} 
\sphinxhref{https://github.com/Quartz/bad-data-guide\#name-order-is-inconsistent}{The Quartz Guide to Bad
Data}
\begin{itemize}
\item {} 
An exhaustive reference to problems seen in real-world data along
with suggestions on how to resolve them

\end{itemize}

\end{itemize}


\subsubsection{Focusing on Machine Learning}
\label{\detokenize{stats_data_ml:focusing-on-machine-learning}}

\paragraph{Basics of Machine Learning}
\label{\detokenize{resource/research/machine_learning/machine_learning_basic:basics-of-machine-learning}}\label{\detokenize{resource/research/machine_learning/machine_learning_basic::doc}}\begin{itemize}
\item {} 
Here is a “short” list of the most popular and up-to-date resources
for learning machine learning and AI related topics.

\item {} 
“You should be practicing!”

\end{itemize}


\bigskip\hrule\bigskip



\subparagraph{Basic Courses}
\label{\detokenize{resource/research/machine_learning/machine_learning_basic:basic-courses}}\begin{itemize}
\item {} 
\sphinxhref{https://see.stanford.edu/Course/CS229}{Stanford CS229 - Machine
Learning}
\begin{itemize}
\item {} 
By Andrew Ng. This course provides a broad introduction to machine
learning and statistical pattern recognition. Videos and lectures
are available freely.

\item {} 
\sphinxhref{https://github.com/Kivy-CN/Stanford-CS-229-CN}{斯坦福机器学习CS229课程讲义的中文翻译}

\end{itemize}

\item {} 
\sphinxhref{http://www.cs.cornell.edu/courses/cs4780/2018fa/}{Cornell CS4780/CS5780 - Machine Learning for Intelligent
Systems}
\begin{itemize}
\item {} 
Lectures and video recordings are available for free.

\end{itemize}

\end{itemize}


\subparagraph{Handy Cheatsheets}
\label{\detokenize{resource/research/machine_learning/machine_learning_basic:handy-cheatsheets}}\begin{itemize}
\item {} 
\sphinxhref{https://github.com/afshinea/stanford-cs-229-machine-learning}{VIP cheatsheets for Stanford’s CS 229 Machine
Learning}
\begin{itemize}
\item {} 
This one has a \sphinxhref{https://github.com/afshinea/stanford-cs-229-machine-learning/tree/master/zh}{Chinese
version}

\end{itemize}

\item {} 
\sphinxhref{https://github.com/kailashahirwar/cheatsheets-ai}{Essential Cheat Sheets for deep learning and machine learning
researchers}

\end{itemize}


\subparagraph{Curated and Awesome List of Resources}
\label{\detokenize{resource/research/machine_learning/machine_learning_basic:curated-and-awesome-list-of-resources}}\begin{itemize}
\item {} 
\sphinxhref{https://github.com/josephmisiti/awesome-machine-learning}{Awesome Machine
Learning}
\begin{itemize}
\item {} 
A curated list of awesome machine learning frameworks, libraries
and software (by language).

\end{itemize}

\item {} 
\sphinxhref{https://github.com/ChristosChristofidis/awesome-deep-learning}{Awesome Deep
Learning}
\begin{itemize}
\item {} 
A curated list of awesome Deep Learning tutorials, projects and
communities

\end{itemize}

\item {} 
\sphinxhref{https://github.com/jtoy/awesome-tensorflow}{Awesome Tensorflow}
\begin{itemize}
\item {} 
A curated list of dedicated resources

\end{itemize}

\end{itemize}


\subparagraph{Papers and Algorithms}
\label{\detokenize{resource/research/machine_learning/machine_learning_basic:papers-and-algorithms}}\begin{itemize}
\item {} 
\sphinxhref{https://github.com/hindupuravinash/the-gan-zoo}{The GAN Zoo - A list of all named
GANs}

\item {} 
\sphinxhref{https://github.com/wiseodd/generative-models}{generative-models - Collection of generative models, e.g. GAN, VAE
in Pytorch and
Tensorflow}

\item {} 
\sphinxhref{https://github.com/PatWie/tensorflow-recipes}{tensorflow-recipes - A collection of TensorFlow (Tensorpack)
implementations of recent deep learning approaches including
pretrained models}

\end{itemize}


\subparagraph{Materials In Chinese}
\label{\detokenize{resource/research/machine_learning/machine_learning_basic:materials-in-chinese}}\begin{itemize}
\item {} 
\sphinxhref{https://github.com/datawhalechina/pumpkin-book}{PumpkinBook -
《机器学习》(西瓜书)公式推导解析}
\begin{itemize}
\item {} 
In Chinese. Read online
\sphinxhref{https://datawhalechina.github.io/pumpkin-book/}{here}

\end{itemize}

\item {} 
\sphinxhref{https://zh.d2l.ai/index.html}{动手学深度学习 -
面向中文读者的能运行、可讨论的深度学习教科书}

\item {} 
\sphinxhref{https://github.com/fengdu78/Coursera-ML-AndrewNg-Notes}{斯坦福大学2014(吴恩达)机器学习教程中文笔记}

\item {} 
\sphinxhref{https://github.com/scutan90/DeepLearning-500-questions}{DeepLearning-500-questions -
深度学习500问}
\begin{itemize}
\item {} 
Deep learning through 500 questions, a Markdown book.

\end{itemize}

\end{itemize}


\subparagraph{Tutorial and Examples}
\label{\detokenize{resource/research/machine_learning/machine_learning_basic:tutorial-and-examples}}\begin{itemize}
\item {} 
\sphinxhref{https://github.com/eriklindernoren/ML-From-Scratch}{Machine Learning From
Scratch}
\begin{itemize}
\item {} 
Bare bones NumPy implementations of machine learning models and
algorithms with a focus on accessibility. Aims to cover everything
from linear regression to deep learning

\end{itemize}

\item {} 
\sphinxhref{https://github.com/trekhleb/homemade-machine-learning}{Homemade Machine
Learning}
\begin{itemize}
\item {} 
Python examples of popular machine learning algorithms with
interactive Jupyter demos and math being explained

\end{itemize}

\item {} 
\sphinxhref{https://github.com/rasbt/deeplearning-models}{Deep Learning
Models}
\begin{itemize}
\item {} 
A collection of various deep learning architectures, models, and
tips for TensorFlow and PyTorch in Jupyter Notebooks

\end{itemize}

\end{itemize}


\subparagraph{More Serious Stuff}
\label{\detokenize{resource/research/machine_learning/machine_learning_basic:more-serious-stuff}}\begin{itemize}
\item {} 
\sphinxhref{https://github.com/floodsung/Deep-Learning-Papers-Reading-Roadmap}{Deep Learning Papers Reading
Roadmap}

\end{itemize}


\paragraph{Collection of Useful Tools for Machine Learning and Artificial Intelligence}
\label{\detokenize{resource/research/machine_learning/machine_learning_tools:collection-of-useful-tools-for-machine-learning-and-artificial-intelligence}}\label{\detokenize{resource/research/machine_learning/machine_learning_tools::doc}}

\subparagraph{Basic Tools}
\label{\detokenize{resource/research/machine_learning/machine_learning_tools:basic-tools}}\begin{itemize}
\item {} 
\sphinxhref{https://github.com/HIPS/autograd}{autograd - Efficiently computes derivatives of numpy
code}
\begin{itemize}
\item {} 
Very useful tool when building statistical and machine learning
tools in Python. \sphinxstylestrong{Autograd} can automatically differentiate
native Python and Numpy code.

\end{itemize}

\end{itemize}


\subparagraph{Machine Learning Framework}
\label{\detokenize{resource/research/machine_learning/machine_learning_tools:machine-learning-framework}}\begin{itemize}
\item {} 
\sphinxhref{https://github.com/tensorflow/tensorflow}{tensorflow - An Open Source Machine Learning Framework for
Everyone}
\begin{itemize}
\item {} 
TensorFlow is an open source software library for numerical
computation using data flow graphs. The graph nodes represent
mathematical operations, while the graph edges represent the
multidimensional data arrays (tensors) that flow between them.
This flexible architecture enables you to deploy computation to
one or more CPUs or GPUs in a desktop, server, or mobile device
without rewriting code.

\item {} 
\sphinxhref{https://www.tensorflow.org/tutorials/}{The official tutorial of
tensorflow}

\item {} 
\sphinxhref{https://github.com/tensorflow/models}{Models and examples built with
TensorFlow}

\item {} 
\sphinxhref{https://github.com/jikexueyuanwiki/tensorflow-zh}{TensorFlow
官方文档中文版}

\item {} 
\sphinxhref{https://github.com/aymericdamien/TensorFlow-Examples}{TensorFlow-Examples - TensorFlow Tutorial and Examples for
Beginners (support TF v1 \&
v2)}

\item {} 
\sphinxhref{https://github.com/jtoy/awesome-tensorflow}{awesome-tensorflow - A curated list of dedicated
resources}

\end{itemize}

\item {} 
\sphinxhref{https://github.com/keras-team/keras}{keras - Deep Learning for
humans}
\begin{itemize}
\item {} 
Keras is a high-level neural networks API, written in Python and
capable of running on top of TensorFlow, CNTK, or Theano. It was
developed with a focus on enabling fast experimentation.

\item {} 
\sphinxhref{https://keras.io/}{Official documents of keras}

\item {} 
\sphinxhref{https://github.com/fchollet/deep-learning-models}{Keras code and weights files for popular deep learning models by
the author}
(A little out of date.)

\item {} 
\sphinxstylestrong{Keras} also lives in \sphinxstylestrong{tensorflow} now: “\sphinxcode{\sphinxupquote{tf.keras}} is
TensorFlow’s implementation of the Keras API specification. This
is a high-level API to build and train models that includes
first-class support for TensorFlow-specific functionality”

\end{itemize}

\item {} 
\sphinxhref{https://github.com/pytorch/pytorch}{pytorch - Tensors and Dynamic neural networks in Python with strong
GPU acceleration}
\begin{itemize}
\item {} 
Usually one uses PyTorch either as 1) a replacement for NumPy to
use the power of GPUs; 2) a deep learning research platform that
provides maximum flexibility and speed.

\end{itemize}

\item {} 
\sphinxhref{https://github.com/scikit-learn/scikit-learn}{scikit-learn - machine learning in
Python}
\begin{itemize}
\item {} 
\sphinxstylestrong{scikit-learn} is a Python module for machine learning built on
top of SciPy

\end{itemize}

\item {} 
\sphinxhref{https://github.com/BVLC/caffe}{caffe - a fast open framework for deep
learning}
\begin{itemize}
\item {} 
Caffe is a deep learning framework made with expression, speed,
and modularity in mind

\end{itemize}

\end{itemize}


\subparagraph{Reinformcement Learning}
\label{\detokenize{resource/research/machine_learning/machine_learning_tools:reinformcement-learning}}\begin{itemize}
\item {} 
\sphinxhref{https://github.com/openai/gym}{gym - A toolkit for developing and comparing reinforcement learning
algorithms}
\begin{itemize}
\item {} 
OpenAI Gym is a toolkit for developing and comparing reinforcement
learning algorithms.

\item {} 
\sphinxhref{https://gym.openai.com/}{gym project website}

\end{itemize}

\end{itemize}


\subparagraph{Visualization}
\label{\detokenize{resource/research/machine_learning/machine_learning_tools:visualization}}\begin{itemize}
\item {} 
\sphinxhref{https://github.com/HarisIqbal88/PlotNeuralNet}{PlotNeuralNet - Latex code for making neural networks
diagrams}
\begin{itemize}
\item {} 
By Haris Iqbal. Latex code for drawing neural networks for reports
and presentation.

\end{itemize}

\end{itemize}


\paragraph{Astrophysical and Cosmological Applications of Machine Learnings}
\label{\detokenize{resource/research/machine_learning/mlearning_astro_application:astrophysical-and-cosmological-applications-of-machine-learnings}}\label{\detokenize{resource/research/machine_learning/mlearning_astro_application::doc}}\begin{itemize}
\item {} 
This is Jun, 2019, and an ADS search of “machine learning” in the
abstract of referred astronomy journals result in \sphinxstylestrong{711} papers, so
it is almost too late to try to “read everything”…but at least we can
try:

\item {} 
And, again, this list only reflects one person’s taste.

\end{itemize}


\bigskip\hrule\bigskip



\subparagraph{Galaxy Morphology}
\label{\detokenize{resource/research/machine_learning/mlearning_astro_application:galaxy-morphology}}\begin{itemize}
\item {} 
\sphinxhref{https://morpheus-project.github.io/morpheus/}{Morpheus: A Deep Learning Framework For Pixel-Level Analysis of
Astronomical Image
Data}
\begin{itemize}
\item {} 
By Ryan Hausen \& Brant Robertson. Based on the \sphinxhref{https://arxiv.org/pdf/1906.11248.pdf}{work
here}. \sphinxhref{https://github.com/morpheus-project/morpheus}{Source code is on
Github}.

\item {} 
\sphinxstylestrong{Morpheus} is a neural network model used to generate
pixel-level morphological classifications for astronomical
sources. This model can be used to generate segmentation maps or
to inform other photometric measurements with granular
morphological information. Built on \sphinxstylestrong{Tensorflow}.

\item {} 
\sphinxhref{https://morpheus-astro.readthedocs.io/en/latest/}{Online document is
here}; and
\sphinxhref{https://github.com/morpheus-project/morpheus/blob/master/examples/example\_array.ipynb}{this
notebook}
is a good place to get started.

\end{itemize}

\item {} 
\sphinxhref{https://ui.adsabs.harvard.edu/abs/2015MNRAS.450.1441D/abstract}{Rotation-invariant convolutional neural networks for galaxy
morphology
prediction}

\end{itemize}


\subparagraph{Photometric Redshift}
\label{\detokenize{resource/research/machine_learning/mlearning_astro_application:photometric-redshift}}\begin{itemize}
\item {} 
\sphinxhref{https://ui.adsabs.harvard.edu/abs/2013MNRAS.432.1483C/abstract}{TPZ: photometric redshift PDFs and ancillary information by using
prediction trees and random
forests}
\begin{itemize}
\item {} 
Decision tree and random forest.

\end{itemize}

\end{itemize}


\subparagraph{Cosmology}
\label{\detokenize{resource/research/machine_learning/mlearning_astro_application:cosmology}}\begin{itemize}
\item {} 
\sphinxhref{https://knaidoo29.github.io/mistreedoc/index.html}{MiSTree - Beyond two-point statistics: using the Minimum Spanning
Tree as a tool for
cosmology}
\begin{itemize}
\item {} 
The \sphinxhref{https://github.com/knaidoo29/mistree}{public code can be found
here}

\end{itemize}

\end{itemize}

\begin{sphinxadmonition}{note}{Note:}
These collections are still highly incomplete and reflect the curator’s limited knowledge on this
topic. Any help will be highly appreciated!
\end{sphinxadmonition}


\chapter{Astronomy, Astrophysics, \& Cosmology}
\label{\detokenize{index:astronomy-astrophysics-cosmology}}

\section{Astro README.md}
\label{\detokenize{resource/astro/astro_readme:astro-readme-md}}\label{\detokenize{resource/astro/astro_readme::doc}}\begin{itemize}
\item {} 
This is a collection of very good intro-level notes/lectures/manuals
that cover a broad range of topics. Although it is not possible to
read all of them before you do research, but it is helpful to know
they exist.

\end{itemize}


\subsection{Online or Free Textbook}
\label{\detokenize{resource/astro/astro_readme:online-or-free-textbook}}\begin{itemize}
\item {} 
\sphinxhref{https://casper.ssl.berkeley.edu/astrobaki/index.php/Main\_Page}{AstroBaki}
\begin{itemize}
\item {} 
AstroBaki is a wiki for current and aspiring scientists to
collaboratively build pedagogical materials such as videos,
lecture notes, and textbooks.

\end{itemize}

\item {} 
\sphinxhref{https://open-astrophysics-bookshelf.github.io/}{Open Astrophysics
Bookshelf}
\begin{itemize}
\item {} 
Open Licensed Astrophysics Texts; Open for Contributions. Now
includes:

\item {} 
\sphinxhref{https://github.com/Open-Astrophysics-Bookshelf/numerical\_exercises}{Computational Hydrodynamics for
Astrophysics}

\item {} 
\sphinxhref{https://github.com/Open-Astrophysics-Bookshelf/star\_formation\_notes}{Notes on Star Formation by Mark
Krumholz}

\item {} 
\sphinxhref{https://github.com/Open-Astrophysics-Bookshelf/astrophysical\_processes\_notes}{Astrophysical Processes
Notes}

\item {} 
\sphinxhref{https://github.com/Open-Astrophysics-Bookshelf/intro-stellar-physics}{Introduction to Stellar Astrophysics for
Undergraduate}

\item {} 
\sphinxhref{https://github.com/Open-Astrophysics-Bookshelf/stellar-physics-notes}{Stellar Astrophysics for Graduate
Student}

\end{itemize}

\item {} 
\sphinxhref{https://ay201b.wordpress.com/}{ISM and Star Formation based on Harvard Astronomy
201B}
\begin{itemize}
\item {} 
Jointly-Edited Online “Book”, first created in 2011 by
Prof. Alyssa Goodman, Teaching Fellow Chris Beaumont, and the 21
Harvard graduate students who took the course at that time.

\end{itemize}

\end{itemize}


\subsection{User Manual for Astronomer}
\label{\detokenize{resource/astro/astro_readme:user-manual-for-astronomer}}\begin{itemize}
\item {} 
Below is a list of (relative short) manual, guide, tutorial on some
specific but important topics in astrophysical research. You don’t
need to read all of them (although you are encouraged to do so…), but
you should know they are available.

\end{itemize}


\subsubsection{Math, Oh Math…}
\label{\detokenize{resource/astro/astro_readme:math-oh-math}}\begin{itemize}
\item {} 
\sphinxhref{https://www.math.uwaterloo.ca/~hwolkowi/matrixcookbook.pdf}{The Matrix
Cookbook}
\begin{itemize}
\item {} 
Quick desktop reference on everything related to matrix.

\end{itemize}

\item {} 
\sphinxhref{https://arxiv.org/abs/0906.4123}{Fisher Matrices and Confidence Ellipses: A Quick-Start Guide and
Software}
\begin{itemize}
\item {} 
Also see \sphinxhref{http://wittman.physics.ucdavis.edu/Fisher-matrix-guide.pdf}{Fisher Matrix for
Beginners}

\item {} 
And \sphinxhref{https://arxiv.org/pdf/1705.01064.pdf}{A Tutorial on Fisher
Information}

\end{itemize}

\item {} 
\sphinxhref{https://users.aalto.fi/~mvermeer/uncertainty.pdf}{Statistical uncertainty and error
propagation}
\begin{itemize}
\item {} 
Quick summary of statistical error propagation.

\end{itemize}

\item {} 
\sphinxhref{https://www.math.uwaterloo.ca/~aghodsib/courses/f06stat890/readings/tutorial\_stat890.pdf}{Dimensionality Reduction A Short
Tutorial}

\end{itemize}


\subsubsection{Data Analysis}
\label{\detokenize{resource/astro/astro_readme:data-analysis}}\begin{itemize}
\item {} 
\sphinxhref{https://arxiv.org/abs/1905.13189}{A Beginner’s Guide to Working with Astronomical
Data}
\begin{itemize}
\item {} 
Basics about astronomical data analysis using \sphinxstylestrong{Python}.

\end{itemize}

\item {} 
\sphinxhref{http://adsabs.harvard.edu/abs/2013pss2.book...35M}{Astronomical
Spectroscopy}
\begin{itemize}
\item {} 
by Philp Massey and Margaret Hanson. Very good introduction about
the spectroscopic instrument and data reduction.

\end{itemize}

\item {} 
\sphinxhref{https://arxiv.org/pdf/1012.3754.pdf}{Dos and don’ts of reduced
chi-squared}
\begin{itemize}
\item {} 
By Rene Andrae, Tim Schulze-Hartung, \& Peter Melchior.

\end{itemize}

\item {} 
\sphinxhref{https://arxiv.org/pdf/1901.07726.pdf}{On Model Selection in
Cosmology}

\end{itemize}


\paragraph{Data Analysis Recipes by David Hogg}
\label{\detokenize{resource/astro/astro_readme:data-analysis-recipes-by-david-hogg}}\begin{itemize}
\item {} 
The GitHub repo for this book is
\sphinxhref{https://github.com/davidwhogg/DataAnalysisRecipes}{here}

\item {} 
\sphinxhref{https://arxiv.org/abs/0807.4820}{Data analysis recipes: Choosing the binning for a
histogram}

\item {} 
\sphinxhref{https://arxiv.org/abs/1008.4686}{Data analysis recipes: Fitting a model to
data}
\begin{itemize}
\item {} 
\sphinxstylestrong{This is really worth reading}

\end{itemize}

\item {} 
\sphinxhref{https://arxiv.org/abs/1205.4446}{Data analysis recipes: Probability calculus for
inference}

\item {} 
\sphinxhref{https://arxiv.org/abs/1710.06068}{Data analysis recipes: Using Markov Chain Monte
Carlo}

\end{itemize}


\subsubsection{Cosmology}
\label{\detokenize{resource/astro/astro_readme:cosmology}}\begin{itemize}
\item {} 
\sphinxhref{https://arxiv.org/abs/astro-ph/9905116}{Distance measures in cosmology by David
Hogg}

\item {} 
\sphinxhref{https://arxiv.org/pdf/astro-ph/0208512.pdf}{Exploring the Expansion History of the Universe by Eirc
Linder}

\item {} 
\sphinxhref{https://arxiv.org/abs/1308.4150}{Damn You, Little h! by Darren
Croton}
\begin{itemize}
\item {} 
The published version is
\sphinxhref{https://www.cambridge.org/core/journals/publications-of-the-astronomical-society-of-australia/article/damn-you-little-h-or-realworld-applications-of-the-hubble-constant-using-observed-and-simulated-data/EB4B786F4500F897A589C3ED980C17F5}{here}

\item {} 
This is a really useful note. \sphinxstylestrong{little-h} can be really tricky!

\item {} 
Also see \sphinxhref{http://www.astro.ljmu.ac.uk/~ikb/research/}{this conversion table by Ivan
Baldry}

\end{itemize}

\item {} 
\sphinxhref{https://arxiv.org/abs/1005.0411}{Dark Matter Halo Mass Profiles}
\begin{itemize}
\item {} 
Notes on basic NFW and Einasto profiles.

\end{itemize}

\end{itemize}


\section{Basic Tools for Astrophysical Research}
\label{\detokenize{resource/astro/astro_research_basic:basic-tools-for-astrophysical-research}}\label{\detokenize{resource/astro/astro_research_basic::doc}}

\subsection{Data format used in astronomy}
\label{\detokenize{resource/astro/astro_research_basic:data-format-used-in-astronomy}}\begin{itemize}
\item {} 
\sphinxhref{https://archive.stsci.edu/fits/}{Flexible Image Transport System
(FITS)}
\begin{itemize}
\item {} 
The default standard for the exchange of data in astronomy now

\item {} 
The \sphinxhref{https://fits.gsfc.nasa.gov/users\_guide/usersguide.pdf}{full document of
FITS}
and \sphinxhref{https://archive.stsci.edu/data\_format.html}{the MAST data format
guidelines}

\item {} 
\sphinxhref{https://heasarc.gsfc.nasa.gov/fitsio/}{CFITSIO} is a library
of C and Fortran subroutines for reading and writing data files in
FITS.

\item {} 
\sphinxhref{https://docs.astropy.org/en/stable/io/fits/}{astropy.io.fits}
can help you deal with FITS files in Python

\end{itemize}

\item {} 
\sphinxhref{https://github.com/spacetelescope/asdf}{Advanced Scientific Data Format
(ASDF)}
\begin{itemize}
\item {} 
Next generation interchange format for scientific data developed
by STScI

\item {} 
It follows the new \sphinxhref{https://asdf-standard.readthedocs.io/en/latest/}{ASDF
Standard}

\end{itemize}

\item {} 
\sphinxhref{https://www.hdfgroup.org/solutions/hdf5/}{HDF5 - High-performance data management and storage
suite}
\begin{itemize}
\item {} 
In the age of big data, HDF5 format is also becoming more popular
in astronomy. For example, many hydro-simulations use HDF5 to
store large and complex data.

\item {} 
\sphinxhref{https://www.h5py.org/}{h5py is a Pythonic interface to the HDF5 binary data
format}

\item {} 
\sphinxhref{https://github.com/telegraphic/fits2hdf}{fits2hdf is a FITS to HDFITS conversion
utility}

\end{itemize}

\item {} 
If you use Python, it is also convenient to use the
\sphinxhref{https://docs.python.org/3/library/pickle.html}{pickle} and
\sphinxhref{https://pypi.org/project/dill/}{dill} (an even better \sphinxstylestrong{pickle})
module for serializing and de-serializing python objects to the
majority of the built-in python types.

\end{itemize}


\subsection{World Coordinate Systems (WCS)}
\label{\detokenize{resource/astro/astro_research_basic:world-coordinate-systems-wcs}}\begin{itemize}
\item {} 
\sphinxhref{https://fits.gsfc.nasa.gov/fits\_wcs.html}{FITS World Coordinate System
(WCS)}
\begin{itemize}
\item {} 
\sphinxhref{http://www.atnf.csiro.au/people/mcalabre/WCS/wcslib/index.html}{wcslib}
is a C library, supplied with a full set of Fortran wrappers, that
implements the “World Coordinate System” (WCS) standard in FITS

\item {} 
All the tools that deals with FITS format can also deal with WCS
information (normally stored in the header). For example, the
\sphinxhref{http://docs.astropy.org/en/stable/wcs/}{astropy.wcs}

\end{itemize}

\item {} 
\sphinxhref{https://gwcs.readthedocs.io/en/latest/}{Generalized World Coordinate
System}
\begin{itemize}
\item {} 
\sphinxhref{https://github.com/spacetelescope/gwcs}{gwcs - provides tools for managing WCS in a general
way}

\end{itemize}

\item {} 
\sphinxhref{https://github.com/spacetelescope/stwcs}{stwcs - WCS based distortion models and coordinate
transformation}

\item {} 
\sphinxhref{https://github.com/spacetelescope/tweakwcs}{tweakwcs - Algorithms for matching and aligning catalogs and for
tweaking the WCS so as to minimize catalog mismatch
errors}
\begin{itemize}
\item {} 
\sphinxstylestrong{tweakwcs} is a package that provides core algorithms for
computing and applying corrections to WCS objects such as to
minimize mismatch between image and reference catalogs. Currently
only aligning images with FITS WCS and JWST gWCS are supported.

\end{itemize}

\end{itemize}


\subsection{Astrometry and Positional Astronomy}
\label{\detokenize{resource/astro/astro_research_basic:astrometry-and-positional-astronomy}}\begin{itemize}
\item {} 
\sphinxhref{https://rhodesmill.org/skyfield/}{Skyfield - Elegant Astronomy for
Python}
\begin{itemize}
\item {} 
\sphinxstylestrong{Skyfield} computes positions for the stars, planets, and
satellites in orbit around the Earth. Its results should agree
with the positions generated by the United States Naval
Observatory and their Astronomical Almanac to within 0.0005
arcseconds. The \sphinxhref{https://github.com/skyfielders/python-skyfield/}{Github version is
here}

\end{itemize}

\item {} 
\sphinxhref{https://github.com/dstndstn/astrometry.net}{Astrometry.net \textendash{} automatic recognition of astronomical
images}
\begin{itemize}
\item {} 
Made by Dustin Lang. The best astrometric calibration tool on the
market.

\end{itemize}

\item {} 
\sphinxhref{http://www.iausofa.org/tandc.html}{SOFA - Standard of Fundamental
Astronomy}
\begin{itemize}
\item {} 
The International Astronomical Union’s SOFA service has the task
of establishing and maintaining an accessible and authoritative
set of algorithms and procedures that implement standard models
used in fundamental astronomy. The library is now in \sphinxhref{http://www.iausofa.org/current.html}{Fortran and
C}

\item {} 
\sphinxhref{https://gitlab.com/deddy/pysofa2}{pysofa2 - A python module for wrapping the International
Astronomical Union’s C SOFA
libraries}

\end{itemize}

\item {} 
\sphinxhref{https://github.com/liberfa/erfa}{ERFA - Essential Routines for Fundamental
Astronomy}
\begin{itemize}
\item {} 
\sphinxstylestrong{ERFA} is a C library containing key algorithms for astronomy,
and is based on the \sphinxhref{http://www.iausofa.org/}{SOFA} library
published by the International Astronomical Union (IAU).

\end{itemize}

\item {} 
\sphinxhref{https://github.com/brandon-rhodes/python-novas}{NOVAS - The United States Naval Observatory NOVAS astronomy
library}
\begin{itemize}
\item {} 
\sphinxstylestrong{NOVAS} is an integrated package of functions for computing
various commonly needed quantities in positional astronomy. At a
lower level, NOVAS also provides astrometric utility
transformations, such as those for precession, nutation,
aberration, parallax, and gravitational deflection of light.

\item {} 
The computations are accurate to better than one milliarcsecond.

\end{itemize}

\item {} 
\sphinxhref{https://github.com/spacetelescope/spherical\_geometry}{spherical\_geometry - A Python package for handling spherical
polygons that represent arbitrary regions of the
sky}
\begin{itemize}
\item {} 
Made by STScI. \sphinxhref{https://spacetelescope.github.io/spherical\_geometry/spherical\_geometry/}{Online document is
here}

\end{itemize}

\end{itemize}


\subsection{Flux standards}
\label{\detokenize{resource/astro/astro_research_basic:flux-standards}}\begin{itemize}
\item {} 
\sphinxhref{http://mips.as.arizona.edu/~cnaw/sun.html}{The Absolute Magnitude of the
Sun}
\begin{itemize}
\item {} 
\sphinxhref{http://mips.as.arizona.edu/~cnaw/sun\_composite.fits}{Link to download the default composite spectrum of the
Sun}

\end{itemize}

\item {} 
\sphinxhref{https://www.eso.org/sci/observing/tools/standards/spectra/stanlis.html}{ESO’s RA Ordered List of Spectrophotometric
Standards}

\item {} 
\sphinxhref{https://www.naoj.org/Observing/Instruments/FOCAS/Detail/UsersGuide/Observing/StandardStar/Spec/SpecStandard.html}{NAOJ’s Optical Spectrophotometric Standard
Stars}

\item {} 
\sphinxhref{http://www.eso.org/sci/observing/tools/standards/Landolt.html}{ESO’s Landolt Equatorial
Standards}

\end{itemize}


\subsection{Plan an observation}
\label{\detokenize{resource/astro/astro_research_basic:plan-an-observation}}\begin{itemize}
\item {} 
\sphinxhref{https://github.com/astropy/astroplan}{astroplan - Python package to help astronomers plan
observations}

\item {} 
\sphinxhref{http://onekilopars.ec/iobserve/}{iObserve}
\begin{itemize}
\item {} 
The free on-line version is
\sphinxhref{https://www.arcsecond.io/iobserve}{here}

\end{itemize}

\item {} 
\sphinxhref{http://catserver.ing.iac.es/staralt/}{Object Visibility by ING}

\item {} 
\sphinxhref{https://www.dartmouth.edu/~physics/labs/skycalc/flyer.html}{JSkyCalc \textendash{} A Convenient, Portable Observing
Aid}

\item {} 
\sphinxhref{https://www2.keck.hawaii.edu/software/obsplan/obsplan.php}{Keck telescopes’ own planning
tool}

\item {} 
\sphinxhref{http://www.eso.org/~ndelmott/obs\_sites.html}{Astronomical Observatory Sites by Latitude and
Longitude}

\end{itemize}


\subsection{Flux unit conversion}
\label{\detokenize{resource/astro/astro_research_basic:flux-unit-conversion}}\begin{itemize}
\item {} 
\sphinxhref{http://www.stsci.edu/hst/nicmos/tools/conversion\_form.html}{NICMOS Units Conversion
Form}

\item {} 
\sphinxhref{http://ssc.spitzer.caltech.edu/warmmission/propkit/pet/magtojy/}{Magnitude/Flux Density Converter: Point
Sources}

\end{itemize}


\subsection{Filter Response Curves}
\label{\detokenize{resource/astro/astro_research_basic:filter-response-curves}}\begin{itemize}
\item {} 
\sphinxhref{https://github.com/bd-j/sedpy/tree/master/sedpy/data/filters}{Response curved collected in sedpy package by Ben
Johnson}

\item {} 
\sphinxhref{https://github.com/bmorris3/tynt}{tynt - Color filter approximations in
Python}
\begin{itemize}
\item {} 
By Brett Morris. \sphinxstylestrong{tynt} is a super lightweight package
containing approximate transmittance curves for more than five
hundred astronomical filters, weighing in at just under 150 KB.
Document can be found
\sphinxhref{https://tynt.readthedocs.io/en/latest/}{here}

\end{itemize}

\end{itemize}


\subsection{Extinction calculator}
\label{\detokenize{resource/astro/astro_research_basic:extinction-calculator}}\begin{itemize}
\item {} 
\sphinxhref{https://ned.ipac.caltech.edu/extinction\_calculator}{NED’s extinction
calculator}

\item {} 
\sphinxhref{http://argonaut.skymaps.info}{3-D Dust Mapping with Pan-STARRS 1}
\begin{itemize}
\item {} 
\sphinxhref{http://argonaut.skymaps.info/query}{Query service}

\end{itemize}

\end{itemize}


\subsection{Coordinate Service}
\label{\detokenize{resource/astro/astro_research_basic:coordinate-service}}\begin{itemize}
\item {} 
\sphinxhref{http://www.astrouw.edu.pl/~jskowron/ra-dec/}{RA, DEC Flexible
converter}

\item {} 
\sphinxhref{https://ned.ipac.caltech.edu/coordinate\_calculator}{NED’s coordinate
calculator}

\end{itemize}


\subsection{Managing Catalogs}
\label{\detokenize{resource/astro/astro_research_basic:managing-catalogs}}\begin{itemize}
\item {} 
\sphinxhref{https://docs.astropy.org/en/stable/table/}{astropy.table} is a
flexible Python module that can handle a variant types of tables.

\item {} 
\sphinxhref{http://www.star.bris.ac.uk/~mbt/topcat/}{TOPCAT - Tool for OPerations on Catalogues And
Tables}
\begin{itemize}
\item {} 
Really powerful GUI tool to deal with tables. It is written as a
Java application.

\item {} 
Have great functions for cross-matching catalogs, querying on-line
databases, and making publication-grade figures.

\item {} 
\sphinxhref{http://www.star.bris.ac.uk/~mbt/stilts/}{STILTS - Starlink Tables Infrastructure Library Tool
Set} provides most of
\sphinxstylestrong{TOPCAT}’s capabilities in command line.

\end{itemize}

\end{itemize}


\subsubsection{Table cross-match}
\label{\detokenize{resource/astro/astro_research_basic:table-cross-match}}\begin{itemize}
\item {} 
\sphinxhref{https://github.com/esheldon/smatch}{smatch - Code to match points on the sphere using the healpix
scheme}
\begin{itemize}
\item {} 
By Erin Sheldon. Very fast cross-match tool, C-code wrapped in
Python.

\end{itemize}

\item {} 
\sphinxhref{https://github.com/JohannesBuchner/nway}{nway - Bayesian cross-matching of astronomical
catalogues}
\begin{itemize}
\item {} 
Bayesian match probabilities based on astronomical sky coordinates
(RA, DEC)

\end{itemize}

\item {} 
\sphinxhref{http://dame.dsf.unina.it/c3.html}{C3 - Command-line Catalogue Cross-matching
Tool}

\item {} 
\sphinxhref{https://git.ias.u-psud.fr/abeelen/xmatch}{xmatch - Cross match of
catalogs}
\begin{itemize}
\item {} 
By Alexandre Beelen.

\end{itemize}

\end{itemize}


\subsection{Cosmology Calculator}
\label{\detokenize{resource/astro/astro_research_basic:cosmology-calculator}}\begin{itemize}
\item {} 
\sphinxhref{http://cosmocalc.icrar.org}{ICRAR’s Cosmology Calculator}
\begin{itemize}
\item {} 
Based on the \sphinxhref{https://github.com/asgr/celestial}{celestial
R-code}

\end{itemize}

\item {} 
\sphinxhref{http://www.astro.ucla.edu/\%7Ewright/CosmoCalc.html}{Ned Wright’s Cosmology
Calculator}

\end{itemize}


\subsection{Simulate Galaxy Spectrum}
\label{\detokenize{resource/astro/astro_research_basic:simulate-galaxy-spectrum}}\begin{itemize}
\item {} 
\sphinxhref{http://specgen.icrar.org}{SpecGen - Mock Galaxy Spectra
Generator}

\end{itemize}


\subsection{Sky Projection Maps of Surveys}
\label{\detokenize{resource/astro/astro_research_basic:sky-projection-maps-of-surveys}}\begin{itemize}
\item {} 
\sphinxhref{http://astromap.icrar.org}{AstroMap - generating sky projection maps for astronomical
surveys}

\end{itemize}


\subsection{List of Observatories}
\label{\detokenize{resource/astro/astro_research_basic:list-of-observatories}}\begin{itemize}
\item {} 
\sphinxhref{https://en.wikipedia.org/wiki/List\_of\_astronomical\_observatories}{List of Astronomical Observatories on
Wiki}

\item {} 
\sphinxhref{https://www.go-astronomy.com/observatories.htm}{Observatories by
GoAstronomy}
\begin{itemize}
\item {} 
Including U.S observatories and observatories worldwide.

\end{itemize}

\end{itemize}


\subsection{Basic Astronomy Database}
\label{\detokenize{resource/astro/astro_research_basic:basic-astronomy-database}}\begin{itemize}
\item {} 
\sphinxhref{http://simbad.u-strasbg.fr/simbad/}{SIMBAD Astronomical Database -
CDS}
\begin{itemize}
\item {} 
The \sphinxstylestrong{SIMBAD} astronomical database provides basic data,
cross-identifications, bibliography and measurements for
astronomical objects outside the solar system.

\end{itemize}

\item {} 
\sphinxhref{http://vizier.u-strasbg.fr/viz-bin/VizieR}{VizieR Catalog
Database}
\begin{itemize}
\item {} 
\sphinxstylestrong{VizieR} provides the most complete library of published
astronomical catalogues -tables and associated data- with verified
and enriched data, accessible via multiple interfaces.

\end{itemize}

\item {} 
\sphinxhref{https://aladin.u-strasbg.fr/aladin.gml\#}{Aladin Sky Atlas}
\begin{itemize}
\item {} 
\sphinxstylestrong{Aladin} is an interactive sky atlas allowing the user to
visualize digitized astronomical images or full surveys,
superimpose entries from astronomical catalogues or databases, and
interactively access related data and information from the Simbad
database, the VizieR service and other archives for all known
astronomical objects in the field.

\end{itemize}

\item {} 
\sphinxhref{https://ned.ipac.caltech.edu/}{NASA/IPAC Extragalactic Database}
\begin{itemize}
\item {} 
\sphinxstylestrong{NED} is a comprehensive database of multiwavelength data for
extragalactic objects, providing a systematic, ongoing fusion of
information integrated from hundreds of large sky surveys and tens
of thousands of research publications.

\end{itemize}

\item {} 
\sphinxhref{https://irsa.ipac.caltech.edu/frontpage/}{NASA/IPAC Infrared Science
Archive}
\begin{itemize}
\item {} 
\sphinxstylestrong{IRSA} is chartered to curate the science products of NASA’s
infrared and submillimeter missions, including many large-area and
all-sky surveys.

\end{itemize}

\item {} 
\sphinxhref{http://archive.stsci.edu/}{MAST - Mikulski Archive for Space
Telescopes}
\begin{itemize}
\item {} 
\sphinxstylestrong{MAST} provides a variety of astronomical archives focused on
scientific data sets in the optical, ultraviolet, and
near-infrared parts of the spectrum.

\item {} 
\sphinxhref{https://mast.stsci.edu/portal/Mashup/Clients/Mast/Portal.html}{The MAST
Portal}
lets you search multiple collections of astronomical datasets all
in one place.

\end{itemize}

\end{itemize}


\section{Resources and Tools for a Specific Topic}
\label{\detokenize{astro_topic:resources-and-tools-for-a-specific-topic}}\label{\detokenize{astro_topic::doc}}
\begin{sphinxadmonition}{warning}{Warning:}
Most of the collections here are still under heavy construction. The contents are likely to be
incomplete.
\end{sphinxadmonition}


\subsection{Data Reduction and Analysis}
\label{\detokenize{astro_topic:data-reduction-and-analysis}}

\subsubsection{Image Redcution and CCD Related}
\label{\detokenize{resource/astro/topics/ccd_reduction:image-redcution-and-ccd-related}}\label{\detokenize{resource/astro/topics/ccd_reduction::doc}}\begin{itemize}
\item {} 
On the general topic of optical/NIR detector and data reduction.

\end{itemize}


\paragraph{Tutorial and Guide}
\label{\detokenize{resource/astro/topics/ccd_reduction:tutorial-and-guide}}\begin{itemize}
\item {} 
\sphinxhref{https://github.com/mwcraig/ccd-reduction-and-photometry-guide}{A guide to CCD data reduction and stellar photometry using astropy
and affiliated
packages}
\begin{itemize}
\item {} 
Only use \sphinxstylestrong{astropy} affiliated Python packages.

\end{itemize}

\end{itemize}


\paragraph{Software}
\label{\detokenize{resource/astro/topics/ccd_reduction:software}}

\subparagraph{Display CCD Images}
\label{\detokenize{resource/astro/topics/ccd_reduction:display-ccd-images}}\begin{itemize}
\item {} 
\sphinxhref{http://ds9.si.edu/site/Home.html}{SAOImage DS9 - Astronomical imaging and data visualization
application}
\begin{itemize}
\item {} 
DS9 supports FITS images and binary tables, multiple frame
buffers, region manipulation, and many scale algorithms and
colormaps. DS9 is actually a very powerful tool for displaying and
manipulating image. \sphinxhref{http://www.jb.man.ac.uk/~gbendo/Sci/Pict/DS9guide.pdf}{Here is a very nice
guide}

\item {} 
\sphinxhref{https://github.com/ericmandel/pyds9}{pyds9 - Python connection to SAOimage DS9 via
XPA}

\end{itemize}

\item {} 
\sphinxhref{https://github.com/spacetelescope/imexam}{imexam by astropy - Python version of the famous imexamine in
IRAF}
\begin{itemize}
\item {} 
\sphinxstylestrong{imexam} is a python tool for simple image examination, and
plotting, with similar functionality to IRAF’s imexamine. \sphinxhref{https://imexam.readthedocs.io/en/latest/}{Online
document is here}

\end{itemize}

\item {} 
\sphinxhref{https://github.com/ejeschke/ginga}{ginga - astronomical FITS file
viewer}
\begin{itemize}
\item {} 
\sphinxstylestrong{Ginga} is a toolkit designed for building viewers for
scientific image data in Python, visualizing 2D pixel data in
numpy arrays. \sphinxhref{https://ginga.readthedocs.io/en/latest/}{Online document is
here}

\end{itemize}

\item {} 
\sphinxhref{https://github.com/astropy/regions}{regions by astropy - Astropy affiliated package for region
handling}

\end{itemize}


\subparagraph{General Reduction}
\label{\detokenize{resource/astro/topics/ccd_reduction:general-reduction}}\begin{itemize}
\item {} 
Modern imaging surveys or major astronomical cameras are often
equipped with speciallized data reduction pipelines. For example:
\begin{itemize}
\item {} 
\sphinxhref{https://www.noao.edu/noao/staff/fvaldes/CPDocPrelim/PL201\_3.html}{DECam Community
Pipeline}

\item {} 
\sphinxhref{https://www.cfht.hawaii.edu/Instruments/Elixir/}{The Elixir System for CFHT
MegaCam}

\item {} 
\sphinxhref{https://hsc.mtk.nao.ac.jp/pipedoc/pipedoc\_6\_e/index.html}{The HSC
Pipeline}

\end{itemize}

\item {} 
\sphinxhref{https://github.com/astropy/ccdproc}{ccdproc - Astropy affiliated package for reducing optical/IR CCD
data}
\begin{itemize}
\item {} 
\sphinxstylestrong{ccdproc} is is an affiliated package for the AstroPy package
for basic data reductions of CCD images. The ccdproc package
provides many of the necessary tools for processing of ccd images
built on a framework to provide error propagation and bad pixel
tracking throughout the reduction process. \sphinxhref{https://ccdproc.readthedocs.io/en/latest/}{Documents can be found
here}

\end{itemize}

\end{itemize}


\subparagraph{Image Detrend and Correction}
\label{\detokenize{resource/astro/topics/ccd_reduction:image-detrend-and-correction}}

\subparagraph{Cosmic Ray Removal}
\label{\detokenize{resource/astro/topics/ccd_reduction:cosmic-ray-removal}}\begin{itemize}
\item {} 
\sphinxhref{https://github.com/astropy/astroscrappy}{astroscrappy by astropy - Speedy Cosmic Ray Annihilation Package in
Python}
\begin{itemize}
\item {} 
\sphinxstylestrong{Astro-SCRAPPY} is designed to detect cosmic rays in images
(numpy arrays), based on Pieter van Dokkum’s \sphinxstylestrong{L.A.Cosmic}
algorithm.

\end{itemize}

\item {} 
\sphinxhref{http://www.astro.yale.edu/dokkum/lacosmic/}{The original L.A.Cosmic code - Laplacian Cosmic Ray
Identification}
\begin{itemize}
\item {} 
\sphinxhref{https://github.com/cmccully/lacosmicx}{lacosmicx - A fast implementation of the LA Cosmic
algorithm}

\end{itemize}

\end{itemize}


\subparagraph{Satellite Trail Removal}
\label{\detokenize{resource/astro/topics/ccd_reduction:satellite-trail-removal}}\begin{itemize}
\item {} 
\sphinxhref{https://github.com/dwkim78/ASTRiDE}{ASTRiDE - Automated Streak Detection for Astronomical
Images}
\begin{itemize}
\item {} 
By Dae-Won Kim. \sphinxstylestrong{ASTRiDE} aims to detect streaks in astronomical
images using a “border” of each object

\end{itemize}

\item {} 
\sphinxhref{https://github.com/guynir42/pyradon}{pyradon - Python tools for streak detection in astronomical images
using the Fast Radon
Transform}
\begin{itemize}
\item {} 
By Guy Nir. Based on \sphinxhref{https://arxiv.org/abs/1806.04204}{Optimal and Efficient Streak Detection in
Astronomical Images}. The
\sphinxhref{https://github.com/guynir42/radon}{Matlab version is here}

\end{itemize}

\end{itemize}


\subparagraph{“Brighter-Fatter” Effect}
\label{\detokenize{resource/astro/topics/ccd_reduction:brighter-fatter-effect}}\begin{itemize}
\item {} 
The “Brighter-Fatter” effect is a direct consequence of the
distortions of the drift electric field sourced by charges
accumulated within the CCD during the exposure and experienced by
forthcoming light-induced charges in the same exposure. It affects
both deep-depleted and thinned CCD sensors.

\item {} 
\sphinxhref{https://ui.adsabs.harvard.edu/abs/2014JInst...9C3048A/abstract}{The brighter-fatter effect and pixel correlations in CCD
sensors}

\item {} 
\sphinxhref{https://ui.adsabs.harvard.edu/abs/2015JInst..10C5015W/abstract}{The Brighter-Fatter and other sensor effects in CCD simulations for
precision
astronomy}

\item {} 
\sphinxhref{https://ui.adsabs.harvard.edu/abs/2018AJ....155..258C/abstract}{Exploring the Brighter-fatter Effect with the Hyper
Suprime-Cam}

\item {} 
\sphinxhref{https://arxiv.org/abs/1906.01846}{Brighter-fatter effect in near-infrared detectors \textendash{} I. Theory of
flat auto-correlations}
\begin{itemize}
\item {} 
\sphinxhref{https://github.com/BJRauscher/nghxrg}{nghxrg - Teledyne HxRG Read Noise
Generator}

\end{itemize}

\item {} 
\sphinxhref{https://arxiv.org/abs/1906.01847}{Brighter-fatter effect in near-infrared detectors \textendash{} II.
Auto-correlation analysis of H4RG-10
flats}

\item {} 
\sphinxhref{https://ui.adsabs.harvard.edu/abs/2017PASP..129h4502B/abstract}{Is Flat fielding Safe for Precision CCD
Astronomy?}

\end{itemize}


\subparagraph{Astrometric Calibration}
\label{\detokenize{resource/astro/topics/ccd_reduction:astrometric-calibration}}\begin{itemize}
\item {} 
\sphinxhref{https://github.com/dstndstn/astrometry.net}{Astrometry.net \textendash{} automatic recognition of astronomical
images}
\begin{itemize}
\item {} 
Made by Dustin Lang. The best astrometric calibration tool on the
market.

\end{itemize}

\item {} 
\sphinxhref{https://www.astromatic.net/software/scamp}{SCAMP from
Astromatic.net}
\begin{itemize}
\item {} 
\sphinxstylestrong{SCAMP} reads SExtractor catalogs and computes astrometric and
photometric solutions for any arbitrary sequence of FITS images in
a completely automatic way.

\end{itemize}

\end{itemize}


\subparagraph{Photometric Calibration}
\label{\detokenize{resource/astro/topics/ccd_reduction:photometric-calibration}}\begin{itemize}
\item {} 
\sphinxhref{https://github.com/erykoff/fgcm}{FGCM - Forward Global Calibration
Method}
\begin{itemize}
\item {} 
Based on the algorithm developed in \sphinxhref{http://adsabs.harvard.edu/abs/2018AJ....155...41B}{Forward Global Photometric
Calibration of the Dark Energy Survey in Burke et
al. 2018}

\item {} 
The \sphinxhref{https://github.com/lsst/fgcmcal/blob/master/cookbook/README.md}{FGCM
Cookbook}
is very good place to start.

\item {} 
The \sphinxhref{https://github.com/lsst/fgcmcal}{Global Photometric Calibration in LSST with
FGCM}

\end{itemize}

\end{itemize}


\subparagraph{Image Reproection and Co-addition}
\label{\detokenize{resource/astro/topics/ccd_reduction:image-reproection-and-co-addition}}\begin{itemize}
\item {} 
\sphinxhref{https://github.com/spacetelescope/drizzle}{drizzle - A package for combining dithered images into a single
image}
\begin{itemize}
\item {} 
The drizzle library is a Python package for combining dithered
images into a single image. This library is derived from code used
in DrizzlePac. Like DrizzlePac, most of the code is implemented in
the C language.

\item {} 
\sphinxhref{https://github.com/spacetelescope/drizzlepac}{The original drizzlepac library for HST
images}

\item {} 
\sphinxhref{https://drizzlepac.readthedocs.io/en/latest/\#}{The online document for
DrizzlePac}

\end{itemize}

\item {} 
\sphinxhref{https://github.com/astropy/reproject}{reproject by astropy - Python-based Astronomical image
reprojection}
\begin{itemize}
\item {} 
By reprojection, we mean the re-gridding of images from one world
coordinate system to another (for example changing the pixel
resolution, orientation, coordinate system).

\end{itemize}

\item {} 
\sphinxhref{https://www.astromatic.net/software/swarp}{SWarp by
Astromatic.net}
\begin{itemize}
\item {} 
\sphinxstylestrong{SWarp} is a program that resamples and co-adds together FITS
images using any arbitrary astrometric projection defined in the
WCS standard

\end{itemize}

\item {} 
\sphinxhref{http://montage.ipac.caltech.edu/}{Montage - Image Mosaic Software for
Astronomers}
\begin{itemize}
\item {} 
\sphinxstylestrong{Montage} is a toolkit for assembling Flexible Image Transport
System (FITS) images into custom mosaics. \sphinxhref{http://montage.ipac.caltech.edu/docs/index.html}{Online document is
here}

\item {} 
It is also on
\sphinxhref{https://github.com/Caltech-IPAC/Montage}{Github}. And there is
an \sphinxhref{http://hachi.ipac.caltech.edu:8080/montage}{Image Mosaic
Service} for 2MASS,
SDSS, WISE images.

\item {} 
\sphinxhref{https://github.com/astropy/montage-wrapper}{montage-wrapper - Python wrapper for the Montage mosaicking
toolkit}
\begin{itemize}
\item {} 
This package provides a python wrapper to the Montage
Astronomical Image Mosaic Engine

\item {} 
\sphinxhref{https://github.com/Caltech-IPAC/MontageNotebooks}{Jupyter notebooks illustrating the use of the Python version
of
Montage}

\end{itemize}

\end{itemize}

\end{itemize}


\subsubsection{Photometric Analysis of Astronomical Targets}
\label{\detokenize{resource/astro/topics/photometry:photometric-analysis-of-astronomical-targets}}\label{\detokenize{resource/astro/topics/photometry::doc}}

\paragraph{General Tools}
\label{\detokenize{resource/astro/topics/photometry:general-tools}}

\subparagraph{Source Extraction and Deblender}
\label{\detokenize{resource/astro/topics/photometry:source-extraction-and-deblender}}\begin{itemize}
\item {} 
\sphinxhref{https://github.com/astropy/photutils}{photutils - Affiliated package for image photometry
utilities}
\begin{itemize}
\item {} 
General Python package. Not the fastest one and is still growing.

\end{itemize}

\item {} 
\sphinxhref{https://github.com/kbarbary/sep/blob/v1.0.x/docs/index.rst}{sep - Python library for Source Extraction and
Photometry}
\begin{itemize}
\item {} 
\sphinxstylestrong{SEP} makes the core algorithms of Source Extractor available as
a library of stand-alone functions and classes.

\end{itemize}

\item {} 
\sphinxhref{http://www.astromatic.net/software/sextractor}{SExtractor}
\begin{itemize}
\item {} 
It is very good software, just remember never call it \sphinxstylestrong{sex},
just don’t.

\item {} 
The best document: \sphinxhref{http://mensa.ast.uct.ac.za/~holwerda/SE/Welcome.html}{Don’t
Panic} by
Benne Holwerda.

\end{itemize}

\item {} 
\sphinxhref{https://github.com/lsst-dm/multiprofit}{multiprofit - Multi-object/band source modelling/galaxy fitting
code}
\begin{itemize}
\item {} 
By Dan Taranu. Still under-development.

\end{itemize}

\item {} 
\sphinxhref{https://github.com/dstndstn/tractor}{The Tractor: measuring astronomical sources via probabilistic
inference}
\begin{itemize}
\item {} 
By Dustin Lang and David Hogg. Used by DECaLS, Probabilistic
astronomical source detection \& measurement.

\end{itemize}

\item {} 
\sphinxhref{https://github.com/esheldon/ngmix}{ngmix - Gaussian mixtures and image processing implemented in
python}
\begin{itemize}
\item {} 
By Erin Sheldon. Gaussian mixture models and other code for
working with for 2d images, implemented in python. Used by DES
data analysis.

\end{itemize}

\end{itemize}


\subparagraph{For Low Surface Brightness or Extremely Faint Targets}
\label{\detokenize{resource/astro/topics/photometry:for-low-surface-brightness-or-extremely-faint-targets}}\begin{itemize}
\item {} 
\sphinxhref{https://github.com/CarolineHaigh/mtobjects}{MTObjects - a tool for detecting sources in astronomical
images}
\begin{itemize}
\item {} 
Working progress, based on the \sphinxstylestrong{C} code by Paul Teeninga’s 2015
work: \sphinxhref{http://fse.studenttheses.ub.rug.nl/12972/1/INF-BA-2015-P.D.Teeninga.pdf}{Improved detection of faint extended astronomical objects
through statistical attribute
filtering}

\end{itemize}

\item {} 
\sphinxhref{https://github.com/danjampro/DeepScan}{DeepScan - source extraction tool designed to identify very low
surface brightness features in large astronomical
data}
\begin{itemize}
\item {} 
Based on the work by \sphinxhref{https://academic.oup.com/mnras/article-abstract/478/1/667/4980941?redirectedFrom=fulltext}{Prole et al. 2018: Automated detectionof
very low surface brightness galaxiesin the Virgo
cluster}

\end{itemize}

\end{itemize}


\subparagraph{Multiband Deblending and Force Photometry}
\label{\detokenize{resource/astro/topics/photometry:multiband-deblending-and-force-photometry}}\begin{itemize}
\item {} 
\sphinxhref{https://github.com/asgr/ProFound}{ProFound - source finding and extraction in
R}

\item {} 
\sphinxhref{https://github.com/fred3m/scarlet}{scarlet - Source separation in multi-band images by Constrained
Matrix Factorization}
\begin{itemize}
\item {} 
By Fred Moolekamp and Peter Melchior. It performs source
separation (aka “deblending”) on multi-band images. Still
under-development.

\end{itemize}

\item {} 
\sphinxhref{https://github.com/AngusWright/LAMBDAR}{LAMBDAR: Lambda Adaptive Multi-Band Deblending Algorithm in
R}

\end{itemize}


\subparagraph{Image Simulation}
\label{\detokenize{resource/astro/topics/photometry:image-simulation}}\begin{itemize}
\item {} 
\sphinxhref{https://github.com/emhuff/Balrog}{Balrog - DES image simulation software: We dug too deeply and too
greedily}
\begin{itemize}
\item {} 
Application in DES: \sphinxhref{https://arxiv.org/abs/1507.08336}{No galaxy left
behind}

\item {} 
\sphinxhref{https://cdcvs.fnal.gov/redmine/projects/des/wiki/des\_balrog\_y3}{Balrog run in DES
Y3}

\end{itemize}

\item {} 
\sphinxhref{https://github.com/lsst/synpipe}{Synpipe - Synthetic Object Pipeline for the LSST
pipeline}
\begin{itemize}
\item {} 
Synpipe provides tasks which make use of the LSST fake object
pipeline to insert realistic galaxies and stars. It also includes
scripts to analyze the results of data which has been processed
with fake objects inserted.

\end{itemize}

\item {} 
\sphinxhref{https://obiwan.readthedocs.io/en/latest/}{Obiwan - Monte Carlo method for adding fake galaxies to Legacy
Survey imaging data}

\end{itemize}


\paragraph{Galaxies or Extended Objects}
\label{\detokenize{resource/astro/topics/photometry:galaxies-or-extended-objects}}

\subparagraph{Tools}
\label{\detokenize{resource/astro/topics/photometry:tools}}\begin{itemize}
\item {} 
\sphinxhref{https://github.com/GalSim-developers/GalSim}{GalSim - The modular galaxy image simulation
toolkit}
\begin{itemize}
\item {} 
\sphinxstylestrong{GalSim} is open-source software for simulating images of
astronomical objects (stars, galaxies) in a variety of ways.

\item {} 
\sphinxstylestrong{GalSim} paper can be found
\sphinxhref{http://adsabs.harvard.edu/abs/2015A\%26C....10..121R}{here}

\end{itemize}

\end{itemize}


\subparagraph{1-D Galaxy Profile Extraction}
\label{\detokenize{resource/astro/topics/photometry:d-galaxy-profile-extraction}}\begin{itemize}
\item {} 
\sphinxhref{http://stsdas.stsci.edu/cgi-bin/gethelp.cgi?ellipse}{stsdas.analysis.isophote - fits elliptical isophotes to galaxy
images}
\begin{itemize}
\item {} 
The OG.

\end{itemize}

\item {} 
\sphinxhref{https://github.com/astropy/photutils/tree/master/photutils/isophote}{photutils.isophote - Python version of the ellipse
code}
\begin{itemize}
\item {} 
Not as fast as the \sphinxstylestrong{IRAF} version, but it is useful. Document is
\sphinxhref{https://photutils.readthedocs.io/en/stable/isophote.html}{here}

\end{itemize}

\item {} 
\sphinxhref{http://abyss.uoregon.edu/~js/archangel/}{ARCHANGEL: Galaxy Photometry
System}
\begin{itemize}
\item {} 
By James Schombert. Used by several projects including
\sphinxhref{http://astroweb.cwru.edu/SPARC/}{SPARC}. \sphinxhref{https://arxiv.org/abs/astro-ph/0703646}{Paper is
here}

\end{itemize}

\end{itemize}


\subparagraph{2-D Galaxy Modeling}
\label{\detokenize{resource/astro/topics/photometry:d-galaxy-modeling}}\begin{itemize}
\item {} 
\sphinxhref{https://www.mpe.mpg.de/~erwin/code/imfit/}{Imfit: Fast, Flexible Multi-component Fitting of Galaxy
Images}
\begin{itemize}
\item {} 
By Peter Erwin. Imfit is a program for fitting astronomical images
\textendash{} especially images of galaxies, though it can in principle be
used for fitting other sources

\item {} 
This is today’s first choice.

\item {} 
\sphinxhref{https://github.com/johnnygreco/pymfit}{pymfit} by Johnny Greco
to use \sphinxstylestrong{imfit} in Python.

\end{itemize}

\item {} 
\sphinxhref{https://users.obs.carnegiescience.edu/peng/work/galfit/galfit.html}{Galfit - Detailed Structural Decomposition of Galaxy
Images}
\begin{itemize}
\item {} 
By Chien Peng. Still the most flexible code. But it has not been
maintained recently.

\item {} 
\sphinxhref{https://users.obs.carnegiescience.edu/peng/work/galfit/TOP10.html}{Top 10 Rules of Thumb for Galaxy
Fitting}:
this is good advice for fitting galaxies in general.

\end{itemize}

\item {} 
\sphinxhref{https://github.com/ICRAR/ProFit}{ProFit - Profile Fitting Code in
R}
\begin{itemize}
\item {} 
ProFit is a Bayesian galaxy fitting tool that uses a fast C++
image generation library and a flexible interface to a large
number of likelihood samplers.

\item {} 
\sphinxhref{https://github.com/ICRAR/libprofit}{libprofit - a low-level C++ library that produces images based on
different luminosity
profiles}. Document is
\sphinxhref{https://libprofit.readthedocs.io/en/latest/}{here}

\item {} 
\sphinxhref{https://github.com/ICRAR/pyprofit}{pyprofit - a python wrapper for
libprofit}
\begin{itemize}
\item {} 
This only provides the tool to generate 2-D galaxy model with
PSF convolution. You need to setup your own optimazation
structue for fitting.

\end{itemize}

\end{itemize}

\item {} 
\sphinxhref{https://github.com/sibirrer/lenstronomy}{lenstronomy - software package for lens model reconstruction of
imaging data}
\begin{itemize}
\item {} 
By \sphinxhref{http://www.astro.ucla.edu/~sibirrer/}{Simon Birrer}.
\sphinxstylestrong{lenstronomy} is a multi-purpose package to model strong
gravitational lenses. The software package is presented in Birrer
\& Amara 2018 and is based on Birrer et al 2015.

\item {} 
It can also be used to fit galaxies like \sphinxstylestrong{Galfit}. See \sphinxhref{https://github.com/sibirrer/lenstronomy\_extensions/blob/master/lenstronomy\_extensions/Notebooks/galfitting.ipynb}{example
here}

\end{itemize}

\end{itemize}


\paragraph{Stellar or PSF Photometry}
\label{\detokenize{resource/astro/topics/photometry:stellar-or-psf-photometry}}\begin{itemize}
\item {} 
\sphinxhref{http://www.star.bris.ac.uk/~mbt/daophot/}{DAOPHOT - Stellar Photometry
Package}
\begin{itemize}
\item {} 
DAOPHOT is a package for stellar photomoetry designed to deal with
crowded fields.

\end{itemize}

\item {} 
\sphinxhref{http://americano.dolphinsim.com/dolphot/}{DOLPHOT - stellar photometry package that was adapted from HSTphot
for general use}
\begin{itemize}
\item {} 
By Andrew Dolphin and Raytheon Company. PSF photometry for HSC.
Now supports WFIRST too.

\item {} 
\sphinxhref{https://github.com/dweisz/pydolphot}{pydolphot - python wrappers for
DOLPHOT} by Dan Weisz.

\end{itemize}

\item {} 
\sphinxhref{https://photutils.readthedocs.io/en/stable/psf.html}{photutils.psf - PSF photometry in
Python}
\begin{itemize}
\item {} 
Provides Python approaches to do \sphinxstylestrong{DAOPhot} or \sphinxstylestrong{IRAF} style PSF
photometry.

\end{itemize}

\end{itemize}


\subparagraph{Useful papers}
\label{\detokenize{resource/astro/topics/photometry:useful-papers}}\begin{itemize}
\item {} 
\sphinxhref{https://arxiv.org/pdf/1902.02374.pdf}{Photometric Biases in Modern Surveys by Portillo, Speagle, \&
Finkbeiner 2019}
\begin{itemize}
\item {} 
Reveals a bias in modern PSF photometry: We show these ML
estimators systematically overestimate the flux as a function of
the signal-to-noise ratio (SNR) and the number of model parameters
involved in the fit.

\item {} 
\sphinxhref{https://github.com/joshspeagle/phot\_bias}{phot-bias - Biases in Maximum-Likelihood
Photometry}

\end{itemize}

\end{itemize}


\subsubsection{Spectroscopic Data Reduction and Analysis}
\label{\detokenize{resource/astro/topics/spectroscopy:spectroscopic-data-reduction-and-analysis}}\label{\detokenize{resource/astro/topics/spectroscopy::doc}}

\paragraph{General Tools}
\label{\detokenize{resource/astro/topics/spectroscopy:general-tools}}\begin{itemize}
\item {} 
\sphinxhref{https://github.com/astropy/specutils}{specutils - An Astropy coordinated package for astronomical
spectroscopy}
\begin{itemize}
\item {} 
Provides a shared set of Python representations of astronomical
spectra and basic tools to operate on these spectra.

\end{itemize}

\item {} 
\sphinxhref{https://github.com/linetools/linetools}{linetools - This is a package for the analysis of 1d astronomical
spectra, especially quasar and galaxy
spectra}
\begin{itemize}
\item {} 
Still under development. Its core developers work primarily on
UV/optical/IR absorption line research, so most of the
functionality is aimed at the identification and analysis of
absorption lines. The eventual goal is to provide a set of tools
useful for both absorption and emission lines.

\item {} 
\sphinxhref{https://linetools.readthedocs.io/en/latest/}{Online document is
here}

\end{itemize}

\end{itemize}


\paragraph{Visualization}
\label{\detokenize{resource/astro/topics/spectroscopy:visualization}}\begin{itemize}
\item {} 
\sphinxhref{https://github.com/spacetelescope/specviz}{specviz - An interactive astronomical 1D spectra analysis
tool}
\begin{itemize}
\item {} 
Made by STScI. An gui-based interactive analysis tool for one
dimensional astronomical data using Python.

\end{itemize}

\item {} 
\sphinxhref{https://github.com/spacetelescope/cubeviz}{cubeviz - Data analysis package for
cubes}
\begin{itemize}
\item {} 
Made by STScI. \sphinxstylestrong{CubeViz}, a visualization and analysis tool for
data cubes from integral field units (IFUs). \sphinxhref{https://cubeviz.readthedocs.io/en/latest/}{Online document is
here}

\end{itemize}

\end{itemize}


\paragraph{Data Reduction Pipeline}
\label{\detokenize{resource/astro/topics/spectroscopy:data-reduction-pipeline}}\begin{itemize}
\item {} 
\sphinxhref{https://github.com/pypeit/PypeIt}{Pypelt - The Python Spectroscopic Data Reduction
Pipeline}
\begin{itemize}
\item {} 
PypeIt is a Python based data reduction pipeline (DRP) written
oringinally for echelle spectroscopy and since expanded to
low-resolution spectrometers.

\item {} 
\sphinxhref{https://github.com/pypeit/arclines}{arclines - Database of arc lines from optical-IR
spectrographs}

\end{itemize}

\item {} 
MaNGA data reduction and analysis pipeline
\begin{itemize}
\item {} 
\sphinxhref{https://github.com/sdss/mangadap}{mangadap - The MaNGA Data Analysis
Pipeline}
\begin{itemize}
\item {} 
The MaNGA data-analysis pipeline (MaNGA DAP) is the survey-led
software package that analyzes the data produced by the MaNGA
data-reduction pipeline (MaNGA DRP) to produced physical
properties derived from the MaNGA spectroscopy.

\end{itemize}

\end{itemize}

\item {} 
\sphinxhref{https://abittner.gitlab.io/thegistpipeline/index.html}{GIST - Galaxy IFU Spectroscopy
Tool}
\begin{itemize}
\item {} 
A convenient, all-in-one framework for the scientific analysis of
fully reduced, (integral-field) spectroscopic data.

\end{itemize}

\item {} 
\sphinxhref{https://github.com/kasperschmidt/TDOSE}{TDOSE - Three Dimensional Optimal Spectral
Extraction}
\begin{itemize}
\item {} 
Python pipeline to extract spectra of both point sources and
extended sources from integral field data cubes

\end{itemize}

\item {} 
\sphinxhref{https://github.com/TheAstroFactory/pydis}{pydis - A simple longslit spectroscopy pipeline in
Python}
\begin{itemize}
\item {} 
The goal of pyDIS is to provide a turn-key solution for reducing
and understanding longslit spectroscopy, which could ideally be
done in real time. Currently we are using many simple assumptions
to get a quick-and-dirty solution, and modeling the workflow after
the robust industry standards set by IRAF.

\end{itemize}

\item {} 
\sphinxhref{https://github.com/dkirkby/speclite}{speclite - Lightweight utilities for working with spectroscopic
data}
\begin{itemize}
\item {} 
By David Kirby. This package provides a set of lightweight
utilities for working with spectroscopic data in astronomy.

\end{itemize}

\end{itemize}


\paragraph{Redshift or Radial Velocity Measurement}
\label{\detokenize{resource/astro/topics/spectroscopy:redshift-or-radial-velocity-measurement}}\begin{itemize}
\item {} 
\sphinxhref{https://github.com/gbrammer/grizli}{grizli - Grism redshift \& line analysis software for space-based
slitless spectroscopy}
\begin{itemize}
\item {} 
By \sphinxhref{http://www.stsci.edu/~brammer/}{Gabriel Brammer}.

\end{itemize}

\item {} 
\sphinxhref{https://github.com/desihub/redrock}{redrock - Redshift fitting for
spectroperfectionism}
\begin{itemize}
\item {} 
This is DESI’s redshift measurement tool

\end{itemize}

\item {} 
\sphinxhref{https://github.com/timahutchinson/redmonster}{redmonster - Python utilieties for redshift
measurement}
\begin{itemize}
\item {} 
\sphinxstylestrong{redmonster} is a project to develop a sophisticated and
flexible set of Python utilities for redshift measurement,
physical parameter measurement, and classification of
one-dimensional astronomical spectra.

\item {} 
Originally designed for eBOSS and the paper can be \sphinxhref{https://arxiv.org/abs/1607.02432}{found
here}

\end{itemize}

\end{itemize}


\subsection{Sub-field in Astronomy and Astrophysics}
\label{\detokenize{astro_topic:sub-field-in-astronomy-and-astrophysics}}

\subsubsection{Exoplanet and Astrobiology}
\label{\detokenize{resource/astro/topics/exoplanet:exoplanet-and-astrobiology}}\label{\detokenize{resource/astro/topics/exoplanet::doc}}\begin{itemize}
\item {} 
\sphinxstylestrong{Just started}

\item {} 
\sphinxstylestrong{Help wanted}: I am not very familiar with this topic, but it
clearly has become one of the most important field in astronomy.

\end{itemize}


\bigskip\hrule\bigskip



\paragraph{Transient Detection}
\label{\detokenize{resource/astro/topics/exoplanet:transient-detection}}\begin{itemize}
\item {} 
\sphinxhref{https://github.com/dfm/exoplanet}{exoplanet - Fast \& scalable MCMC for all your exoplanet
needs}
\begin{itemize}
\item {} 
\sphinxstylestrong{exoplanet} is a toolkit for probabilistic modeling of transit
and/or radial velocity observations of exoplanets and other
astronomical time series using PyMC3.

\item {} 
By \sphinxhref{https://dfm.io/}{Dan Foreman-Mackey}.

\end{itemize}

\item {} 
\sphinxhref{https://github.com/rodluger/pysyzygy}{pysyzygy - A fast and general planet transit (syzygy) code written
in C and in Python}

\item {} 
\sphinxhref{https://github.com/hippke/tls}{tls - An optimized transit-fitting algorithm to search for periodic
transits of small planets}
\begin{itemize}
\item {} 
By Michael Hippke. Based on this
\sphinxhref{https://ui.adsabs.harvard.edu/abs/2019A\%26A...623A..39H/abstract}{paper}

\item {} 
\sphinxstylestrong{TLS} searches for transit-like features with stellar
limb-darkening and including the effects of planetary ingress and
egress

\end{itemize}

\item {} 
\sphinxhref{https://github.com/rodluger/planetplanet}{planetplanet - A general photodynamical code for exoplanet light
curves}
\begin{itemize}
\item {} 
By \sphinxhref{https://rodluger.github.io/}{Rodrigo Luger}. A general
photodynamical code for modeling exoplanet transits, secondary
eclipses, phase curves, and exomoons, as well as eclipsing
binaries, circumbinary planets, and more

\end{itemize}

\item {} 
\sphinxhref{https://github.com/hpparvi/ldtk}{PyLDTk - Python Limb Darkening
Toolkit}
\begin{itemize}
\item {} 
a Python toolkit for calculating stellar limb darkening profiles
and model-specific coefficients for arbitrary passbands using the
stellar spectrum model library by Husser et al (2013)

\end{itemize}

\item {} 
\sphinxhref{https://github.com/timothydmorton/VESPA}{VESPA - Validation of Exoplanet Signals using a Probabilistic
Algorithm}
\begin{itemize}
\item {} 
By Timothy Morton. It calculats false positive probabilities for
transit signals

\end{itemize}

\item {} 
\sphinxhref{https://github.com/nespinoza/juliet}{juliet - a versatile modelling tool for transiting and
non-transiting exoplanetary
systems}
\begin{itemize}
\item {} 
By Néstor Espinoza. Based on the paper \sphinxhref{https://arxiv.org/abs/1812.08549}{Espinoza et
al. 2018}

\end{itemize}

\item {} 
\sphinxhref{https://github.com/nespinoza/exonailer}{exonailer - Tools for fitting transiting exoplanet lightcurves and
radial velocities}
\begin{itemize}
\item {} 
By Néstor Espinoza. The EXOplanet traNsits and rAdIal veLocity
fittER (\sphinxstylestrong{EXO-NAILER}), is an easy-to-use code that allows you to
efficiently fit exoplanet transit lightcurves, radial velocities
(RVs) or both.

\end{itemize}

\item {} 
\sphinxhref{http://astro.uchicago.edu/~kreidberg/batman/}{batman - Bad-Ass Transit Model
cAlculatioN}
\begin{itemize}
\item {} 
\sphinxstylestrong{batman}, a Python package for fast calculation of exoplanet
transit light curves.

\item {} 
Based on \sphinxhref{https://arxiv.org/abs/1507.08285}{the paper by Laura
Kreidberg}

\end{itemize}

\end{itemize}


\paragraph{Direct Imaging}
\label{\detokenize{resource/astro/topics/exoplanet:direct-imaging}}\begin{itemize}
\item {} 
\sphinxhref{https://github.com/PynPoint/PynPoint}{PynPoint - Pipeline for processing and analysis of high-contrast
imaging data}
\begin{itemize}
\item {} 
\sphinxstylestrong{PynPoint} is a generic, end-to-end pipeline for the data
reduction and analysis of high-contrast imaging data of planetary
and substellar companions, as well as circumstellar disks in
scattered light.

\item {} 
\sphinxhref{https://pynpoint.readthedocs.io/en/latest/}{Online document can be found
here}

\end{itemize}

\item {} 
\sphinxhref{https://github.com/sblunt/orbitize}{orbitize - Orbit-fitting for directly imaged
objects}

\end{itemize}


\paragraph{Radial Velocity}
\label{\detokenize{resource/astro/topics/exoplanet:radial-velocity}}\begin{itemize}
\item {} 
\sphinxhref{https://github.com/megbedell/wobble}{wobble - precise data-driven RV fitting, now with
tellurics}
\begin{itemize}
\item {} 
By \sphinxhref{https://bedell.space/}{Megan Bedell} \sphinxstylestrong{wobble} is an
open-source python package for analyzing time-series spectra. It
was designed with stabilized extreme precision radial velocity
(EPRV) spectrographs in mind, but is highly flexible and
extensible to a variety of applications.

\end{itemize}

\item {} 
\sphinxhref{https://github.com/California-Planet-Search/radvel}{radvel - General Toolkit for Modeling Radial Velocity
Data}
\begin{itemize}
\item {} 
Based on the publication by \sphinxhref{https://arxiv.org/abs/1801.01947}{Fulton et
al. 2018}

\end{itemize}

\end{itemize}


\paragraph{Planet Atmosphere}
\label{\detokenize{resource/astro/topics/exoplanet:planet-atmosphere}}\begin{itemize}
\item {} 
\sphinxhref{https://github.com/natashabatalha/PandExo}{PandExo - A Community Tool for Transiting Exoplanet Science with the
JWST \& HST}
\begin{itemize}
\item {} 
Tools to help the community with planning exoplanet observations

\item {} 
By \sphinxhref{https://natashabatalha.github.io/}{Natasha Batalha}

\end{itemize}

\end{itemize}


\subsubsection{Extragalactic Astrophysics}
\label{\detokenize{resource/astro/topics/extragalactic_astronomy:extragalactic-astrophysics}}\label{\detokenize{resource/astro/topics/extragalactic_astronomy::doc}}\begin{itemize}
\item {} 
\sphinxstylestrong{Just started}

\end{itemize}


\paragraph{HST Surveys and Large Projects}
\label{\detokenize{resource/astro/topics/extragalactic_astronomy:hst-surveys-and-large-projects}}\begin{itemize}
\item {} 
\sphinxhref{https://3dhst.research.yale.edu/Home.html}{3D-HST - A Spectroscopic Galaxy Evolution Survey with the Hubble
Space Telescope}
\begin{itemize}
\item {} 
3D-HST surveyed \textasciitilde{}600 square arcminutes of well-studied
extragalactic survey fields (AEGIS, COSMOS, GOODS-S, UKIDSS-UDS)
with two orbits of primary WFC3/G141 grism coverage and two to
four orbits with ACS/G800L coverage. The short acquisition images,
taken in the WFC3/F140W and ACS/F814W filters used for wavelength
reference for the spectra, are also deeper than most ground-based
observations.

\end{itemize}

\end{itemize}


\paragraph{Ground-base Specoscopic Surveys}
\label{\detokenize{resource/astro/topics/extragalactic_astronomy:ground-base-specoscopic-surveys}}\begin{itemize}
\item {} 
\sphinxhref{http://www.astro.ljmu.ac.uk/~ikb/research/galaxy-redshift-surveys.html}{List of spectroscopic surveys by Ivan
Baldry}
\begin{itemize}
\item {} 
Covers all the “classic” ones, but some links might no longer work
any more.

\end{itemize}

\item {} 
\sphinxhref{https://primus.ucsd.edu}{PRIMUS: PRIsm MUlti-object Survey}
\begin{itemize}
\item {} 
A spectroscopic survey to z=1 with \textasciitilde{}120,000 robust galaxy
redshifts covering \textgreater{}9 sq. deg. of the sky, focusing on regions
with deep Spitzer, optical, GALEX and X-ray data. \sphinxhref{https://primus.ucsd.edu/version1.html}{Data release
available}

\end{itemize}

\item {} 
\sphinxhref{http://www.mpia.de/home/legac/}{Lega-C: Large Early Galaxy Astrophysics
Census}
\begin{itemize}
\item {} 
VIMOS survey of 0.6 \textless{} z \textless{} 1.0 massive galaxies in the COSMOS
field. \sphinxhref{http://www.mpia.de/home/legac/}{Data release available}

\end{itemize}

\item {} 
\sphinxhref{http://cesam.lam.fr/zCosmos/}{zCOSMOS: spectroscopic redshift survey in the COSMOS
field}
\begin{itemize}
\item {} 
VIMOS survey of 28,000 galaxies at 0.2 \textless{} z \textless{} 1.2 and 12,000
galaxies at 1.2 \textless{} z \textless{} 3. \sphinxhref{http://cesam.lam.fr/zCosmos/search/download}{Data release
available}

\item {} 
Due to the special role of the COSMOS field for spec-z and photo-z
calibration, this dataset is also very important.

\end{itemize}

\item {} 
\sphinxhref{https://cesam.lam.fr/vuds/DR1/}{VUDS: VIMOS Ultra Deep Survey}
\begin{itemize}
\item {} 
VIMOS survey of \textasciitilde{}10.000 very faint galaxies to study the major
phase of galaxy assembly 2 \textless{} z ≲ 6. \sphinxhref{https://cesam.lam.fr/vuds/DR1/}{Data release
available}

\end{itemize}

\item {} 
\sphinxhref{http://mosdef.astro.berkeley.edu}{MOSDEF: The MOSFIRE Deep Evolution Field
Survey}
\begin{itemize}
\item {} 
Charting the evolution of the rest-frame optical spectra for \textasciitilde{}1500
galaxies in three distinct redshift intervals spanning 1.4 \textless{} z \textless{}
3.8 in AEGIS, COSMOS, and GOODS-N.

\end{itemize}

\item {} 
\sphinxhref{http://zfire.swinburne.edu.au/index.html}{ZFIRE: A Keck/MOSFIRE SPECTROSCOPIC SURVEY OF STAR-FORMING GALAXIES
IN RICH ENVIORNMENTS DURING
COSMIC-NOON}

\end{itemize}


\subparagraph{IFU Surveys}
\label{\detokenize{resource/astro/topics/extragalactic_astronomy:ifu-surveys}}\begin{itemize}
\item {} 
\sphinxhref{https://musewide.aip.de/project/}{MUSE-Wide project}
\begin{itemize}
\item {} 
MUSE Survey of GOODS-S/CDFS and CANDELS-COSMOS areas.

\end{itemize}

\item {} 
\sphinxhref{http://www.mpe.mpg.de/~forster/SINS/sins\_nmfs.html}{SINS: Spectroscopic Imaging survey in the Near-infrared with
SINFONI}

\item {} 
\sphinxhref{http://adsabs.harvard.edu/abs/2017MNRAS.471.1280T}{KDS: KMOS Deep
Survey}
\begin{itemize}
\item {} 
A KMOS study of the gas kinematics and metallicity in 77 SFGs with
a median redshift of z ≃ 3.5, probing a representative section of
the galaxy main-sequence.

\end{itemize}

\item {} 
\sphinxhref{https://ui.adsabs.harvard.edu/\#abs/arXiv:1708.00454}{KCS: KMOS Cluster
Survey}
\begin{itemize}
\item {} 
A 30-night KMOS GTO program performing deep absorption-line
spectroscopy in four main overdensities at 1.39 \textless{} z \textless{} 1.8 and one
lower-priority overdensity at z = 1.04 to bridge our high-redshift
observations with the local sample.

\end{itemize}

\end{itemize}


\paragraph{Groud-base Imaging Surveys}
\label{\detokenize{resource/astro/topics/extragalactic_astronomy:groud-base-imaging-surveys}}\begin{itemize}
\item {} 
\sphinxhref{https://zenodo.org/record/2225161\#.XBnQji2ZOV5}{Deep Hyper Suprime-Cam Images and a Forced Photometry Catalog in
W-CDF-S}
\begin{itemize}
\item {} 
Both catalogs and reduced HSC data are available here

\item {} 
Associated RNASS note can be found
\sphinxhref{https://arxiv.org/abs/1812.07565}{here}

\end{itemize}

\item {} 
\sphinxhref{http://svo2.cab.inta-csic.es/vocats/alhambra/index.php}{ALHAMBRA: Advance Large Homogeneous Area Medium Band Redshift
Astronomical}

\item {} 
\sphinxhref{https://www.pausurvey.org}{PAU: Narrow-band Survey using the
PAUCam}

\end{itemize}


\paragraph{MIR or FIR Surveys}
\label{\detokenize{resource/astro/topics/extragalactic_astronomy:mir-or-fir-surveys}}\begin{itemize}
\item {} 
\sphinxhref{http://astroweb.cwru.edu/SPARC/}{SPARCS - Spitzer Photometry and Accurate Rotation
Curves}
\begin{itemize}
\item {} 
SPARC is a database of 175 late-type galaxies (spirals and
irregulars) with Spitzer photometry at 3.6 mu (tracing the stellar
mass distribution) and high-quality HI/Ha rotation curves

\end{itemize}

\end{itemize}


\paragraph{Radio Surveys}
\label{\detokenize{resource/astro/topics/extragalactic_astronomy:radio-surveys}}\begin{itemize}
\item {} 
\sphinxhref{http://www.iram.fr/~phibss2/Home.html}{PHIBSS: Plateau de Bure high-z Blue Sequence
Survey}

\item {} 
\sphinxhref{https://empiresurvey.webstarts.com/}{EMPIRE - EMIR Multiline Probe of the ISM Regulating Galaxy
Evolution}
\begin{itemize}
\item {} 
The first wide-area multi-line mapping survey targeting tracers of
dense gas across the entire molecular disks of 9 nearby
star-forming galaxies.

\item {} 
EMPIRE provides, for the first time, resolved (1-2kpc resolution)
maps of a suite of density-sensitive transitions in the 3mm
atmospheric window, including HCN (1-0), HCO+ (1-0) and HNC (1-0).

\end{itemize}

\end{itemize}


\paragraph{X-ray or High-Energy Surveys}
\label{\detokenize{resource/astro/topics/extragalactic_astronomy:x-ray-or-high-energy-surveys}}

\paragraph{Useful dataset and tools}
\label{\detokenize{resource/astro/topics/extragalactic_astronomy:useful-dataset-and-tools}}\begin{itemize}
\item {} 
\sphinxhref{http://lasd.lyman-alpha.com}{LASD - Lyman Alpha Spectral
Database}
\begin{itemize}
\item {} 
Lyman Alpha Spectral Database will facilitate comparison of Lya
spectral modeling and observations.

\end{itemize}

\end{itemize}


\paragraph{Simulation and other model predictions}
\label{\detokenize{resource/astro/topics/extragalactic_astronomy:simulation-and-other-model-predictions}}\begin{itemize}
\item {} 
\sphinxhref{https://www.simonsfoundation.org/semi-analytic-forecasts-for-jwst/}{Semi-Analytic Forecasts for JWST using Santa Cruz
SAM}
\begin{itemize}
\item {} 
We provide predictions for galaxy populations over a wide range of
rest UV luminosity from MUV ∼ −6 to ∼ −24 between z = 4 \textendash{} 10.

\end{itemize}

\end{itemize}


\subsubsection{Galactic Astronomy and the Milky Way}
\label{\detokenize{resource/astro/topics/galactic_astronomy:galactic-astronomy-and-the-milky-way}}\label{\detokenize{resource/astro/topics/galactic_astronomy::doc}}\begin{itemize}
\item {} 
\sphinxstylestrong{Just started}

\item {} 
Focusing on the dynamic and stellar population sides.

\end{itemize}


\paragraph{Galactic Dynamics}
\label{\detokenize{resource/astro/topics/galactic_astronomy:galactic-dynamics}}

\subparagraph{Courses}
\label{\detokenize{resource/astro/topics/galactic_astronomy:courses}}\begin{itemize}
\item {} 
\sphinxhref{http://astro.utoronto.ca/~bovy/AST1420/notes/index.html\#}{Dynamics and Astrophysics of Galaxies by Jo
Bovy}

\item {} 
\sphinxhref{http://www.astro.yale.edu/vdbosch/galdyn.html}{Dynamics of Collisionless Systems by Frank van den
Bosch}

\end{itemize}


\subparagraph{Tools for Galactic Dynamic}
\label{\detokenize{resource/astro/topics/galactic_astronomy:tools-for-galactic-dynamic}}\begin{itemize}
\item {} 
\sphinxhref{https://github.com/adrn/gala}{gala - Galactic and gravitational dynamics in
Python}
\begin{itemize}
\item {} 
By \sphinxhref{http://adrian.pw/}{Adrian Price-Whelan}. Provide a
class-based and user-friendly interface to fast (C or
Cython-optimized) implementations of common operations such as
gravitational potential and force evaluation, orbit integration,
dynamical transformations, and chaos indicators for nonlinear
dynamics.

\item {} 
Online \sphinxhref{http://gala.adrian.pw/en/latest/}{document is here}

\end{itemize}

\item {} 
\sphinxhref{https://github.com/jobovy/galpy}{galpy - Galactic Dynamics in
Python}
\begin{itemize}
\item {} 
By \sphinxhref{http://astro.utoronto.ca/~bovy/}{Jo Bovy}. It supports
orbit integration in a variety of potentials, evaluating and
sampling various distribution functions, and the calculation of
action-angle coordinates for all static potentials.

\item {} 
Online \sphinxhref{https://galpy.readthedocs.io/en/v1.4.0/}{document is
here}

\end{itemize}

\item {} 
\sphinxhref{https://github.com/adrn/thejoker}{thejoker - A custom Monte Carlo sampler for the (gravitational)
two-body problem}
\begin{itemize}
\item {} 
By Adrian Price-Whelan. The Joker {[}1{]} is a custom Monte Carlo
sampler for the two-body problem and is therefore useful for
constraining star-star or star-planet systems

\end{itemize}

\item {} 
\sphinxhref{https://github.com/jobovy/wendy}{wendy - A one-dimensional gravitational N-body
code}
\begin{itemize}
\item {} 
By Jo Bovy. \sphinxstylestrong{wendy} solves the one-dimensional gravitational
N-body problem to machine precision with an efficient algorithm
{[}O(log N) / particle-collision{]}. Alternatively, it can solve the
problem with approximate integration, but with exact forces.

\end{itemize}

\item {} 
\sphinxhref{http://www.physics.rutgers.edu/galaxy/}{GALAXY N-body simulation
package}
\begin{itemize}
\item {} 
By Jerry Sellwood. The
\sphinxhref{http://www.physics.rutgers.edu/~sellwood/manual.pdf}{manual}
is here.

\end{itemize}

\item {} 
\sphinxhref{https://github.com/GalacticDynamics-Oxford/Agama}{Agama - Action-based galaxy modeling
framework}
\begin{itemize}
\item {} 
\sphinxstylestrong{Agama} is a library intended for a broad range of tasks within
the field of stellar dynamics.

\end{itemize}

\item {} 
\sphinxhref{https://teuben.github.io/nemo/}{NEMO - A Stellar Dynamics
Toolbox}
\begin{itemize}
\item {} 
By Peter Teuben. Also available on
\sphinxhref{https://github.com/teuben/nemo}{github}

\end{itemize}

\end{itemize}


\subparagraph{Jeans Modeling}
\label{\detokenize{resource/astro/topics/galactic_astronomy:jeans-modeling}}\begin{itemize}
\item {} 
\sphinxhref{http://www-astro.physics.ox.ac.uk/~mxc/software/\#jam}{JAM - Jeans Anisotropic MGE modeling
method}
\begin{itemize}
\item {} 
By Michelle Cappellari. Available in Python and IDL.

\end{itemize}

\item {} 
\sphinxhref{https://github.com/lauralwatkins/cjam}{CJAM - Jeans Anisotropic MGE modelling code written in
C}
\begin{itemize}
\item {} 
By Laura Watkins. Extended JAM models to calculate all three (x,
y, z) first moments and all six (xx, yy, zz, xy, xz, yz) second
moments.

\end{itemize}

\item {} 
\sphinxhref{https://github.com/adwasser/slomo}{slomo - Jeans modeling with multiple tracer
populations}
\begin{itemize}
\item {} 
By Asher Wasserman. There is also a Julia version:
\sphinxhref{https://github.com/adwasser/Slomo.jl}{Slomo.jl}

\end{itemize}

\end{itemize}


\subparagraph{Schwarzschild Modeling}
\label{\detokenize{resource/astro/topics/galactic_astronomy:schwarzschild-modeling}}\begin{itemize}
\item {} 
\sphinxhref{https://github.com/remcovandenbosch/TriaxSchwarzschild}{TriaxSchwarzschild by Remco van den
Bosch}
\begin{itemize}
\item {} 
Fortran source codes.

\end{itemize}

\end{itemize}


\paragraph{Galactic Chemical Evolution}
\label{\detokenize{resource/astro/topics/galactic_astronomy:galactic-chemical-evolution}}

\subparagraph{Models of Chemical Evolution}
\label{\detokenize{resource/astro/topics/galactic_astronomy:models-of-chemical-evolution}}\begin{itemize}
\item {} 
\sphinxhref{https://github.com/jobovy/kimmy}{kimmy - Galactic chemical evolution in
python}
\begin{itemize}
\item {} 
By Jo Bovy. \sphinxstylestrong{kimmy} contains simple tools to study chemical
evolution in galaxies.

\end{itemize}

\end{itemize}


\paragraph{Spectra of Milky Way Stars}
\label{\detokenize{resource/astro/topics/galactic_astronomy:spectra-of-milky-way-stars}}

\subparagraph{Spectroscopic Surveys}
\label{\detokenize{resource/astro/topics/galactic_astronomy:spectroscopic-surveys}}\begin{itemize}
\item {} 
\sphinxhref{https://www.sdss.org/surveys/apogee/}{APOGEE - The APO Galactic Evolution
Experiment}

\item {} 
\sphinxhref{https://galah-survey.org/}{GALAH - GALactic Archaeology with
HERMES}
\begin{itemize}
\item {} 
Using HERMES on AAT. Resolution: R\textasciitilde{}28,000.

\end{itemize}

\item {} 
\sphinxhref{https://www.rave-survey.org/project/}{RAVE \textendash{} the Radial Velocity
Experiment}

\item {} 
\sphinxhref{http://dr4.lamost.org/}{LAMOST - Large Sky Area Multi-Object Fiber Spectroscopic
Telescope}

\end{itemize}


\subparagraph{Tools}
\label{\detokenize{resource/astro/topics/galactic_astronomy:tools}}\begin{itemize}
\item {} 
\sphinxhref{https://github.com/jobovy/apogee}{apogee - Tools for dealing with APOGEE
data}

\end{itemize}


\paragraph{Stellar structures in the Milky Way}
\label{\detokenize{resource/astro/topics/galactic_astronomy:stellar-structures-in-the-milky-way}}\begin{itemize}
\item {} 
\sphinxhref{https://github.com/cmateu/galstreams}{galstreams - Milky Way Streams Footprint Library and Toolkit for
Python}
\begin{itemize}
\item {} 
It includes a compilation of spatial information for known stellar
streams and overdensities in the Milky Way (MW) and Python tools
for visualizing them.

\end{itemize}

\end{itemize}


\paragraph{\sphinxstylestrong{Gaia} related}
\label{\detokenize{resource/astro/topics/galactic_astronomy:gaia-related}}

\subparagraph{Python tools:}
\label{\detokenize{resource/astro/topics/galactic_astronomy:python-tools}}\begin{itemize}
\item {} 
\sphinxhref{https://github.com/adrn/pyia}{pyia - A Python package for working with Gaia
data}

\item {} 
\sphinxhref{https://github.com/jobovy/gaia\_tools}{gaia\_tools - Tools for working with the ESA/Gaia data and related
data sets}

\item {} 
\sphinxhref{https://github.com/jobovy/tgas-completeness}{The Gaia DR1 TGAS completeness and a new stellar inventory of the
solar neighborhood}

\end{itemize}


\subsubsection{Observational and Theoretical Cosmology}
\label{\detokenize{resource/astro/topics/cosmology_tools:observational-and-theoretical-cosmology}}\label{\detokenize{resource/astro/topics/cosmology_tools::doc}}

\paragraph{Cosmology Parameters and Model Optimization}
\label{\detokenize{resource/astro/topics/cosmology_tools:cosmology-parameters-and-model-optimization}}\begin{itemize}
\item {} 
\sphinxhref{https://github.com/LSSTDESC/CCL}{CCL - DESC Core Cosmology Library: cosmology routines with validated
numerical accuracy}
\begin{itemize}
\item {} 
On top of \sphinxstylestrong{CCL}, there is \sphinxstylestrong{firecrown}:

\item {} 
\sphinxhref{https://github.com/LSSTDESC/firecrown}{firecrown: the “c” is for
“cosmology”}

\end{itemize}

\item {} 
\sphinxhref{https://github.com/cmbant/CosmoMC}{CosmoMC - MCMC parameter sampling
code}
\begin{itemize}
\item {} 
\sphinxstylestrong{CosmoMC} is a Fortran 2008 Markov-Chain Monte-Carlo (MCMC)
engine for exploring cosmological parameter space, together with
Fortran and python code for analysing Monte-Carlo samples and
importance sampling (plus a suite of scripts for building grids of
runs, plotting and presenting results).

\end{itemize}

\item {} 
\sphinxhref{https://github.com/cosmo-ethz/CosmoHammer}{CosmoHammer - Cosmological parameter estimation with the MCMC
Hammer}
\begin{itemize}
\item {} 
A paper describing the software can be \sphinxhref{https://arxiv.org/abs/1212.1721}{found
here}

\end{itemize}

\item {} 
\sphinxhref{https://github.com/CobayaSampler/cobaya}{Cobaya - Code for Bayesian Analysis in
Cosmology}
\begin{itemize}
\item {} 
\sphinxstylestrong{Cobaya} is a framework for sampling and statistical modelling:
it allows you to explore an arbitrary prior or posterior using a
range of Monte Carlo samplers (including the advanced MCMC sampler
from \sphinxstylestrong{CosmoMC}, and the advanced nested sampler \sphinxstylestrong{PolyChord}).

\end{itemize}

\item {} 
\sphinxhref{https://baudren.github.io/montepython.html}{MontePython - The Monte Carlo code for class in
Python}
\begin{itemize}
\item {} 
\sphinxstylestrong{MontePython} is a Monte Carlo code for Cosmological Parameter
extraction. It contains likelihood codes of most recent
experiments, and interfaces with the Boltzmann code \sphinxstylestrong{Class} for
computing the cosmological observables.

\end{itemize}

\end{itemize}


\paragraph{Supernova related}
\label{\detokenize{resource/astro/topics/cosmology_tools:supernova-related}}\begin{itemize}
\item {} 
\sphinxhref{https://github.com/sncosmo/sncosmo}{sncosmo - Python library for supernova
cosmology}
\begin{itemize}
\item {} 
\sphinxstylestrong{SNCosmo} is a Python library for supernova cosmology analysis.
It aims to make such analysis both as flexible and clear as
possible. \sphinxhref{https://sncosmo.readthedocs.io/en/v2.0.x/}{Online document is
here}

\item {} 
\sphinxhref{https://github.com/sncosmo/sndatasets}{sndatasets - Download and normalize published supernova
photometric data}

\end{itemize}

\end{itemize}


\paragraph{CMB related}
\label{\detokenize{resource/astro/topics/cosmology_tools:cmb-related}}\begin{itemize}
\item {} 
\sphinxhref{https://github.com/cmbant/CAMB}{CAMB - Code for Anisotropies in the Microwave
Background}
\begin{itemize}
\item {} 
\sphinxstylestrong{CAMB} is a cosmology code for calculating cosmological
observables, including CMB, lensing, source count and 21cm angular
power spectra, matter power spectra, transfer functions and
background evolution

\end{itemize}

\item {} 
\sphinxhref{https://github.com/nickhand/classylss}{CLassylss - a lightweight Python binding of the CLASS CMB Boltzmann
code}
\begin{itemize}
\item {} 
A very nice gateway to \sphinxstylestrong{CLASS}

\end{itemize}

\item {} 
\sphinxhref{http://class-code.net/}{CLASS - Cosmic Linear Anisotropy Solving
System}
\begin{itemize}
\item {} 
The purpose of \sphinxstylestrong{CLASS} is to simulate the evolution of linear
perturbations in the universe and to compute CMB and large scale
structure observables.

\item {} 
A public repository is available on
\sphinxhref{https://github.com/lesgourg/class\_public}{github}

\end{itemize}

\end{itemize}


\paragraph{Correlation Functions}
\label{\detokenize{resource/astro/topics/cosmology_tools:correlation-functions}}\begin{itemize}
\item {} 
\sphinxhref{https://github.com/manodeep/Corrfunc}{Corrfunc - Blazing fast correlation functions on the
CPU}

\item {} 
\sphinxhref{https://github.com/rmjarvis/TreeCorr}{TreeCorr - Code for efficiently computing 2-point and 3-point
correlation functions}
\begin{itemize}
\item {} 
It can compute correlations of regular number counts, weak lensing
shears, or scalar quantities such as convergence or CMB
temperature fluctutations.

\end{itemize}

\end{itemize}


\paragraph{Weak Lensing}
\label{\detokenize{resource/astro/topics/cosmology_tools:weak-lensing}}\begin{itemize}
\item {} 
Both \sphinxstylestrong{corrfunc} and \sphinxstylestrong{treecorr} can be used for shear-shear or
galaxy-shear analysis

\item {} 
\sphinxhref{https://github.com/apetri/LensTools}{LensTools - collects together a suite of widely used analysis tools
in Weak Gravitational
Lensing}

\item {} 
\sphinxhref{https://github.com/rmjarvis/DESWL}{DESWL - A collection of scripts and software related to DES weak
lensing analysis}
\begin{itemize}
\item {} 
By Marc Jarvis

\end{itemize}

\end{itemize}


\subparagraph{Cluster Lensing}
\label{\detokenize{resource/astro/topics/cosmology_tools:cluster-lensing}}\begin{itemize}
\item {} 
\sphinxhref{https://github.com/jesford/cluster-lensing}{cluster-lensing - Galaxy Cluster and Weak Lensing
Tools}
\begin{itemize}
\item {} 
By Jes Ford. \sphinxhref{https://iopscience.iop.org/article/10.3847/1538-3881/152/6/228/meta}{Paper can be found
here}

\end{itemize}

\item {} 
\sphinxhref{https://github.com/tmcclintock/cluster\_toolkit}{cluster\_toolkit - Tools for analyzing galaxy
clusters}
\begin{itemize}
\item {} 
by \sphinxhref{https://tmcclintock.github.io/}{Tom McClintock}. Contains
routines used in the Dark Energy Survey Year 1 stacked cluster
weak lensing analysis.

\end{itemize}

\end{itemize}


\paragraph{IGM Related (e.g. Lya Forrest)}
\label{\detokenize{resource/astro/topics/cosmology_tools:igm-related-e-g-lya-forrest}}\begin{itemize}
\item {} 
\sphinxhref{https://igmhub.github.io/}{IGMHUB - IGM analysis tools}
\begin{itemize}
\item {} 
\sphinxhref{https://github.com/igmhub/baofit}{baofit - Fits cosmological data to measure baryon acoustic
oscillations}
\begin{itemize}
\item {} 
\sphinxstylestrong{baofit} is a software package for analyzing cosmological
correlation functions to estimate parameters related to baryon
acoustic oscillations and redshift-space distortions

\end{itemize}

\end{itemize}

\end{itemize}


\paragraph{Dark Matter Halos}
\label{\detokenize{resource/astro/topics/cosmology_tools:dark-matter-halos}}\begin{itemize}
\item {} 
\sphinxhref{https://github.com/astropy/halotools}{Halotools - Python package for studying large scale structure,
cosmology, and galaxy evolution using N-body simulations and halo
models}
\begin{itemize}
\item {} 
\sphinxstylestrong{Halotools} is a specialized python package for building and
testing models of the galaxy-halo connection, and analyzing
catalogs of dark matter halos.

\end{itemize}

\item {} 
\sphinxhref{http://www.benediktdiemer.com/code/colossus/}{Colossus - a python toolkit for calculations pertaining to
cosmology, the large-scale structure of the universe, and the
properties of dark matter
halos}

\item {} 
\sphinxhref{https://correacamila.com/code/commah/}{COMMAH - COncentration-Mass relation and Mass Accretion
History}
\begin{itemize}
\item {} 
By Camila Correa; calculates dark matter halo concentrations as a
function of halo mass and redshift. The \sphinxhref{https://github.com/astroduff/commah}{source code is available
on Github}

\item {} 
Based on the works of \sphinxhref{https://arxiv.org/abs/1409.5228}{Correa et
al. 2015a}; \sphinxhref{https://arxiv.org/abs/1502.00391}{Correa et
al. 2015c}

\end{itemize}

\end{itemize}


\paragraph{Emulators:}
\label{\detokenize{resource/astro/topics/cosmology_tools:emulators}}\begin{itemize}
\item {} 
Increasingly popular way to study cosmology based on a limit set of
N-body simulations.

\end{itemize}


\subparagraph{Key Technique}
\label{\detokenize{resource/astro/topics/cosmology_tools:key-technique}}\begin{itemize}
\item {} 
A suite of N-body cosmology simulations
\begin{itemize}
\item {} 
2nd order Lagrangian perturbation theory (2LPT) initial conditions
\begin{itemize}
\item {} 
e.g. by \sphinxhref{http://cosmo.nyu.edu/roman/2LPT/}{2LPTIC} or on
Github \sphinxhref{https://github.com/manodeep/2LPTic}{here}

\end{itemize}

\item {} 
Input power spectrum. e.g. by \sphinxhref{https://camb.info}{CAMB: Code for Anisotropies in the
Microwave Background}

\end{itemize}

\item {} 
Sampling the cosmological parameters:
\begin{itemize}
\item {} 
Latin Hypercube Designs (LHDs)

\item {} 
\sphinxhref{https://www.asc.ohio-state.edu/statistics/comp\_exp/jour.club/optimal\_sliced\_lhd\_ba2015.pdf}{Maximin-distance “sliced” LHD
(SLHD)}
\begin{itemize}
\item {} 
\sphinxhref{https://pythonhosted.org/pyDOE/index.html}{A Python
implementation}

\item {} 
\sphinxhref{https://smt.readthedocs.io/en/latest/index.html}{SMT - Surrogate Modeling
Toolbox}

\item {} 
\sphinxhref{https://github.com/sahilm89/lhsmdu}{Another Python version}

\end{itemize}

\end{itemize}

\item {} 
Principle Component Analysis (PCA)
\begin{itemize}
\item {} 
e.g. \sphinxhref{https://github.com/sbailey/empca}{empca} by Stephen
Bailey

\end{itemize}

\item {} 
Gaussian process emulator
\begin{itemize}
\item {} 
e.g. \sphinxhref{http://dfm.io/george/current/}{george} by Dan
Foreman-Mackey

\end{itemize}

\end{itemize}


\subparagraph{Available Emulators}
\label{\detokenize{resource/astro/topics/cosmology_tools:available-emulators}}\begin{itemize}
\item {} 
\sphinxhref{https://aemulusproject.github.io}{Aemulus Project led by
Stanford}
\begin{itemize}
\item {} 
The basic structure of the code:
\sphinxhref{https://github.com/AemulusProject/Aemulator}{Aemulator}

\item {} 
Emulator of \sphinxhref{https://github.com/AemulusProject/hmf\_emulator}{halo mass
function} and
\sphinxhref{https://github.com/AemulusProject/bias\_emulator}{halo bias}

\item {} 
\sphinxhref{https://arxiv.org/abs/1804.05865}{The Aemulus Project I: Numerical Simulations for Precision
Cosmology}

\item {} 
\sphinxhref{https://arxiv.org/abs/1804.05866}{The Aemulus Project II: Emulating the Halo Mass
Function}

\item {} 
\sphinxhref{https://arxiv.org/abs/1804.05867}{The Aemulus Project III: Emulation of the Galaxy Correlation
Function}

\item {} 
Documents for \sphinxhref{https://aemulus-data.readthedocs.io/en/latest/}{data release
1}

\end{itemize}

\item {} 
\sphinxhref{http://www.hep.anl.gov/cosmology/CosmicEmu/emu.html}{CosmicEmu led by
ANL}
\begin{itemize}
\item {} 
Code can be found \sphinxhref{https://github.com/lanl/CosmicEmu}{here}

\item {} 
\sphinxstylestrong{CosmicEmu} produces predictions for the matter power spectrum
based on eight cosmological parametersand redshift.

\item {} 
Based on the \sphinxhref{https://arxiv.org/abs/1508.02654}{Mira-Titan
simulations}

\item {} 
Also related to the Coyote Universe emulator: \sphinxhref{https://arxiv.org/abs/0812.1052}{Paper
I}, \sphinxhref{https://arxiv.org/abs/0902.0429}{Paper
II}, \sphinxhref{https://arxiv.org/abs/0912.4490}{Paper
III}, and
\sphinxhref{https://arxiv.org/abs/1304.7849}{Extended}

\item {} 
Paper about the \sphinxhref{https://arxiv.org/abs/1311.6444}{emulated
power-spectrum}

\item {} 
Paper about the \sphinxhref{https://arxiv.org/abs/1210.1576}{emulated halo mass-concentration
relation}

\end{itemize}

\item {} 
ACME Emulator led by OSU
\begin{itemize}
\item {} 
Paper by Ben Wibking: \sphinxhref{http://adsabs.harvard.edu/doi/10.1093/mnras/sty2258}{Emulating galaxy clustering and
galaxy-galaxy lensing into the deeply nonlinear
regime}

\item {} 
Use the \sphinxhref{https://lgarrison.github.io/AbacusCosmos/}{AbacusCosmos suite of
simulations}
\begin{itemize}
\item {} 
The code used for the simulation is
\sphinxhref{https://github.com/lgarrison/AbacusCosmos}{here}

\item {} 
The \sphinxhref{https://arxiv.org/abs/1712.05768}{AbacusCosmos description
paper}

\end{itemize}

\end{itemize}

\item {} 
Dark Emulator led by IPMU
\begin{itemize}
\item {} 
Based on the Dark Quest suite of simulations.

\item {} 
\sphinxhref{http://adsabs.harvard.edu/abs/2018arXiv181109504N}{Dark Quest. I. Fast and Accurate Emulation of Halo Clustering
Statistics and Its Application to Galaxy
Clustering}

\end{itemize}

\end{itemize}


\subsubsection{Stellar Physics and Stellar Population}
\label{\detokenize{resource/astro/topics/stellar_and_spops:stellar-physics-and-stellar-population}}\label{\detokenize{resource/astro/topics/stellar_and_spops::doc}}

\paragraph{Stellar Evolution Code}
\label{\detokenize{resource/astro/topics/stellar_and_spops:stellar-evolution-code}}\begin{itemize}
\item {} 
\sphinxhref{http://mesa.sourceforge.net/}{MESA - Modules for Experiments in Stellar
Astrophysics}
\begin{itemize}
\item {} 
The most important \sphinxstylestrong{open} 1-D stellar evolution code on the
market.

\end{itemize}

\item {} 
\sphinxhref{https://www.ast.cam.ac.uk/~stars/}{stars - A Stellar Evolution
Code}
\begin{itemize}
\item {} 
The classic.

\end{itemize}

\item {} 
\sphinxhref{https://www.ast.cam.ac.uk/~rgi/binstar.html}{BINSTAR - a detailed binary stellar evolution
code}

\end{itemize}


\subparagraph{Hydro-simulation}
\label{\detokenize{resource/astro/topics/stellar_and_spops:hydro-simulation}}\begin{itemize}
\item {} 
\sphinxhref{https://github.com/AMReX-Astro/MAESTROeX}{MAEXTROeX - A C++/F90 low Mach number stellar hydrodynamics
code}
\begin{itemize}
\item {} 
\sphinxhref{https://amrex-astro.github.io/MAESTROeX/index.html}{MAESTROeX}
solves the equations of low Mach number hydrodynamics for
stratified atmospheres/stars with a general equation of state.

\end{itemize}

\end{itemize}


\paragraph{Isochrones}
\label{\detokenize{resource/astro/topics/stellar_and_spops:isochrones}}

\subparagraph{Softwares}
\label{\detokenize{resource/astro/topics/stellar_and_spops:softwares}}\begin{itemize}
\item {} 
\sphinxhref{https://github.com/timothydmorton/isochrones}{isochrones - Provides simple interface for interacting with stellar
model grids}
\begin{itemize}
\item {} 
The central goal of isochrones is to standardize model-grid-based
stellar parameter inference, and to enable such inference under
different sets of stellar models.

\item {} 
Online \sphinxhref{https://isochrones.readthedocs.io/en/latest/}{documents is
here}

\end{itemize}

\end{itemize}


\subparagraph{Libraries}
\label{\detokenize{resource/astro/topics/stellar_and_spops:libraries}}\begin{itemize}
\item {} 
\sphinxhref{http://waps.cfa.harvard.edu/MIST/}{MIST - MESA Isochrones \& Stellar
Tracks}
\begin{itemize}
\item {} 
Key papers: \sphinxhref{http://adsabs.harvard.edu/abs/2016ApJ...823..102C}{Choi et
al. 2016}

\end{itemize}

\item {} 
\sphinxhref{http://stev.oapd.inaf.it/cgi-bin/cmd}{PARSEC: stellar tracks and isochrones with the PAdova and TRieste
Stellar Evolution Code}
\begin{itemize}
\item {} 
Key papers: \sphinxhref{https://arxiv.org/abs/1208.4498}{Bressan et
al. 2012}

\item {} 
\sphinxhref{https://philrosenfield.github.io/padova\_tracks/}{Interpolated PARSEC Stellar Evolution
Tracks}

\end{itemize}

\item {} 
\sphinxhref{http://pleiadi.pd.astro.it/}{Padova database of stellar evolutionary tracks and
isochrones}

\item {} 
\sphinxhref{http://www.astro.yale.edu/yapsi/}{YaPSI - Yale-Postdam Stellar
Isochrones}

\item {} 
\sphinxhref{http://obswww.unige.ch/~ekstrom/WWW/evol/recherche/database.html}{Geneva Grids of Stellar Evolution
Models}
\begin{itemize}
\item {} 
\sphinxhref{https://www.unige.ch/sciences/astro/evolution/en/database/syclist/}{Interactive online tool to generate
isochrones}

\end{itemize}

\item {} 
\sphinxhref{http://basti.oa-teramo.inaf.it/}{BaSTI - A Bag of Stellar Tracks and
Isocrhones}

\end{itemize}


\paragraph{Stellar Spectra Libraries and SED}
\label{\detokenize{resource/astro/topics/stellar_and_spops:stellar-spectra-libraries-and-sed}}

\subparagraph{Observed}
\label{\detokenize{resource/astro/topics/stellar_and_spops:observed}}\begin{itemize}
\item {} 
\sphinxhref{https://www.eso.org/sci/facilities/paranal/decommissioned/isaac/tools/lib.html}{Pickles et al. 1998
library}
\begin{itemize}
\item {} 
The spectral range is 1150-25000 Anstroms, sampling 5 Angstroms.
131 stellar spectra.

\item {} 
The \sphinxhref{http://www.stsci.edu/hst/observatory/crds/pickles\_atlas.html}{Pickles library can also be found
here}

\end{itemize}

\item {} 
\sphinxhref{https://www.noao.edu/cflib/}{The Indo-U.S. Library of Coudé Feed Stellar
Spectra}
\begin{itemize}
\item {} 
1273 stars obtained with the 0.9m Coudé Feed telescope at Kitt
Peak National Observatory

\item {} 
Original dispersion of 0.44 Angstroms/pixel, at a resolution of \textasciitilde{}1
Angstroms FWHM.

\item {} 
Cover the entire wavelength range of 3460 Angstroms to 9464
Angstroms

\end{itemize}

\item {} 
\sphinxhref{http://www.iac.es/proyecto/miles/pages/stellar-libraries.php}{MILES Stellar
Library}
\begin{itemize}
\item {} 
\sphinxstylestrong{MILES}: \textasciitilde{}1000 stars spanning a large range in atmospheric
parameters. The spectra were obtained at the 2.5m INT telescope
and cover the range 3525-7500 Å at 2.5 Å (FWHM) spectral
resolution

\item {} 
\sphinxstylestrong{CaII Triplet Library}: 700 stars with spectra around the Ca II
triplet region. Cover the spectral range between 8350-9020 Å at
1.5 Å (FWHM)

\end{itemize}

\item {} 
\sphinxhref{http://xsl.u-strasbg.fr/}{The X-Shooter Spectral Library}
\begin{itemize}
\item {} 
The current release contains more than 200 stars. 3000\textendash{}25000 Å all
stellar spectra observed at a resolving power of R = λ/Δλ \textasciitilde{} 10000
with the medium-resolution spectrograph X-Shooter

\end{itemize}

\item {} 
\sphinxhref{https://www.sdss.org/surveys/mastar/}{MaStar - MaNGA Stellar
Library}
\begin{itemize}
\item {} 
A stellar spectral library with a very comprehensive stellar
parameter coverage, a large sample size, and high quality
calibrations, using the same instrument as used by the MaNGA
survey

\item {} 
Wavelength: 362-1035 nm, resolution R\textasciitilde{}2000; More than 8000 stars
in wide areas of the sky.

\end{itemize}

\item {} 
\sphinxhref{http://svo2.cab.inta-csic.es/vocats/v2/stelib/}{STELIB Stellar
Library}
\begin{itemize}
\item {} 
A stellar Library for stellar population synthesis models

\item {} 
An homogeneous library of stellar spectra in the visible range
(3200 to 9500A), including stars of all spectral types, luminosity
classes and metallicity. The spectral resolution of our Stellar
Library is about 3A FWHM.

\end{itemize}

\item {} 
\sphinxhref{http://atlas.obs-hp.fr/elodie/}{ÉLODIE Stellar Library}
\begin{itemize}
\item {} 
A stellar database of 1959 spectra for 1503 stars, observed with
the echelle spectrograph ÉLODIE on the 193 cm telescope at the
Observatoire de Haute Provence.

\item {} 
Wavelength range is 400-680 nm. For the purpose of population
synthesis, the original resolution R=42 000 has been reduced to
R=10 000 at lambda=550 nm, or more precisely to a gaussian
instrumental profile of FWHM\textasciitilde{}0.55 A over the whole range of
wavelengths.

\end{itemize}

\item {} 
\sphinxhref{https://lco.global/~apickles/INGS/}{INGS Stellar Library}
\begin{itemize}
\item {} 
INGS is a compendium of 143 stellar-type spectra formed from
spectra of stars of similar type from 3 sources: 1) IUE: 1153A to
3201A, 2A/pixel, 2) NGSL v2: 1600A to 11000A; 3) SpeX/IRTF: 8110A
to 25000+A, 2.5,3,5A/pixel

\end{itemize}

\item {} 
\sphinxhref{https://www.blancocuaresma.com/s/benchmarkstars}{The Gaia FGK Benchmark
Stars}
\begin{itemize}
\item {} 
Library of high resolution and high signal to noise ratio stellar
spectra

\item {} 
A homogeneous library in the visual range (480-680 nm) of high
resolution and signal to noise ratio (S/N) spectra corresponding
to the 34 Benchmark Stars and 5 metal-poor candidates.

\end{itemize}

\end{itemize}


\subparagraph{Near-Mid Infrared}
\label{\detokenize{resource/astro/topics/stellar_and_spops:near-mid-infrared}}\begin{itemize}
\item {} 
\sphinxhref{http://irtfweb.ifa.hawaii.edu/~spex/IRTF\_Extended\_Spectral\_Library/}{IRTF Extended Spectral
Library}
\begin{itemize}
\item {} 
A collection of 0.7-2.5 μm (with a subset from 0.8 to 5.2 μm)
mostly stellar spectra observed at a resolving power of R ≡ λ/Δλ \textasciitilde{}
2000 with the medium-resolution spectrograph, SpeX. Please see
\sphinxhref{https://arxiv.org/abs/1705.08906}{Villaume et al. 2017 for
details}

\item {} 
The original \sphinxhref{http://irtfweb.ifa.hawaii.edu/~spex/IRTF\_Spectral\_Library/}{IRTF spectral library is available
here}

\end{itemize}

\item {} 
\sphinxhref{http://pono.ucsd.edu/~adam/browndwarfs/spexprism/}{SpeX Prism
Library}
\begin{itemize}
\item {} 
A repository of low-resolution, near-infrared spectra, primarily
of low-temperature dwarf stars and brown dwarfs

\item {} 
Resolutions are R\textasciitilde{}75-200. Wavelength coverage is 0.65-2.55 microns
in a single order

\end{itemize}

\end{itemize}


\subparagraph{Theoretical}
\label{\detokenize{resource/astro/topics/stellar_and_spops:theoretical}}\begin{itemize}
\item {} 
\sphinxhref{http://marcs.astro.uu.se/}{MARCS - a grid of one-dimensional, hydrostatic, plane-parallel and
spherical LTE model atmospheres}
\begin{itemize}
\item {} 
The MARCS site contains about 52,000 stellar atmospheric models of
spectral types F, G and K in 3 different formats and also flux
sample files indicating rough surface fluxes.

\end{itemize}

\item {} 
\sphinxhref{http://cdsarc.u-strasbg.fr/viz-bin/qcat?J/A+A/442/1127}{SYNTHE - synthetic library of stellar
spectra}
\begin{itemize}
\item {} 
Covers the wavelength region from 2500 Å to 1.05 μm at a spectral
resolution of σ=6.4 km/s, R=20,000.

\end{itemize}

\item {} 
\sphinxhref{https://www.inaoep.mx/~modelos/uvblue/uvblue.html}{UVBLUE - A High-resolution Theoretical Library of Stellar
Spectra}
\begin{itemize}
\item {} 
A high-resolution library of synthetic spectra of stars covering
the ultraviolet wavelength range.

\item {} 
Stellar spectra cover the wavelength interval from 850 to 4700 Å,
at a spectral resolving power R = λ/Δ λ = 50,000. The grid
consists of 1770 SEDs

\end{itemize}

\item {} 
\sphinxhref{http://specmodels.iag.usp.br/}{Paula Coelho’s Theoretical Spectra of Stars and Stellar
Populations}
\begin{itemize}
\item {} 
A new theoretical library which covers 3000 \textless{}= Teff \textless{}= 25 000 K,
-0.5 \textless{}= log g \textless{}= 5.5 and 12 chemical mixtures covering 0.0017 \textless{}= Z
\textless{}= 0.049 at both scaled-solar and alpha-enhanced compositions.

\end{itemize}

\end{itemize}


\subparagraph{Tools for Stellar Physics}
\label{\detokenize{resource/astro/topics/stellar_and_spops:tools-for-stellar-physics}}\begin{itemize}
\item {} 
\sphinxhref{https://github.com/BEAST-Fitting/beast}{BEAST - Bayesian Extinction and Stellar
Tool}
\begin{itemize}
\item {} 
Fits the ultraviolet to near-infrared photometric SEDs of stars to
extract stellar and dust extinction parameters. See \sphinxhref{http://adsabs.harvard.edu/abs/2016ApJ...826..104G}{Gordon et
al. 2016}
for details.

\item {} 
Online document is
\sphinxhref{https://beast.readthedocs.io/en/latest/}{here}

\end{itemize}

\item {} 
\sphinxhref{https://github.com/pacargile/ThePayne}{ThePayne - Artificial Neural-Net compression and fitting of
synthetic spectral grids}
\begin{itemize}
\item {} 
By \sphinxhref{https://www.cfa.harvard.edu/~pcargile/}{Phillip Cargile}
and \sphinxhref{https://www.sns.ias.edu/~ting/}{Yuan-Sen Ting}. Artificial
Neural-Net compression and fitting of ab initio synthetic spectral
grids.

\end{itemize}

\item {} 
\sphinxhref{https://github.com/annayqho/TheCannon}{TheCannon - a data-driven method for determining stellar parameters
and abundances from stellar
spectra}
\begin{itemize}
\item {} 
By \sphinxhref{https://annayqho.github.io/}{Anna Ho}. A data-driven method
for determining stellar labels (physical parameters and chemical
abundances) from stellar spectra in the context of large
spectroscopic surveys.

\end{itemize}

\item {} 
\sphinxhref{https://github.com/joshspeagle/brutus}{brutus - Modeling stellar photometry with “brute force”
methods}
\begin{itemize}
\item {} 
By Josh Speagle. A Pure Python package whose core modules involve
using “brute force” Bayesian inference to derive distances,
reddenings, and stellar properties from photometry using a grid of
stellar models.

\end{itemize}

\item {} 
\sphinxhref{https://github.com/iancze/Starfish}{Starfish - Tools for Flexible Spectroscopic
Inference}
\begin{itemize}
\item {} 
By \sphinxhref{http://iancze.github.io/}{Ian Czekala}. Starfish is a set
of tools used for spectroscopic inference. We designed the package
to robustly determine stellar parameters using high resolution
spectral models

\end{itemize}

\item {} 
\sphinxhref{http://www.as.utexas.edu/~chris/moog.html}{MOOG - a code that performs a variety of LTE line analysis and
spectrum synthesis
tasks}
\begin{itemize}
\item {} 
Old fashion but classic.

\item {} 
If you use Python, try Andy Casey’s \sphinxhref{https://github.com/andycasey/moog}{Installing MOOG the Easy
Way}

\end{itemize}

\end{itemize}


\paragraph{Stellar Population Synthesis or SED Fitting}
\label{\detokenize{resource/astro/topics/stellar_and_spops:stellar-population-synthesis-or-sed-fitting}}\begin{itemize}
\item {} 
This \sphinxhref{http://www.sedfitting.org/Models.html}{sedfitting.org page}
is a very good one-stop shopping place for all SED related resources.
\begin{itemize}
\item {} 
There is also a \sphinxhref{http://www.sedfitting.org/Paper\_vs1.0\_online/walcher\_ms.html}{review
paper}

\end{itemize}

\end{itemize}


\subparagraph{Tools for Generating or Manipulating Stellar Population Model}
\label{\detokenize{resource/astro/topics/stellar_and_spops:tools-for-generating-or-manipulating-stellar-population-model}}\begin{itemize}
\item {} 
\sphinxhref{https://github.com/cconroy20/fsps}{FSPS - Flexible Stellar Population
Synthesis}
\begin{itemize}
\item {} 
If you want to see how sausage is made, this is it, including
every details of stellar population synthesis. Original code in
\sphinxstylestrong{Frotran}. Supports different isochrones and libraries.

\item {} 
\sphinxhref{http://dfm.io/python-fsps/current/}{python-fsps} can help you
use it in \sphinxstylestrong{Python}

\item {} 
\sphinxhref{https://github.com/nell-byler/cloudyfsps}{cloudyfsps - Python interface between FSPS and
Cloudy}

\end{itemize}

\item {} 
\sphinxhref{https://github.com/bd-j/sedpy}{sedpy - Utilities for astronomical spectral energy
distributions}
\begin{itemize}
\item {} 
By Ben Johnson. Modules for storing and operating on astronomical
source spectral energy distributions.

\item {} 
Has nice function to handle filters and measure SED from spectrum.

\end{itemize}

\item {} 
\sphinxhref{https://github.com/astropy/PopStar}{PopStar - generating simple stellar populations from synthetic
evolution and atmosphere
models}
\begin{itemize}
\item {} 
PopStar generates single-age, single-metallicity populations
(i.e. star clusters). It supports different initial mass
functions, multiplicity perscriptions, reddening laws, filter
functions, atmosphere models, and evolution models.

\item {} 
Support a large variety of theoretical models.

\end{itemize}

\end{itemize}


\subparagraph{Tools for SED or Spectral Fitting}
\label{\detokenize{resource/astro/topics/stellar_and_spops:tools-for-sed-or-spectral-fitting}}\begin{itemize}
\item {} 
\sphinxhref{https://github.com/bd-j/prospector}{prospector - Python code for Stellar Population Inference from
Spectra and SEDs}
\begin{itemize}
\item {} 
By Ben Johnson. Conduct principled inference of stellar population
properties from photometric and/or spectroscopic data.

\item {} 
Bayesian method, can use \sphinxstylestrong{emcee}, \sphinxstylestrong{nestle}, and \sphinxstylestrong{dynesty} as
sampling tool

\item {} 
Can fit spectrum and/or SED.

\end{itemize}

\item {} 
\sphinxhref{https://github.com/cschreib/fastpp}{fastpp - C++ version of the SED fitting code FAST (Kriek et
al. 2009);}
\begin{itemize}
\item {} 
By Corentin Schreiber. “it’s faster, uses less memory, and has
more features.”

\item {} 
Based on model grid.

\item {} 
Can fit spectrum and/or SED.

\end{itemize}

\item {} 
\sphinxhref{http://www.starlight.ufsc.br/}{STARLIGHT - Spectra decomposition
code}
\begin{itemize}
\item {} 
Written in Fortran, using simulated annealing algorithm with
Markov chains.

\item {} 
\sphinxhref{http://minerva.ufsc.br/starlight/files/papers/Manual\_StCv04.pdf}{The manual written by Cid
Fernandes}
is a very good introduction of the SSP decomposition business.

\item {} 
Mostly used for spectral fitting.

\end{itemize}

\item {} 
\sphinxhref{http://www-astro.physics.ox.ac.uk/~mxc/software/\#ppxf}{pPXF - Penalized
Pixel-Fitting}
\begin{itemize}
\item {} 
By Michelle Cappellari. Extract the stellar or gas kinematics and
stellar population from galaxy spectra via full spectrum fitting.

\item {} 
Available in Python and IDL. Can fit spectrum.

\end{itemize}

\item {} 
\sphinxhref{https://github.com/moustakas/impro}{iSEDfit - IDL routines to fit
SED}
\begin{itemize}
\item {} 
By John Moustakas. Part of the \sphinxstylestrong{impro} suite. \sphinxhref{http://www.sos.siena.edu/~jmoustakas/isedfit/}{Website for
downloading library and documents is
here}

\item {} 
Based on model grid, only fit SED.

\end{itemize}

\item {} 
\sphinxhref{https://github.com/FireflySpectra/firefly\_release}{Firefly \textendash{} A Full Spectral Fitting
Code}
\begin{itemize}
\item {} 
FIREFLY is a chi-squared minimisation fitting code for deriving
the stellar population properties of stellar systems, be these
observed galaxy or star cluster spectra, or model spectra from
simulations. Document can be \sphinxhref{http://www.icg.port.ac.uk/firefly/}{found
here}

\end{itemize}

\item {} 
\sphinxhref{https://gitlab.lam.fr/cigale/cigale/}{cigale - Python version of the Code Investigating GALaxy
Emission}
\begin{itemize}
\item {} 
\sphinxhref{https://cigale.lam.fr/}{Full document can be found here}

\end{itemize}

\item {} 
\sphinxhref{http://www.astroscu.unam.mx/~sfsanchez/FIT3D/}{FIT3D - a tool for fitting stellar populations and emission lines in
optical
spectroscopy}
\begin{itemize}
\item {} 
FIT3D is a package for fitting optical spectra to deblend the
underlying stellar population and the ionized gas, and extract
physical information from each component. Focusing on IFU surveys.

\item {} 
Fit full spectrum. In Python or Perl

\end{itemize}

\item {} 
\sphinxhref{http://www.jacopochevallard.org/beagle/}{BEAGLE - BayEsian Analysis of GaLaxy
sEds}
\begin{itemize}
\item {} 
A new-generation tool to model and interpret galaxy spectral
energy distributions (SEDs) developed by Jacopo Chevallard (ESA)
and Stephane Charlot (IAP).

\end{itemize}

\end{itemize}


\subparagraph{Stellar Population Models:}
\label{\detokenize{resource/astro/topics/stellar_and_spops:stellar-population-models}}\begin{itemize}
\item {} 
\sphinxhref{http://miles.iac.es/}{MILES - Population synthesis for the 21st
Century}
\begin{itemize}
\item {} 
The new extended MILES
(\sphinxhref{http://adsabs.harvard.edu/abs/2016MNRAS.463.3409V}{E-MILES})
models covering from 1680Å to 5.0μm

\item {} 
Provides useful \sphinxhref{http://www.iac.es/proyecto/miles/pages/webtools.php}{online
tools} to
use the stellar library and SSP models.

\end{itemize}

\item {} 
\sphinxhref{https://bpass.auckland.ac.nz/}{BPASS - Binary Population and Spectral Synthesis
code}
\begin{itemize}
\item {} 
Best binary population model on the market.

\end{itemize}

\item {} 
\sphinxhref{http://astro.u-strasbg.fr/~morgan/PEGASE.html}{PEGASE - Projet d’Etude des GAlaxies par Synthese
Evolutive}

\end{itemize}


\subsubsection{Transients and Other Time Series Science}
\label{\detokenize{resource/astro/topics/transient_and_time_domain:transients-and-other-time-series-science}}\label{\detokenize{resource/astro/topics/transient_and_time_domain::doc}}\begin{itemize}
\item {} 
Focusing on the detection and analysis of any object that shows flux
variation.
\begin{itemize}
\item {} 
For example: variable stars, TDE, and eclipsing binaries.

\item {} 
\sphinxstylestrong{Exoplanet} and \sphinxstylestrong{supernova} are also trasients, but they have
much broader impact therefore they have their own topics.

\end{itemize}

\end{itemize}


\paragraph{Transient Notification}
\label{\detokenize{resource/astro/topics/transient_and_time_domain:transient-notification}}\begin{itemize}
\item {} 
\sphinxhref{https://github.com/lpsinger/pygcn}{pycgn - Python package for processing Gamma-ray Coordinates Network
(GCN) notices and circulars}
\begin{itemize}
\item {} 
Anonymous VOEvent client for receiving GCN/TAN notices in XML
format

\end{itemize}

\end{itemize}


\paragraph{Data Access}
\label{\detokenize{resource/astro/topics/transient_and_time_domain:data-access}}\begin{itemize}
\item {} 
\sphinxhref{https://github.com/MickaelRigault/ztfquery}{ztfquery - Access ZTF data from
Python}
\begin{itemize}
\item {} 
By Mickael Rigault. \sphinxstylestrong{ztfquery} is a python tool to download ZTF
(and SEDM) data

\end{itemize}

\end{itemize}


\paragraph{Differential Photometry}
\label{\detokenize{resource/astro/topics/transient_and_time_domain:differential-photometry}}\begin{itemize}
\item {} 
\sphinxhref{https://github.com/vterron/lemon}{lemon - Differential photometry for humans (and
astronomers)}
\begin{itemize}
\item {} 
By Víctor Terrón. \sphinxstylestrong{LEMON} is a differential-photometry pipeline,
written in Python, that determines the changes in the brightness
of astronomical objects over time and compiles their measurements
into light curves.

\end{itemize}

\item {} 
\sphinxhref{https://github.com/acbecker/hotpants}{hotpants - High Order Transform of Psf ANd Template Subtraction
code}
\begin{itemize}
\item {} 
By Andy Becker.

\end{itemize}

\item {} 
\sphinxhref{https://github.com/waqasbhatti/astrobase}{astrobase - Python modules for light curve work and variable star
astronomy}
\begin{itemize}
\item {} 
By \sphinxhref{https://wbhatti.org/}{Waqas Bhatti}. It includes
implementations of several period-finding algorithms, batch work
drivers for working on large collections of light curves, and a
small web-app useful for reviewing and classifying light curves by
stellar variability type.

\end{itemize}

\end{itemize}


\paragraph{Trasient Identification and Classification}
\label{\detokenize{resource/astro/topics/transient_and_time_domain:trasient-identification-and-classification}}\begin{itemize}
\item {} 
\sphinxhref{https://github.com/kboone/avocado}{avocado - Photometric Classification of Astronomical Transients and
Variables With Biased Spectroscopic
Samples}
\begin{itemize}
\item {} 
\sphinxstylestrong{Avocado} is a general photometric classification code that is
designed to produce classifications of arbitrary astronomical
transients and variable objects.

\item {} 
The original codebase of avocado was developed for and won the
\sphinxhref{https://www.kaggle.com/c/PLAsTiCC-2018}{2018 Kaggle PLAsTiCC
challenge}.

\end{itemize}

\item {} 
\sphinxhref{https://github.com/daniel-muthukrishna/astrorapid}{astrorapid - Real-time Automated Photometric IDentification (RAPID)
of astronomical transients using deep
learning}
\begin{itemize}
\item {} 
By \sphinxhref{http://www.danielmuthukrishna.com/}{Daniel Muthukrishna}.
\sphinxstylestrong{RAPID} (Real-time Automated Photometric IDentification) can
classify multiband photometric light curves into several different
transient classes. It uses a deep recurrent neural network to
produce time-varying classifications.

\end{itemize}

\item {} 
\sphinxhref{https://github.com/daniel-muthukrishna/astrodash}{astrodash - Deep learning for the automated spectral classification
of supernovae}
\begin{itemize}
\item {} 
By \sphinxhref{http://www.danielmuthukrishna.com/}{Daniel Muthukrishna}.
Software to classify the type, age, redshift and host for any
supernova spectra. Two platforms exists: a python library that
enables a user to classify several spectra (can classify thousands
of spectra in seconds), and also a graphical interface that
enables a user to view and classify a spectrum.

\end{itemize}

\item {} 
\sphinxhref{https://github.com/dwkim78/upsilon}{UPSILoN- Automated Classification of Periodic Variable Stars Using
Machine Learning}
\begin{itemize}
\item {} 
\sphinxstylestrong{UPSILoN} (AUtomated Classification of Periodic Variable Stars
using MachIne LearNing)

\end{itemize}

\item {} 
\sphinxhref{https://github.com/supernnova/SuperNNova/}{SuperNNova - Open Source Photometric
classification}
\begin{itemize}
\item {} 
Using recurrent network technique. Based on \sphinxhref{https://arxiv.org/abs/1901.06384}{SuperNNova: an
open-source framework for Bayesian, Neural Network based supernova
classification}

\end{itemize}

\end{itemize}


\paragraph{Lightcurve and Exoplanet}
\label{\detokenize{resource/astro/topics/transient_and_time_domain:lightcurve-and-exoplanet}}\begin{itemize}
\item {} 
\sphinxhref{https://github.com/KeplerGO/lightkurve}{lightkurve - A friendly package for Kepler \& TESS time series
analysis in Python}
\begin{itemize}
\item {} 
\sphinxstylestrong{Lightkurve} is a community-developed, open-source Python
package which offers a beautiful and user-friendly way to analyze
astronomical flux time series data, in particular the pixels and
lightcurves obtained by NASA’s Kepler and TESS exoplanet missions.

\end{itemize}

\item {} 
\sphinxhref{https://github.com/afeinstein20/eleanor}{eleanor - light curves from
TESS}
\begin{itemize}
\item {} 
\sphinxstylestrong{eleanor} is a python package to extract target pixel files from
TESS Full Frame Images and produce systematics-corrected light
curves for any star observed by the TESS mission.

\end{itemize}

\item {} 
\sphinxhref{https://github.com/rodluger/starry}{starry - Analytic occultation light curves for
astronomy}
\begin{itemize}
\item {} 
By \sphinxhref{https://rodluger.github.io/}{Rodrigo Luger}. Based on \sphinxhref{https://docs.google.com/viewer?url=https://github.com/rodluger/starry/raw/master-pdf/tex/starry.pdf}{a
very nice
paper}

\end{itemize}

\item {} 
\sphinxhref{https://github.com/rodluger/everest}{everest - EPIC Variability Extraction and Removal for Exoplanet
Science Targets}
\begin{itemize}
\item {} 
A pipeline for de-trending K2 light curves with pixel level
decorrelation and Gaussian processes.{]}

\end{itemize}

\item {} 
\sphinxhref{https://github.com/hippke/wotan}{wotan - Automagically remove trends from time-series
data}
\begin{itemize}
\item {} 
By \sphinxhref{http://www.jaekle.info/}{Michael Hippke}. Offers free and
open source algorithms to automagically remove trends from
time-series data.

\end{itemize}

\end{itemize}


\subsubsection{Interstellar and Intra-Galactic Medium, Dust}
\label{\detokenize{resource/astro/topics/ism_and_dust:interstellar-and-intra-galactic-medium-dust}}\label{\detokenize{resource/astro/topics/ism_and_dust::doc}}

\paragraph{Mily Way dust extinction map}
\label{\detokenize{resource/astro/topics/ism_and_dust:mily-way-dust-extinction-map}}

\subparagraph{Maps}
\label{\detokenize{resource/astro/topics/ism_and_dust:maps}}\begin{itemize}
\item {} 
\sphinxhref{https://lambda.gsfc.nasa.gov/product/foreground/fg\_ebv\_map.cfm}{SFD98 Dust
Map}
\begin{itemize}
\item {} 
Schlegel, Finkbeiner \& Davis (see 1998 reference below) derived an
all-sky map of Galactic reddening, E(B-V), from a composite 100
micron map formed from IRAS/ISSA maps calibrated using DIRBE
observations.

\end{itemize}

\item {} 
\sphinxhref{http://argonaut.skymaps.info/}{3D Dust Mapping with Pan-STARRS 1, 2MASS and
Gaia}
\begin{itemize}
\item {} 
These dust maps are based on Pan-STARRS 1 photometry of 800
million stars, 2MASS photometry of 200 million stars, and Gaia
parallaxes of 500 million stars.

\end{itemize}

\item {} 
\sphinxhref{https://arxiv.org/abs/1507.01005}{Green et al. 2016 - A Three-Dimensional Map of Milky-Way
Dust}

\item {} 
\sphinxhref{https://dataverse.harvard.edu/dataset.xhtml?persistentId=doi:10.7910/DVN/AV9GXO}{Green et al. 2019 - High-Galactic-latitude 2D dust map using
Gaussian
Processes}

\end{itemize}


\subparagraph{Tools}
\label{\detokenize{resource/astro/topics/ism_and_dust:tools}}\begin{itemize}
\item {} 
\sphinxhref{https://github.com/adrn/SFD}{SFD - Schlegel, Finkbeiner, David dust maps in
Python}

\item {} 
\sphinxhref{https://github.com/gregreen/dustmaps}{dustmaps - A uniform interface for a number of 2D and 3D maps of
interstellar dust
reddening/extinction}
\begin{itemize}
\item {} 
By \sphinxhref{http://greg.ory.gr/een\#papers}{Gregory Green}. Including
multiple dust maps.

\end{itemize}

\item {} 
\sphinxhref{https://github.com/jobovy/mwdust}{mwdust - Dust in 3D in the Milky
Way}
\begin{itemize}
\item {} 
By Jo Bovy. Including multiple dust maps.

\end{itemize}

\item {} 
\sphinxhref{https://github.com/karllark/dust\_extinction}{dust\_extinction - Astronomical Dust
Extinction}
\begin{itemize}
\item {} 
This package provides astronomical interstellar dust extinction
curves implemented using the astropy.modeling framework.

\end{itemize}

\end{itemize}


\paragraph{Interstellar Dust Model}
\label{\detokenize{resource/astro/topics/ism_and_dust:interstellar-dust-model}}\begin{itemize}
\item {} 
\sphinxhref{https://www.ias.u-psud.fr/themis/THEMIS\_model.html}{THEMIS - The Heterogeneous dust Evolution Model for Interstellar
Solids}

\item {} 
\sphinxhref{http://www.inasan.rssi.ru/~khramtsova/SHIVA.html}{SHIVA - a numerical tool for evolution of dust in the
ISM}
\begin{itemize}
\item {} 
\sphinxstylestrong{SHIVA} simulates the dust destruction in warm neutral, warm
ionized, and hot ionized media under the influence of
photo-processing, sputtering, and shattering. Based on \sphinxhref{https://arxiv.org/pdf/1906.11308.pdf}{this
work}

\end{itemize}

\end{itemize}


\paragraph{Extragalactic ISM and Dust Dataset}
\label{\detokenize{resource/astro/topics/ism_and_dust:extragalactic-ism-and-dust-dataset}}\begin{itemize}
\item {} 
\sphinxhref{http://dustpedia.com/}{DustPedia: A Definitive Study of Cosmic Dust in the Local
Universe}

\end{itemize}


\subsubsection{High Energy Astrophysics and Astroparticle Physics}
\label{\detokenize{resource/astro/topics/high_energy_astrophy:high-energy-astrophysics-and-astroparticle-physics}}\label{\detokenize{resource/astro/topics/high_energy_astrophy::doc}}
\sphinxstylestrong{Just started}


\bigskip\hrule\bigskip

\begin{itemize}
\item {} 
\sphinxhref{https://github.com/StingraySoftware/stingray}{Stingray - X-Ray Spectral Timing Made
Easy}
\begin{itemize}
\item {} 
\sphinxstylestrong{Stingray} is an in-development spectral-timing software package
for astrophysical X-ray (and more) data

\item {} 
Based on the work by \sphinxhref{https://arxiv.org/abs/1901.07681}{D. Huppenkothen et al. 2019: Stingray: A
Modern Python Library For Spectral
Timing}

\end{itemize}

\item {} 
\sphinxhref{http://gamma-sky.net/}{gamma-sky - Portal to the gamma-ray sky}
\begin{itemize}
\item {} 
\sphinxstylestrong{gamma-sky.net} is a novel interactive website designed for
exploring the gamma-ray sky. It is geared towards both
professional astronomers and the general public.

\end{itemize}

\item {} 
\sphinxhref{https://github.com/gammapy/gammapy}{gammapy - A Python package for gamma-ray
astronomy}
\begin{itemize}
\item {} 
\sphinxstylestrong{Gammapy} is a community-developed, open-source Python package
for gamma-ray astronomy. It is a prototype for the CTA science
tools

\item {} 
\sphinxhref{https://docs.gammapy.org/0.12/}{Online document can be found
here}

\end{itemize}

\item {} 
\sphinxhref{https://github.com/gammapy/enrico}{Enrico - helps you with your Fermi data
analysis}
\begin{itemize}
\item {} 
Produce spectra (model fit and flux points), maps and lightcurves
for your target simply by editing a config file and running a
python script which executes the Fermi science tool chain.

\item {} 
\sphinxhref{https://enrico.readthedocs.io/en/latest/}{Online document can be found
here}

\end{itemize}

\item {} 
\sphinxhref{https://github.com/gammapy/gamma-cat/tree/master/gammacat}{gamma-cat - An open data collection and source catalog for gamma-ray
astronomy}
\begin{itemize}
\item {} 
\sphinxstylestrong{Gamma-Cat} provides the gamma-ray data as 1) a full data
collection (often multiple measurements, e.g. spectra, for a given
source); 2) a source catalog (a simple table with one source per
row, and a small subset of available data)

\end{itemize}

\item {} 
\sphinxhref{https://github.com/threeML/threeML}{threeML - The Multi-Mission Maximum Likelihood framework
(3ML)}
\begin{itemize}
\item {} 
Depends on \sphinxhref{https://github.com/threeML/astromodels}{astromodels - Spatial and spectral models for
astrophysics}

\item {} 
A framework for multi-wavelength/multi-messenger analysis for
astronomy/astrophysics

\item {} 
Can help deal with X-ray and Gamma-ray observations.

\end{itemize}

\end{itemize}


\subsubsection{Radio and Sub-millimeter Astronomy}
\label{\detokenize{resource/astro/topics/radio_submm_astrophy:radio-and-sub-millimeter-astronomy}}\label{\detokenize{resource/astro/topics/radio_submm_astrophy::doc}}
\sphinxstylestrong{Just started}


\bigskip\hrule\bigskip

\begin{itemize}
\item {} 
\sphinxhref{https://github.com/casacore/casacore}{casacore - Suite of C++ libraries for radio astronomy data
processing}
\begin{itemize}
\item {} 
The \sphinxstylestrong{casacore} package contains the core libraries of the old
\sphinxstylestrong{AIPS++/CASA} package. This split was made to get a better
separation of core libraries and applications. \sphinxstylestrong{CASA} is now
built on top of \sphinxstylestrong{Casacore}.

\item {} 
\sphinxstylestrong{python-casacore}: Python bindings for casacore, a library used
in radio astronomy. Document can be \sphinxhref{http://casacore.github.io/python-casacore/}{found
here}

\end{itemize}

\item {} 
\sphinxhref{https://github.com/RadioAstronomySoftwareGroup/pyuvdata}{pyuvdata - A pythonic interface for radio astronomy interferometry
data (uvfits, miriad,
others)}
\begin{itemize}
\item {} 
\sphinxstylestrong{pyuvdata} defines a pythonic interface to interferometric data
sets. Currently pyuvdata supports reading and writing of miriad,
uvfits, and uvh5 files and reading of \sphinxstylestrong{CASA} measurement sets
and FHD (Fast Holographic Deconvolution) visibility save files.

\item {} 
Online document is
\sphinxhref{https://pyuvdata.readthedocs.io/en/latest/}{here}

\end{itemize}

\item {} 
\sphinxhref{https://github.com/RadioAstronomySoftwareGroup/pyuvsim}{pyuvsim - A comprehensive simulation package for radio
interferometers in
python}
\begin{itemize}
\item {} 
A number of analysis tools are available to simulate the output of
a radio interferometer (CASA, OSKAR, FHD, PRISim, et al), however
each makes numerical approximations to enable speed ups.

\end{itemize}

\item {} 
\sphinxhref{https://github.com/flomertens/wise}{wise - Wavelet Image Segmentation and Evaluation
tool}
\begin{itemize}
\item {} 
\sphinxstylestrong{wise} is developed to address the issue of detecting
significant features in radio interferometric images and obtaining
reliable velocity field from cross-correlation of these regions in
multi-epoch observations.

\item {} 
Detection of structural information is performed using the
segmented wavelet decomposition (SWD) method.

\end{itemize}

\item {} 
\sphinxhref{https://github.com/gausspy/gausspy}{GaussPy - A Python tool for implementing Autonomous Gaussian
Decomposition}
\begin{itemize}
\item {} 
Based on the work of \sphinxhref{https://arxiv.org/abs/1409.2840}{Autonomous Gaussian
Decomposition}

\end{itemize}

\item {} 
\sphinxhref{https://github.com/mriener/gausspyplus}{gausspyplus - Fully automated Gaussian decomposition package for
emission line spectra}
\begin{itemize}
\item {} 
\sphinxstylestrong{GaussPy+} is based on \sphinxstylestrong{GaussPy}: A python tool for
implementing the Autonomous Gaussian Decomposition algorithm.

\end{itemize}

\item {} 
\sphinxhref{https://github.com/mhvk/scintillometry}{Scintillometry - A Package for Radio Baseband Data
Reduction}
\begin{itemize}
\item {} 
\sphinxstylestrong{Scintillometry} is a package for reduction and analysis of
radio baseband data, optimized for pulsar scintillometry science.
\sphinxhref{https://scintillometry.readthedocs.io/en/latest/}{Online document is
here}

\end{itemize}

\item {} 
\sphinxhref{https://github.com/SoFiA-Admin/SoFiA}{SoFiA - The HI Source Finding
Application}
\begin{itemize}
\item {} 
\sphinxstylestrong{SoFiA}, the Source Finding Application, is a new HI source
finding pipeline intended to find and parametrise galaxies in HI
data cubes.

\item {} 
The version 2 which is reimplemented in \sphinxstylestrong{C} can be \sphinxhref{https://github.com/SoFiA-Admin/SoFiA-2}{found
here}

\end{itemize}

\item {} 
\sphinxhref{https://github.com/mtazzari/galario}{galario - Gpu Accelerated Library for Analysing Radio Interferometer
Observations}
\begin{itemize}
\item {} 
\sphinxstylestrong{galario} is a library that exploits the computing power of
modern graphic cards (GPUs) to accelerate the comparison of model
predictions to radio interferometer observations.

\end{itemize}

\end{itemize}


\subsubsection{Astrophysical and Cosmological Simulations}
\label{\detokenize{resource/astro/topics/simulations:astrophysical-and-cosmological-simulations}}\label{\detokenize{resource/astro/topics/simulations::doc}}

\paragraph{Tools}
\label{\detokenize{resource/astro/topics/simulations:tools}}

\subparagraph{Analysing Data from Simulation}
\label{\detokenize{resource/astro/topics/simulations:analysing-data-from-simulation}}\begin{itemize}
\item {} 
\sphinxhref{https://yt-project.org/}{The yt project}
\begin{itemize}
\item {} 
\sphinxstylestrong{yt} is an open-source, permissively-licensed python package for
analyzing and visualizing volumetric data. \sphinxhref{https://github.com/yt-project/yt}{Source code is
available on Github}

\item {} 
There is a list of useful \sphinxhref{https://yt-project.org/extensions.html}{extensions of
yt}

\item {} 
\sphinxhref{https://github.com/yt-project/unyt}{unyt - Handle, manipulate, and convert data with units in
Python}

\end{itemize}

\item {} 
\sphinxhref{https://pynbody.github.io/pynbody/index.html}{Pynbody - an analysis package for astrophysical N-body and Smooth
Particle Hydrodynamics
simulations}
\begin{itemize}
\item {} 
\sphinxhref{https://github.com/pynbody/pynbody}{Source code available on
Github}

\end{itemize}

\item {} 
\sphinxhref{https://github.com/franciscovillaescusa/Pylians}{Pylians - Libraries to analyze numerical
simulations}
\begin{itemize}
\item {} 
\sphinxstylestrong{Pylians} stands for Python libraries for the analysis of
numerical simulations. They are a set of python libraries, written
in python, cython and C, whose purposes is to facilitate the
analysis of numerical simulations (both N-body and hydro).

\end{itemize}

\item {} 
\sphinxhref{https://github.com/N-BodyShop/tipsy}{tipsy - The Theoretical Image Processing SYstem for
visualizing/analyzing n-body
simulations}

\end{itemize}


\subparagraph{Halo and Subhalo Finder}
\label{\detokenize{resource/astro/topics/simulations:halo-and-subhalo-finder}}\begin{itemize}
\item {} 
For an overview and comparison of current algorithms:
\begin{itemize}
\item {} 
\sphinxhref{https://arxiv.org/abs/1104.0949}{Haloes gone MAD: The Halo-Finder Comparison
Project}

\item {} 
\sphinxhref{https://arxiv.org/abs/1203.3695}{Subhaloes going Notts: the subhalo-finder comparison
project}

\item {} 
\sphinxhref{https://arxiv.org/abs/1210.2578}{Galaxies going MAD: the Galaxy-Finder Comparison
Project}

\end{itemize}

\item {} 
\sphinxhref{https://bitbucket.org/gfcstanford/rockstar/src/master/}{Rockstar - Robust Overdensity Calculation using K-Space
Topologically Adaptive
Refinement}
\begin{itemize}
\item {} 
By Peter Behroozi. Based on \sphinxhref{https://arxiv.org/abs/1110.4372}{Phase-Space Temporal Halo Finder and
the Velocity Offsets of Cluster
Cores}

\item {} 
\sphinxstylestrong{rockstar} identifies dark matter halos, substructure, and tidal
features. The approach is based on adaptive hierarchical
refinement of friends-of-friends groups in six phase-space
dimensions and one time dimension, which allows for robust
(grid-independent, shape-independent, and noise-resilient)
tracking of substructure.

\end{itemize}

\item {} 
\sphinxhref{https://faculty.washington.edu/trq/hpcc/tools/fof.html}{FoF - Friends-of-friends method to find
groups}
\begin{itemize}
\item {} 
A particle belongs to a friends-of-friends group if it is within
some linking length of any other particle in the group. After all
such groups are found, those with less than a specified minimum
number of group members are rejected.

\end{itemize}

\end{itemize}


\subparagraph{Merger Tree Construction}
\label{\detokenize{resource/astro/topics/simulations:merger-tree-construction}}\begin{itemize}
\item {} 
For an overview and comparison of current algorithms:
\begin{itemize}
\item {} 
\sphinxhref{https://arxiv.org/abs/1307.3577}{Sussing Merger Trees: The Merger Trees Comparison
Project}

\item {} 
\sphinxhref{https://arxiv.org/abs/1604.01463}{Sussing Merger Trees: Stability and
Convergence}

\end{itemize}

\item {} 
\sphinxhref{https://bitbucket.org/pbehroozi/consistent-trees/src/master/}{consistent-trees - Gravitationally Consistent Merger
Trees}
\begin{itemize}
\item {} 
By Peter Behroozi. Based on \sphinxhref{https://arxiv.org/abs/1110.4370}{Gravitationally Consistent Halo
Catalogs and Merger Trees for Precision
Cosmology}

\end{itemize}

\item {} 
\sphinxhref{https://github.com/pelahi/VELOCIraptor-STF}{VELOCIraptor - Galaxy/(sub)Halo finder for N-body
simulations}
\begin{itemize}
\item {} 
By Pascal Jahan Elahi.

\item {} 
Also see \sphinxhref{https://github.com/pelahi/TreeFrog}{TreeFrog - Software to build Halo Merger Trees/compare
halo catalogs}

\item {} 
And \sphinxhref{https://github.com/pelahi/VELOCIraptor\_Python\_Tools}{VELOCIraptor\_Python\_Tools - python tools for manipulating
velociraptor
data}

\end{itemize}

\end{itemize}


\subparagraph{N-body Simulation}
\label{\detokenize{resource/astro/topics/simulations:n-body-simulation}}\begin{itemize}
\item {} 
\sphinxhref{https://xgitlab.cels.anl.gov/hacc/HACCKernels/tree/master}{HACC - Hardware/Hybrid Accelerated Cosmology
Code}
\begin{itemize}
\item {} 
A recently developed and evolving cosmology N-body code framework,
designed to run efficiently on diverse computing architectures and
to scale to millions of cores and beyond. See \sphinxhref{https://arxiv.org/abs/1410.2805}{publication here
for details}

\item {} 
\sphinxhref{https://xgitlab.cels.anl.gov/hacc}{Some relevant codes are available on
Gitlab}

\end{itemize}

\item {} 
\sphinxhref{https://academic.oup.com/pasj/article/61/6/1319/1462224}{GreeM - Massively Parallel TreePM Code for Large Cosmological N-body
Simulations}

\item {} 
\sphinxhref{https://github.com/HAWinther/MG-PICOLA-PUBLIC}{COLA - COmoving Lagrangian
Acceleration}
\begin{itemize}
\item {} 
Based on the work: \sphinxhref{https://arxiv.org/abs/1703.00879}{COLA with scale-dependent growth: applications
to screened modified gravity
models}

\item {} 
\sphinxhref{https://github.com/junkoda/cola\_halo}{Parallel COLA cosmological simulation + 2LPT initial condition
generator + FoF halo
finder}

\end{itemize}

\item {} 
\sphinxhref{https://github.com/franciscovillaescusa/Quijote-simulations}{Quijote-simulations}
\begin{itemize}
\item {} 
The Quijote simulations are a set of 34500 N-body simulations.
They are designed for two main tasks: 1) Quantify the information
content on cosmological observables; 2) Provide enough statistics
to train machine learning algorithms

\end{itemize}

\item {} 
\sphinxhref{http://www.unitsims.org/}{UNIT - Universe N-body simulations for the Investigation of
Theoretical models from galaxy surveys}
\begin{itemize}
\item {} 
Based on the work by \sphinxhref{https://arxiv.org/pdf/1811.02111.pdf}{Chia-Hsun Chuang et
al. 2018}

\item {} 
\sphinxstylestrong{Unit} uses FastPM (Feng et al. 2016) to generate the paired
initial conditions with fixed-amplitude.

\end{itemize}

\item {} 
\sphinxhref{http://maia.ice.cat/mice/}{MICE - Marenostrum Institut de Ciències de l’Espai
Simulations}
\begin{itemize}
\item {} 
A suit of cosmological simulations. Lots of data are already
available in public.

\end{itemize}

\end{itemize}


\subparagraph{Hydrodynamic and MHD Simulation}
\label{\detokenize{resource/astro/topics/simulations:hydrodynamic-and-mhd-simulation}}

\subparagraph{SPH: Smoothed Particle Hydrodynamics}
\label{\detokenize{resource/astro/topics/simulations:sph-smoothed-particle-hydrodynamics}}\begin{itemize}
\item {} 
\sphinxhref{https://wwwmpa.mpa-garching.mpg.de/gadget/}{Gadget-2 - A code for cosmological simulations of structure
formation}
\begin{itemize}
\item {} 
\sphinxstylestrong{Gadget-2} is a freely available code for cosmological
N-body/SPH simulations on massively parallel computers with
distributed memory. There are multiple spin-off of \sphinxstylestrong{Gadget} now.

\item {} 
\sphinxhref{https://github.com/MP-Gadget/MP-Gadget}{MP-Gadget - massively-parallel cosmology
simulator}
\begin{itemize}
\item {} 
This version of Gadget is derived from main P-Gadget /
Gadget-2. It is the source code used to run the BlueTides
simulation

\end{itemize}

\end{itemize}

\item {} 
\sphinxhref{http://swift.dur.ac.uk}{SWIFT - SPH With Inter-dependent Fine-grained
Tasking}
\begin{itemize}
\item {} 
\sphinxstylestrong{SWIFT} is a hydrodynamics and gravity code for astrophysics and
cosmology.

\item {} 
\sphinxhref{https://gitlab.cosma.dur.ac.uk/swift/swiftsim}{Source codes can be found on
GitLab}

\end{itemize}

\item {} 
\sphinxhref{http://www.tapir.caltech.edu/~phopkins/Site/GIZMO.html}{GIZMO by Phil
Hopkins}
\begin{itemize}
\item {} 
\sphinxstylestrong{GIZMO} is a flexible, massively-parallel, multi-physics
simulation code. The \sphinxhref{https://bitbucket.org/phopkins/gizmo-public/src/default/}{public version code can be found
here}

\item {} 
It introduces new Lagrangian Godunov-type methods that allow you
to solve the fluid equations with a moving particle distribution
that is automatically adaptive in resolution and avoids the
advection errors, angular momentum conservation errors, and
excessive diffusion problems that limit the applicability of
“adaptive mesh” (AMR) codes, while simultaneously avoiding the
low-order errors inherent to simpler methods like
smoothed-particle hydrodynamics (SPH).

\end{itemize}

\item {} 
\sphinxhref{https://gasoline-code.com/}{Gasoline - Particle hydrodynamics have never been
smoother}
\begin{itemize}
\item {} 
Gasoline is a modern SPH simulation code for astrophysical
problems. \sphinxhref{https://github.com/N-BodyShop/gasoline}{Source code is available
publicly}

\end{itemize}

\item {} 
\sphinxhref{https://github.com/laristra/flecsph}{flecsph - A Parallel and Distributed SPH Implementation Based on the
FleCSI}
\begin{itemize}
\item {} 
This project implements smoothed particles hydrodynamics (SPH)
method of simulating fluids and gases using the FleCSI framework.
Currently, particle affinity and gravitation is handled using the
parallel implementation of the octree data structure provided by
FleCSI.

\end{itemize}

\end{itemize}


\subparagraph{AMR: Adaptive Mesh Refinement}
\label{\detokenize{resource/astro/topics/simulations:amr-adaptive-mesh-refinement}}\begin{itemize}
\item {} 
\sphinxhref{https://amrex-codes.github.io/}{AMReX-Codes - Block-Structured AMR Software Framework and
Applications}
\begin{itemize}
\item {} 
\sphinxhref{https://amrex-codes.github.io/amrex/}{AMReX - A software framework for massively parallel,
block-structured adaptive mesh refinement (AMR)
applications}

\item {} 
\sphinxhref{https://amrex-astro.github.io/}{AMReX Astrophysics - An Astrophysical Hydrodynamics Code
Suite}
\begin{itemize}
\item {} 
\sphinxstylestrong{AMReX} Astrophysics codes can model subsonic convection and
compressible flows in stars, explosive burning in stellar
environments, and large scale structure on cosmological scales.
They share a common design and an open development model.

\item {} 
\sphinxhref{https://github.com/AMReX-Astro/Castro}{Castro - An adaptive mesh, astrophysical radiation
hydrodynamics simulation
code}

\item {} 
\sphinxhref{https://github.com/AMReX-Astro/MAESTRO}{MAESTRO - A low Mach number stellar hydrodynamics
code}

\item {} 
\sphinxhref{https://github.com/AMReX-Astro/Nyx}{Nyx - An adaptive mesh, N-body hydro cosmological simulation
code}

\end{itemize}

\end{itemize}

\item {} 
\sphinxhref{https://enzo-project.github.io/}{ENZO - adaptive mesh-refinement simulation
code}
\begin{itemize}
\item {} 
\sphinxstylestrong{Enzo} is a community-developed adaptive mesh refinement
simulation code, designed for rich, multi-physics hydrodynamic
astrophysical calculations. \sphinxhref{https://github.com/enzo-project/enzo-dev}{Source codes are available on
Github}

\end{itemize}

\item {} 
\sphinxhref{https://github.com/ECP-Astro/FLASH5}{FLASH5 - multiphysics, multiscale simulation
code}

\end{itemize}


\subparagraph{Moving Mesh Approach}
\label{\detokenize{resource/astro/topics/simulations:moving-mesh-approach}}\begin{itemize}
\item {} 
\sphinxhref{https://www.h-its.org/arepo/}{Arepo - Galilean-invariant cosmological hydrodynamical simulations
on a moving mesh}
\begin{itemize}
\item {} 
The \sphinxstylestrong{AREPO} code is a cosmological hydrodynamical simulation
code on a fully dynamic unstructured mesh. Code is not publicly
available yet.

\end{itemize}

\end{itemize}


\paragraph{Projects}
\label{\detokenize{resource/astro/topics/simulations:projects}}

\subparagraph{Cosmological Simulations}
\label{\detokenize{resource/astro/topics/simulations:cosmological-simulations}}

\subparagraph{N-Body Simulations}
\label{\detokenize{resource/astro/topics/simulations:n-body-simulations}}\begin{itemize}
\item {} 
\sphinxhref{https://www.cosmosim.org/}{CosmoSim Database}
\begin{itemize}
\item {} 
The Spanish MultiDark Consolider project supports efforts to
identify and detect matter, including dark matter simulations of
the universe. Including \sphinxstylestrong{SMDPL}, \sphinxstylestrong{MDPL}, \sphinxstylestrong{MDPL2},
\sphinxstylestrong{BigMDPL}, \sphinxstylestrong{Bolshoi}, and \sphinxstylestrong{BolshoiP} simulations.

\end{itemize}

\item {} 
\sphinxhref{https://cosmology.alcf.anl.gov/}{HACC Simulation Data Portal}
\begin{itemize}
\item {} 
This webpage provides access to results from large cosmological
simulations carried out with \sphinxstylestrong{HACC}, the Hardware/Hybrid
Accelerated Cosmology Code, developed primarily at Argonne

\item {} 
\sphinxhref{https://cosmology.alcf.anl.gov/transfer/miratitan}{Mira-Titan Universe
Simulations}
\begin{itemize}
\item {} 
A suite of eleven cosmological models, evolving almost 33
billion particles each in a (2.1Gpc)\textasciicircum{}3 volume.

\end{itemize}

\item {} 
\sphinxhref{https://cosmology.alcf.anl.gov/outerrim}{Outer Rim Simulation}
\begin{itemize}
\item {} 
A LCDM simulation evolving more than 1 trillion particles in a
(4.225Gpc)\textasciicircum{}3 volume.

\end{itemize}

\item {} 
\sphinxhref{https://cosmology.alcf.anl.gov/transfer/qcontinuum}{QContinuum
Simulation}
\begin{itemize}
\item {} 
A LCDM simulation evolving more than 0.5 trillion particles in
a (1.3Gpc)3 volume.

\end{itemize}

\end{itemize}

\item {} 
\sphinxhref{https://wwwmpa.mpa-garching.mpg.de/galform/virgo/}{VIRGO: Cosmological N-Body
Simulations}
\begin{itemize}
\item {} 
The VIRGO Consortium is an international grouping of scientists
carrying out supercomputer simulations of the formation of
galaxies, galaxy clusters, large-scale structure, and of the
evolution of the intergalactic medium.

\item {} 
\sphinxhref{https://wwwmpa.mpa-garching.mpg.de/galform/virgo/millennium/index.shtml}{The Millennium Simulation
Project}

\item {} 
\sphinxhref{https://wwwmpa.mpa-garching.mpg.de/galform/virgo/hubble/index.shtml}{The Hubble Volume
Project}

\end{itemize}

\end{itemize}


\subparagraph{Hydrodynamic or MHB Simulations}
\label{\detokenize{resource/astro/topics/simulations:hydrodynamic-or-mhb-simulations}}\begin{itemize}
\item {} 
\sphinxhref{http://www.illustris-project.org/}{The Illustris Simulation - Towards a predictive theory of galaxy
formation}
\begin{itemize}
\item {} 
The Illustris project is a large cosmological simulation of galaxy
formation, completed in late 2013, using a state of the art
numerical code and a comprehensive physical model.

\item {} 
All the data have been \sphinxhref{http://www.illustris-project.org/data/}{released to
public}. See the {[}About
page{]} for general information.

\end{itemize}

\item {} 
\sphinxhref{http://www.tng-project.org/}{The IllustrisTNG Project}
\begin{itemize}
\item {} 
The IllustrisTNG project is an ongoing series of large,
cosmological magnetohydrodynamical simulations of galaxy
formation.

\item {} 
Some of the IllustrisTNG data have been \sphinxhref{http://www.tng-project.org/data/}{released to the
public}

\end{itemize}

\item {} 
\sphinxhref{http://icc.dur.ac.uk/Eagle/}{The EAGLE Project - Evolution and Assembly of GaLaxies and their
Environments}
\begin{itemize}
\item {} 
\sphinxhref{http://icc.dur.ac.uk/Eagle/database.php}{Public data release is available
here}

\end{itemize}

\item {} 
\sphinxhref{https://wwwmpa.mpa-garching.mpg.de/auriga/}{Auriga project - High resolution disc galaxy simulations in a
cosmological context}
\begin{itemize}
\item {} 
The Auriga Project is a large suite of high-resolution
magneto-hydrodynamical simulations of Milky Way-sized galaxies,
simulated in a fully cosmological environment by means of the
‘zoom-in’ technique. It is simulated with the state-of-the-art
hydrodynamic moving mesh code AREPO, and includes a comprehensive
galaxy formation model based on the successful cosmological
simulation Illustris.

\end{itemize}

\item {} 
\sphinxhref{http://bluetides-project.org/}{BlueTides Simulation}
\begin{itemize}
\item {} 
BlueTides was run on the BlueWaters super computer at NCSA with an
allocation of 2.6 million node-hours. It simulated the universe
from z=99 to z=8.0.

\item {} 
Bluetides is the largest hydrodynamic simulation ever performed at
such high redshift.

\end{itemize}

\item {} 
\sphinxhref{http://www.astro.ljmu.ac.uk/~igm/BAHAMAS/}{The BAHAMAS Project - BAryons and HAloes of MAssive
Systems}
\begin{itemize}
\item {} 
BAHAMAS is a first attempt to do large-scale structure (LSS)
cosmology using self-consistent full cosmological hydrodynamical
simulations.

\end{itemize}

\end{itemize}


\subsubsection{Solar Physics and Space Science}
\label{\detokenize{resource/astro/topics/solar_physics:solar-physics-and-space-science}}\label{\detokenize{resource/astro/topics/solar_physics::doc}}
\sphinxstylestrong{Just started}


\bigskip\hrule\bigskip



\paragraph{Mission and Data}
\label{\detokenize{resource/astro/topics/solar_physics:mission-and-data}}\begin{itemize}
\item {} 
\sphinxhref{https://sdo.gsfc.nasa.gov/}{Solar Dynamics Observatory}

\item {} 
\sphinxhref{http://iris.lmsal.com/mission.html}{Interface Region Imaging Spectrograph
(IRIS)}
\begin{itemize}
\item {} 
The primary goal of the Interface Region Imaging Spectrograph
(IRIS) explorer is to understand how the solar atmosphere is
energized

\end{itemize}

\end{itemize}


\paragraph{Tools}
\label{\detokenize{resource/astro/topics/solar_physics:tools}}

\subparagraph{General Data Analysis Tools}
\label{\detokenize{resource/astro/topics/solar_physics:general-data-analysis-tools}}\begin{itemize}
\item {} 
\sphinxhref{https://sohowww.nascom.nasa.gov/solarsoft/}{SolarSoftWare (SSW)}
\begin{itemize}
\item {} 
The \sphinxstylestrong{SolarSoft} system is a set of integrated software
libraries, data bases, and system utilities which provide a common
programming and data analysis environment for Solar Physics.
Mostly written in \sphinxstylestrong{IDL}.

\end{itemize}

\item {} 
\sphinxhref{https://sunpy.org/}{SunPy - Python for Solar Physics}
\begin{itemize}
\item {} 
The community-developed, free and open-source solar data analysis
environment for Python. \sphinxhref{https://github.com/sunpy/sunpy}{Source code on
Github}

\item {} 
\sphinxhref{https://docs.sunpy.org/en/stable/}{Document is here} and here
is the \sphinxhref{https://docs.sunpy.org/en/stable/guide/tour.html}{New comers’ guide to
SunPy}

\item {} 
\sphinxhref{https://docs.sunpy.org/en/stable/guide/ssw.html}{SSWIDL to SunPy cheat
sheet}

\end{itemize}

\item {} 
\sphinxhref{https://github.com/sunpy/sunkit-image}{sunkit-image - A image processing toolbox for Solar
Physics}
\begin{itemize}
\item {} 
\sphinxstylestrong{sunkit-image} is a a open-source toolbox for solar physics
image processing. Currently it is experimental library for various
solar physics specific image processing routines.

\end{itemize}

\end{itemize}


\subparagraph{Telescope or Mission Specific Tools}
\label{\detokenize{resource/astro/topics/solar_physics:telescope-or-mission-specific-tools}}\begin{itemize}
\item {} 
\sphinxhref{https://github.com/sunpy/irispy}{irispy - A SunPy-affiliated package which provides tools to analyze
data from IRIS}
\begin{itemize}
\item {} 
\sphinxstylestrong{IRISPy} is a SunPy-affiliated package that provides the tools
to read in and analyze data from the IRIS solar-observing
satellite in Python.

\end{itemize}

\item {} 
\sphinxhref{https://github.com/sunpy/drms}{drms - Access HMI, AIA and MDI data with
Python}
\begin{itemize}
\item {} 
The \sphinxstylestrong{drms} module provides an easy-to-use interface for
accessing HMI, AIA and MDI data with Python. It uses the publicly
accessible JSOC DRMS server by default

\end{itemize}

\item {} 
\sphinxhref{https://github.com/DKISTDC/dkist}{dkist - A Python library for obtaining, processing and interacting
with calibrated DKIST data}
\begin{itemize}
\item {} 
The \sphinxhref{https://dkist.nso.edu/}{Daniel K. Inouye Solar Telescope}
(DKIST, formerly the Advanced Technology Solar Telescope, ATST)

\end{itemize}

\end{itemize}


\subparagraph{Other Useful Tools}
\label{\detokenize{resource/astro/topics/solar_physics:other-useful-tools}}\begin{itemize}
\item {} 
\sphinxhref{https://github.com/sunpy/solarbextrapolation}{solarbextrapolation - Extrapolation framework for Solar Magnetic
Fields}
\begin{itemize}
\item {} 
\sphinxstylestrong{solarbextrapolation} is a library for extrapolating 3D magnetic
fields from line-of-sight magnetograms

\end{itemize}

\item {} 
\sphinxhref{https://github.com/ianan/demreg}{demreg - Calculate the Differential Emission Measure (DEM) from
solar data using regularised
inversion}
\begin{itemize}
\item {} 
Written in IDL. More on the \sphinxhref{http://www.astro.gla.ac.uk/~iain/demreg/map/}{Regularized DEM
Maps}

\end{itemize}

\item {} 
\sphinxhref{https://github.com/fraserwatson/MagneticFragmentation}{MagneticFragmentation}
\begin{itemize}
\item {} 
The algorithms required to divide SDO/HMI magnetograms into
individual magnetic fragments, based on a ‘downhill’ segmentation.

\end{itemize}

\item {} 
\sphinxhref{https://github.com/pingswept/pysolar}{Pysolar - a collection of Python libraries for simulating the
irradiation of any point on earth by the
sun}

\item {} 
\sphinxhref{https://github.com/rice-solar-physics/EBTEL}{EBTEL - Enthalpy-Based Thermal Evolution of Loops
model}
\begin{itemize}
\item {} 
Written in IDL, for \sphinxhref{https://arxiv.org/abs/0710.0185}{Highly Efficient Modeling of Dynamic Coronal
Loops}

\item {} 
\sphinxhref{https://github.com/rice-solar-physics/ebtelPlusPlus}{ebtel++ - C++ implementation of the two-fluid EBTEL
model}

\end{itemize}

\item {} 
\sphinxhref{https://github.com/rice-solar-physics/IonPopSolver}{IonPopSolver - Code for computing effective temperature and ion
population fractions from temperature and density
timeseries}
\begin{itemize}
\item {} 
Based on \sphinxhref{https://www.aanda.org/component/article?access=bibcode\&bibcode=\&bibcode=2009A\%2526A...502..409BFUL}{A numerical tool for the calculation of non-equilibrium
ionisation states in the solar corona and other astrophysical
plasma
environments}

\end{itemize}

\item {} 
\sphinxhref{https://github.com/sunpy/eitwave}{eitwave - EIT Wave Detection (AGU
2011)}

\end{itemize}


\subparagraph{Atomic Database}
\label{\detokenize{resource/astro/topics/solar_physics:atomic-database}}\begin{itemize}
\item {} 
\sphinxhref{http://www.chiantidatabase.org/}{CHIANTI - An Atomic Database for Spectroscopic Diagnostics of
Astrophysical Plasmas}
\begin{itemize}
\item {} 
This is a general astrophysical database, but it is frequently
used in solar physics. It is \sphinxhref{http://www.chiantidatabase.org/instructions.html}{distributed within
SSW}.

\item {} 
\sphinxhref{https://github.com/wtbarnes/fiasco}{fiasco - A Python interface to the CHIANTI atomic
database}
\begin{itemize}
\item {} 
\sphinxhref{https://fiasco.readthedocs.io/en/latest/}{Online document is
here}

\end{itemize}

\item {} 
\sphinxhref{https://github.com/chianti-atomic/ChiantiPy}{ChiantiPy - the Python interface to the CHIANTI atomic database
for astrophysical
spectroscopy}

\end{itemize}

\end{itemize}


\subsubsection{Planet Science and Small Bodies in the Solar System}
\label{\detokenize{resource/astro/topics/solar_system:planet-science-and-small-bodies-in-the-solar-system}}\label{\detokenize{resource/astro/topics/solar_system::doc}}
\sphinxstylestrong{Just started}


\bigskip\hrule\bigskip



\paragraph{Data}
\label{\detokenize{resource/astro/topics/solar_system:data}}\begin{itemize}
\item {} 
\sphinxhref{https://minorplanetcenter.net/iau/mpc.html}{IAU Minor Planet
Center}
\begin{itemize}
\item {} 
The MPC is responsible for the designation of minor bodies in the
solar system: minor planets; comets; and natural satellites. The
MPC is also responsible for the efficient collection, computation,
checking and dissemination of astrometric observations and orbits
for minor planets and comets

\item {} 
For \sphinxhref{https://minorplanetcenter.net/data}{data available from the minor planet
center}

\item {} 
\sphinxhref{https://minorplanetcenter.net//web\_service}{MPC Web Service}:
is providing a web service interface to users, allowing them to
programmatically fetch minor planet properties data from the MPC’s
database.

\item {} 
\sphinxhref{https://www.minorplanetcenter.net/iau/MPEph/MPEph.html}{Minor Planet \& Comet Ephemeris
Service}

\end{itemize}

\item {} 
\sphinxhref{https://planetarynames.wr.usgs.gov/}{Gazetteer of Planetary
Nomenclature}
\begin{itemize}
\item {} 
The Gazetteer of Planetary Nomenclature is maintained by the
Planetary Geomatics Group of the USGS Astrogeology Science Center.

\end{itemize}

\end{itemize}


\paragraph{Tools}
\label{\detokenize{resource/astro/topics/solar_system:tools}}\begin{itemize}
\item {} 
\sphinxhref{https://github.com/NASA-Planetary-Science/sbpy}{sbpy - A Python package for small bodies
research}
\begin{itemize}
\item {} 
\sphinxstylestrong{sbpy} is a community effort to build a Python package for
small-body planetary astronomy in the form of an astropy
affiliated package. See the \sphinxhref{http://mommermi.github.io/}{sbpy website for
more}

\item {} 
Include functionalities for planning observations, modeling
photometry, fitting astrometry and orbit, analysing spectroscopic
data, simulating and analysing cometary gas and dust coma,
estimating size/albedo of asteroid, enhancing images, and
analysing lightcurve.

\item {} 
\sphinxhref{https://github.com/NASA-Planetary-Science/sbpy-tutorial}{sbpy Tutorials and Workshops
Materials}

\end{itemize}

\end{itemize}


\subparagraph{Solar System Geometry}
\label{\detokenize{resource/astro/topics/solar_system:solar-system-geometry}}\begin{itemize}
\item {} 
\sphinxhref{https://naif.jpl.nasa.gov/naif/}{SPICE - An Observation Geometry System for Space Science
Missions}
\begin{itemize}
\item {} 
NASA’s Navigation and Ancillary Information Facility (NAIF) offers
NASA flight projects and NASA funded researchers an observation
geometry information system named “SPICE” to assist scientists in
planning and interpreting scientific observations from space-based
instruments aboard robotic planetary spacecraft.

\item {} 
\sphinxhref{https://naif.jpl.nasa.gov/naif/toolkit.html}{The SPICE
Toolkit} is
available in C, Fortran, IDL, and Matlab.

\item {} 
\sphinxhref{https://github.com/AndrewAnnex/SpiceyPy}{SpiceyPy - Python wrapper for the NAIF C SPICE
Toolkit}
\begin{itemize}
\item {} 
\sphinxhref{https://spiceypy.readthedocs.io/en/master/}{Online document can be found
here}

\end{itemize}

\end{itemize}

\item {} 
\sphinxhref{https://github.com/esaSPICEservice/spiops}{spiops - Extension of SPICE functionalities for ESA
Missons}
\begin{itemize}
\item {} 
\sphinxstylestrong{spiops} is a library aimed to help scientists and engineers
that deal with Solar System Geometry mainly for planetary science

\end{itemize}

\end{itemize}


\subparagraph{Object Detection}
\label{\detokenize{resource/astro/topics/solar_system:object-detection}}\begin{itemize}
\item {} 
\sphinxhref{https://github.com/NASA-Planetary-Science/SALTAD}{SALTAD}
\begin{itemize}
\item {} 
\sphinxstylestrong{SALTAD} is comprised of a series of C language function modules
for image processing multiframe image data to detect moving
asteroids in a star cluttered background.

\end{itemize}

\end{itemize}


\subparagraph{Minor Planet Center Related}
\label{\detokenize{resource/astro/topics/solar_system:minor-planet-center-related}}\begin{itemize}
\item {} 
\sphinxhref{https://minorplanetcenter.net/mpc-fetch.py}{mpc-fetch.py - fetches properties of all objects that match search
params}

\item {} 
\sphinxhref{https://astroquery.readthedocs.io/en/latest/mpc/mpc.html}{astroquery.mpc - Minor Planet Center
Queries}

\item {} 
\sphinxhref{https://github.com/qdonnellan/mpc-client}{mpc-client - A python client for the Minor Planet Center’s MPC web
service}

\end{itemize}


\subparagraph{Modeling and Simulating Planet}
\label{\detokenize{resource/astro/topics/solar_system:modeling-and-simulating-planet}}\begin{itemize}
\item {} 
\sphinxhref{https://github.com/physicsguy42/BEAM\_beta}{Beam - A Monte Carlo Radiative Transfer Code for studies of Saturn’s
Rings}

\end{itemize}

\begin{sphinxadmonition}{note}{Note:}
Right now, the selection of these topics only reflect the curator’s interests and they are
clearly biased toward extragalactic and cosmological topics.
Any help to make them more complete and balanced are highly welcomed!
\end{sphinxadmonition}


\subsection{More Specific Topics}
\label{\detokenize{astro_topic:more-specific-topics}}

\subsubsection{All about Lensing: Strong, Weak, and Micro}
\label{\detokenize{resource/astro/topics/lensing:all-about-lensing-strong-weak-and-micro}}\label{\detokenize{resource/astro/topics/lensing::doc}}

\paragraph{Strong Lensing}
\label{\detokenize{resource/astro/topics/lensing:strong-lensing}}

\subparagraph{Modeling}
\label{\detokenize{resource/astro/topics/lensing:modeling}}\begin{itemize}
\item {} 
\sphinxhref{https://github.com/Jammy2211/PyAutoLens}{PyAutoLens - Automated Strong Gravitational Lens
Modeling}
\begin{itemize}
\item {} 
From James Nightingale. Based on \sphinxhref{https://arxiv.org/abs/1412.7436}{Adaptive Semi-linear Inversion
of Strong Gravitational Lens
Imaging} and \sphinxhref{https://arxiv.org/abs/1708.07377}{AutoLens:
Automated Modeling of a Strong Lens’s Light, Mass and
Source}

\end{itemize}

\item {} 
\sphinxhref{https://github.com/sibirrer/lenstronomy}{lenstronomy - Software package for lens model reconstruction of
imaging data}
\begin{itemize}
\item {} 
By Simon Birrer. \sphinxstylestrong{lenstronomy} is a multi-purpose package to
model strong gravitational lenses.

\item {} 
Based on \sphinxhref{https://arxiv.org/abs/1803.09746v1}{Lenstronomy: multi-purpose gravitational lens modelling
software package}

\end{itemize}

\item {} 
\sphinxhref{https://github.com/jspilker/visilens}{vislens - Module for modeling gravitational lensing systems in which
the data are not images but interferometric
visibilities}
\begin{itemize}
\item {} 
\sphinxstylestrong{Visilens} is a python module for modeling gravitational lensing
systems observed by a radio/mm interferometer like ALMA or ATCA.

\end{itemize}

\end{itemize}


\paragraph{Weak Lensing}
\label{\detokenize{resource/astro/topics/lensing:weak-lensing}}\begin{itemize}
\item {} 
Some of the tools overlap with the cosmology tools section.

\end{itemize}


\paragraph{Micro Lensing}
\label{\detokenize{resource/astro/topics/lensing:micro-lensing}}

\subsubsection{Photometric Redshift Tools and Algorithms}
\label{\detokenize{resource/astro/topics/photoz:photometric-redshift-tools-and-algorithms}}\label{\detokenize{resource/astro/topics/photoz::doc}}\begin{itemize}
\item {} 
In this new era of extragalactic and cosmological surveys,
photometric redshift becomes increasingly more important. And in
recent years, there have been many exciting new developments for the
photo-z algorithms.

\end{itemize}


\paragraph{Template Based}
\label{\detokenize{resource/astro/topics/photoz:template-based}}\begin{itemize}
\item {} 
\sphinxhref{https://github.com/gbrammer/eazy-photoz}{EAZY - photometric redshift
code}
\begin{itemize}
\item {} 
\sphinxstylestrong{EAZY} is a photometric redshift code designed to produce
high-quality redshifts for situations where complete spectroscopic
calibration samples are not available. See \sphinxhref{http://adsabs.harvard.edu/abs/2008ApJ...686.1503B}{Brammer, van Dokkum \&
Coppi 2008}
for details.

\item {} 
\sphinxhref{https://github.com/gbrammer/eazy-py}{eazy-py - Pythonic photometric redshift tools based on
EAZY}

\end{itemize}

\end{itemize}


\paragraph{Machine Learning}
\label{\detokenize{resource/astro/topics/photoz:machine-learning}}\begin{itemize}
\item {} 
\sphinxhref{https://github.com/IftachSadeh/ANNZ}{ANNZ - Machine learning methods for
astrophysics}
\begin{itemize}
\item {} 
By Iftach Sadeh. ANNZ uses both regression and classification
techniques for estimation of single-value photo-z (or any
regression problem) solutions and PDFs. Includes several
“traditional” machine learning algorithms (ANN, BDT, KNN)

\end{itemize}

\item {} 
\sphinxhref{https://github.com/mgckind/MLZ}{MLZ - Machine Learning for
photo-Z}
\begin{itemize}
\item {} 
MLZ is a python code that computes photometric redshift PDFs using
machine learning techniques, providing optional extra information.

\end{itemize}

\item {} 
\sphinxhref{https://github.com/joshspeagle/frankenz}{frankenz - A photometric redshift
monstrosity}
\begin{itemize}
\item {} 
By Josh Speagle. \sphinxstylestrong{frankenz} is a Pure Python implementation of a
variety of methods to quickly yet robustly perform (hierarchical)
Bayesian inference using large (but discrete) sets of (possibly
noisy) models with (noisy) photometric data.

\end{itemize}

\end{itemize}


\paragraph{Gaussian Process}
\label{\detokenize{resource/astro/topics/photoz:gaussian-process}}\begin{itemize}
\item {} 
\sphinxhref{https://github.com/ixkael/Delight}{Delight - Photometric redshift via Gaussian processes with physical
kernels}
\begin{itemize}
\item {} 
By \sphinxhref{https://ixkael.github.io/}{Boris Leistedt}. See \sphinxhref{https://arxiv.org/abs/1612.00847}{Leistedt \&
Hogg 2016} for details.

\end{itemize}

\end{itemize}


\paragraph{Clustering}
\label{\detokenize{resource/astro/topics/photoz:clustering}}\begin{itemize}
\item {} 
\sphinxhref{https://github.com/morriscb/the-wizz}{the-wizz - A clustering redshift estimation code for us
folks}
\begin{itemize}
\item {} 
By Chris Morrison. \sphinxstylestrong{the-wizz} is a clustering redshift
estimating tool designed with ease of use for end users in mind.

\item {} 
For information on the method see Schmidt et al. 2013, Menard et
al. 2013, and Rahman et al. 2015, 2016b. Details on this
implementation can be found in Morrison et al. 2017

\end{itemize}

\end{itemize}


\subsubsection{Group or Cluster Detection Algorithms}
\label{\detokenize{resource/astro/topics/cluster_finder:group-or-cluster-detection-algorithms}}\label{\detokenize{resource/astro/topics/cluster_finder::doc}}\begin{itemize}
\item {} 
This topic has become more and more important for the study of
galaxy-halo connection.

\end{itemize}


\paragraph{Based on clustering of galaxies}
\label{\detokenize{resource/astro/topics/cluster_finder:based-on-clustering-of-galaxies}}\begin{itemize}
\item {} 
\sphinxhref{https://github.com/sfarrens/sfof}{sfof - Friends-of-Friends Galaxy Cluster Detection
Algorithm}
\begin{itemize}
\item {} 
By \sphinxhref{https://sfarrens.github.io/}{Samuel Farrens}. SFoF is a
friends-of-friends galaxy cluster detection algorithm that
operates in either spectroscopic or photometric redshift space.
The linking parameters, both transverse and along the
line-of-sight, change as a function of redshift to account for
selection effects.

\end{itemize}

\end{itemize}


\paragraph{Based on red-sequence of the cluster members}
\label{\detokenize{resource/astro/topics/cluster_finder:based-on-red-sequence-of-the-cluster-members}}

\subparagraph{redMaPPer: red-sequence Matched-filter Probabilistic Percolation}
\label{\detokenize{resource/astro/topics/cluster_finder:redmapper-red-sequence-matched-filter-probabilistic-percolation}}\begin{itemize}
\item {} 
\sphinxhref{https://github.com/erykoff/redmapper}{redMaPPer - The Red-sequence Cluster
Finder}
\begin{itemize}
\item {} 
By \sphinxhref{https://github.com/erykoff}{Eli Rykoff}. This is the
open-source, python version of the red-sequence matched-filter
Probabilistic Percolation (redMaPPer) cluster finder, originally
described in Rykoff et al. (2014), with updates described in Rozo
et al. (2015) and Rykoff et al. (2016).

\item {} 
One of the most commonly used algorithm in recent years.

\end{itemize}

\end{itemize}


\subparagraph{Papers:}
\label{\detokenize{resource/astro/topics/cluster_finder:papers}}\begin{itemize}
\item {} 
\sphinxhref{https://ui.adsabs.harvard.edu/abs/2014ApJ...785..104R/abstract}{redMaPPer. I. Algorithm and SDSS DR8
Catalog}

\item {} 
\sphinxhref{https://ui.adsabs.harvard.edu/abs/2014ApJ...783...80R/abstract}{redMaPPer II: X-Ray and SZ Performance Benchmarks for the SDSS
Catalog}

\item {} 
\sphinxhref{https://ui.adsabs.harvard.edu/abs/2015MNRAS.450..592R/abstract}{redMaPPer - III. A detailed comparison of the Planck 2013 and SDSS
DR8 redMaPPer cluster
catalogues}

\item {} 
\sphinxhref{https://ui.adsabs.harvard.edu/abs/2015MNRAS.453...38R/abstract}{redMaPPer - IV. Photometric membership identification of red cluster
galaxies with 1 per cent
precision}

\end{itemize}


\subparagraph{Calibration:}
\label{\detokenize{resource/astro/topics/cluster_finder:calibration}}

\subparagraph{Issues:}
\label{\detokenize{resource/astro/topics/cluster_finder:issues}}

\subparagraph{CAMIRA}
\label{\detokenize{resource/astro/topics/cluster_finder:camira}}

\paragraph{Test cluster finder in mock catalog}
\label{\detokenize{resource/astro/topics/cluster_finder:test-cluster-finder-in-mock-catalog}}\begin{itemize}
\item {} 
\sphinxhref{https://github.com/sfarrens/pycymatch}{pycymatch}
\begin{itemize}
\item {} 
\sphinxstylestrong{Pycymatch} is a cylindrical matching code for identifying
matches between a catalogue of simulated dark matter haloes
populated with galaxies and the results of a cluster detection
algorithm run on said catalogue.

\end{itemize}

\end{itemize}


\subsubsection{Models for Galaxy Formation and Evolution}
\label{\detokenize{resource/astro/topics/galaxy_formation_model:models-for-galaxy-formation-and-evolution}}\label{\detokenize{resource/astro/topics/galaxy_formation_model::doc}}

\paragraph{Reviews}
\label{\detokenize{resource/astro/topics/galaxy_formation_model:reviews}}\begin{itemize}
\item {} 
\sphinxhref{https://ui.adsabs.harvard.edu/abs/2010PhR...495...33B/abstract}{Galaxy Formation
Thory}
\begin{itemize}
\item {} 
By Andrew Benson. A 58 pages review article in 2010. Although some
aspects of it need to be updated. This remains as a wonderful
place to start learning about galaxy formation.

\end{itemize}

\item {} 
\sphinxhref{https://arxiv.org/abs/1612.06891}{Theoretical Challenges in Galaxy
Formation}
\begin{itemize}
\item {} 
ARA\&A review by Thorsten Naab \& Jeremiah Ostriker with a
theoretical point of view.

\end{itemize}

\item {} 
\sphinxhref{https://arxiv.org/abs/1403.2783}{The Evolution of Galaxy Structure over Cosmic
Time}
\begin{itemize}
\item {} 
ARA\&A review by Christopher Conselice with an observational point
of view.

\end{itemize}

\item {} 
\sphinxhref{https://arxiv.org/abs/1804.03097}{The Connection Between Galaxies and Their Dark Matter
Halos}
\begin{itemize}
\item {} 
ARA\&A review by Risa Wechsler \& Jeremy Tinker from a galaxy-halo
connection point of view.

\end{itemize}

\item {} 
\sphinxhref{https://www.mdpi.com/2075-4434/7/2/56}{Lighting Up Dark Matter
Haloes}
\begin{itemize}
\item {} 
Review article by Gabriella De Lucia on \sphinxstyleemphasis{Galaxies}.

\end{itemize}

\end{itemize}


\paragraph{Semi-Analytic Model}
\label{\detokenize{resource/astro/topics/galaxy_formation_model:semi-analytic-model}}\begin{quote}

“Semi-analytic galaxy formation models are established tools for
connecting the predicted hierarchical growth of dark matter haloes to
the observed properties of the galaxy population. Semi-analytic
models employ a forward-modelling approach and are constructed such
that they contain as much as possible of the baryonic physics that is
thought to be relevant to galaxy evolution, albeit at a simplified,
macroscopic level. The simplified, macroscopic nature of
semi-analytic models means that they are computationally inexpensive
to evaluate.” - \sphinxhref{https://arxiv.org/pdf/1709.08647.pdf}{Mitchell et
al. 2017}
\end{quote}
\begin{itemize}
\item {} 
There are a whole bunch of SAM available now, we will focus on the
ones with source codes in public or the ones with a clear trace of
publications.

\item {} 
\sphinxhref{https://bitbucket.org/galacticusdev/galacticus/wiki/Home}{Galacticus - A Semi-Analytic Model of Galaxy
Formation}
\begin{itemize}
\item {} 
Based on the paper by Andrew Benson: \sphinxhref{https://arxiv.org/abs/1008.1786}{Galacticus: A Semi-Analytic
Model of Galaxy Formation}

\item {} 
Written in Fortran. Currently only support Linux OS.

\item {} 
Mock galaxy catalog for the \sphinxstylestrong{MDPL2} N-body simulation can be
\sphinxhref{http://skiesanduniverses.org/page/page-3/page-22/}{found in the MultiDark-Galaxies
project}

\item {} 
The tools for interacting with and analyzing the outputs of
\sphinxstylestrong{Galacticus} is written in \sphinxstylestrong{Perl}. \sphinxhref{https://users.obs.carnegiescience.edu/abenson/galacticus/GalacticusAnalysisPerl.pdf}{A PDF document of these
tools can be found
here}

\end{itemize}

\item {} 
\sphinxhref{https://github.com/ICRAR/shark}{Shark - A new, flexible semi-analytic model of galaxy
formation}
\begin{itemize}
\item {} 
Based on the work led by Claudia Lagos: \sphinxhref{https://ui.adsabs.harvard.edu/abs/2018MNRAS.481.3573L/abstract}{Shark: introducing an
open source, free, and flexible semi-analytic model of galaxy
formation}

\item {} 
\sphinxhref{https://shark-sam.readthedocs.io/en/latest/}{Online document is
here}

\item {} 
Written in C++ with \sphinxstylestrong{OpenMP} support.

\item {} 
\sphinxstylestrong{Shark} has been implemented with several models for gas
cooling, active galactic nuclei, stellar and photo-ionization
feedback, and star formation (SF).

\item {} 
\sphinxstylestrong{Shark} is based on the new cluster finder \sphinxstylestrong{VELOCIRAPTOR} and
merger tree builder \sphinxstylestrong{TreeFrog} developed by the same group.

\end{itemize}

\item {} 
\sphinxhref{https://github.com/sage-home/sage-model}{SAGE - Semi-Analytic Galaxy Evolution galaxy formation
model}
\begin{itemize}
\item {} 
\sphinxstylestrong{SAGE} is a publicly available code-base for modelling galaxy
formation in a cosmological context. A description of the model
and its default calibration results can be found in \sphinxhref{https://arxiv.org/abs/1601.04709}{Croton et al.
(2016)}. These calibration
results can also be explored in an \sphinxhref{https://github.com/darrencroton/sage/blob/master/output/SAGE\_MM.ipynb}{iPython
notebook}
showcasing the key figures here. SAGE is a significant update to
that previously used in \sphinxhref{https://arxiv.org/abs/astro-ph/0508046}{Croton et al.
(2006)}.

\item {} 
\sphinxstylestrong{SAGE} is written in C and was built to be modular and
customisable. It will run on any N-body simulation whose trees are
organised in a supported format and contain a minimum set of basic
halo properties.

\item {} 
Mock galaxy catalog for the \sphinxstylestrong{MDPL2} N-body simulation can be
\sphinxhref{http://skiesanduniverses.org/page/page-3/page-22/}{found in the MultiDark-Galaxies
project}

\item {} 
\sphinxhref{https://github.com/jacobseiler/rsage}{rsage - The Reionization using Semi-Analytic Galaxy Evolution
model}
\begin{itemize}
\item {} 
An augmented version of the \sphinxstylestrong{SAGE} model that
self-consistently couples galaxy evolution with the evolution
of ionized gas during the Epoch of Reionization.

\end{itemize}

\end{itemize}

\item {} 
\sphinxhref{http://galformod.mpa-garching.mpg.de/public/LGalaxies/index.php}{L-Galaxies - Munich Galaxy Formation
Model}
\begin{itemize}
\item {} 
The public version is \sphinxhref{https://github.com/LGalaxiesPublicRelease/LGalaxies\_PublicRepository}{available on
GitHub}

\item {} 
The most recent reference is the paper led by \sphinxhref{http://galformod.mpa-garching.mpg.de/public/LGalaxies/Henriques2014a.pdf}{Bruno
Henriques}.
A supplementary material is \sphinxhref{http://galformod.mpa-garching.mpg.de/public/LGalaxies/munich\_sam.pdf}{also
available}
that contains more details of the model. And here is a
\sphinxhref{http://galformod.mpa-garching.mpg.de/public/LGalaxies/LGalaxies\_slides.pdf}{presentation that helps you understand the key recipe of the
mode}

\item {} 
The \sphinxstylestrong{L-Galaxies} model galaxy catalog for the Millennium
simulations can be \sphinxhref{http://gavo.mpa-garching.mpg.de/Millennium/}{found
here}

\item {} 
\sphinxhref{https://github.com/scottclay/Lgalaxies\_Dust}{L-Galaxies Dust
Analysis} -
Implementation of detailed dust modelling into the Henriques 2015
version of L-Galaxies. The data analysis pipeline is \sphinxhref{https://github.com/scottclay/Lgalaxies\_Analysis}{available
here}

\end{itemize}

\item {} 
\sphinxhref{http://star-www.dur.ac.uk/~cole/merger\_trees/galform\_2000/}{GALFORM: Galactic
Modeling}
\begin{itemize}
\item {} 
Originally based on the classic paper by \sphinxhref{https://ui.adsabs.harvard.edu/abs/2000MNRAS.319..168C/abstract}{Shaun Cole et
al. 2010}.
The new version of \sphinxstylestrong{GALFORM} model is presented in the work led
by \sphinxhref{https://ui.adsabs.harvard.edu/abs/2016MNRAS.462.3854L/abstract}{Cedric Lacey et
al. 2016}

\item {} 
This is also known as the “Durham” galaxy formation model.

\end{itemize}

\item {} 
The “Santa Cruz” Galaxy Formation Model
\begin{itemize}
\item {} 
Originally based on the work by \sphinxhref{https://ui.adsabs.harvard.edu/abs/1999MNRAS.310.1087S/abstract}{Rachel Somerville \& Joel Primack
1998}

\item {} 
The updated version is presented in the work: \sphinxhref{https://ui.adsabs.harvard.edu/abs/2015MNRAS.453.4337S/abstract}{Star formation in
semi-analytic galaxy formation models with multiphase gas by
Somerville, Popping, \& Trager
2015}

\end{itemize}

\item {} 
\sphinxstylestrong{GAEA} - GAlaxy Evolution and Assembly model
\begin{itemize}
\item {} 
Based on the work \sphinxhref{https://ui.adsabs.harvard.edu/abs/2016MNRAS.461.1760H/abstract}{“Galaxy assembly, stellar feedback and metal
enrichment: the view from the GAEA model” by Michaela Hirschmann,
Gabriella De Lucia, \& Fabio Fontanot
2015}

\end{itemize}

\item {} 
\sphinxhref{http://adlibitum.oats.inaf.it/monaco/Homepage/morgana.html}{MORGANA - MOdel for the Rise of GAlaxies aNd Active
nuclei}
\begin{itemize}
\item {} 
Originally based on the work led by \sphinxhref{https://ui.adsabs.harvard.edu/abs/2007MNRAS.375.1189M/abstract}{Pierluigi Monaco et
al. 2007}.

\end{itemize}

\item {} 
\sphinxstylestrong{GalICS} - Galaxies In Cosmological Simulations
\begin{itemize}
\item {} 
Originally based on the work led by \sphinxhref{https://ieeexplore.ieee.org/document/8148635}{Steve Hatton et
al. 2005}. The
updated \sphinxstylestrong{V2.0} is presented in the work led by \sphinxhref{https://ui.adsabs.harvard.edu/abs/2017MNRAS.471.1401C/abstract}{Cattaneo et
al. 2017}

\end{itemize}

\end{itemize}


\paragraph{Semi-Empirical Model (SEM)}
\label{\detokenize{resource/astro/topics/galaxy_formation_model:semi-empirical-model-sem}}\begin{itemize}
\item {} 
This is a relative new approach for modeling galaxy formation. The
key difference with SAM is that it does not focus on the detailed
recipe for physical processes involved in galaxy formation. Instead,
it relies on the assumption that \sphinxstylestrong{empirical} relation can be
established between the star formation rate and the halo accretion
rate (or between stellar mass and halo mass). One key advantage of
this approach is that it can run on N-body simulations much faster
and hence can be directly constrained by a series of observations at
different epochs through Bayesian analysis.

\item {} 
Some important early works on this topic include: \sphinxhref{https://ui.adsabs.harvard.edu/abs/2010ApJ...717..379B/abstract}{Behroozi et
al. 2010},
\sphinxhref{https://ui.adsabs.harvard.edu/abs/2010ApJ...710..903M/abstract}{Moster et
al. 2010}.

\item {} 
\sphinxhref{https://bitbucket.org/pbehroozi/universemachine/src/master/}{UniverseMachine - Empirical Model for Galaxy
Formation}
\begin{itemize}
\item {} 
Developed by Peter Behroozi. Based on the work \sphinxhref{https://arxiv.org/abs/1806.07893}{“UniverseMachine:
The Correlation between Galaxy Growth and Dark Matter Halo
Assembly from z=0-10”}

\item {} 
The \sphinxstylestrong{UniverseMachine} applies simple empirical models of galaxy
formation to dark matter halo merger trees. For each model, it
generates an entire mock universe, which it then observes in the
same way as the real Universe to calculate a likelihood function.
It includes an advanced MCMC algorithm to explore the allowed
parameter space of empirical models that are consistent with
observations.

\item {} 
The \sphinxhref{https://www.peterbehroozi.com/data.html}{data release by benchmark UniverseMachine model can be found
here}

\end{itemize}

\item {} 
\sphinxhref{https://ui.adsabs.harvard.edu/abs/2018MNRAS.477.1822M/abstract}{Emerge - Empirical ModEl for the foRmation of
GalaxiEs}
\begin{itemize}
\item {} 
\sphinxstylestrong{emerge} is an empirical model for the formation of galaxies
since z\textasciitilde{}10

\item {} 
Based on the publication by \sphinxhref{https://ui.adsabs.harvard.edu/abs/2018MNRAS.477.1822M/abstract}{Benjamin Moster, Thorsten Naab, \&
Simon White
2018}

\end{itemize}

\item {} 
\sphinxhref{https://ui.adsabs.harvard.edu/abs/2017MNRAS.470..651R/abstract}{Model by Aldo
Rodríguez-Puebla}
\begin{itemize}
\item {} 
Based on the publication \sphinxhref{https://ui.adsabs.harvard.edu/abs/2017MNRAS.470..651R/abstract}{“Constraining the galaxy-halo connection
over the last 13.3 Gyr: star formation histories, galaxy mergers
and structural properties” by Rodríguez-Puebla et
al. 2017}

\end{itemize}

\item {} 
\sphinxhref{https://ui.adsabs.harvard.edu/abs/2019MNRAS.483.2506G/abstract}{STEEL - a STatistical sEmi-Empirical
modeL}
\begin{itemize}
\item {} 
Based on the publication \sphinxhref{https://ui.adsabs.harvard.edu/abs/2019MNRAS.483.2506G/abstract}{“A statistical semi-empirical model:
satellite galaxies in groups and clusters” by Grylls et
al. 2019}

\end{itemize}

\end{itemize}


\paragraph{Other Model}
\label{\detokenize{resource/astro/topics/galaxy_formation_model:other-model}}\begin{itemize}
\item {} 
Here is a list of model for galaxy formation that can not be easily
classied as SAM or SEM.

\item {} 
\sphinxhref{https://ui.adsabs.harvard.edu/abs/2019arXiv190610135S/abstract}{The Iκϵα model of feedback-regulated galaxy
formation}
\begin{itemize}
\item {} 
In \sphinxstylestrong{Iκϵα}, a galaxy’s star formation rate is set by the balance
between energy injected by feedback from massive stars and energy
lost by the deepening of the potential of its host dark matter
halo due to cosmological accretion.

\end{itemize}

\end{itemize}


\section{Resources about Important Astrophysical and Cosmological Projects}
\label{\detokenize{astro_project:resources-about-important-astrophysical-and-cosmological-projects}}\label{\detokenize{astro_project::doc}}
\begin{sphinxadmonition}{warning}{Warning:}
Most of the collections here are still under heavy construction. The contents are likely to be
highly incomplete. Any help will be highly appreciated!
\end{sphinxadmonition}


\subsection{Important Ground-base Telescope and Instrument}
\label{\detokenize{astro_project:important-ground-base-telescope-and-instrument}}

\subsubsection{Optical and Near-Infrared}
\label{\detokenize{astro_project:optical-and-near-infrared}}

\subsection{Important Space Telescope and Instrument}
\label{\detokenize{astro_project:important-space-telescope-and-instrument}}

\subsubsection{Gaia}
\label{\detokenize{resource/astro/projects/gaia:gaia}}\label{\detokenize{resource/astro/projects/gaia::doc}}\begin{itemize}
\item {} 
\sphinxstyleemphasis{Gaia} is an ambitious mission to chart a three-dimensional map of
our Galaxy, the Milky Way, in the process revealing the composition,
formation and evolution of the Galaxy. \sphinxstyleemphasis{Gaia} will provide
unprecedented positional and radial velocity measurements with the
accuracies needed to produce a stereoscopic and kinematic census of
about one billion stars in our Galaxy and throughout the Local Group.
This amounts to about 1 per cent of the Galactic stellar population.

\item {} 
Right now we are at \sphinxstyleemphasis{Gaia} DR2 (06/2019)

\end{itemize}


\paragraph{Basic Information:}
\label{\detokenize{resource/astro/projects/gaia:basic-information}}\begin{itemize}
\item {} 
\sphinxhref{http://sci.esa.int/gaia/}{Gaia website at ESA}

\item {} 
\sphinxhref{https://www.cosmos.esa.int/web/gaia/data}{Gaia data and tools summary
page}

\item {} 
All data are available from the ESA \sphinxhref{http://gea.esac.esa.int/archive/}{Gaia Data
Archive}
\begin{itemize}
\item {} 
Also at \sphinxhref{http://cdsweb.u-strasbg.fr/gaia}{Gaia at CDS}; \sphinxhref{http://gaia.ari.uni-heidelberg.de/}{Gaia
at ARI}; \sphinxhref{https://gaia.aip.de/}{Gaia at
AIP}; and \sphinxhref{http://gaiaportal.asdc.asi.it/}{Gaia at
ASDC}

\item {} 
And all \sphinxstyleemphasis{Gaia} data can be directly downloaded from \sphinxhref{http://cdn.gea.esac.esa.int/Gaia/}{here in .csv
format}

\end{itemize}

\end{itemize}


\subparagraph{Data Release:}
\label{\detokenize{resource/astro/projects/gaia:data-release}}\begin{itemize}
\item {} 
\sphinxhref{http://sci.esa.int/gaia/58275-data-release-1/}{Gaia Data Release
1}

\item {} 
\sphinxhref{http://sci.esa.int/gaia/60243-data-release-2/}{Gaia Data Release
2}
\begin{itemize}
\item {} 
\sphinxhref{http://gea.esac.esa.int/archive/documentation/GDR2/index.html}{Online
documents}

\item {} 
\sphinxhref{http://gea.esac.esa.int/archive/documentation/GDR2/Gaia\_archive/chap\_datamodel/}{Gaia DR2 data
models}

\item {} 
\sphinxhref{https://www.cosmos.esa.int/web/gaia/dr2-known-issues}{Gaia DR2 known
issues}

\item {} 
\sphinxhref{http://gea.esac.esa.int/archive-help/index.html}{Official tutorials to use Gaia DR2
data}

\item {} 
\sphinxhref{https://www.cosmos.esa.int/web/gaia/guide-to-scientists\#video6}{Video guide for
scientists}

\end{itemize}

\end{itemize}


\subparagraph{Using \sphinxstyleemphasis{Gaia} Data}
\label{\detokenize{resource/astro/projects/gaia:using-gaia-data}}\begin{itemize}
\item {} 
\sphinxhref{http://www.mpia.de/~calj/gdr2\_distances/main.html}{Estimating distances from parallaxes. IV. Distances to 1.33 billion
stars in Gaia data release
2}
\begin{itemize}
\item {} 
The “BailerJones” catalog of distance.

\end{itemize}

\item {} 
\sphinxhref{https://github.com/agabrown/astrometry-inference-tutorials}{Tutorials on the use of (Gaia) astrometry in astronomical data
analysis or inference
problems}
\begin{itemize}
\item {} 
By Anthony Brown. This is a \sphinxstylestrong{must read}

\item {} 
Also see his \sphinxhref{https://github.com/agabrown/gaiadr2-6dgold-example}{example of the construction of a sample of sources
from Gaia DR2 with all recommend data quality filtering
applied}

\item {} 
And \sphinxhref{https://github.com/agabrown/gaiadr2-ruwe-tools}{Python code for calculating the Renormalized Unit Weight
Error from the Gaia DR2 table
columns}

\end{itemize}

\item {} 
Slides about \sphinxhref{https://www.cosmos.esa.int/documents/29201/1770596/Lindegren\_GaiaDR2\_Astrometry\_extended.pdf/1ebddb25-f010-6437-cb14-0e360e2d9f09}{Gaia DR2
astrometry}

\item {} 
Slides about \sphinxhref{https://www.cosmos.esa.int/documents/915837/915858/2016\_11\_02\_dr1Workshop\_AlcioneMora.pdf/39b2492b-e74c-4b28-bd1f-71258b25fd08}{Working with Gaia
Data}

\item {} 
The \sphinxstyleemphasis{Gaia} TAP+ catalog can be accessed through \sphinxstylestrong{astroquery.gaia}
module. Here is a \sphinxhref{https://gea.esac.esa.int/archive-help/tutorials/python\_cluster/index.html}{Python example for
DR2}

\item {} 
\sphinxhref{http://www.euro-vo.org/sites/default/files/documents/tutorial-topcat-stilts\_2018Nov.pdf}{Exploring Gaia data with TOPCAT and
STILTS}

\item {} 
\sphinxhref{https://gaia-kepler.fun/}{Gaia-Kepler.fun}
\begin{itemize}
\item {} 
This website provides cross-matched catalogs of Gaia data for
stars observed by Kepler/K2

\end{itemize}

\end{itemize}


\subparagraph{Social Media}
\label{\detokenize{resource/astro/projects/gaia:social-media}}\begin{itemize}
\item {} 
{[}Twitter handle: @ESAGaia{]}(\sphinxurl{https://twitter.com/ESAGaia})

\end{itemize}


\paragraph{Important Documents:}
\label{\detokenize{resource/astro/projects/gaia:important-documents}}

\subparagraph{Special Issues:}
\label{\detokenize{resource/astro/projects/gaia:special-issues}}\begin{itemize}
\item {} 
There are too many important publications about the \sphinxstyleemphasis{Gaia} mission,
instrument, data reduction, and data release. Most of these
“official” publications are compiled into two special issues of the
Astronomy \& Astrophysics:
\begin{itemize}
\item {} 
\sphinxhref{https://www.aanda.org/component/toc/?task=topic\&id=641}{A\&A Special Issue: Gaia Data Release
1}

\item {} 
\sphinxhref{https://www.aanda.org/component/toc/?task=topic\&id=922}{A\&A Special Issue: Gaia Data Release
2}

\end{itemize}

\end{itemize}


\paragraph{Important Publications:}
\label{\detokenize{resource/astro/projects/gaia:important-publications}}\begin{itemize}
\item {} 
For using DR2 data, the most important ones are:
\begin{itemize}
\item {} 
\sphinxhref{https://www.aanda.org/articles/aa/pdf/2018/08/aa33955-18.pdf}{Gaia Data Release
2}

\item {} 
\sphinxhref{https://www.aanda.org/articles/aa/pdf/2018/08/aa33051-18.pdf}{Gaia Data Release 2 - Summary of the contents and survey
properties}

\item {} 
\sphinxhref{https://www.aanda.org/articles/aa/pdf/2018/08/aa32964-18.pdf}{Gaia Data Release 2 - Using Gaia
parallaxes}

\item {} 
\sphinxhref{https://www.aanda.org/articles/aa/pdf/2019/01/aa34142-18.pdf}{Gaia Data Release 2 - Cross-match with external catalogues:
algorithms and
results}

\end{itemize}

\end{itemize}


\paragraph{Softwares and Tools:}
\label{\detokenize{resource/astro/projects/gaia:softwares-and-tools}}\begin{itemize}
\item {} 
\sphinxhref{https://github.com/agabrown/PyGaia}{PyGaia - Python toolkit for basic Gaia data simulation,
manipulation, and analysis}
\begin{itemize}
\item {} 
By Anthony Brown. PyGaia provides python modules for the
simulation of Gaia data and their errors, as well modules for the
manipulation and analysis of the Gaia catalogue data

\end{itemize}

\item {} 
{[}\_\_gaia\_tools\_\_ - Tools for working with the @ESAGaia data and
related data sets{]}(\sphinxurl{https://github.com/jobovy/gaia\_tools})
\begin{itemize}
\item {} 
By Jo Bovy. Tools for working with the ESA/Gaia data and related
data sets (APOGEE, GALAH, LAMOST DR2, and RAVE).

\end{itemize}

\item {} 
\sphinxhref{https://github.com/GalacticDynamics-Oxford/GaiaTools}{GaiaTools - A collection of scripts for analyzing the data from the
Gaia
satellite}
\begin{itemize}
\item {} 
By the Oxford’s Galactic Dynamics group.

\end{itemize}

\item {} 
\sphinxhref{https://astroquery.readthedocs.io/en/latest/gaia/gaia.html}{Gaia TAP+ through
astroquery.gaia}

\item {} 
\sphinxhref{http://cds.u-strasbg.fr/resources/doku.php?id=cdsclient:doc-find\_gaia\_dr2}{find\_gaia\_dr2.py}
\begin{itemize}
\item {} 
By CDS. Depends on \sphinxstylestrong{python-cdsclient}

\end{itemize}

\item {} 
\sphinxhref{https://zah.uni-heidelberg.de/institutes/ari/gaia/outreach/gaiasky/}{Gaia Sky - a real-time, 3D, astronomy visualisation
software}

\end{itemize}


\paragraph{Conferences and Sprints}
\label{\detokenize{resource/astro/projects/gaia:conferences-and-sprints}}\begin{itemize}
\item {} 
\sphinxhref{http://gaia.lol/}{Gaia Sprints}
\begin{itemize}
\item {} 
A project to support exploration and scientific use of the Gaia
Data Releases.

\item {} 
There are a lot of interesting discussions and ideas can be found
in the sprint website.

\end{itemize}

\item {} 
\sphinxhref{https://www.kitp.ucsb.edu/activities/gaia19}{KITP Project 2019: Dynamical Models for Stars and Gas in Galaxies in
the Gaia Era}
\begin{itemize}
\item {} 
Most of the \sphinxhref{http://online.kitp.ucsb.edu/online/gaia19/}{talks can be found on
line}

\item {} 
There is also a associated conference: \sphinxhref{https://www.kitp.ucsb.edu/activities/gaia-c19}{In the Balance: Stasis and
Disequilibrium in the Milky
Way}

\end{itemize}

\item {} 
\sphinxhref{http://gaia.ari.uni-heidelberg.de/gaia-workshop-2018/}{Gaia Data Workshop - Heidelberg
2018}
\begin{itemize}
\item {} 
A lot of useful slides are available here.

\end{itemize}

\item {} 
\sphinxhref{https://www.gaia.ac.uk/science/workshops/gaia-dr2-london-may18}{Gaia Data Release 2 London
workshop}

\item {} 
\sphinxhref{https://www.physik.uzh.ch/events/sf2019/}{49th Saas-Fee Advanced Course - The Milky Way in the Gaia
Era}
\begin{itemize}
\item {} 
Video lectures and course materials are available.

\end{itemize}

\end{itemize}


\subsubsection{James Webb Space Telescope}
\label{\detokenize{resource/astro/projects/jwst:james-webb-space-telescope}}\label{\detokenize{resource/astro/projects/jwst::doc}}

\paragraph{Basic Information}
\label{\detokenize{resource/astro/projects/jwst:basic-information}}\begin{itemize}
\item {} 
\sphinxhref{https://www.jwst.nasa.gov/}{The James Webb Space Telescope (JWST) homepage at
NASA}

\end{itemize}


\subsection{Important Sky Survey}
\label{\detokenize{astro_project:important-sky-survey}}

\subsubsection{Dark Energy Survey (DES) and Dark Energy Camera (DECam)}
\label{\detokenize{resource/astro/projects/des:dark-energy-survey-des-and-dark-energy-camera-decam}}\label{\detokenize{resource/astro/projects/des::doc}}

\paragraph{Basic Information}
\label{\detokenize{resource/astro/projects/des:basic-information}}\begin{itemize}
\item {} 
\sphinxhref{https://www.darkenergysurvey.org/}{The Dark Energey Survey official
website}

\end{itemize}


\subsubsection{Dark Energy Spectroscopic Instrument}
\label{\detokenize{resource/astro/projects/desi:dark-energy-spectroscopic-instrument}}\label{\detokenize{resource/astro/projects/desi::doc}}\begin{itemize}
\item {} 
The Dark Energy Spectroscopic Instrument (DESI) will measure the
effect of dark energy on the expansion of the universe. It will
obtain optical spectra for tens of millions of galaxies and quasars,
constructing a 3D map spanning the nearby universe to 11 billion
light years.

\end{itemize}


\paragraph{Basic Information}
\label{\detokenize{resource/astro/projects/desi:basic-information}}\begin{itemize}
\item {} 
\sphinxhref{https://www.desi.lbl.gov/}{The Dark Energy Spectroscopic Instrument (DESI)
Webpage}

\end{itemize}


\subparagraph{Imaging Surveys:}
\label{\detokenize{resource/astro/projects/desi:imaging-surveys}}\begin{itemize}
\item {} 
\sphinxhref{http://www.legacysurvey.org/}{DECam Legacy Surveys}

\end{itemize}


\subparagraph{Social Media:}
\label{\detokenize{resource/astro/projects/desi:social-media}}\begin{itemize}
\item {} 
{[}@desisurvey: DESI’s twitter
handle{]}(\sphinxurl{https://twitter.com/desisurvey?lang=en})

\end{itemize}


\paragraph{Important Documents:}
\label{\detokenize{resource/astro/projects/desi:important-documents}}\begin{itemize}
\item {} 
\sphinxhref{https://arxiv.org/pdf/1611.00036.pdf}{Science Final Design
Report}

\item {} 
\sphinxhref{https://arxiv.org/pdf/1611.00037.pdf}{Instrument Final Design
Report}

\end{itemize}


\subparagraph{Policy and Community:}
\label{\detokenize{resource/astro/projects/desi:policy-and-community}}\begin{itemize}
\item {} 
\sphinxhref{https://desi.lbl.gov/trac/attachment/wiki/PublicPages/bylaws-v11.pdf}{DESI collaboration bylaws: collaboration, membership,
governance​}

\item {} 
\sphinxhref{https://desi.lbl.gov/trac/attachment/wiki/PublicPages/datamanplan.pdf}{DESI data management plan
(DMP)}

\item {} 
\sphinxhref{https://desi.lbl.gov/trac/attachment/wiki/PublicPages/membership.approved.v1.pdf}{DESI Policy on Membership Effort and Builder
Status​}

\item {} 
\sphinxhref{https://desi.lbl.gov/trac/attachment/wiki/PublicPages/continuing\_participants.approved.v1.pdf}{DESI Policy on Ongoing Participation By Those Leaving DESI
Institutions}

\item {} 
\sphinxhref{https://desi.lbl.gov/trac/attachment/wiki/PublicPages/DESI\_Publication\_Policy\_v1.0.pdf}{DESI Publication
Policy}

\item {} 
\sphinxhref{https://desi.lbl.gov/trac/attachment/wiki/PublicPages/external\_collaborator.approved\_v1.pdf}{DESI External Collaborator
Policy}

\end{itemize}


\paragraph{Important Publications:}
\label{\detokenize{resource/astro/projects/desi:important-publications}}\begin{itemize}
\item {} 
\sphinxhref{https://arxiv.org/abs/1807.09287}{Overview of the Dark Energy Spectroscopic
Instrument}

\end{itemize}


\subparagraph{Imaging Survey:}
\label{\detokenize{resource/astro/projects/desi:imaging-survey}}\begin{itemize}
\item {} 
\sphinxhref{https://ui.adsabs.harvard.edu/abs/2019AJ....157..168D/abstract}{Overview of the DESI Legacy Imaging
Surveys}

\item {} 
\sphinxhref{https://ui.adsabs.harvard.edu/abs/2019ApJS..242....8Z/abstract}{Photometric Redshifts and Stellar Masses for Galaxies from the DESI
Legacy Imaging
Surveys}

\end{itemize}


\subparagraph{Technical Papers:}
\label{\detokenize{resource/astro/projects/desi:technical-papers}}\begin{itemize}
\item {} 
\sphinxhref{https://arxiv.org/abs/1807.09383}{Dark Energy Spectroscopic Instrument (DESI) Fiber Positioner Thermal
and Wind Disturbance Test}

\item {} 
\sphinxhref{https://ui.adsabs.harvard.edu/abs/2018SPIE10707E..1DH/abstract}{The DESI instrument control systems: status and early
testing}

\item {} 
\sphinxhref{https://ui.adsabs.harvard.edu/abs/2018SPIE10706E..69L/abstract}{Dark energy spectroscopic instrument (DESI) fiber positioner
production}

\item {} 
\sphinxhref{https://ui.adsabs.harvard.edu/abs/2018SPIE10706E..43D/abstract}{DESI focal plate
alignment}

\item {} 
\sphinxhref{https://ui.adsabs.harvard.edu/abs/2018SPIE10706E..0XM/abstract}{Fabrication of the DESI corrector
lenses}

\item {} 
\sphinxhref{https://ui.adsabs.harvard.edu/abs/2018SPIE10702E..7OP/abstract}{The DESI fiber
system}

\item {} 
\sphinxhref{https://ui.adsabs.harvard.edu/abs/2018SPIE10702E..7GE/abstract}{The DESI spectrograph system and
production}

\item {} 
\sphinxhref{https://ui.adsabs.harvard.edu/abs/2018SPIE10702E..7DF/abstract}{A predictive optical sky background model for
DESI}

\end{itemize}


\subparagraph{Cosmology Related:}
\label{\detokenize{resource/astro/projects/desi:cosmology-related}}

\subparagraph{Impacts of Fiber Assignments:}
\label{\detokenize{resource/astro/projects/desi:impacts-of-fiber-assignments}}\begin{itemize}
\item {} 
\sphinxhref{https://ui.adsabs.harvard.edu/abs/2018MNRAS.481.2338B/abstract}{Unbiased clustering estimates with the DESI fibre
assignment}

\item {} 
\sphinxhref{https://ui.adsabs.harvard.edu/abs/2017JCAP...04..008P/abstract}{Imprint of DESI fiber assignment on the anisotropic power spectrum
of emission line
galaxies}

\item {} 
\sphinxhref{https://ui.adsabs.harvard.edu/abs/2017JCAP...03..001B/abstract}{Mitigating the impact of the DESI fiber assignment on galaxy
clustering}

\end{itemize}


\paragraph{Software and Pipeline:}
\label{\detokenize{resource/astro/projects/desi:software-and-pipeline}}\begin{itemize}
\item {} 
\sphinxhref{https://github.com/desihub}{desihub - Public code associated with the Dark Energy Spectroscopic
Instrument}

\end{itemize}


\subparagraph{Key Components:}
\label{\detokenize{resource/astro/projects/desi:key-components}}\begin{itemize}
\item {} 
\sphinxhref{https://github.com/desihub/desisurvey}{desisurvey: Code for desi survey planning and
implementation}

\item {} 
\sphinxhref{https://github.com/desihub/surveysim}{surveysim: Simulate afternoon planning and nightly tile scheduling
for the DESI survey}

\item {} 
\sphinxhref{https://github.com/desihub/specter}{specter: A toolkit for simulating multi-object
spectrographs}

\item {} 
\sphinxhref{https://github.com/desihub/desitarget}{desitarget: DESI
targeting}

\item {} 
\sphinxhref{https://github.com/desihub/desispec}{desispec: DESI spectral
pipeline}

\item {} 
\sphinxhref{https://github.com/desihub/specsim}{specsim: Quick simulations of spectrograph
response}

\item {} 
\sphinxhref{https://github.com/desihub/desisim}{desisim: DESI simulations}

\item {} 
\sphinxhref{https://github.com/desihub/fiberassign}{fiberassign: Fiber assignment code for
DESI}

\item {} 
\sphinxhref{https://github.com/desihub/redrock}{redrock: Redshift fitting for
spectroperfectionism}

\item {} 
\sphinxhref{https://github.com/desihub/desietc}{desietc: Online exposure time
calculator}

\end{itemize}


\subsubsection{Large Synoptic Survey Telescope (LSST)}
\label{\detokenize{resource/astro/projects/lsst:large-synoptic-survey-telescope-lsst}}\label{\detokenize{resource/astro/projects/lsst::doc}}\begin{itemize}
\item {} 
Opening a Window of Discovery on the Dynamic Universe

\end{itemize}


\paragraph{Basic Information:}
\label{\detokenize{resource/astro/projects/lsst:basic-information}}\begin{itemize}
\item {} 
\sphinxhref{https://www.lsst.org/}{LSST Organization Website}
\begin{itemize}
\item {} 
\sphinxhref{https://www.lsst.org/content/lsst-information-scientists}{Information for
scientists}

\end{itemize}

\item {} 
\sphinxhref{https://community.lsst.org/categories}{LSST Community Forum}

\item {} 
\sphinxhref{https://www.lsstcorporation.org/}{The LSST Corporation (LSSTC)}
\begin{itemize}
\item {} 
LSSTC is a not-for-profit 501(c)3 corporation formed to initiate
the LSST Project and advance the science of astronomy and physics.

\end{itemize}

\item {} 
\sphinxhref{https://www.lsst.org/about/fact-sheets}{LSST Fact Sheets}

\item {} 
\sphinxhref{https://zenodo.org/search?page=1\&size=20\&q=\%22LSST\%22\%20\%22data\%20management\%22}{LSST Data Management on
Zenodo}

\item {} 
\sphinxhref{https://lsst.slac.stanford.edu/\#science-goals}{LSST Camera Website at
SLAC}

\item {} 
\sphinxhref{https://project.lsst.org/groups/sac/welcome}{LSST Science Advisory
Committee}

\item {} 
\sphinxhref{http://ast.noao.edu/lsst/}{NOAO \& Large Synoptic Survey Telescope
(LSST)}

\end{itemize}


\subparagraph{LSST Science Collaborations}
\label{\detokenize{resource/astro/projects/lsst:lsst-science-collaborations}}\begin{itemize}
\item {} 
\sphinxhref{http://galaxies.science.lsst.org/}{Galaxies Collaboration}

\item {} 
\sphinxhref{http://milkyway.science.lsst.org/}{Stars, Milky Way, and Local
Volume}

\item {} 
\sphinxhref{http://solarsystem.science.lsst.org/}{Solar System Collaboration}

\item {} 
\sphinxhref{http://lsst-desc.org/}{Dark Energy Collaboration}

\item {} 
\sphinxhref{https://agn.science.lsst.org/}{Active Galactic Nuclei
Collaboration}

\item {} 
\sphinxhref{https://tvs.science.lsst.org/}{Transients and Variable Stars
Collaboration}

\item {} 
\sphinxhref{https://sites.google.com/view/lsst-stronglensing}{Strong Lensing
Collaboration}

\item {} 
\sphinxhref{https://issc.science.lsst.org/}{Informatics and Statistics
Collaboration}

\end{itemize}


\subparagraph{LSST Dark Matter}
\label{\detokenize{resource/astro/projects/lsst:lsst-dark-matter}}\begin{itemize}
\item {} 
\sphinxhref{https://lsstdarkmatter.github.io/}{LSST Dark Matter - Probing the fundamental physics of dark matter
with LSST}
\begin{itemize}
\item {} 
\sphinxhref{https://lsstdarkmatter.github.io/dark-matter-graph}{Dark Matter
Graphics} -
This graphic is intended to help conceptually organize the LSST
dark matter program and to serve as a road map for future
scientific investigations.

\end{itemize}

\end{itemize}


\subparagraph{LSST Dark Energy Science Collaboration}
\label{\detokenize{resource/astro/projects/lsst:lsst-dark-energy-science-collaboration}}\begin{itemize}
\item {} 
\sphinxhref{https://lsstdesc.org/}{LSST Dark Energy Science Collaboration}
\begin{itemize}
\item {} 
LSST DESC is the large international collaboration that will make
high accuracy measurements of fundamental cosmological parameters
using data from the Large Synoptic Survey Telescope

\item {} 
\sphinxhref{https://lsstdesc.org/assets/pdf/docs/DESC\_SRM\_latest.pdf}{LSST DESC Science
Roadmap}

\item {} 
\sphinxhref{https://lsstdesc.org/pages/DESchool.html}{LSST Dark Energy
School}

\end{itemize}

\end{itemize}


\subparagraph{Social Media}
\label{\detokenize{resource/astro/projects/lsst:social-media}}\begin{itemize}
\item {} 
\sphinxhref{https://www.youtube.com/channel/UCoKgqaxFrkwnffyfyhKhC2w}{LSST Youtube
Channel}

\item {} 
\sphinxhref{https://twitter.com/lsst?lang=en}{LSST Twitter Handle}

\item {} 
\sphinxhref{https://www.linkedin.com/company/large-synoptic-survey-telescope-lsst-}{LSST
LinkedIn}

\end{itemize}


\paragraph{Important Documents:}
\label{\detokenize{resource/astro/projects/lsst:important-documents}}\begin{itemize}
\item {} 
\sphinxhref{https://www.lsst.org/scientists/scibook}{LSST Science Book Version
2.0}
\begin{itemize}
\item {} 
596 pages, over 50 Mb. The \sphinxhref{https://arxiv.org/abs/0912.0201}{arXiv version is
here}

\end{itemize}

\item {} 
\sphinxhref{https://arxiv.org/abs/1708.04058}{Science-Driven Optimization of the LSST Observing
Strategy}
\begin{itemize}
\item {} 
The GitHub repo is
\sphinxhref{https://github.com/LSSTScienceCollaborations/ObservingStrategy}{here}

\end{itemize}

\item {} 
\sphinxhref{https://docushare.lsstcorp.org/docushare/dsweb/Get/LPM-151/}{Summary of Data Management
Principles}

\item {} 
\sphinxhref{https://docushare.lsstcorp.org/docushare/dsweb/Get/LSE-163/}{Data Products Definition
Document}

\end{itemize}


\subparagraph{Important Publications:}
\label{\detokenize{resource/astro/projects/lsst:important-publications}}\begin{itemize}
\item {} 
\sphinxhref{https://arxiv.org/abs/0805.2366}{LSST: from Science Drivers to Reference Design and Anticipated Data
Products}

\item {} 
\sphinxhref{https://arxiv.org/abs/1902.01055}{Probing the Fundamental Nature of Dark Matter with the Large
Synoptic Survey Telescope}

\end{itemize}


\subparagraph{Planning and Scheduling Observations:}
\label{\detokenize{resource/astro/projects/lsst:planning-and-scheduling-observations}}\begin{itemize}
\item {} 
\sphinxhref{https://ui.adsabs.harvard.edu/abs/2019AJ....157..151N/abstract}{A Framework for Telescope Schedulers: With Applications to the Large
Synoptic Survey
Telescope}

\item {} 
\sphinxhref{https://arxiv.org/abs/1812.00515}{Optimizing the LSST Observing Strategy for Dark Energy Science: DESC
Recommendations for the Wide-Fast-Deep
Survey}

\item {} 
\sphinxhref{https://ui.adsabs.harvard.edu/abs/2019ApJ...874...88C/abstract}{LSST Target-of-opportunity Observations of Gravitational-wave
Events: Essential and
Efficient}

\item {} 
\sphinxhref{https://arxiv.org/abs/1906.08235}{The Prospects of Observing Tidal Disruption Events with the
LSST}

\end{itemize}


\subparagraph{Technical Papers:}
\label{\detokenize{resource/astro/projects/lsst:technical-papers}}\begin{itemize}
\item {} 
\sphinxhref{https://ui.adsabs.harvard.edu/abs/2018arXiv181203248B/abstract}{An Overview of the LSST Image Processing
Pipelines}

\item {} 
\sphinxhref{https://ui.adsabs.harvard.edu/abs/2019arXiv190311756K/abstract}{Models and Simulations for the Photometric LSST Astronomical Time
Series Classification Challenge
(PLAsTiCC)}

\item {} 
\sphinxhref{https://ui.adsabs.harvard.edu/abs/2018ApJS..234...36M/abstract}{DESCQA: An Automated Validation Framework for Synthetic Sky
Catalogs}

\end{itemize}


\subparagraph{Synergy with Other Projects}
\label{\detokenize{resource/astro/projects/lsst:synergy-with-other-projects}}\begin{itemize}
\item {} 
\sphinxhref{https://arxiv.org/abs/1904.10439}{Enhancing LSST Science with Euclid
Synergy}

\item {} 
\sphinxhref{https://ui.adsabs.harvard.edu/abs/2018arXiv181203298C/abstract}{The Gaia-LSST Synergy: resolved stellar populations in selected
Local Group stellar
systems}

\end{itemize}


\subparagraph{Need for Spectroscopy}
\label{\detokenize{resource/astro/projects/lsst:need-for-spectroscopy}}\begin{itemize}
\item {} 
\sphinxhref{https://ui.adsabs.harvard.edu/abs/2019BAAS...51c.358N/abstract}{Deep Multi-object Spectroscopy to Enhance Dark Energy Science from
LSST}

\item {} 
\sphinxhref{https://ui.adsabs.harvard.edu/abs/2019BAAS...51c.369H/abstract}{Single-object Imaging and Spectroscopy to Enhance Dark Energy
Science from
LSST}

\item {} 
\sphinxhref{https://ui.adsabs.harvard.edu/abs/2019BAAS...51c.363M/abstract}{Wide-field Multi-object Spectroscopy to Enhance Dark Energy Science
from
LSST}

\end{itemize}


\subparagraph{Science and System Requirements:}
\label{\detokenize{resource/astro/projects/lsst:science-and-system-requirements}}\begin{itemize}
\item {} 
\sphinxhref{https://docushare.lsstcorp.org/docushare/dsweb/Get/LPM-17}{Science Requirements
Document}

\item {} 
\sphinxhref{https://docushare.lsstcorp.org/docushare/dsweb/Get/LSE-29}{LSST System Requirements
(LSR)}

\item {} 
\sphinxhref{https://docushare.lsstcorp.org/docushare/dsweb/Get/LSE-30}{Observatory System Specifications
(OSS)}

\item {} 
\sphinxhref{https://docushare.lsstcorp.org/docushare/dsweb/Get/LSE-61}{Data Management System (DMS)
Requirements}

\item {} 
\sphinxhref{https://docushare.lsstcorp.org/docushare/dsweb/Get/LSE-60}{Telescope and Site (TS)
Requirements}

\item {} 
\sphinxhref{https://docushare.lsstcorp.org/docushare/dsweb/Get/LSE-59}{Camera Subsystem
Requirements}

\item {} 
\sphinxhref{https://docushare.lsstcorp.org/docushare/dsweb/Get/LSE-62}{Observatory Control System
Requirements}

\end{itemize}


\subparagraph{Policy and Community}
\label{\detokenize{resource/astro/projects/lsst:policy-and-community}}\begin{itemize}
\item {} 
\sphinxhref{https://community.lsst.org/faq}{LSST Community Guidelines}

\item {} 
\sphinxhref{https://docushare.lsstcorp.org/docushare/dsweb/Get/Document-24920}{LSST Communications Code of
Conduct}

\item {} 
\sphinxhref{https://docushare.lsst.org/docushare/dsweb/Get/Document-28973/NoContent2461049852211920670.txt}{LSST Meetings Code of
Conduct}

\item {} 
\sphinxhref{https://docushare.lsstcorp.org/docushare/dsweb/Get/LPM-171/BullyingHarassmentComplaintsProcedure.pdf}{Bullying and Harassment Complaints Procedural
Manual}

\item {} 
\sphinxhref{https://docushare.lsstcorp.org/docushare/dsweb/Get/Document-17995/}{Template for Science Collaboration Publication
Policy}

\item {} 
\sphinxhref{https://github.com/lsst-pst/LSSTreferences/}{How to Cite LSST
Papers}

\end{itemize}


\paragraph{Software and Pipeline:}
\label{\detokenize{resource/astro/projects/lsst:software-and-pipeline}}\begin{itemize}
\item {} 
\sphinxhref{https://pipelines.lsst.io/}{The LSST Science Pipelines}
\begin{itemize}
\item {} 
\sphinxhref{https://github.com/lsst}{GitHub Repositories}

\item {} 
\sphinxhref{https://www.lsst.io/}{LSST Documentation Hub}

\item {} 
\sphinxhref{http://doxygen.lsst.codes/stack/doxygen/x\_masterDoxyDoc/}{Doxygen
Documents}

\item {} 
\sphinxhref{https://github.com/lsst/pipelines\_lsst\_io/blob/master/.github/CONTRIBUTING.rst}{Documentation contribution
guidelines}

\end{itemize}

\item {} 
\sphinxhref{https://github.com/lsst-dm}{LSST Data Management Team Github
Organization}

\item {} 
\sphinxhref{https://github.com/LSSTDESC}{LSST Dark Energy Science Collaboration Github
Organization}

\item {} 
\sphinxhref{https://github.com/lsstdarkmatter}{LSST Dark Matter Github
Organization}

\end{itemize}


\subparagraph{Important Components:}
\label{\detokenize{resource/astro/projects/lsst:important-components}}\begin{itemize}
\item {} 
\sphinxhref{https://github.com/lsst/afw}{afw: LSST data management: pipeline library code and primitives
including images and tables}

\item {} 
\sphinxhref{https://github.com/lsst/lsst\_build}{lsst-build: a builder and continuous integration tool for
LSST}

\item {} 
\sphinxhref{https://github.com/lsst/pipe\_base}{pipe\_base: LSST Data Management: base classes for data processing
tasks}

\item {} 
\sphinxhref{https://github.com/lsst/pipe\_tasks}{pipe\_tasks: LSST Data Management: astronomical data processing
tasks}

\item {} 
\sphinxhref{https://github.com/lsst/pipe\_drivers}{pipe\_drivers: LSST Data Management: high level task coordination
scripts}

\item {} 
\sphinxhref{https://github.com/lsst/daf\_butler}{daf\_butler: Prototype for data access framework described in
DMTN-056}

\item {} 
\sphinxhref{https://github.com/lsst/daf\_persistence}{daf\_persistence: Data access interface (the Butler) and deprecated
persistence framework for LSST Data
Management}

\item {} 
\sphinxhref{https://github.com/lsst/daf\_base}{daf\_base: Low-level data structures, including memory-management
helpers (Citizen), mappings (PropertySet, PropertyList), and
DateTime}

\item {} 
\sphinxhref{https://github.com/lsst/skymap}{skymap: Sky pixelization interfaces and implementations used by LSST
Data Management}

\item {} 
\sphinxhref{https://github.com/lsst/jointcal}{jointcal: Simultaneous astrometry and
photometry}

\item {} 
\sphinxhref{https://github.com/lsst/meas\_mosaic}{meas\_mosaic: determine consistent astrometry and photometry for
multiple images}

\item {} 
\sphinxhref{https://github.com/lsst/meas\_modelfit}{meas\_modelfit: LSST Data Management: model fitting
algorithms}

\item {} 
\sphinxhref{https://github.com/lsst/meas\_extensions\_shapeHSM}{meas\_extentions\_shapeHSM: LSST Data Management: HSM shape
measurement}

\item {} 
\sphinxhref{https://github.com/lsst/meas\_extensions\_psfex}{meas\_extensions\_psfex: LSST Data Management: PSF
Estimation}

\item {} 
\sphinxhref{https://github.com/lsst/meas\_base}{meas\_base: LSST Data Management: core astronomical measurement
algorithms}

\item {} 
\sphinxhref{https://github.com/lsst/meas\_algorithms}{meas\_algorithms: LSST Data Management: astronomical measurement
algorithm}

\item {} 
\sphinxhref{https://github.com/lsst/meas\_deblender}{meas\_deblender: LSST Data Management: astronomical source
deblender}

\item {} 
\sphinxhref{https://github.com/lsst/meas\_astrom}{meas\_astrom: LSST Data Management: astrometric measurement
algorithms}

\item {} 
\sphinxhref{https://github.com/lsst/ip\_isr}{ip\_isr: LSST data management: instrument signature removal
(detrending) for astronomical
images}

\item {} 
\sphinxhref{https://github.com/lsst/ip\_diffim}{ip\_diffim: LSST data management: astronomical image
differencing}

\item {} 
\sphinxhref{https://github.com/lsst/geom}{geom: Low-level geometry primitives for LSST Data
Management}

\item {} 
\sphinxhref{https://github.com/lsst/coadd\_utils}{coadd\_utils: LSST data management: base classes for coadding
(stacking) astronomical
images}

\item {} 
\sphinxhref{https://github.com/lsst/throughputs}{throughputs: LSST Simulations repository for baseline evaluation
information}

\end{itemize}


\subparagraph{Simulation Related:}
\label{\detokenize{resource/astro/projects/lsst:simulation-related}}\begin{itemize}
\item {} 
\sphinxhref{https://github.com/LSSTDESC/imSim}{imSim: GalSim based Large Synoptic Survey Telescope (LSST) image
simulation package}
\begin{itemize}
\item {} 
It produces simulated images from the 3.25 Gigapixel camera which
are suitable to be processed through the LSST Data Management
pipeline.

\end{itemize}

\item {} 
\sphinxhref{https://bitbucket.org/phosim/phosim\_release/wiki/Home}{phoSim: Photon
Simulator}
\begin{itemize}
\item {} 
A set of extremely fast photon Monte Carlo codes used to calculate
the physics of the atmosphere and a telescope \& camera in order to
simulate realistic optical/IR astronomical images.

\end{itemize}

\item {} 
\sphinxhref{https://github.com/lsst/sims\_utils}{sims\_utils: LSST Simulations package for simulation utility
functions}

\item {} 
\sphinxhref{https://github.com/lsst/sims\_photUtils}{sims\_photUtils: LSST Simulations package for photometric utility
functions}

\item {} 
\sphinxhref{https://github.com/lsst/sims\_survey\_fields}{sims\_survey\_fields: A package for retrieving LSST survey
fields}

\item {} 
\sphinxhref{https://github.com/lsst/sims\_skybrightness}{sims\_skybrightness: Tool to predict the background sky emission SED
at an arbitrary RA, Dec, and
MJD}

\end{itemize}


\subsubsection{Euclid Satellite}
\label{\detokenize{resource/astro/projects/euclid:euclid-satellite}}\label{\detokenize{resource/astro/projects/euclid::doc}}

\paragraph{Basic Information}
\label{\detokenize{resource/astro/projects/euclid:basic-information}}\begin{itemize}
\item {} 
\sphinxhref{https://www.euclid-ec.org/}{Euclid Consortium - A Space mission to map the Dark
Universe}

\end{itemize}


\subsubsection{Wide Field Infrared Survey Telescope}
\label{\detokenize{resource/astro/projects/wfirst:wide-field-infrared-survey-telescope}}\label{\detokenize{resource/astro/projects/wfirst::doc}}\begin{itemize}
\item {} 
The Wide Field Infrared Survey Telescope (WFIRST) is a NASA
observatory designed to unravel the secrets of dark energy and dark
matter, search for and image exoplanets, and explore many topics in
infrared astrophysics. WFIRST has a 2.4m telescope, the same size as
Hubble’s, but with a view 100 times greater than Hubble’s. WFIRST was
the top-ranked large space mission in the 2010 Decadal Survey of
Astronomy and Astrophysics, and is slated to launch in the mid-2020s.

\end{itemize}


\paragraph{Basic Information:}
\label{\detokenize{resource/astro/projects/wfirst:basic-information}}\begin{itemize}
\item {} 
\sphinxhref{https://wfirst.gsfc.nasa.gov/}{WFIRST website}

\item {} 
\sphinxhref{https://wfirst.ipac.caltech.edu/sims/Param\_db.html}{Spacecraft and Instrument Parameters (Cycle
7)}
\begin{itemize}
\item {} 
These parameters refer to the telescope design as specified in
Design Cycle 7 or the 2015 \sphinxhref{https://wfirst.ipac.caltech.edu/docs/WFIRST-AFTA\_SDT\_Report\_150310\_Final.pdf}{WFIRST-AFTA Science Definition Team
and WFIRST Study Office
Report}.

\end{itemize}

\item {} 
\sphinxhref{https://wfirst.gsfc.nasa.gov/science/WFIRST\_Reference\_Information.html}{WFIRST Reference Information (Cycle
7)}
\begin{itemize}
\item {} 
Current (06/2019) design information of WFIRST instruments.

\end{itemize}

\end{itemize}


\subparagraph{Social Media}
\label{\detokenize{resource/astro/projects/wfirst:social-media}}\begin{itemize}
\item {} 
{[}Twitter handle: @NASAWFIRST{]}(\sphinxurl{https://twitter.com/nasawfirst?lang=en})

\end{itemize}


\paragraph{Important Documents:}
\label{\detokenize{resource/astro/projects/wfirst:important-documents}}\begin{itemize}
\item {} 
A lot of historical documents about the design and science of WFIRST
can be found in the \sphinxhref{https://wfirst.gsfc.nasa.gov/library.html}{document
library}

\item {} 
\sphinxhref{http://www.stsci.edu/files/live/sites/www/files/home/wfirst/\_documents/WFIRST-FactSheet-001.pdf}{WFIRST Fact
Sheet}

\item {} 
\sphinxhref{http://www.stsci.edu/files/live/sites/www/files/home/wfirst/\_documents/WFIRST-ScienceSheet-001.pdf}{WFIRST Science
Summary}

\item {} 
\sphinxhref{https://wfirst.ipac.caltech.edu/docs/WFIRST-AFTA\_SDT\_Report\_150310\_Final.pdf}{WFIRST-AFTA Science Definition Team and WFIRST Study Office
Report}

\end{itemize}


\subparagraph{Important Publications:}
\label{\detokenize{resource/astro/projects/wfirst:important-publications}}

\subparagraph{Prediction and Plans:}
\label{\detokenize{resource/astro/projects/wfirst:prediction-and-plans}}\begin{itemize}
\item {} 
\sphinxhref{https://arxiv.org/abs/1709.02763}{Solar system science with the Wide-Field InfraRed Survey Telescope
(WFIRST)}

\item {} 
\sphinxhref{https://arxiv.org/abs/1808.10458}{Photometric Redshift Calibration Requirements for WFIRST
Weak-lensing Cosmology: Predictions from
CANDELS}

\item {} 
\sphinxhref{https://arxiv.org/abs/1808.02490}{Predictions of the WFIRST Microlensing Survey. I. Bound Planet
Detection Rates}

\item {} 
\sphinxhref{https://ui.adsabs.harvard.edu/abs/2019AJ....157..132L/abstract}{Characterization of Exoplanet Atmospheres with the Optical
Coronagraph on
WFIRST}

\item {} 
\sphinxhref{https://ui.adsabs.harvard.edu/abs/2018ApJ...867...23H/abstract}{Simulations of the WFIRST Supernova Survey and Forecasts of
Cosmological
Constraints}

\end{itemize}


\subparagraph{Technical Reports (after 2018):}
\label{\detokenize{resource/astro/projects/wfirst:technical-reports-after-2018}}\begin{itemize}
\item {} 
\sphinxhref{http://www.stsci.edu/files/live/sites/www/files/home/wfirst/\_documents/WFIRST-STSCI-TR1901.pdf}{WFIRST Wavelength Calibration: A Strategy with
M67}

\item {} 
\sphinxhref{http://www.stsci.edu/files/live/sites/www/files/home/wfirst/\_documents/WFIRST-STScI-TR1801.pdf}{Will Gaia be precise enough to solve for the geometric distortion of
the
WFI?}

\end{itemize}


\paragraph{Software and Pipeline:}
\label{\detokenize{resource/astro/projects/wfirst:software-and-pipeline}}\begin{itemize}
\item {} 
\sphinxhref{http://www.stsci.edu/wfirst/science-planning-toolbox}{WFIRST Software and Simulation
Tools}
\begin{itemize}
\item {} 
\sphinxhref{http://www.stsci.edu/files/live/sites/www/files/home/wfirst/\_documents/WFIRST-at-STScI-001.pdf}{One page summary of available
softwares}

\end{itemize}

\item {} 
\sphinxhref{https://github.com/spacetelescope/wfirst-tools}{wfirst-tools - Instructions and tutorials for WFIRST software tools
distributed by STScI for the science
community}

\item {} 
\sphinxhref{https://github.com/spacetelescope/STScI-STIPS}{STScI-STIPS - the Space Telescope Imaging Product
Simulator}
\begin{itemize}
\item {} 
It is designed to create simulations of full-detector
post-pipeline astronomical scenes for any telescope. Now works for
both JWST and WFIRST.

\end{itemize}

\item {} 
\sphinxhref{https://wfirst.gsfc.nasa.gov/science/etc14.html}{Chris Hirata’s WFIRST Galaxy Survey Exposure Time
Calculator}

\end{itemize}


\section{Reading List for a Specific Topic}
\label{\detokenize{astro_reading:reading-list-for-a-specific-topic}}\label{\detokenize{astro_reading::doc}}
\begin{sphinxadmonition}{warning}{Warning:}
Just started this project. Here are just a few demo pages.
\end{sphinxadmonition}

\begin{sphinxadmonition}{note}{Note:}
Right now, the selection of these topics only reflect the curator’s interests and they are
clearly biased toward extragalactic and cosmological topics.
Any help to make them more complete and balanced are highly welcomed!
\end{sphinxadmonition}


\subsection{Important Reviews}
\label{\detokenize{astro_reading:important-reviews}}

\subsubsection{A Collection of Interesting Revew Articles}
\label{\detokenize{resource/astro/reference/reviews:a-collection-of-interesting-revew-articles}}\label{\detokenize{resource/astro/reference/reviews::doc}}
\sphinxstylestrong{Just started, Help needed}
\begin{itemize}
\item {} 
A (personal) collection of useful or interesting review articles
classified based on the \sphinxstylestrong{astro-ph} categories for arXiv.

\item {} 
Will try to make this list up-to-date by only keeping the reviews
from the last five years.

\item {} 
Right now, will only keep review articles published (or posted) after
2017.

\end{itemize}


\paragraph{Astrophysics of Galaxies}
\label{\detokenize{resource/astro/reference/reviews:astrophysics-of-galaxies}}\begin{itemize}
\item {} 
\sphinxhref{https://arxiv.org/abs/1904.12890}{Supermassive Black Holes in the Early
Universe}

\item {} 
\sphinxhref{https://arxiv.org/abs/1901.00008}{Halo Concentrations and the Fundamental Plane of Galaxy
Clusters}

\end{itemize}


\paragraph{Cosmology and Nongalactic Astrophysics}
\label{\detokenize{resource/astro/reference/reviews:cosmology-and-nongalactic-astrophysics}}\begin{itemize}
\item {} 
\sphinxhref{https://arxiv.org/abs/1905.06082}{Modelling baryonic feedback for survey
cosmology}

\item {} 
\sphinxhref{https://arxiv.org/abs/1902.10837}{The galaxy cluster mass scale and its impact on cosmological
constraints from the cluster
population}

\end{itemize}


\paragraph{Earth and Planetary Astrophysics}
\label{\detokenize{resource/astro/reference/reviews:earth-and-planetary-astrophysics}}\begin{itemize}
\item {} 
\sphinxhref{https://arxiv.org/abs/1905.08892}{From Centaurs to comets — 40
years}

\item {} 
\sphinxhref{https://arxiv.org/abs/1905.07158}{Trans-Neptunian objects and Centaurs at thermal
wavelengths}

\item {} 
\sphinxhref{https://arxiv.org/abs/1904.02980}{Kuiper belt: formation and
evolution}

\item {} 
\sphinxhref{https://arxiv.org/abs/1904.03190}{Exoplanetary Atmospheres: Key Insights, Challenges and
Prospects}

\item {} 
\sphinxhref{https://arxiv.org/abs/1812.03793}{Exoplanet Clouds}

\end{itemize}


\paragraph{High Energy Astrophysical Phenomena}
\label{\detokenize{resource/astro/reference/reviews:high-energy-astrophysical-phenomena}}\begin{itemize}
\item {} 
\sphinxhref{https://arxiv.org/abs/1906.10212}{Multi-Messenger Astrophysics}

\item {} 
\sphinxhref{https://arxiv.org/abs/1904.11918}{Multi-Messenger Physics with the Pierre Auger
Observatory}

\item {} 
\sphinxhref{https://arxiv.org/abs/1903.07644}{Multi-Messenger Astrophysics with Pulsar Timing
Arrays}

\item {} 
\sphinxhref{https://arxiv.org/abs/1906.05878}{Fast Radio Bursts: An Extragalactic
Enigma}

\item {} 
\sphinxhref{https://arxiv.org/abs/1904.11067}{Neutrino Emission as Diagnostics of Core-Collapse
Supernovae}

\item {} 
\sphinxhref{https://arxiv.org/abs/1903.11704}{Observing Black Holes Spin}

\end{itemize}


\paragraph{Instrumentation and Methods for Astrophysics}
\label{\detokenize{resource/astro/reference/reviews:instrumentation-and-methods-for-astrophysics}}

\paragraph{Solar and Stellar Astrophysics}
\label{\detokenize{resource/astro/reference/reviews:solar-and-stellar-astrophysics}}\begin{itemize}
\item {} 
\sphinxhref{https://arxiv.org/abs/1907.00115}{Pulsating white dwarfs: new
insights}

\item {} 
\sphinxhref{https://arxiv.org/abs/1906.12262}{Asteroseismology of solar-type
stars}

\item {} 
\sphinxhref{https://arxiv.org/abs/1905.13036}{An observer’s view on the future of
asteroseismology}

\item {} 
\sphinxhref{https://arxiv.org/abs/1812.06728}{The role of magnetic fields in the formation of protostellar
discs}

\end{itemize}


\subsection{Extragalactic Astrophysics and Cosmology}
\label{\detokenize{astro_reading:extragalactic-astrophysics-and-cosmology}}

\subsubsection{Constraining Cosmology using Galaxy Clusters}
\label{\detokenize{resource/astro/reference/cluster_cosmology:constraining-cosmology-using-galaxy-clusters}}\label{\detokenize{resource/astro/reference/cluster_cosmology::doc}}

\paragraph{Reviews}
\label{\detokenize{resource/astro/reference/cluster_cosmology:reviews}}
\begin{figure}[htbp]
\centering
\capstart

\noindent\sphinxincludegraphics{{Allen2011_1}.png}
\caption{Allen2011}\label{\detokenize{resource/astro/reference/cluster_cosmology:id1}}\end{figure}
\begin{itemize}
\item {} 
\sphinxhref{http://adsabs.harvard.edu/abs/2011ARA\%26A..49..409A}{Cosmological Parameters from Observations of Galaxy Clusters - ARA\&A
review by Allen, Evrard \& Mantz
2011}

\item {} 
\sphinxhref{http://adsabs.harvard.edu/abs/2015APh....63...23H}{Huterer et al. 2015 - Growth of cosmic structure: Probing dark
energy beyond
expansion}
\begin{itemize}
\item {} 
See Section 4.2: Cluster Abundance

\item {} 
The primary importance of clusters in the context of dark energy
is their complementarity to geometric probes, i.e. their \sphinxstylestrong{ability
to distinguish between modified gravity and dark energy models
with degenerate expansion histories}. Galaxy clusters are
statistically competitive with and often better than other probes.
(Fig 5, right panel, the “sweet spot” is z=0.3 - 0.8.

\item {} 
The basic physics behind cluster abundances as a cosmological
probe are \sphinxstylestrong{conceptually simple}; and A cluster abundance
experiment is \sphinxstylestrong{conceptually very simple}.

\item {} 
The dependence of the number of galaxy clusters on the \sphinxstylestrong{variance
of the linear density field} that allows us to utilize galaxy
clusters to constrain the growth of structure.

\item {} 
Future surveys will almost certainly rely on \sphinxstylestrong{weak lensing mass
calibration} to estimate cluster masses.

\item {} 
Other usages of clusters as a cosmological tool:
\begin{itemize}
\item {} 
Galaxy clusters can probe dark energy in other ways as well,
most notably by \sphinxstylestrong{comparing cluster mass estimates from weak
lensing and dynamical methods such as galaxy velocity
dispersions}

\item {} 
Because the growth of structure is also impacted by non-zero
neutrino mass, galaxy cluster abundances can provide
\sphinxstylestrong{competitive constraints on the sum of neutrino masses}.

\end{itemize}

\item {} 
Difficulties and limitations:
\begin{itemize}
\item {} 
To achieve the goal of DES Stage III (IV) requirement, we must
be able to measure cluster masses with \sphinxstylestrong{5\% (2\%) precision}.
See \sphinxhref{http://adsabs.harvard.edu/abs/2014MNRAS.439...48A}{Applegate et al.
(2014)}
(Currently can achieve 7\%, but with 20\% systematic differences
among methods).

\item {} 
Systematic errors in shear measurements tend to be less
critical for cluster abundance work than for cosmic shear work
(only care about circular averaged tangential shear).

\item {} 
Key systematic is the \sphinxstylestrong{calibration not only of the mean
relation between cluster observables (optical, X-ray, or mm
signals) and cluster mass, but also the scatter (shape and
amplitude) about the mean.}

\item {} 
\sphinxstylestrong{Cluster centering} remains an important systematic in
optical and/or low resolution experiments (e.g. Planck). (See
\sphinxhref{http://adsabs.harvard.edu/abs/2011PhRvD..83b3008O}{Oguri \& Takada 2011 about
self-calibration}).

\item {} 
Optical observations benefit from a lower mass detection
threshold than X-ray/mm over a large redshift range, which in
turn result in improved statistical constraints.

\item {} 
\sphinxstylestrong{The synergistic nature of multi-wavelength cluster cosmology
will necessarily play a key role in future cluster abundance
experiments.} A balanced multi-wavelength approach will be
critical to the success of cluster cosmology over the next
10\textendash{}20 years.

\item {} 
\sphinxstylestrong{Self-calibration}: the cluster-clustering signal is itself
an observable that one can use to calibrate cluster masses, and
which is insensitive to all of the above systematic effects.

\end{itemize}

\end{itemize}

\item {} 
\sphinxhref{http://adsabs.harvard.edu/abs/2018RPPh...81a6901H}{Huterer \& Shafer 2018 - Dark energy two decades after: observables,
probes, consistency
tests}
\begin{itemize}
\item {} 
See Section 5.5 Galaxy Clusters

\item {} 
Clusters are \sphinxstylestrong{versatile probes of cosmology and astrophysics}
and have had an important role in the development of modern
cosmology

\item {} 
Recent cluster observations typically do not have enough
signal-to-noise to determine the cluster masses directly; instead,
\sphinxstylestrong{forward-modeling} can be applied to the mass function to recast
the theory in the space of observable quantities (e.g. see Evrard
et al. 2014).

\item {} 
The mass function’s near-exponential dependence on the power
spectrum in the high-mass limit is at the root of the power of
clusters to probe the growth of density fluctuations.

\item {} 
(The clusters’) two-point correlation function probes the matter
power spectrum as well as the growth and geometry factors
sensitive to dark energy.

\item {} 
Clusters can also be correlated with background galaxies to probe
the growth (Cluster-Galaxy lensing, see \sphinxhref{http://adsabs.harvard.edu/abs/2011PhRvD..83b3008O}{Oguri \& Takada
2011}

\item {} 
The most important uncertainty is typically tied to parameters
that describe the scaling relations between mass and observable
properties of the cluster (e.g. Flux, temperature).

\end{itemize}

\item {} 
\sphinxhref{http://adsabs.harvard.edu/abs/2013PhR...530...87W}{Weinberg 2013, PhR - Observational probes of cosmic
acceleration}
\begin{itemize}
\item {} 
See Section 6: Clusters of galaxies

\end{itemize}

\end{itemize}


\paragraph{Key Papers}
\label{\detokenize{resource/astro/reference/cluster_cosmology:key-papers}}

\subparagraph{Theoretical}
\label{\detokenize{resource/astro/reference/cluster_cosmology:theoretical}}

\subparagraph{Predictions}
\label{\detokenize{resource/astro/reference/cluster_cosmology:predictions}}\begin{itemize}
\item {} 
\sphinxhref{http://adsabs.harvard.edu/abs/2018MNRAS.481..613P}{Pillepich et al. 2018 - Forecasts on dark energy from the X-ray
cluster survey with eROSITA: constraints from counts and
clustering}
\begin{itemize}
\item {} 
Fisher information is extracted from the number density and
spatial clustering of a photon-count-limited sample of clusters of
galaxies up to z ˜ 2. We consider different scenarios for the
availability of (i) X-ray follow-up observations, (ii) photometric
and spectroscopic redshifts, and (iii) accurate knowledge of the
observable - mass relation down to the scale of galaxy groups, but
no additional observation-related systematics are taken into
account.

\end{itemize}

\end{itemize}


\subparagraph{Early stage}
\label{\detokenize{resource/astro/reference/cluster_cosmology:early-stage}}\begin{itemize}
\item {} 
\sphinxhref{http://adsabs.harvard.edu/abs/1998ApJ...508..483W}{Wang \& Steinhardt 1998 - Cluster Abundance Constraints for
Cosmological Models with a Time-varying, Spatially Inhomogeneous
Energy Component with Negative
Pressure}

\item {} 
\sphinxhref{http://adsabs.harvard.edu/abs/2001ApJ...553..545H}{Haiman, Mohr \& Holder 2001, ApJ - Constraints on Cosmological
Parameters from Future Galaxy Cluster
Surveys}
\begin{itemize}
\item {} 
\sphinxstylestrong{Important}: “Our results indicate a formal statistical
uncertainty of \textasciitilde{}3\% (68\% confidence) on both Ωm and w when the SZE
survey is combined with either the CMB or SN data; a large number
of clusters in the X-ray survey further suppresses the degeneracy
between w and both Ωm and h.”

\end{itemize}

\item {} 
\sphinxhref{http://adsabs.harvard.edu/abs/2001ApJ...550..547F}{Fan \& Chiueh 2001 - Determining the Geometry and the Cosmological
Parameters of the Universe through Sunyaev-Zeldovich Effect Cluster
Counts}

\item {} 
\sphinxhref{http://adsabs.harvard.edu/abs/2001ApJ...560L.111H}{Holder, Haiman \& Mohr 2001 - Constraints on Ωm, ΩΛ, and σ8 from
Galaxy Cluster Redshift
Distributions}

\item {} 
\sphinxhref{http://adsabs.harvard.edu/abs/2002PASP..114...29N}{Newman et al. 2002 - Measuring the Cosmic Equation of State with
Galaxy Clusters in the DEEP2 Redshift
Survey}

\item {} 
\sphinxhref{http://adsabs.harvard.edu/abs/2002A\%26A...390....1R}{Refregier et al. 2002 - Cosmology with galaxy clusters in the XMM
large-scale structure
survey}

\item {} 
\sphinxhref{http://adsabs.harvard.edu/abs/2002ApJ...577..569L}{Levine et al. 2002 - Future Galaxy Cluster Surveys: The Effect of
Theory Uncertainty on Constraining Cosmological
Parameters}

\end{itemize}


\subparagraph{Systematic, (Self-)Calibration}
\label{\detokenize{resource/astro/reference/cluster_cosmology:systematic-self-calibration}}\begin{itemize}
\item {} 
\sphinxhref{http://adsabs.harvard.edu/abs/2004PhRvD..70d3504L}{Lima \& Hu 2004 - Self-calibration of cluster dark energy studies:
Counts in
cells}
\begin{itemize}
\item {} 
\sphinxstylestrong{Important} Self-calibration (Using noise as signal!): \sphinxstylestrong{The
excess variance of counts due to the clustering of clusters}
provides such an opportunity and can be measured from the survey
without additional observational cost.

\end{itemize}

\item {} 
\sphinxhref{http://adsabs.harvard.edu/abs/2005PhRvD..72d3006L}{Lima \& Hu 2005 - Self-calibration of cluster dark energy studies:
Observable-mass
distribution}
\begin{itemize}
\item {} 
Given the shape of the actual mass function, the properties of the
distribution may be internally monitored by the shape of the
observable mass function.

\end{itemize}

\item {} 
\sphinxhref{http://adsabs.harvard.edu/abs/2006PhRvD..73f7301H}{Hu \& Cohn 2006 - Likelihood methods for cluster dark energy
surveys}

\item {} 
\sphinxhref{http://adsabs.harvard.edu/abs/2007PhRvD..76l3013L}{Lima \& Hu 2007 - Photometric redshift requirements for
self-calibration of cluster dark energy
studies}
\begin{itemize}
\item {} 
Self-calibration in combination with external mass inferences
helps reduce photo-z requirements and provides important
consistency checks for future cluster surveys.

\end{itemize}

\item {} 
\sphinxhref{http://adsabs.harvard.edu/abs/2003ApJ...585..603M}{Majumdar \& Mohr 2003 - Importance of Cluster Structural Evolution in
Using X-Ray and Sunyaev-Zeldovich Effect Galaxy Cluster Surveys to
Study Dark
Energy}
\begin{itemize}
\item {} 
We show that for a particular X-ray survey (Sunyaev-Zeldovich
effect {[}SZE{]} survey), the constraints on w degrade by roughly a
factor of 3 (factor of 2) when one accounts for the possibility of
nonstandard cluster evolution.

\end{itemize}

\item {} 
\sphinxhref{http://adsabs.harvard.edu/abs/2004ApJ...613...41M}{Majumdar \& Mohr 2004 - Self-Calibration in Cluster Studies of Dark
Energy: Combining the Cluster Redshift Distribution, the Power
Spectrum, and Mass
Measurements}
\begin{itemize}
\item {} 
The best constraints are obtained when one combines both the power
spectrum constraints and the mass measurements with the cluster
redshift distribution; when using the survey to extract the
parameters and evolution of the mass-observable relations, we
estimate uncertainties on w of \textasciitilde{}4\%-6\%

\end{itemize}

\item {} 
\sphinxhref{http://adsabs.harvard.edu/abs/2003PhRvD..67h1304H}{Hu 2003 - Self-consistency and calibration of cluster number count
surveys for dark
energy}
\begin{itemize}
\item {} 
“we find that the ambiguity from the normalization of the
mass-observable relationships, or an extrapolation of external
mass-observable determinations from higher masses, can be largely
eliminated with a sufficiently deep survey, even allowing for an
arbitrary evolution”

\end{itemize}

\item {} 
\sphinxhref{http://adsabs.harvard.edu/abs/2008ApJ...688..729W}{Wu, Rozo \& Wechsler 2008 - The Effects of Halo Assembly Bias on
Self-Calibration in Galaxy Cluster
Surveys}
\begin{itemize}
\item {} 
\sphinxstylestrong{Halo assembly bias}: the clustering amplitude of halos depends
not only on the halo mass, but also on various secondary
variables.

\item {} 
The impact of the secondary dependence is determined by (1) the
scatter in the observable-mass relation and (2) the correlation
between observable and secondary variables. \sphinxstylestrong{Could be important
to DES and LSST like survey}

\end{itemize}

\item {} 
\sphinxhref{http://adsabs.harvard.edu/abs/2009PhRvD..79f3009C}{Cunha 2009 - Cross-calibration of cluster mass
observables}
\begin{itemize}
\item {} 
We use a \sphinxstylestrong{Fisher matrix analysis} to study the improvements in
the joint dark energy and cluster mass-observables constraints
resulting from combining cluster counts and clustering abundances
measured with different techniques.

\item {} 
The \sphinxstylestrong{cross-calibrated constraints} are less sensitive to
variations in the mass threshold or maximum redshift range.

\end{itemize}

\item {} 
\sphinxhref{http://adsabs.harvard.edu/abs/2009PhRvD..80f3532C}{Cunha, Huterer \& Frieman 2009 - Constraining dark energy with
clusters: Complementarity with other
probes}
\begin{itemize}
\item {} 
We find that optimally combined optical and Sunyaev-Zeldovich
effect cluster surveys should improve the Dark Energy Task Force
figure of merit

\end{itemize}

\item {} 
\sphinxhref{http://adsabs.harvard.edu/abs/2010ApJ...713.1207W}{Wu, Rozo \& Wechsler 2010 - Annealing a Follow-up Program:
Improvement of the Dark Energy Figure of Merit for Optical Galaxy
Cluster
Surveys}
\begin{itemize}
\item {} 
Considering clusters selected from optical imaging in the Dark
Energy Survey, we find that approximately 200 low-redshift X-ray
clusters or massive Sunyaev-Zel’dovich clusters can improve the
dark energy figure of merit by 50\%, provided that the follow-up
mass measurements involve no systematic error.

\item {} 
The scatter in the optical richness\textendash{}mass distribution, which needs
to be made as tight as possible to improve the efficacy of
follow-up observations

\end{itemize}

\item {} 
\sphinxhref{http://adsabs.harvard.edu/abs/2011PhRvD..83b3008O}{Oguri \& Takada 2011, PhRvD - Combining cluster observables and
stacked weak lensing to probe dark energy: Self-calibration of
systematic
uncertainties}

\item {} 
\sphinxhref{http://adsabs.harvard.edu/abs/2011ApJ...735..118R}{Rozo et al. 2011, ApJ - Stacked Weak Lensing Mass Calibration:
Estimators, Systematics, and Impact on Cosmological Parameter
Constraints}

\item {} 
\sphinxhref{http://adsabs.harvard.edu/abs/2014MNRAS.441.3562E}{Evrard, Arnault, Huterer \& Farahi 2014 - A model for multiproperty
galaxy cluster
statistics}
\begin{itemize}
\item {} 
We derive closed-form expressions for the space density of haloes
as a function of multiple observables as well as forms for the
low-order moments of properties of observable-selected samples.

\end{itemize}

\end{itemize}


\subparagraph{Other issues}
\label{\detokenize{resource/astro/reference/cluster_cosmology:other-issues}}\begin{itemize}
\item {} 
\sphinxhref{http://adsabs.harvard.edu/abs/2007NJPh....9..446T}{Takada \& Bridle 2007, NJPh - Probing dark energy with cluster counts
and cosmic shear power spectra: including the full
covariance}

\item {} 
\sphinxhref{http://adsabs.harvard.edu/abs/2012PhRvD..85f3521I}{Ichiki \& Takada 2012, PhRvD - Impact of massive neutrinos on the
abundance of massive
clusters}

\item {} 
\sphinxhref{http://adsabs.harvard.edu/abs/2014MNRAS.441.2456T}{Takada \& Spergel 2014, MNRAS - Joint analysis of cluster number
counts and weak lensing power spectrum to correct for the
super-sample
covariance}

\end{itemize}


\subparagraph{Observational}
\label{\detokenize{resource/astro/reference/cluster_cosmology:observational}}\begin{itemize}
\item {} 
\sphinxstylestrong{Chandra Cluster Cosmology Project}
\begin{itemize}
\item {} 
\sphinxhref{http://adsabs.harvard.edu/abs/2009ApJ...692.1033V}{Vilhlinin et al. 2009a - Chandra Cluster Cosmology Project. II.
Samples and X-Ray Data
Reduction}

\item {} 
\sphinxhref{http://adsabs.harvard.edu/abs/2009ApJ...692.1060V}{Vilhlinin et al. 2009b - Chandra Cluster Cosmology Project III:
Cosmological Parameter
Constraints}
\begin{itemize}
\item {} 
\sphinxstylestrong{37} Chandra clusters at = 0.55 from ROSAT and \sphinxstylestrong{49}
brightest z=0.05 clusters

\end{itemize}

\end{itemize}

\item {} 
\sphinxstylestrong{The observed growth of massive galaxy clusters using
ROSAT/Chandra}
\begin{itemize}
\item {} 
\sphinxhref{http://adsabs.harvard.edu/abs/2010MNRAS.406.1759M}{Mantz, Allen, Rapetti \& Ebeling 2010a - I. Statistical methods
and cosmological
constraints}
\begin{itemize}
\item {} 
\sphinxstylestrong{238} clusters from RASS; \sphinxstylestrong{94} Chandra follow-up.

\end{itemize}

\item {} 
\sphinxhref{http://adsabs.harvard.edu/abs/2010MNRAS.406.1773M}{Mantz, Allen, Ebeling, Rapetti \& Drlica-Wagner 2010 - II. X-ray
scaling
relations}

\item {} 
\sphinxhref{http://adsabs.harvard.edu/abs/2010MNRAS.406.1796R}{Rapetti, Allen, Mantz \& Ebeling 2010 - III. Testing general
relativity on cosmological
scales}

\item {} 
\sphinxhref{http://adsabs.harvard.edu/abs/2010MNRAS.406.1805M}{Mantz, Allen \& Rapetti 2010 - IV. Robust constraints on neutrino
properties}

\end{itemize}

\item {} 
\sphinxstylestrong{maxBCG clusters}
\begin{itemize}
\item {} 
\sphinxhref{http://adsabs.harvard.edu/abs/2010ApJ...708..645R}{Rozo et al. 2010 - Cosmological Constraints from the Sloan
Digital Sky Survey maxBCG Cluster
Catalog}
\begin{itemize}
\item {} 
\sphinxstylestrong{SDSS-maxBCG}: fully consistent with the WMAP five-year data,
and in a joint analysis we find σ8 = 0.807 \(\pm\) 0.020 and Ωm =
0.265 \(\pm\) 0.016

\end{itemize}

\item {} 
\sphinxhref{http://adsabs.harvard.edu/abs/2014MNRAS.439.1628Z}{Zu et al. 2014, MNRAS - Cosmological constraints from the
large-scale weak lensing of SDSS MaxBCG
clusters}

\end{itemize}

\item {} 
\sphinxhref{http://adsabs.harvard.edu/abs/2012ApJ...745...16T}{Tinker et al. 2012 - Cosmological Constraints from Galaxy Clustering
and the Mass-to-number Ratio of Galaxy
Clusters}
\begin{itemize}
\item {} 
\sphinxstylestrong{SDSS 2PCF + mass-to-galaxy number ratio within cluster}

\end{itemize}

\item {} 
\sphinxstylestrong{Cosmology and astrophysics from relaxed galaxy clusters in Chandra
\& ROSAT}
\begin{itemize}
\item {} 
\sphinxhref{http://adsabs.harvard.edu/abs/2015MNRAS.449..199M}{Mantz et al. 2015 - I. Sample
selection}

\item {} 
\sphinxhref{http://adsabs.harvard.edu/abs/2014MNRAS.440.2077M}{Mantz et al. 2014 - II. Cosmological
constraints}

\item {} 
\sphinxhref{http://adsabs.harvard.edu/abs/2016MNRAS.456.4020M}{Mantz et al. 2016 - III. Thermodynamic profiles and scaling
relations}

\item {} 
\sphinxhref{http://adsabs.harvard.edu/abs/2016MNRAS.457.1522A}{Applegate et al. 2016 - IV. Robustly calibrating hydrostatic
masses with weak
lensing}

\item {} 
\sphinxhref{http://adsabs.harvard.edu/abs/2016MNRAS.462..681M}{Mantz et al. 2016 - V. Consistency with cold dark matter
structure
formation}

\end{itemize}

\item {} 
\sphinxhref{http://adsabs.harvard.edu/abs/2016ApJ...832...95D}{de Haan et al. 2016 - Cosmological Constraints from Galaxy Clusters
in the 2500 Square-degree SPT-SZ
Survey}
\begin{itemize}
\item {} 
\sphinxstylestrong{377} clusters at z\textgreater{}0.2 from 2500 square-degree South Pole
Telescope SZ survey

\end{itemize}

\end{itemize}


\subparagraph{Cluster mass calibration and scaling relations}
\label{\detokenize{resource/astro/reference/cluster_cosmology:cluster-mass-calibration-and-scaling-relations}}\begin{itemize}
\item {} 
\sphinxhref{http://adsabs.harvard.edu/abs/2011ApJ...740...25B}{Becker \& Kravtsov 2011 - On the Accuracy of Weak-lensing Cluster
Mass
Reconstructions}
\begin{itemize}
\item {} 
\sphinxstylestrong{Important}: We find that correlated large-scale structure
within several virial radii of clusters contributes a smaller, but
nevertheless significant, amount to the scatter. The intrinsic
scatter due to these physical sources is ≈20\% for massive clusters
and can be as high as ≈30\% for group-sized systems.

\item {} 
We find that WL mass measurements can have a small, ≈5\%-10\%, but
non-negligible amount of bias.

\end{itemize}

\item {} 
\sphinxhref{http://adsabs.harvard.edu/abs/2012NJPh...14e5018R}{Rasia et al. 2012 - Lensing and x-ray mass estimates of clusters
(simulations)}
\begin{itemize}
\item {} 
We confirm previous results showing that lensing masses obtained
from the fit of the cluster tangential shear profiles with
Navarro-Frenk-White functionals are \sphinxstylestrong{biased low by ˜5-10\% with a
large scatter (˜10-25\%)}

\end{itemize}

\item {} 
\sphinxhref{http://adsabs.harvard.edu/abs/2014MNRAS.438...78R}{Rozo, Bartlett, Evrard \& Rykoff 2014 - Closing the loop: a
self-consistent model of optical, X-ray and Sunyaev-Zel’dovich
scaling relations for clusters of
Galaxies}
\begin{itemize}
\item {} 
We find that scaling relations derived from optical and X-ray
selected cluster samples are consistent with one another. These
cluster scaling relations satisfy several non- trivial spatial
abundance constraints and closure relations.

\end{itemize}

\end{itemize}


\subparagraph{Using Velocity Distribution Function}
\label{\detokenize{resource/astro/reference/cluster_cosmology:using-velocity-distribution-function}}\begin{itemize}
\item {} 
\sphinxhref{https://ui.adsabs.harvard.edu/abs/2017ApJ...835..106N/abstract}{The Velocity Distribution Function of Galaxy Clusters as a
Cosmological
Probe}

\item {} 
\sphinxhref{https://arxiv.org/abs/1906.07729}{Cluster Cosmology with the Velocity Distribution Function of the
HeCS-SZ Sample}

\item {} 
\sphinxstylestrong{On the Cluster Physics of Sunyaev-Zeldovich and X-Ray Surveys}
\begin{itemize}
\item {} 
\sphinxhref{http://adsabs.harvard.edu/abs/2012ApJ...758...74B}{Battaglia, Bond, Pfommer \& Sievers 2012a - I. The Influence of
Feedback, Non-thermal Pressure, and Cluster Shapes on Y-M Scaling
Relations}

\item {} 
\sphinxhref{http://adsabs.harvard.edu/abs/2012ApJ...758...75B}{Battaglia, Bond, Pfommer \& Sievers 2012b - II. Deconstructing the
Thermal SZ Power
Spectrum}

\item {} 
\sphinxhref{http://adsabs.harvard.edu/abs/2013ApJ...777..123B}{Battaglia, Bond, Pfommer \& Sievers 2013 - III. Measurement Biases
and Cosmological Evolution of Gas and Stellar Mass
Fractions}

\end{itemize}

\item {} 
\sphinxhref{http://adsabs.harvard.edu/abs/2017A\%26A...604A..89P}{Penna-Lima et al. 2017 - Calibrating the Planck cluster mass scale
with CLASH}
\begin{itemize}
\item {} 
\sphinxstylestrong{1 - b\_sz = 0.73 +/- 0.10}

\end{itemize}

\end{itemize}


\paragraph{Lectures and Conferences}
\label{\detokenize{resource/astro/reference/cluster_cosmology:lectures-and-conferences}}\begin{itemize}
\item {} 
\sphinxhref{http://cosmology.lbl.gov/talks/Rozo\_13.pdf}{Ushering in DES Cluster Cosmology with redMaPPer by Eduardo
Rozo}
\begin{itemize}
\item {} 
Galaxy clusters are the most massive gravitationally bound
structures in the Universe.

\item {} 
Number of galaxy clusters as a function of halo mass measures the
amount of structure in the Universe (sigma\_8).

\item {} 
Optical selection allows detection of low mass systems; more
abundant == better weak lensing halo mass == Better cosmology.
Finding clusters in the optical maximizes the cosmological
information that can be drawn from clusters.

\item {} 
Centering cluster is hard!

\end{itemize}

\item {} 
\sphinxhref{https://cmb-s4.org/wiki/index.php/SLAC-2017:Clusters}{SLAC-2017 Conference on Cluster
Cosmology}

\item {} 
\sphinxhref{http://www.star.bris.ac.uk/bjm/lectures/cluster-cosmology/}{Cosmology with Clusters of Galaxies by Ben Maughan (Undergraduate
Level)}

\item {} 
\sphinxhref{http://online.kitp.ucsb.edu/online/gclusters\_c11/}{KITP Conference: Astrophysics and Cosmology with Galaxy
Clusters}

\end{itemize}


\paragraph{Important References}
\label{\detokenize{resource/astro/reference/cluster_cosmology:important-references}}\begin{itemize}
\item {} 
Galaxy clusters have been recognized as powerful cosmological probes
\begin{itemize}
\item {} 
Henry et al. 2009; Vikhlinin et al. 2009; Mantz et al. 2010; Rozo
et al. 2010; Clerc et al. 2012; Benson et al. 2013; Hasselfield et
al. 2013).

\end{itemize}

\item {} 
Early optical cluster finders can be divided into roughly two classes
\begin{enumerate}
\def\theenumi{\arabic{enumi}}
\def\labelenumi{\theenumi .}
\makeatletter\def\p@enumii{\p@enumi \theenumi .}\makeatother
\item {} 
Those based on photometric redshifts
\begin{itemize}
\item {} 
Kepner et al. 1999; van Breukelen \& Clewley 2009; Milkeraitis
et al. 2010; Durret et al. 2011; Szabo et al. 2011;
Soares-Santos et al. 2011; Wen et al. 2012

\end{itemize}

\item {} 
Those utilizing the cluster red sequence
\begin{itemize}
\item {} 
Annis et al. 1999; Gladders \& Yee 2000; Koester et al. 2007a;
Gladders et al. 2007; Gal et al. 2009; Thanjavur et al. 2009;
Hao et al. 2010; Murphy et al. 2012

\end{itemize}

\end{enumerate}

\end{itemize}


\subsubsection{Splashback Radius of Dark Matter Halo}
\label{\detokenize{resource/astro/reference/halo_splashback_radius:splashback-radius-of-dark-matter-halo}}\label{\detokenize{resource/astro/reference/halo_splashback_radius::doc}}

\paragraph{Introduction}
\label{\detokenize{resource/astro/reference/halo_splashback_radius:introduction}}\begin{itemize}
\item {} 
Read the introduction by \sphinxhref{http://www.benediktdiemer.com/research/splashback/}{Benedikt Diemer on his
webpage}: \textgreater{}
“…A more natural halo boundary is provided by the splashback radius,
\(R_{\rm sp}\), the radius where particles reach the apocenter of
their first orbit after infall…Particles at the apocenter of their
first orbit pile up due to their low radial velocity, creating a
caustic that manifests itself as a sharp drop in the density profile.
This so-called splashback radius represents a clear boundary between
matter orbiting in the halo and matter on a first infall toward the
halo.”

\end{itemize}

\sphinxincludegraphics{{04f3d3d255a50bdd5804cec16a71527233747b8b}.png}


\paragraph{Reviews}
\label{\detokenize{resource/astro/reference/halo_splashback_radius:reviews}}\begin{itemize}
\item {} 
\sphinxhref{https://arxiv.org/pdf/1810.00890.pdf}{Walker et al. 2018 - The physics of galaxy cluster
outskirts}
\begin{itemize}
\item {} 
See Section 1.1 and Figure 1.

\end{itemize}

\end{itemize}


\paragraph{Theoretical consideration}
\label{\detokenize{resource/astro/reference/halo_splashback_radius:theoretical-consideration}}\begin{itemize}
\item {} 
\sphinxhref{http://adsabs.harvard.edu/abs/2014ApJ...789....1D}{Dimmer \& Kravtsov 2014 - Dependence of the Outer Density Profiles of
Halos on Their Mass Accretion
Rate}
\begin{itemize}
\item {} 
\sphinxstylestrong{Important}

\end{itemize}

\item {} 
\sphinxhref{http://adsabs.harvard.edu/abs/2014JCAP...11..019A}{Adhikari, Dalal \& Chamberlain 2014 - Splashback in accreting dark
matter halos}

\item {} 
\sphinxhref{http://adsabs.harvard.edu/abs/2015ApJ...810...36M}{More, Diemer \& Kravtsov - The Splashback Radius as a Physical Halo
Boundary and the Growth of Halo
Mass}

\item {} 
\sphinxhref{http://adsabs.harvard.edu/abs/2016MNRAS.459.3711S}{Shi 2016 - The outer profile of dark matter haloes: an analytical
approach}

\item {} 
\sphinxhref{http://adsabs.harvard.edu/abs/2017ApJS..231....5D}{Diemer 2017 - The Splashback Radius of Halos from Particle Dynamics.
I. The SPARTA
Algorithm}

\item {} 
\sphinxhref{http://adsabs.harvard.edu/abs/2017ApJ...843..140D}{Diemer et al. 2017 - The Splashback Radius of Halos from Particle
Dynamics. II. Dependence on Mass, Accretion Rate, Redshift, and
Cosmology}
\begin{itemize}
\item {} 
\sphinxstylestrong{Important}

\end{itemize}

\item {} 
\sphinxhref{http://adsabs.harvard.edu/abs/2017ApJ...841...34M}{Mansfield, Kravtsov \& Diemer 2017 - Splashback Shells of Cold Dark
Matter Halos}
\begin{itemize}
\item {} 
\sphinxhref{https://github.com/phil-mansfield/shellfish}{Shellfish: SHELL Finding In Spheroidal
Halos}

\end{itemize}

\item {} 
\sphinxhref{http://adsabs.harvard.edu/abs/2017MNRAS.470.4767B}{Busch \& White 2017 - Assembly bias and splashback in galaxy
clusters}
\begin{itemize}
\item {} 
\sphinxstylestrong{Important}

\end{itemize}

\item {} 
\sphinxhref{http://adsabs.harvard.edu/abs/2017MNRAS.472.2694S}{Snaith et al. 2017 - Haloes at the ragged edge: the importance of
the splashback
radius}

\item {} 
\sphinxhref{http://adsabs.harvard.edu/abs/2018arXiv180604302A}{Adhikari et al. 2018 - Splashback in galaxy clusters as a probe of
cosmic expansion and
gravity}

\item {} 
\sphinxhref{http://adsabs.harvard.edu/abs/2018PhRvD..98b3523O}{Okumura et al. 2018 - Splashback radius of nonspherical dark matter
halos from cosmic density and velocity
fields}

\item {} 
\sphinxhref{http://adsabs.harvard.edu/abs/2018MNRAS.479.1100R}{Renneby, Hilbert \& Angulo 2018 - Halo mass and weak galaxy-galaxy
lensing profiles in rescaled cosmological N-body
simulations}

\end{itemize}


\paragraph{Observational constraints}
\label{\detokenize{resource/astro/reference/halo_splashback_radius:observational-constraints}}
\sphinxincludegraphics{{chang2018_1}.png} (Figure from Chang et al. 2018. Claimed detection of
splashback radius around redMaPPer clusters using DES)
\begin{itemize}
\item {} 
\sphinxhref{http://adsabs.harvard.edu/abs/2015PASJ...67..103N}{Niikura et al. 2015 - Detection of universality of dark matter
profile from Subaru weak lensing measurements of 50 massive
clusters}
\begin{itemize}
\item {} 
No detection of splashback signature.

\end{itemize}

\item {} 
\sphinxhref{http://adsabs.harvard.edu/abs/2016ApJ...825...39M}{More et al. 2016 - Detection of the Splashback Radius and Halo
Assembly Bias of Massive Galaxy
Clusters}
\begin{itemize}
\item {} 
Claim detection: “We show that the \sphinxstylestrong{projected number density
profiles of Sloan Digital Sky Survey photometric galaxies around
galaxy clusters} display strong evidence for the splashback
radius, a sharp halo edge corresponding to the location of the
first orbital apocenter of satellite galaxies after their infall.”

\end{itemize}

\item {} 
\sphinxhref{http://adsabs.harvard.edu/abs/2017ApJ...836..231U}{Umetsu \& Diemer 2017 - Lensing Constraints on the Mass Profile Shape
and the Splashback Radius of Galaxy
Clusters}
\begin{itemize}
\item {} 
Place lower limit.

\end{itemize}

\item {} 
\sphinxhref{http://adsabs.harvard.edu/abs/2017ApJ...841...18B}{Baxter et al. 2017 - The Halo Boundary of Galaxy Clusters in the
SDSS}
\begin{itemize}
\item {} 
Unclear, results are model sensitive

\end{itemize}

\item {} 
\sphinxhref{http://adsabs.harvard.edu/abs/2018PASJ...70S..24N}{Nishizawa et al. 2018 - First results on the cluster galaxy
population from the Subaru Hyper Suprime-Cam survey. II. Faint end
color-magnitude diagrams and radial profiles of red and blue galaxies
at 0.1 \textless{} z \textless{}
1.1}
\begin{itemize}
\item {} 
Detection using galaxy number density profiles (Fig 6)

\end{itemize}

\item {} 
\sphinxhref{http://adsabs.harvard.edu/abs/2018ApJ...864...83C}{Chang et al. 2018 - The Splashback Feature around DES Galaxy
Clusters: Galaxy Density and Weak Lensing
Profiles}
\begin{itemize}
\item {} 
\sphinxstylestrong{Important} Claim detection: “we find strong evidence of a
splashback-like steepening of the galaxy density profile”

\end{itemize}

\item {} 
\sphinxhref{https://arxiv.org/abs/1809.10045}{Contigiani, Hoekstra \& Bahe 2018 - Weak lensing constraints on
splashback around massive
clusters}
\begin{itemize}
\item {} 
Unclear, “We do not detect a significant steepening”, but gives a
constraint.

\end{itemize}

\item {} 
\sphinxhref{https://arxiv.org/abs/1811.06081}{Shin et al. 2018 arXiv:1811 - Measurement of the Splashback Feature
around SZ-selected Galaxy Clusters with DES, SPT and
ACT}
\begin{itemize}
\item {} 
\sphinxstylestrong{Detection} using galaxy number density profile around SZ
selected clusters

\item {} 
\sphinxstylestrong{R\_sp} consistent with prediction from simulation, in different
with optically selected clusters

\item {} 
Blue galaxies show just power law profile without splashback
signiture, consistent with first infall

\end{itemize}

\end{itemize}


\paragraph{About accretion rate}
\label{\detokenize{resource/astro/reference/halo_splashback_radius:about-accretion-rate}}\begin{itemize}
\item {} 
\sphinxhref{http://adsabs.harvard.edu/abs/2016ApJ...818..188D}{De Boni et al. 2016 - The Mass Accretion Rate of Galaxy Clusters: A
Measurable
Quantity}

\item {} 
\sphinxhref{http://adsabs.harvard.edu/abs/2016Galax...4...79D}{De Boni 2016 - Using the Outskirts of Galaxy Clusters to Determine
Their Mass Accretion
Rate}

\end{itemize}


\paragraph{General discussion about mass density profiles}
\label{\detokenize{resource/astro/reference/halo_splashback_radius:general-discussion-about-mass-density-profiles}}\begin{itemize}
\item {} 
\sphinxhref{http://adsabs.harvard.edu/abs/2016MNRAS.461.1804S}{Shi 2016b - Locations of accretion shocks around galaxy clusters and
the ICM properties: insights from self-similar spherical collapse
with arbitrary mass accretion
rates}

\item {} 
\sphinxhref{http://adsabs.harvard.edu/abs/2017MNRAS.471L..47T}{Trevisan, Mamon \& Stalder 2017 - Group galaxy number density
profiles far out: Is the \_\_one-halo’ term NFW out to \textgreater{}10 virial
radii?}

\item {} 
\sphinxhref{http://adsabs.harvard.edu/abs/2018MNRAS.478.5366F}{Fong et al. 2018 - Prospects for determining the mass distributions
of galaxy clusters on large scales using weak gravitational
lensing}

\item {} 
\sphinxhref{http://adsabs.harvard.edu/abs/2018MNRAS.477.2141O}{Osato et al. 2018 - Strong orientation dependence of surface mass
density profiles of dark haloes at large
scales}
\begin{itemize}
\item {} 
“While the orientation dependence at small scales is ascribed to
the halo triaxiality, our results indicate even stronger
orientation dependence in the so-called two-halo regime”

\end{itemize}

\end{itemize}


\subsubsection{Intrinsic Alignment of Galaxies and Impacts on Weak Lensing}
\label{\detokenize{resource/astro/reference/wl_intrinsic_alignment:intrinsic-alignment-of-galaxies-and-impacts-on-weak-lensing}}\label{\detokenize{resource/astro/reference/wl_intrinsic_alignment::doc}}

\paragraph{Key Points}
\label{\detokenize{resource/astro/reference/wl_intrinsic_alignment:key-points}}\begin{itemize}
\item {} 
Spurious (non-cosmological) correlations between galaxies, known as
intrinsic alignments (IAs) have long been known to affect cosmic
shear estimates. {[}Samuroff et al. 2018{]}

\item {} 
The particular challenge posed by IA modelling is in large part down
to the nature of the contamination: IA correlations are a real
astrophysical signal, which enters much the same angular scales as
cosmic shear itself. {[}Samuroff et al. 2018{]}

\end{itemize}


\paragraph{Reviews}
\label{\detokenize{resource/astro/reference/wl_intrinsic_alignment:reviews}}\begin{itemize}
\item {} 
\sphinxhref{http://adsabs.harvard.edu/abs/2015PhR...558....1T}{Troxel \& Ishak 2015 - The intrinsic alignment of galaxies and its
impact on weak gravitational lensing in an era of precision
cosmology}
\begin{itemize}
\item {} 
\sphinxstylestrong{Important}

\end{itemize}

\item {} 
\sphinxstylestrong{Galaxy Alignments}
\begin{itemize}
\item {} 
\sphinxhref{http://adsabs.harvard.edu/abs/2015SSRv..193....1J}{Joachimi et al. 2015 -
Overview}

\item {} 
\sphinxhref{http://adsabs.harvard.edu/abs/2015SSRv..193..139K}{Kirk et al. 2015 - Observations and Impact on
Cosmology}

\item {} 
\sphinxhref{http://adsabs.harvard.edu/abs/2015SSRv..193...67K}{Kiessling et al. 2015 - Theory, Modelling \&
Simulation}

\end{itemize}

\end{itemize}


\paragraph{Key Paper}
\label{\detokenize{resource/astro/reference/wl_intrinsic_alignment:key-paper}}\begin{itemize}
\item {} 
Alignment of the intrinsic shapes of physically close pairs of
galaxies, known as II correlations (\sphinxstylestrong{Intrinsic-Intrinsic})
\begin{itemize}
\item {} 
\sphinxhref{http://adsabs.harvard.edu/abs/2001MNRAS.320L...7C}{Catelan et al. 2001 - Intrinsic and extrinsic galaxy
alignment}
\begin{itemize}
\item {} 
\sphinxstylestrong{Tidal interaction}: Long-range correlations in the tidal
field will thus lead to long-range ellipticity-ellipticity
correlations that mimic the shear correlations arising from
weak gravitational lensing.

\end{itemize}

\item {} 
\sphinxhref{http://adsabs.harvard.edu/abs/2002MNRAS.332..788M}{Mackey et al. 2002 - Theoretical estimates of intrinsic galaxy
alignment}
\begin{itemize}
\item {} 
\sphinxstylestrong{Rotational torquing}: We present a new model relating
ellipticity to angular momentum, which can be calculated in
linear theory.

\end{itemize}

\end{itemize}

\item {} 
An often more pervasive effect comes from the fact that the same
foreground matter experiences local gravitational interactions over
short spatial scales, and also induces lensing of background
galaxies. This generates correlations in shape between foreground
galaxies and background sources (\sphinxstylestrong{Gravitational-Intrinsic})
\begin{itemize}
\item {} 
\sphinxhref{http://adsabs.harvard.edu/abs/2004PhRvD..70f3526H}{Hirata \& Seljak 2004 - Intrinsic alignment-lensing interference
as a contaminant of cosmic
shear}

\end{itemize}

\item {} 
Total IA contamination to cosmological shear can be as high as 10\% in
modern surveys
\begin{itemize}
\item {} 
\sphinxhref{http://adsabs.harvard.edu/abs/2000ApJ...545..561C}{Croft \& Metzler 2000 - Weak-Lensing Surveys and the Intrinsic
Correlation of Galaxy
Ellipticities}

\item {} 
\sphinxhref{http://adsabs.harvard.edu/abs/2012MNRAS.424.1647K}{Kirk et al. 2012 - The cosmological impact of intrinsic alignment
model choice for cosmic
shear}

\item {} 
\sphinxhref{http://adsabs.harvard.edu/abs/2016MNRAS.456..207K}{Krause et al. 2016 - The impact of intrinsic alignment on current
and future cosmic shear
surveys}

\end{itemize}

\item {} 
The IA signal, if correctly modelled, can in principle be used as a
probe of:
\begin{itemize}
\item {} 
\sphinxstylestrong{Cosmology}: \sphinxhref{http://adsabs.harvard.edu/abs/2013JCAP...12..029C}{Chisari \& Dvokin 2013 - Cosmological information
in the intrinsic alignments of luminous red
galaxies}
\begin{itemize}
\item {} 
We make forecasts for the ability of current and future
spectroscopic surveys to constrain local primordial
non-Gaussianity and Baryon Acoustic Oscillations (BAO) in the
cross-correlation function of intrinsic alignments and the
galaxy density field

\end{itemize}

\item {} 
\sphinxstylestrong{Primordial non-Gaussianity}: \sphinxhref{http://adsabs.harvard.edu/abs/2016PhRvD..94l3507C}{Chisari et al. 2016 -
Multitracing anisotropic non-Gaussianity with galaxy
shapes}
\begin{itemize}
\item {} 
We study the use of two different shape estimators as a
multitracer probe of intrinsic alignments. We show, by means of
a Fisher analysis, that this technique promises a significant
improvement on anisotropic non-Gaussianity constraints over a
single-tracer method.

\end{itemize}

\item {} 
\sphinxstylestrong{Galaxy formation}: \sphinxhref{http://adsabs.harvard.edu/abs/2018JCAP...07..030S}{Schmitz et al. 2018 - Time evolution of
intrinsic alignments of
galaxies}
\begin{itemize}
\item {} 
In particular, advection of galaxies due to peculiar velocities
alters the impact of IA, because galaxy positions when observed
are generally different from their positions at the epoch when
IA is believed to be set.

\end{itemize}

\end{itemize}

\item {} 
A number of mitigation techniques have been proposed, involving:
\begin{itemize}
\item {} 
\sphinxstylestrong{Discarding physically close pairs of galaxies}: \sphinxhref{http://adsabs.harvard.edu/abs/2001MNRAS.320L...7C}{Catelan et
al. 2001}
and \sphinxhref{http://adsabs.harvard.edu/abs/2015SSRv..193..139K}{Kirk et al. 2015 - Galaxy Alignments: Observations and Impact
on
Cosmology}

\item {} 
\sphinxstylestrong{Downweighting}:
\begin{itemize}
\item {} 
\sphinxhref{http://adsabs.harvard.edu/abs/2003A\%26A...398...23K}{King \& Schneider 2003 - Separating cosmic shear from intrinsic
galaxy alignments: Correlation function
tomography}

\item {} 
\sphinxhref{http://adsabs.harvard.edu/abs/2003MNRAS.339..711H}{Heymans \& Heavens 2003 - Weak gravitational lensing: reducing
the contamination by intrinsic
alignments}
\begin{itemize}
\item {} 
We show how distance information, either spectroscopic or
photometric redshifts, can be used to downweight nearby
pairs in an optimized way, to reduce the errors in the shear
signal arising from intrinsic alignments. Provided a
conservatively large intrinsic alignment is assumed, the
optimized weights will essentially remove all traces of
contamination

\end{itemize}

\end{itemize}

\item {} 
\sphinxstylestrong{Nulling}: \sphinxhref{http://adsabs.harvard.edu/abs/2011MNRAS.415.1681H}{Heavens \& Joachimi 2011 - Cosmic magnification:
nulling intrinsic
clustering}

\item {} 
All of these methods depend on the existence of accurate redshift
information to allow galaxies to be located relative to each other
along the line of sight. Significantly, they are also ineffective
in mitigating GI correlations {[}Samuroff et al. 2018{]}

\end{itemize}

\item {} 
It has been noted in both simulations and data that the choice of
galaxy shape estimation method can alter the magnitude of the IA
signal by an overall scale-independent factor
\begin{itemize}
\item {} 
\sphinxhref{http://adsabs.harvard.edu/abs/2016MNRAS.457.2301S}{Singh \& Mandelbaum 2016 - Intrinsic alignments of BOSS LOWZ
galaxies - II. Impact of shape measurement
methods}
\begin{itemize}
\item {} 
We observe environment dependence of ellipticity, with
brightest galaxies in groups being rounder on average compared
to satellite and field galaxies. We also study the anisotropy
in intrinsic alignments measurements introduced by projected
shapes, finding effects consistent with predictions of the
non-linear alignment model and hydrodynamic simulations.

\end{itemize}

\item {} 
\sphinxhref{http://adsabs.harvard.edu/abs/2018MNRAS.479.1412L}{Leonard \& Mandelbaum 2018 - Measuring the scale dependence of
intrinsic alignments using multiple shear
estimates}

\end{itemize}

\item {} 
Attempts to place observational constraints on the alignment
properties of galaxy samples more representative of the sort used for
cosmological lensing measurements
\begin{itemize}
\item {} 
\sphinxhref{http://adsabs.harvard.edu/abs/2012JCAP...05..041B}{Blazek et al. 2012 - Separating intrinsic alignment and
galaxy-galaxy
lensing}

\item {} 
\sphinxhref{http://adsabs.harvard.edu/abs/2018PASJ...70...41T}{Tonegawa et al. 2017 - The Subaru FMOS galaxy redshift survey
(FastSound). V. Intrinsic alignments of emission-line galaxies at
z\textasciitilde{}1.4}

\end{itemize}

\end{itemize}


\paragraph{Recent Measurements}
\label{\detokenize{resource/astro/reference/wl_intrinsic_alignment:recent-measurements}}\begin{itemize}
\item {} 
\sphinxhref{https://arxiv.org/abs/1811.06989}{Samuroff et al. 2018 - Dark Energy Survey Year 1 Results:
Constraints on Intrinsic Alignments and their Colour Dependence from
Galaxy Clustering and Weak
Lensing}

\item {} 
\sphinxstylestrong{Intrinsic alignments in redMaPPer clusters}:
\begin{itemize}
\item {} 
\sphinxhref{http://adsabs.harvard.edu/abs/2016MNRAS.463..222H}{Huang et al. 2018a - I. Central galaxy alignments and angular
segregation of
satellites}
\begin{itemize}
\item {} 
We find that central galaxies are more aligned with their
member galaxy distributions in clusters that are more elongated
and have higher richness, and for central galaxies with larger
physical size, higher luminosity and centering probability, and
redder color. Satellites with redder color, higher luminosity,
located closer to the central galaxy, and with smaller
ellipticity show a stronger angular segregation towards their
central galaxy major axes.

\end{itemize}

\item {} 
\sphinxhref{http://adsabs.harvard.edu/abs/2018MNRAS.474.4772H}{Huang et al. 2018b - II. Radial alignment of satellites towards
cluster
centres}
\begin{itemize}
\item {} 
\sphinxstylestrong{Satellites alignments (SA)}: We detect the strongest SA
signal in isophotal shapes, followed by de Vaucouleurs shapes.
While no net SA signal is detected using re-Gaussianization
shapes across the entire sample

\item {} 
We find that the measured SA signal is strongest for satellites
with the following characteristics: higher luminosity, smaller
distance to the cluster centre, rounder in shape, higher bulge
fraction, and distributed preferentially along the major axis
directions of their centrals.

\end{itemize}

\end{itemize}

\item {} 
\sphinxhref{http://adsabs.harvard.edu/abs/2017MNRAS.468.4502V}{van Uitert \& Joachimi 2017 - Intrinsic alignment of redMaPPer
clusters: cluster shape-matter density
correlation}
\begin{itemize}
\item {} 
We obtain tentative evidence that the signal increases towards
higher richness and lower redshift.

\item {} 
Comparing our results to the IA measurements of luminous red
galaxies, we find that the IA amplitude of galaxy clusters forms a
smooth extension towards a higher mass.

\end{itemize}

\end{itemize}


\paragraph{In Simulations}
\label{\detokenize{resource/astro/reference/wl_intrinsic_alignment:in-simulations}}\begin{itemize}
\item {} 
\sphinxstylestrong{Illustris and IllustrisTNG}
\begin{itemize}
\item {} 
\sphinxhref{http://adsabs.harvard.edu/abs/2017MNRAS.468..790H}{Hilbert et al. 2017 -
http://adsabs.harvard.edu/abs/2016MNRAS.457.2301S}
\begin{itemize}
\item {} 
The correlations considered include the matter
density-intrinsic ellipticity (mI), galaxy density-intrinsic
ellipticity (dI), gravitational shear-intrinsic ellipticity
(GI) and intrinsic ellipticity-intrinsic ellipticity (II)
correlations. We find stronger correlations for more massive
and more luminous galaxies, as well as for earlier photometric
types.

\item {} 
We also find that the GI contributions to the observed
ellipticity correlations could be inferred directly from
measurements of galaxy density-intrinsic ellipticity
correlations, except on small scales, where systematic
differences between mI and dI correlations are large.

\end{itemize}

\end{itemize}

\item {} 
\sphinxstylestrong{Horizon-AGN}
\begin{itemize}
\item {} 
\sphinxhref{http://adsabs.harvard.edu/abs/2015MNRAS.454.2736C}{Chisari et al. 2015 - Intrinsic alignments of galaxies in the
Horizon-AGN cosmological hydrodynamical
simulation}
\begin{itemize}
\item {} 
We find that spheroidal galaxies in the simulation show a
tendency to be aligned radially towards overdensities in the
dark matter density field and other spheroidals.

\item {} 
Disc galaxies show a tendency to be oriented tangentially
around spheroidals in three dimensions.

\end{itemize}

\end{itemize}

\item {} 
\sphinxstylestrong{MassiveBlackII}
\begin{itemize}
\item {} 
\sphinxhref{http://adsabs.harvard.edu/abs/2017ApJ...834..169T}{Tenneti et al. 2017 - Impact of Baryonic Physics on Intrinsic
Alignments}
\begin{itemize}
\item {} 
We explore the parameter space of the subgrid star formation
and feedback model and find remarkable robustness of the
observable statistical measures to the details of subgrid
physics.

\end{itemize}

\item {} 
\sphinxhref{http://adsabs.harvard.edu/abs/2016MNRAS.462.2668T}{Tenneti et al. 2016 - Intrinsic alignments of disc and elliptical
galaxies in the MassiveBlack-II and Illustris
simulations}
\begin{itemize}
\item {} 
We find that simulated disc galaxies are more oblate in shape
and more misaligned with the shape of their host dark matter
subhalo when compared with ellipticals. The disc major axis is
found to be oriented towards the location of nearby elliptical
galaxies.

\end{itemize}

\end{itemize}

\end{itemize}


\chapter{Programing and Developing Software}
\label{\detokenize{index:programing-and-developing-software}}

\section{Everything about Python}
\label{\detokenize{python:everything-about-python}}\label{\detokenize{python::doc}}\begin{itemize}
\item {} 
Since \_\_Python\_\_ has become the new favorite among astronomers and cosmologists, here we collect a list
of basic resources for learning and using \_\_Python\_\_ in research.

\end{itemize}

\begin{sphinxadmonition}{note}{Note:}
If you are looking for \_\_Python\_\_ tool for a specific topic or field, please check out the
section for “Specific Topics in Astronomy”.
\end{sphinxadmonition}


\subsection{Basic Reference for Learning Python}
\label{\detokenize{resource/programing/python_basic:basic-reference-for-learning-python}}\label{\detokenize{resource/programing/python_basic::doc}}
There are clearly too many guide and tutorial for Python, some of which
are quite good. Here is just a personal selection.


\bigskip\hrule\bigskip



\subsubsection{Basic}
\label{\detokenize{resource/programing/python_basic:basic}}\begin{itemize}
\item {} 
\sphinxhref{https://github.com/realpython/python-guide}{Hitchhiker’s Guide to
Python}
\begin{itemize}
\item {} 
The actual guide book is \sphinxhref{docs.python-guide.org}{here}

\end{itemize}

\item {} 
\sphinxhref{https://python.swaroopch.com/}{A Byte of Python}
\begin{itemize}
\item {} 
“A Byte of Python” is a free book on programming using the Python
language. It serves as a tutorial or guide to the Python language
for a beginner audience.

\end{itemize}

\item {} 
\sphinxhref{https://github.com/vinta/awesome-python}{Awesome Python}
\begin{itemize}
\item {} 
A curated list of awesome Python frameworks, libraries, software
and resources.

\end{itemize}

\end{itemize}


\subsubsection{Dig Deeper}
\label{\detokenize{resource/programing/python_basic:dig-deeper}}\begin{itemize}
\item {} 
\sphinxhref{https://github.com/TheAlgorithms/Python}{All Algorithms implemented in
Python}
\begin{itemize}
\item {} 
These implementations are for learning purposes. They may be less
efficient than the implementations in the Python standard library.

\end{itemize}

\item {} 
Python Standard Library is a very good place to start learning
Python:
\begin{itemize}
\item {} 
There are a lot of useful tools in the standard library.

\item {} 
\sphinxhref{https://docs.python.org/3/tutorial/stdlib.html\#operating-system-interface}{A brief tour of the standard
library}

\item {} 
\sphinxhref{https://docs.python.org/3/library/}{The Reference of the Python Standard
Library}

\end{itemize}

\end{itemize}


\subsubsection{Managing Python Packages}
\label{\detokenize{resource/programing/python_basic:managing-python-packages}}

\paragraph{Anaconda}
\label{\detokenize{resource/programing/python_basic:anaconda}}\begin{itemize}
\item {} 
\sphinxstylestrong{Anaconda} distribution is the easiest way to perform Python/R data
science and machine learning on all platforms. It can help you create
environments with different Python versions, and manage libraries and
dependencies in Python.
\begin{itemize}
\item {} 
It is an easy way to start with Python programming without
worrying too much about installing packages all by yourself.

\item {} 
\sphinxhref{https://docs.anaconda.com/anaconda/user-guide/getting-started/}{Getting started with
Anaconda}
and the \sphinxhref{https://docs.anaconda.com/anaconda/navigator/tutorials/}{Anaconda
tutorials}
are good places to start.

\end{itemize}

\item {} 
\sphinxhref{https://astroconda.readthedocs.io/en/latest/}{AstroConda - Conda for
astronomers}
\begin{itemize}
\item {} 
\sphinxstylestrong{AstroConda} is a free Conda channel maintained by the Space
Telescope Science Institute (STScI) \#\#\# \sphinxstylestrong{pip}: Python Package
Installer

\end{itemize}

\item {} 
One important reason to use \sphinxstylestrong{Python} in science is that there are
already a huge number of great tools prepared to make your life
easier.

\item {} 
The most useful tool is \sphinxhref{https://pip.pypa.io/en/stable/}{pip - The Python Package
Installer}.
\begin{itemize}
\item {} 
\sphinxstylestrong{pip} is the package installer for Python. You can use pip to
install packages from the Python Package Index and other indexes

\item {} 
\sphinxhref{https://pip.pypa.io/en/stable/reference/}{This complete reference
guide} is very
useful.

\item {} 
\sphinxhref{https://realpython.com/what-is-pip/}{What Is Pip? A Guide for New
Pythonistas}

\end{itemize}

\item {} 
\sphinxhref{https://github.com/jazzband/pip-tools}{pip-tools - A set of tools to keep your pinned Python dependencies
fresh}
\begin{itemize}
\item {} 
A set of command line tools to help you keep your pip-based
packages fresh, even when you’ve pinned them.

\end{itemize}

\end{itemize}


\subsubsection{Start-up Package}
\label{\detokenize{resource/programing/python_basic:start-up-package}}\begin{itemize}
\item {} 
The most basic packages you want to use on daily bases.

\item {} 
\sphinxstylestrong{Scipy} ecosystem:
\begin{itemize}
\item {} 
\sphinxhref{https://www.numpy.org/}{Numpy - Base N-dimensional array
package}

\item {} 
\sphinxhref{https://github.com/scipy/scipy/}{SciPy - Fundamental library for scientific
computing}

\item {} 
\sphinxhref{https://www.sympy.org/en/index.html}{SymPy - a Python library for symbolic
mathematics}

\item {} 
\sphinxhref{https://docs.scipy.org/doc/}{Numpy and Scipy Documentation}

\item {} 
\sphinxhref{https://www.numpy.org/devdocs/user/quickstart.html}{Quickstart tutorial of
Numpy}

\item {} 
\sphinxhref{https://scipy-lectures.org/}{Scipy Lecture Notes - One document to learn numerics, science,
and data with Python}

\item {} 
\sphinxhref{https://jalammar.github.io/visual-numpy/}{A Visual Intro to NumPy and Data
Representation}

\end{itemize}

\item {} 
\sphinxhref{http://pandas.pydata.org/}{pandas - Python Data Analysis Library}
\begin{itemize}
\item {} 
\sphinxstylestrong{pandas} is an open source, BSD-licensed library providing
high-performance, easy-to-use data structures and data analysis
tools for the Python programming language.

\item {} 
\sphinxhref{http://pandas.pydata.org/pandas-docs/stable/}{Online document of
pandas}

\end{itemize}

\item {} 
\sphinxhref{https://docs.scipy.org/doc/}{matplotlib - Comprehensive 2D
Plotting}
\begin{itemize}
\item {} 
\sphinxstylestrong{Matplotlib} is a Python 2D plotting library which produces
publication quality figures in a variety of hardcopy formats and
interactive environments across platforms.

\item {} 
\sphinxhref{https://matplotlib.org/users/index.html}{Online document of
matplotlib}; and the
\sphinxhref{https://matplotlib.org/gallery/index.html}{matplotlib gallery}
are very helpful resources to learn.

\end{itemize}

\item {} 
\sphinxhref{https://bokeh.pydata.org/en/latest/}{Bokeh -}
\begin{itemize}
\item {} 
\sphinxstylestrong{Bokeh} is an interactive visualization library that targets
modern web browsers for presentation. Its goal is to provide
elegant, concise construction of versatile graphics, and to extend
this capability with high-performance interactivity over very
large or streaming datasets.

\item {} 
\sphinxhref{https://bokeh.pydata.org/en/latest/docs/user\_guide.html\#userguide}{Bokeh user
guide}
and the \sphinxhref{https://bokeh.pydata.org/en/latest/docs/reference.html\#refguide}{reference
guide}
are very useful. The \sphinxhref{https://bokeh.pydata.org/en/latest/docs/gallery.html}{gallery of
examples}
is also a good place to start.

\end{itemize}

\item {} 
Interactive Python computing:
\begin{itemize}
\item {} 
\sphinxhref{https://jupyter.org/}{Jupyter environment}
\begin{itemize}
\item {} 
Project \sphinxstylestrong{Jupyter} exists to develop open-source software,
open-standards, and services for interactive computing across
dozens of programming languages.

\item {} 
The \sphinxstylestrong{Jupyter} Notebook App is a server-client application
that allows editing and running notebook documents via a web
browser.

\end{itemize}

\item {} 
\sphinxhref{http://ipython.org/}{IPython - Interactive computing}
\begin{itemize}
\item {} 
\sphinxstylestrong{IPython} provides a rich architecture for interactive
computing

\item {} 
\sphinxhref{https://nbviewer.jupyter.org/}{nbviewer - A simple way to share Jupyter
notebooks}

\end{itemize}

\item {} 
\sphinxhref{https://jupyter-notebook-beginner-guide.readthedocs.io/en/latest/index.html}{Jupyter/IPython Notebook Quick Start
Guide}

\end{itemize}

\item {} 
\sphinxhref{https://github.com/scikit-learn/scikit-learn}{scikit-learn - Machine learning in
Python}
\begin{itemize}
\item {} 
\sphinxstylestrong{scikit-learn} is a Python module for machine learning built on
top of SciPy.

\item {} 
\sphinxhref{https://scikit-learn.org/stable/user\_guide.html}{Online document of
scikit-learn}
and \sphinxhref{https://scikit-learn.org/stable/tutorial/index.html}{scikit-learn
Tutorials}

\end{itemize}

\item {} 
\sphinxhref{https://scikit-image.org/}{scikit-image - Image processing in
Python}
\begin{itemize}
\item {} 
\sphinxstylestrong{scikit-image} is a collection of algorithms for image
processing.

\item {} 
\sphinxhref{https://github.com/scikit-image/skimage-tutorials}{Tutorials of
skimage} and
\sphinxhref{https://scikit-image.org/docs/dev/auto\_examples/}{gallery of
examples} are
very useful.

\end{itemize}

\item {} 
\sphinxhref{https://www.astropy.org/}{astropy - Community Python library for
astronomer}
\begin{itemize}
\item {} 
The \sphinxstylestrong{Astropy} Project is a community effort to develop a common
core package for Astronomy in Python and foster an ecosystem of
interoperable astronomy packages.

\item {} 
\sphinxhref{http://docs.astropy.org/en/stable/index.html}{Online document of astropy is
here}

\item {} 
\sphinxhref{http://learn.astropy.org/}{Learn.Astropy - Tutorials, documentation, and examples of
astropy}

\end{itemize}

\item {} 
\sphinxhref{https://scrapy.org/}{Scrapy - A fast high-level web crawling \& scraping framework for
Python}
\begin{itemize}
\item {} 
An open source and collaborative framework for extracting the data
you need from websites. In a fast, simple, yet extensible way.

\end{itemize}

\end{itemize}


\subsubsection{Tricks and Tips}
\label{\detokenize{resource/programing/python_basic:tricks-and-tips}}

\subsection{Performance Optimization for Python}
\label{\detokenize{resource/programing/python_performance:performance-optimization-for-python}}\label{\detokenize{resource/programing/python_performance::doc}}

\subsubsection{Basics}
\label{\detokenize{resource/programing/python_performance:basics}}\begin{itemize}
\item {} 
\sphinxhref{https://wiki.python.org/moin/PythonSpeed/PerformanceTips}{Basic tips for optimizing the performace of your Python
code}

\item {} 
\sphinxhref{https://github.com/mynameisfiber/high\_performance\_python}{Code for the book “High Performance Python” by Micha Gorelick and
Ian Ozsvald with
OReilly}

\end{itemize}


\subsubsection{Interfacing with C/C++}
\label{\detokenize{resource/programing/python_performance:interfacing-with-c-c}}\begin{itemize}
\item {} 
\sphinxhref{https://scipy-lectures.org/advanced/interfacing\_with\_c/interfacing\_with\_c.html\#introduction}{Interfacing with C from Scipy lecture
notes}
\begin{itemize}
\item {} 
Very nice overview and examples of four approaches.

\end{itemize}
\begin{enumerate}
\def\theenumi{\arabic{enumi}}
\def\labelenumi{\theenumi .}
\makeatletter\def\p@enumii{\p@enumi \theenumi .}\makeatother
\item {} 
\sphinxhref{https://docs.python.org/3/c-api/index.html}{Python/C API}
\begin{itemize}
\item {} 
\sphinxhref{https://docs.python.org/3/extending/extending.html}{Extending Python3.7 with C or
C++}

\end{itemize}

\item {} 
\sphinxhref{https://docs.python.org/3/library/ctypes.html}{ctypes - A foreign function library for
Python}
\begin{itemize}
\item {} 
It provides C compatible data types, and allows calling
functions in DLLs or shared libraries. It can be used to wrap
these libraries in pure Python.

\end{itemize}

\item {} 
\sphinxhref{http://www.swig.org/exec.html}{SWIG}
\begin{itemize}
\item {} 
\sphinxstylestrong{SWIG} is an interface compiler that connects programs
written in C and C++ with scripting languages such as Python

\end{itemize}

\item {} 
\sphinxhref{https://cython.org/}{Cython - C-Extensions for Python}
\begin{itemize}
\item {} 
\sphinxhref{https://cython.readthedocs.io/en/latest/src/tutorial/cython\_tutorial.html}{Basic tutorial of
Cython}

\item {} 
\sphinxhref{https://github.com/adrn/cython-tutorial}{Tutorial on how to use Cython to optimize Python code by
Adrian
Price-Whelan}

\item {} 
\sphinxhref{https://python.g-node.org/python-summerschool-2011/\_media/materials/cython/cython-slides.pdf}{Cython tutorial by Pauli Virtanen from
2011}

\end{itemize}

\end{enumerate}

\end{itemize}


\paragraph{Only for C++}
\label{\detokenize{resource/programing/python_performance:only-for-c}}\begin{itemize}
\item {} 
\sphinxhref{https://www.boost.org/doc/libs/1\_70\_0/libs/python/doc/html/index.html}{Boost.Python}
\begin{itemize}
\item {} 
\sphinxstylestrong{Boost.Python}, a C++ library which enables seamless
interoperability between C++ and the Python programming language.

\end{itemize}

\item {} 
\sphinxhref{https://github.com/pybind/pybind11}{pybind11 - Seamless operability between C++11 and
Python}
\begin{itemize}
\item {} 
\sphinxstylestrong{pybind11} is a lightweight header-only library that exposes C++
types in Python and vice versa, mainly to create Python bindings
of existing C++ code.

\item {} 
This is used by the \sphinxstylestrong{LSST} developers, please see the \sphinxhref{https://developer.lsst.io/pybind11/style.html}{DM
Pybind11 style
guide} for
details.

\end{itemize}

\end{itemize}


\subsubsection{Using Just-in-Time (JIT) Compiler}
\label{\detokenize{resource/programing/python_performance:using-just-in-time-jit-compiler}}\begin{itemize}
\item {} 
\sphinxhref{http://pypy.org/}{pypy - a fast, compliant alternative implementation of the Python
language}

\item {} 
\sphinxhref{http://numba.pydata.org/}{numba - makes Python code fast}
\begin{itemize}
\item {} 
\sphinxstylestrong{Numba} is an open source JIT compiler that translates a subset
of Python and \sphinxstylestrong{NumPy} code into fast machine code.
\_\_Numba\_\_only supports LLVM.

\item {} 
\sphinxstylestrong{Numba} offers a range of options for parallelizing your code
for CPUs and GPUs, often with only minor code changes.

\item {} 
\sphinxhref{http://numba.pydata.org/numba-doc/0.15.1/numpy\_support.html}{Numpy supports in
Numba}

\item {} 
\sphinxstylestrong{Numba} has a \sphinxstylestrong{vectorize} and \sphinxstylestrong{guvectorize} decorators that
can be very useful.

\end{itemize}

\item {} 
\sphinxhref{https://github.com/jakeret/hope}{hope - A Python Just-In-Time compiler for astrophysical
computations}
\begin{itemize}
\item {} 
\sphinxstylestrong{hope} is a specialized method-at-a-time JIT compiler written in
Python for translating Python source code into C++ and compiles
this at runtime.

\item {} 
Has not been updated for three yeears.

\end{itemize}

\end{itemize}


\paragraph{Tutorial and notes}
\label{\detokenize{resource/programing/python_performance:tutorial-and-notes}}\begin{itemize}
\item {} 
\sphinxhref{https://jakevdp.github.io/blog/2015/02/24/optimizing-python-with-numpy-and-numba/}{Optimizing Python in the Real World: NumPy, Numba, and the NUFFT by
Jake
VanderPlas}

\end{itemize}


\paragraph{Making Numpy faster}
\label{\detokenize{resource/programing/python_performance:making-numpy-faster}}\begin{itemize}
\item {} 
\sphinxhref{https://github.com/google/jax}{jax - GPU- and TPU-backed NumPy with differentiation and JIT
compilation by Google}
\begin{itemize}
\item {} 
JAX is Autograd and XLA, brought together for high-performance
machine learning research.

\end{itemize}

\item {} 
\sphinxhref{https://github.com/HIPS/autograd}{autograd - Efficiently computes derivatives of numpy
code}

\end{itemize}


\paragraph{Other packages}
\label{\detokenize{resource/programing/python_performance:other-packages}}\begin{itemize}
\item {} 
\sphinxhref{https://scikit-learn.org/stable/developers/performance.html}{How to speed up Python by
scikit-learn}

\end{itemize}


\subsubsection{Parallel computing in Python}
\label{\detokenize{resource/programing/python_performance:parallel-computing-in-python}}

\paragraph{Tutorial}
\label{\detokenize{resource/programing/python_performance:tutorial}}\begin{itemize}
\item {} 
\sphinxhref{https://sebastianraschka.com/Articles/2014\_multiprocessing.html}{An introduction to parallel programming using Python’s
multiprocessing
module}

\item {} 
\sphinxhref{https://towardsdatascience.com/speed-up-your-algorithms-part-3-parallelization-4d95c0888748}{Speed Up Your Algorithms Part
3 — Parallel-ization}

\end{itemize}


\paragraph{Software}
\label{\detokenize{resource/programing/python_performance:software}}

\subparagraph{Common tools:}
\label{\detokenize{resource/programing/python_performance:common-tools}}\begin{itemize}
\item {} 
\sphinxhref{https://github.com/mpi4py/mpi4py}{mpi4py - Python bindings for
MPI}

\item {} 
\sphinxhref{https://github.com/joblib/joblib}{joblib - a set of tools to provide lightweight pipelining in
Python}

\item {} 
\sphinxhref{https://github.com/tomMoral/loky}{loky - Robust and reusable Executor for
joblib}

\item {} 
\sphinxhref{https://github.com/adrn/schwimmbad}{schwimmbad - A common interface to processing
pools}

\end{itemize}


\subparagraph{More “Big Data” approach:}
\label{\detokenize{resource/programing/python_performance:more-big-data-approach}}\begin{itemize}
\item {} 
\sphinxhref{https://dask.org/}{Dask - Parallel computing with task
scheduling}
\begin{itemize}
\item {} 
\sphinxstylestrong{Dask} provides advanced parallelism for analytics, enabling
performance at scale for the tools you love

\item {} 
\sphinxstylestrong{Dask} is open source and freely available. It is developed in
coordination with other community projects like \sphinxstylestrong{Numpy},
\sphinxstylestrong{Pandas}, and \sphinxstylestrong{Scikit-Learn}.

\item {} 
\sphinxhref{https://github.com/dask/dask-tutorial}{Official Dask tutorial using Jupyter
notebook}

\end{itemize}

\end{itemize}


\subsection{Python Tools for Model Fitting and General Optimization}
\label{\detokenize{resource/programing/python_optimization:python-tools-for-model-fitting-and-general-optimization}}\label{\detokenize{resource/programing/python_optimization::doc}}

\subsubsection{Optimization Algorithms}
\label{\detokenize{resource/programing/python_optimization:optimization-algorithms}}\begin{itemize}
\item {} 
\sphinxhref{https://github.com/100/Solid}{Solid - A Python framework for gradient-free
optimization}
\begin{itemize}
\item {} 
Include algorithms like Genetic Algorithm, Stimulated Annealing,
Particle Swarm, Harmony Search, and Stochastic hill Climb

\end{itemize}

\item {} 
\sphinxhref{https://github.com/PytLab/gaft}{GAFT - A Genetic Algorithm Framework in
Python}
\begin{itemize}
\item {} 
有一系列中文blog介绍背后设计和算法,非常适合学习遗传算法

\end{itemize}

\item {} 
\sphinxhref{https://github.com/ljvmiranda921/pyswarms}{pyswarms - A research toolkit for particle swarm optimization in
Python}
\begin{itemize}
\item {} 
Wikipedia introduction of \sphinxhref{https://en.wikipedia.org/wiki/Particle\_swarm\_optimization}{PSO: Particle Swarm
Optimization}

\item {} 
Document includes a \sphinxhref{https://pyswarms.readthedocs.io/en/latest/intro.html}{very nice introducation of the PSO
algorithm}

\end{itemize}

\item {} 
\sphinxhref{https://en.wikipedia.org/wiki/CMA-ES}{The Covariance Matrix Adaptation Evolution Strategy (CMA-ES)
Algorihm}
\begin{itemize}
\item {} 
A stochastic derivative-free numerical optimization algorithm for
difficult (non-convex, ill-conditioned, multi-modal, rugged,
noisy) optimization problems in continuous search spaces.

\item {} 
\sphinxhref{https://github.com/CMA-ES/pycma}{pycma - A Python implementation of
CMA-ES}

\end{itemize}

\item {} 
\sphinxhref{https://github.com/pmelchior/proxmin}{proxmin - Proximal Minimization in
Python}
\begin{itemize}
\item {} 
Based on the paper by \sphinxhref{https://link.springer.com/article/10.1007\%2Fs11081-018-9380-y}{Fred Moolekamp \& Peter Melchoir
(2018)}

\item {} 
Includes algorithms like Alternating Direction Method of
Multipliers (ADMM) and Block-Simultaneous Direction Method of
Multipliers (bSDMM)

\item {} 
Used as the backend of multi-band deblending code
\sphinxhref{https://github.com/fred3m/scarlet}{scarlet}

\end{itemize}

\item {} 
\sphinxhref{https://github.com/esa/pagmo2}{pagmo2 - A C++ / Python platform to perform parallel computations of
optimisation tasks (global and local) via the asynchronous
generalized island model}
\begin{itemize}
\item {} 
A a scientific library for massively parallel optimization used by
ESA.

\item {} 
Document can be found \sphinxhref{https://esa.github.io/pagmo2/}{here}

\item {} 
Includes \sphinxhref{https://esa.github.io/pagmo2/docs/algorithm\_list.html}{a long list of global and local optimization
algorithms}

\end{itemize}

\item {} 
\sphinxhref{https://github.com/tBuLi/symfit}{symfit - Symbolic Fitting; fitting as it should
be}
\begin{itemize}
\item {} 
Pythonic way to do fitting.

\end{itemize}

\item {} 
\sphinxhref{https://github.com/zfit/zfit}{zfit - scalable pythonic fitting}
\begin{itemize}
\item {} 
The \sphinxstylestrong{zfit} package is a model manipulation and fitting library
based on TensorFlow and optimised for simple and direct
manipulation of probability density functions. Its main focus is
on scalability, parallelisation and user friendly experience.

\end{itemize}

\end{itemize}


\paragraph{Bayesian Optimization}
\label{\detokenize{resource/programing/python_optimization:bayesian-optimization}}\begin{itemize}
\item {} 
\sphinxhref{https://github.com/fmfn/BayesianOptimization}{BayesianOptimization - Pure Python implementation of bayesian global
optimization with gaussian
processes}
\begin{itemize}
\item {} 
Bayesian optimization works by constructing a posterior
distribution of functions (gaussian process) that best describes
the function you want to optimize.

\end{itemize}

\item {} 
\sphinxhref{https://github.com/moews/gaussbock}{gaussbock - Fast parallel-iterative cosmological parameter
estimation with Bayesian
nonparametrics}

\item {} 
\sphinxhref{https://github.com/lacerbi/vbmc}{vbmc - Variational Bayesian Monte Carlo (VBMC) algorithm for
posterior and model inference}
\begin{itemize}
\item {} 
VBMC is a novel approximate inference method designed to fit and
evaluate computational models with a limited budget of likelihood
evaluations (e.g., for computationally expensive models).

\item {} 
A Python implementation is planned.

\end{itemize}

\end{itemize}


\subsection{Statistical Analysis and Model in Python}
\label{\detokenize{resource/programing/python_statistics:statistical-analysis-and-model-in-python}}\label{\detokenize{resource/programing/python_statistics::doc}}

\subsubsection{Error Propagation}
\label{\detokenize{resource/programing/python_statistics:error-propagation}}\begin{itemize}
\item {} 
\sphinxhref{http://docs.astropy.org/en/stable/uncertainty/}{astropy.uncertainty}
\begin{itemize}
\item {} 
Provides a \sphinxstylestrong{Distribution} object to represent statistical
distributions in a form that acts as a drop-in replacement for
\sphinxstylestrong{Quantity} or a regular \sphinxstylestrong{numpy.ndarray}. Still work in
progress.

\end{itemize}

\item {} 
\sphinxhref{https://github.com/lebigot/uncertainties}{uncertainties - Transparent calculations with uncertainties on the
quantities involved}
\begin{itemize}
\item {} 
The \sphinxstylestrong{uncertainties} package is a free, cross-platform program
that transparently handles calculations with numbers with
uncertainties (like 3.14\(\pm\)0.01). It can also yield the derivatives
of any expression.

\end{itemize}

\end{itemize}


\subsubsection{Modeling Tool}
\label{\detokenize{resource/programing/python_statistics:modeling-tool}}\begin{itemize}
\item {} 
\sphinxhref{https://github.com/thouska/spotpy}{spotpy - A Statistical Parameter Optimization
Tool}
\begin{itemize}
\item {} 
SPOTPY is a Python framework that enables the use of Computational
optimization techniques for calibration, uncertainty and
sensitivity analysis techniques of almost every (environmental-)
model.

\end{itemize}

\item {} 
\sphinxhref{https://github.com/fmfn/BayesianOptimization}{BayesianOptimization - A Python implementation of global
optimization with gaussian
processes}
\begin{itemize}
\item {} 
This is a constrained global optimization package built upon
bayesian inference and gaussian process, that attempts to find the
maximum value of an unknown function in as few iterations as
possible.

\end{itemize}

\end{itemize}


\subsubsection{Sampling Tools and Bayesian Analysis}
\label{\detokenize{resource/programing/python_statistics:sampling-tools-and-bayesian-analysis}}\begin{itemize}
\item {} 
\sphinxhref{https://github.com/dfm/emcee}{emcee - The Python ensemble sampling toolkit for affine-invariant
MCMC}
\begin{itemize}
\item {} 
By Dan Foreman-Mackey. \sphinxstylestrong{emcee} is a stable, well tested Python
implementation of the affine-invariant ensemble sampler for Markov
chain Monte Carlo (MCMC) proposed by Goodman \& Weare (2010).

\end{itemize}

\item {} 
\sphinxhref{https://github.com/joshspeagle/dynesty}{dynesty - Dynamic Nested Sampling package for computing Bayesian
posteriors and evidences}
\begin{itemize}
\item {} 
By \sphinxhref{https://joshspeagle.github.io/}{Josh Speagle}. A Dynamic
Nested Sampling package for computing Bayesian posteriors and
evidences. Pure Python.

\end{itemize}

\item {} 
\sphinxhref{https://github.com/kbarbary/nestle}{nestle - Pure Python, MIT-licensed implementation of nested sampling
algorithms for evaluating Bayesian
evidence}
\begin{itemize}
\item {} 
By \sphinxhref{http://kylebarbary.com/}{Kyle Barbary}

\end{itemize}

\item {} 
\sphinxhref{https://github.com/adammoss/nnest}{nnest - Neural network accelerated nested and MCMC
sampling}
\begin{itemize}
\item {} 
By Adam Moss. Based on \sphinxhref{https://arxiv.org/abs/1903.10860}{this
paper}

\end{itemize}

\item {} 
\sphinxhref{https://github.com/mcleonard/sampyl}{sampyl - MCMC samplers for Bayesian estimation in Python, including
Metropolis-Hastings, NUTS, and
Slice}
\begin{itemize}
\item {} 
\sphinxstylestrong{Sampyl} is a package for sampling from probability
distributions using MCMC methods. Similar to \sphinxstylestrong{PyMC3} using
theano to compute gradients, Sampyl uses autograd to compute
gradients.

\end{itemize}

\item {} 
\sphinxhref{https://github.com/pymc-devs/pymc3}{PyMC3 - Probabilistic Programming in Python: Bayesian Modeling and
Probabilistic Machine Learning with
Theano}
\begin{itemize}
\item {} 
\sphinxstylestrong{PyMC3} is a Python package for Bayesian statistical modeling
and Probabilistic Machine Learning focusing on advanced Markov
chain Monte Carlo (MCMC) and variational inference (VI)
algorithms. Its flexibility and extensibility make it applicable
to a large suite of problems.

\item {} 
\sphinxhref{https://docs.pymc.io/notebooks/getting\_started}{Getting started with
PyMC3} and the
\sphinxhref{https://docs.pymc.io/nb\_examples/index.html}{Example
Notebooks} are
good places to get started.

\end{itemize}

\item {} 
\sphinxhref{https://github.com/pymc-devs/pymc4}{PyMC4 - A high-level probabilistic programming interface for
TensorFlow Probability}

\end{itemize}


\subsubsection{Gaussian Process}
\label{\detokenize{resource/programing/python_statistics:gaussian-process}}\begin{itemize}
\item {} 
A full introduction to the theory of Gaussian Processes is available
for free online: \sphinxhref{http://www.gaussianprocess.org/gpml/}{Rasmussen \& Williams
(2006)}.

\item {} 
\sphinxhref{https://astrostatistics.psu.edu/su14/lectures/penn-gp.pdf}{An Astronomer’s Introduction to Gaussian
Processes}
\begin{itemize}
\item {} 
Very good introduction by Dan Foreman-Mackey.

\end{itemize}

\item {} 
\sphinxhref{https://scikit-learn.org/stable/modules/gaussian\_process.html}{sklearn.gaussian\_process - The Gaussian Processes module in
scikit-learn}

\item {} 
\sphinxhref{https://github.com/sheffieldml/gpy}{GPy - Gaussian processes framework in
python}
\begin{itemize}
\item {} 
Gaussian processes underpin range of modern machine learning
algorithms. In \sphinxhref{http://sheffieldml.github.io/GPy/}{GPy}, we’ve
used python to implement a range of machine learning algorithms
based on GPs. \sphinxhref{https://gpy.readthedocs.io/en/deploy/}{Online document is
here}

\item {} 
\sphinxhref{https://nbviewer.jupyter.org/github/SheffieldML/notebook/blob/master/GPy/index.ipynb}{Jupyter notebooks to introduce
GPy}

\end{itemize}

\item {} 
\sphinxhref{https://github.com/GPflow/GPflow}{gpflow - Gaussian processes in
TensorFlow}
\begin{itemize}
\item {} 
\sphinxstylestrong{GPflow} is a package for building Gaussian process models in
python, using \sphinxstylestrong{TensorFlow}.

\item {} 
\sphinxstylestrong{GPflow} implements modern Gaussian process inference for
composable kernels and likelihoods.

\item {} 
\sphinxstylestrong{GPflow} uses TensorFlow for running computations, which allows
fast execution on GPUs, and uses Python 3.5 or above.

\item {} 
\sphinxhref{https://gpflow.readthedocs.io/en/develop/}{Online document is
here}

\end{itemize}

\item {} 
\sphinxhref{https://github.com/cornellius-gp/gpytorch}{gpytorch - A highly efficient and modular implementation of Gaussian
Processes in PyTorch}
\begin{itemize}
\item {} 
\sphinxstylestrong{GPyTorch} is a Gaussian process library implemented using
\sphinxstylestrong{PyTorch}. \sphinxstylestrong{GPyTorch} is designed for creating scalable,
flexible, and modular Gaussian process models with ease.

\end{itemize}

\item {} 
\sphinxhref{https://github.com/dfm/george}{george - Fast and flexible Gaussian Process regression in
Python}
\begin{itemize}
\item {} 
By Dan Foreman-Mackey. \sphinxstylestrong{George} is a fast and flexible Python
library for Gaussian Process (GP) Regression.

\item {} 
Unlike some other GP implementations, \sphinxstylestrong{george} is focused on
efficiently evaluating the marginalized likelihood of a dataset
under a GP prior, even as this dataset gets Big

\item {} 
Example applications:
\begin{itemize}
\item {} 
\sphinxhref{https://github.com/tmcclintock/AReconstructionTool}{ART - A Reconstruction
Tool}

\item {} 
\sphinxhref{https://github.com/rodluger/everest}{everest - De-trending of K2 Light
curves}

\end{itemize}

\end{itemize}

\item {} 
\sphinxhref{https://github.com/dfm/celerite}{celerite - Scalable 1D Gaussian Processes in C++, Python, and
Julia}
\begin{itemize}
\item {} 
By Dan Foreman-Mackey. \sphinxhref{https://celerite.readthedocs.io/en/stable/}{Online document is
here}

\item {} 
Based on \sphinxhref{https://arxiv.org/abs/1703.09710}{Fast and scalable Gaussian process modeling with
applications to astronomical time
series}

\end{itemize}

\end{itemize}


\subsubsection{Survival Analysis}
\label{\detokenize{resource/programing/python_statistics:survival-analysis}}\begin{itemize}
\item {} 
Traditionally, \sphinxhref{https://en.wikipedia.org/wiki/Survival\_analysis}{survival
analysis} was
developed to measure lifespans of individuals. The analysis can be
further applied to not just traditional births and deaths, but any
duration.

\item {} 
\sphinxstylestrong{Survival function}: the survival function defines the probability
the death event has not occured yet at time t, or equivalently, the
probability of surviving past time t

\item {} 
\sphinxstylestrong{Hazard curve}: the probability of the death event occurring at
time t, given that the death event has not occurred until time t.
Hazard function is non-parametric.

\item {} 
\sphinxstylestrong{Kaplan-Meier estimator for survival function}: Survival analysis
assumes that upper limits have the same underlying distribution as
the data, and the Kaplan-Meier esti- mator further assumes that
detections and upper limits are mutually independent

\item {} 
\sphinxhref{https://lifelines.readthedocs.io/en/latest/}{lifelines - implementation of survival analysis in
Python}
\begin{itemize}
\item {} 
Handles right-censored data.

\item {} 
Example of astrophysical usage: \sphinxhref{https://arxiv.org/abs/1812.03392}{radio SED of high-z SF
galaxies}

\end{itemize}

\end{itemize}


\subsection{Visualization using Python}
\label{\detokenize{resource/programing/python_visualization:visualization-using-python}}\label{\detokenize{resource/programing/python_visualization::doc}}

\subsubsection{Basics}
\label{\detokenize{resource/programing/python_visualization:basics}}\begin{itemize}
\item {} 
\sphinxhref{https://journals.plos.org/ploscompbiol/article/file?id=10.1371/journal.pcbi.1003833\&type=printable}{Ten simple rules for better
figures}
\begin{itemize}
\item {} 
{[}1{]}: Know Your Audience; {[}2{]}: Identify Your Message; {[}3{]}: Adapt
the Figure to the Support Medium; {[}4{]}: Captions are not Optional;
{[}5{]}: Do Not trust the Defaults; {[}6{]}: Use Color Effectively; {[}7{]}:
Do Not Mislead the Reader; {[}8{]}: Avoid ‘‘Chartjunk’’; {[}9{]}: Message
Trumps Beauty; {[}10{]}: Get the Right Tool

\end{itemize}

\item {} 
\sphinxhref{https://realpython.com/python-matplotlib-guide/}{Basic guide for plotting with
Matplotlib}
\begin{itemize}
\item {} 
This is a very good introduction for Matplotlib

\end{itemize}

\end{itemize}


\subsubsection{Great Examples}
\label{\detokenize{resource/programing/python_visualization:great-examples}}\begin{itemize}
\item {} 
\sphinxhref{http://www.astroml.org/book\_figures/index.html}{Textbook Figures for Statistics, Data Mining, and Machine Learning
in Astronomy}

\end{itemize}


\paragraph{SciPy John Hunter Excellence in Plotting Contest}
\label{\detokenize{resource/programing/python_visualization:scipy-john-hunter-excellence-in-plotting-contest}}\begin{itemize}
\item {} 
\sphinxhref{http://conference.scipy.org/jhepc2013/index.html}{The 2013
gallery} - Many
are astronomy related and excellent examples

\item {} 
\sphinxhref{http://members.cbio.mines-paristech.fr/~nvaroquaux/jhepc/index.html}{The 2014
gallery}

\item {} 
\sphinxhref{https://scipy2016.scipy.org/ehome/115969/276538/}{The 2015
gallery} Link
not working…

\item {} 
\sphinxhref{http://droettboom.com/jhepc2018-judge-packet/}{The 2018 entries}

\end{itemize}


\subsubsection{Color and Colour}
\label{\detokenize{resource/programing/python_visualization:color-and-colour}}\begin{itemize}
\item {} 
\sphinxhref{https://blog.datawrapper.de/colorguide/}{Your Friendly Guide to Colors in Data
Visualisation}
\begin{itemize}
\item {} 
Pretty much the only thing you need to read about color.

\end{itemize}

\item {} 
\sphinxhref{https://blogs.egu.eu/divisions/gd/2017/08/23/the-rainbow-colour-map/}{Why rainbow colormap is
harmful}

\end{itemize}


\paragraph{Matplotlib}
\label{\detokenize{resource/programing/python_visualization:matplotlib}}\begin{itemize}
\item {} 
\sphinxhref{https://matplotlib.org/examples/color/named\_colors.html}{Gallery of named colors in
Matplotlib}

\item {} 
\sphinxhref{https://matplotlib.org/examples/color/colormaps\_reference.html}{Gallery of colormaps in
Matplotlib}

\end{itemize}


\paragraph{Fancier and Better}
\label{\detokenize{resource/programing/python_visualization:fancier-and-better}}\begin{itemize}
\item {} 
\sphinxhref{http://colorbrewer2.org}{ColorBrewer2 - Interactive colormap
builder}

\item {} 
\sphinxhref{https://jiffyclub.github.io/palettable/}{Palettable - Color palettes for
Python}

\item {} 
\sphinxhref{http://www.fabiocrameri.ch/visualisation.php}{Perceptually uniform color
maps}

\item {} 
\sphinxhref{http://nipponcolors.com}{Nippon Colors - Fancy traditional Japanese
colors}

\end{itemize}


\subsection{Write a Python package}
\label{\detokenize{resource/programing/python_write_your_project:write-a-python-package}}\label{\detokenize{resource/programing/python_write_your_project::doc}}

\subsubsection{Developer Guide}
\label{\detokenize{resource/programing/python_write_your_project:developer-guide}}\begin{itemize}
\item {} 
\sphinxhref{https://developer.lsst.io}{LSST DM Developer Guide}
\begin{itemize}
\item {} 
LSST的数据管理系统的开发指南是天文领域里非常具有参考价值的

\item {} 
Python编程风格:\sphinxhref{https://developer.lsst.io/python/style.html}{DM Python style
guide}

\item {} 
类似的还有 \sphinxhref{https://developerskatelescopeorg.readthedocs.io/en/latest/}{SKA develop
portal}
(建设中…)

\end{itemize}

\end{itemize}


\subsubsection{Structure}
\label{\detokenize{resource/programing/python_write_your_project:structure}}\begin{itemize}
\item {} 
“Dress for the job you want, not the job you have.”

\item {} 
\sphinxhref{https://docs.python-guide.org/writing/structure/}{Structuring Your
Project}
\begin{itemize}
\item {} 
从组织文件到代码结构,比较好的入门阅读

\end{itemize}

\item {} 
\sphinxhref{https://python-packaging.readthedocs.io/en/latest/index.html}{How To Package Your Python
Code}
\begin{itemize}
\item {} 
可读性也很强: aims to put forth an opinionated and specific
pattern to make trouble-free packages for community use

\end{itemize}

\item {} 
\sphinxhref{https://drivendata.github.io/cookiecutter-data-science/}{Cookiecutter - A logical, reasonably standardized, but flexible
project structure for doing and sharing data science
work}
\begin{itemize}
\item {} 
一个很好的组织基于数据的项目的例子,其中关于“Keep secrets and
configuration out of version control”的部分很有用。

\end{itemize}

\end{itemize}


\subsubsection{Code Format}
\label{\detokenize{resource/programing/python_write_your_project:code-format}}\begin{itemize}
\item {} 
It is good practice to follow well-established code format. Not only
it can help you write codes that are nice looking and easy to
maintain, it will help others to read and contribute to the code.

\item {} 
For Python, \sphinxhref{https://www.python.org/dev/peps/pep-0008/}{the PEP8 style
guide} is the most
important one. Some of these rules feel unecessary and annoying, but
there are always good reasons behind them.

\item {} 
\sphinxhref{https://github.com/hhatto/autopep8}{autopep8: A tool that automatically formats Python code to conform
to the PEP 8 style guide}

\item {} 
\sphinxhref{https://github.com/python/black}{black: The uncompromising Python code
formatter}
\begin{itemize}
\item {} 
Blackened code looks the same regardless of the project you’re
reading. Formatting becomes transparent after a while and you can
focus on the content instead.

\end{itemize}

\item {} 
\sphinxhref{https://github.com/google/yapf}{yapf: A formatter for Python files from
Google}

\end{itemize}


\subsubsection{\sphinxstylestrong{setup.py}}
\label{\detokenize{resource/programing/python_write_your_project:setup-py}}\begin{itemize}
\item {} 
\sphinxhref{https://github.com/kennethreitz/setup.py}{A Human’s Ultimate Guide to
setup.py}
\begin{itemize}
\item {} 
This is very good template for using \sphinxstylestrong{setup.py}

\end{itemize}

\end{itemize}


\subsubsection{Readme}
\label{\detokenize{resource/programing/python_write_your_project:readme}}\begin{itemize}
\item {} 
\sphinxhref{https://github.com/noffle/art-of-readme}{Art of README}
\begin{itemize}
\item {} 
\sphinxhref{https://github.com/noffle/art-of-readme/blob/master/README-zh.md}{中文版}
基本上,想学习如何写好Readme看这篇就够了

\end{itemize}

\item {} 
\sphinxhref{https://github.com/kefranabg/readme-md-generator}{readme-md-generator - CLI that generates beautiful README.md
files}
\begin{itemize}
\item {} 
\sphinxstylestrong{readme-md-generator} will suggest you default answers by
reading your package.json and git configuration.

\end{itemize}

\end{itemize}


\subsubsection{Document}
\label{\detokenize{resource/programing/python_write_your_project:document}}

\paragraph{General instructions}
\label{\detokenize{resource/programing/python_write_your_project:general-instructions}}\begin{itemize}
\item {} 
\sphinxhref{https://developer.lsst.io/project-docs/change-controlled-docs.html}{Writing change-controlled
documentation}
\begin{itemize}
\item {} 
Manual provided by LSST DM team

\end{itemize}

\item {} 
\sphinxhref{https://developer.lsst.io/python/numpydoc.html\#py-docstring-short-summary}{LSST DM的Documenting Python APIs with
Docstrings}
\begin{itemize}
\item {} 
Also very good example by LSST DM. LSST adopts the \sphinxstylestrong{Numpydoc}
format.

\end{itemize}

\end{itemize}


\paragraph{Tools}
\label{\detokenize{resource/programing/python_write_your_project:tools}}\begin{itemize}
\item {} 
\sphinxhref{https://www.sphinx-doc.org/en/1.5/index.html}{sphinx - Python documentation
generator}
\begin{itemize}
\item {} 
\sphinxstylestrong{Sphinx} is a tool that makes it easy to create intelligent and
beautiful documentation.

\item {} 
\sphinxstylestrong{Sphinx} uses
\sphinxhref{http://docutils.sourceforge.net/rst.html}{reStructuredText} as
its markup language, and many of its strengths come from the power
and straightforwardness of \sphinxstylestrong{reStructuredText} and its parsing
and translating suite, the
\sphinxhref{http://docutils.sourceforge.net/}{Docutils}.

\item {} 
\sphinxhref{https://www.sphinx-doc.org/en/1.5/tutorial.html}{First steps with
sphinx}

\item {} 
\sphinxhref{https://gist.github.com/dupuy/1855764}{On Markdown v.s.
reStructuredText}:
\sphinxstylestrong{Markdown} is easy to use; \sphinxstylestrong{reStructuredText} is more
extensible and powerful.

\item {} 
\sphinxhref{https://buildmedia.readthedocs.org/media/pdf/brandons-sphinx-tutorial/latest/brandons-sphinx-tutorial.pdf}{Brandon’s Sphinx Tutorial from PyCon
2013}

\item {} 
\sphinxhref{https://sphinx-tutorial.readthedocs.io/start/}{Sphinx Tutorial by Eric
Holscher} is the
best place to start. The \sphinxhref{https://github.com/ericholscher/sphinx-tutorial}{GitHub repo
itself} is a
very good example.

\item {} 
\sphinxhref{https://sphinx-themes.org/}{Sphinx Themes}

\end{itemize}

\item {} 
\sphinxhref{https://pandoc.org/}{pandoc - A universal document converter}
\begin{itemize}
\item {} 
If you need to convert files from one markup format into another,
\sphinxstylestrong{pandoc} is your swiss-army knife. e.g. It can convert
\sphinxstylestrong{reStructuredText} to/from \sphinxstylestrong{Markdown}.

\end{itemize}

\item {} 
\sphinxhref{https://github.com/brechtm/rinohtype}{rinohtype - The Python document
processor}
\begin{itemize}
\item {} 
\sphinxstylestrong{Rinohtype} is a document processor in the style of \sphinxstylestrong{LaTeX}.
It renders structured documents to PDF based on a document
template and a style sheet.

\item {} 
\sphinxhref{https://medium.com/@richdayandnight/a-simple-tutorial-on-how-to-document-your-python-project-using-sphinx-and-rinohtype-177c22a15b5b}{A Simple Tutorial on How to document your Python Project using
Sphinx and
Rinohtype}

\end{itemize}

\item {} 
\sphinxhref{https://github.com/numpy/numpydoc}{numpydoc \textendash{} Numpy’s Sphinx
extensions}
\begin{itemize}
\item {} 
\sphinxhref{https://numpydoc.readthedocs.io/en/latest/format.html}{The numpydoc docstring format
guide}

\end{itemize}

\item {} 
\sphinxhref{http://www.doxygen.nl/}{Doxygen - Generate documentation from source
code}
\begin{itemize}
\item {} 
\sphinxstylestrong{Doxygen} is the de facto standard tool for generating
documentation from annotated C++ sources, but it also supports
other popular programming languages such as C, Objective-C, C\#,
PHP, Java, Python, IDL.

\item {} 
\sphinxhref{http://galsim-developers.github.io/GalSim/index.html}{The Doxygen document site for Galsim is a very good
example}

\end{itemize}

\item {} 
\sphinxhref{https://readthedocs.org/}{Read the Docs - Technical documentation lives
here}
\begin{itemize}
\item {} 
Read the Docs simplifies software documentation by automating
building, versioning, and hosting of your docs for you.

\end{itemize}

\end{itemize}


\subsubsection{Test}
\label{\detokenize{resource/programing/python_write_your_project:test}}\begin{itemize}
\item {} 
\sphinxhref{https://docs.python-guide.org/writing/tests/}{Testing Your Code from the Hitchhiker’s Guide to
Python}
\begin{itemize}
\item {} 
A nice summary of multiple approaches of unit test in Python.

\end{itemize}

\item {} 
\sphinxhref{https://realpython.com/python-testing/}{Getting Started With Testing in Python from
RealPython}
\begin{itemize}
\item {} 
Another very nice introduction, convering \sphinxstylestrong{unittest},
\sphinxstylestrong{pytest}, and \sphinxstylestrong{nose}.

\end{itemize}

\item {} 
\sphinxhref{https://developer.lsst.io/python/testing.html}{LSST DM: Python Unit Testing
Guide}
\begin{itemize}
\item {} 
LSST DM standard is a very good example:LSST tests should be
written using the \sphinxstylestrong{unittest} framework, with default test
discovery, and should support being run using the \sphinxstylestrong{pytest} test
runner

\end{itemize}

\item {} 
\sphinxhref{https://docs.python.org/3/library/unittest.html}{unittest — Unit testing
framework}
\begin{itemize}
\item {} 
Basic unit test in Python. The \sphinxhref{https://docs.python.org/3/library/unittest.html\#assert-methods}{list of assertion methods is
here}

\end{itemize}

\item {} 
\sphinxhref{https://docs.pytest.org/en/latest/}{pytest - helps you write better
programs}
\begin{itemize}
\item {} 
The \sphinxstylestrong{pytest} framework makes it easy to write small tests, yet
scales to support complex functional testing for applications and
libraries.

\item {} 
\sphinxhref{https://docs.pytest.org/en/latest/example/}{Examples and customization tricks for
pytest}: this is
very useful.

\end{itemize}

\end{itemize}

-\sphinxhref{https://github.com/nose-devs/nose2}{nose2 - Nicer testing for
Python} * \sphinxstylestrong{nose2}’s purpose
is to extend unittest to make testing nicer and easier to understand.


\paragraph{Code Coverage}
\label{\detokenize{resource/programing/python_write_your_project:code-coverage}}\begin{itemize}
\item {} 
\sphinxhref{https://en.wikipedia.org/wiki/Code\_coverage}{Code coverage}:

\end{itemize}
\begin{quote}

In computer science, test coverage is a measure used to describe the
degree to which the source code of a program is executed when a
particular test suite runs. A program with high test coverage,
measured as a percentage, has had more of its source code executed
during testing, which suggests it has a lower chance of containing
undetected software bugs compared to a program with low test
coverage. \textendash{} Wikipedia
\end{quote}
\begin{itemize}
\item {} 
\sphinxhref{https://github.com/nedbat/coveragepy}{Coverage.py - Code coverage testing for
Python}
\begin{itemize}
\item {} 
\sphinxstylestrong{Coverage.py} measures code coverage, typically during test
execution. It uses the code analysis tools and tracing hooks
provided in the Python standard library to determine which lines
are executable, and which have been executed.

\item {} 
\sphinxhref{https://coverage.readthedocs.io/en/v4.5.x/\#quick-start}{Quick start
guide}

\item {} 
\sphinxhref{https://pytest-cov.readthedocs.io/en/latest/}{pytest has a pytest-cov
plugin}

\end{itemize}

\item {} 
\sphinxhref{https://github.com/codecov}{Codecov - Empower developers with tools to improve code quality and
testing}
\begin{itemize}
\item {} 
It is web service that improves your code review workflow and
quality. Free for open source. Plans starting at \$2.50/month per
repository. You can login with your \sphinxstylestrong{GitHub} or \sphinxstylestrong{Bitbucket}
account.

\item {} 
\sphinxhref{https://github.com/codecov/example-python}{Here is a Python example for
Codecov}

\end{itemize}

\end{itemize}


\subsubsection{Optimization}
\label{\detokenize{resource/programing/python_write_your_project:optimization}}\begin{itemize}
\item {} 
\sphinxhref{http://www.scipy-lectures.org/advanced/optimizing/}{Optimizing Python Code - Scipy Lecture
Notes}
\begin{itemize}
\item {} \begin{enumerate}
\def\theenumi{\arabic{enumi}}
\def\labelenumi{\theenumi .}
\makeatletter\def\p@enumii{\p@enumi \theenumi .}\makeatother
\item {} 
Make it work; 2: Make it work reliably; 3: Optimization

\end{enumerate}

\item {} 
No optimization without measuring: profiling and timing

\item {} 
Moving computation or memory allocation outside a for loop;
Vectorizing for loops; Broadcasting; Use in place operations; Be
easy on the memory: use views, and not copies;

\end{itemize}

\item {} 
\sphinxhref{https://developer.lsst.io/python/profiling.html}{LSST DM Python performance
profiling}
\begin{itemize}
\item {} 
Very good guide.

\end{itemize}

\item {} 
\sphinxhref{https://docs.python.org/3/library/profile.html}{The Python
Profilers}
\begin{itemize}
\item {} 
Python comes with a series of profiling tools. The most useful
ones are \sphinxstylestrong{cProfile}, \sphinxstylestrong{profile}, and \sphinxstylestrong{pstats} (convert
profiling results into a report)

\end{itemize}

\item {} 
\sphinxhref{https://julien.danjou.info/guide-to-python-profiling-cprofile-concrete-case-carbonara/}{Profiling Python using cProfile: a concrete
case}
\begin{itemize}
\item {} 
\sphinxstylestrong{cProfile} 对于发现程序中的瓶颈很有帮助

\end{itemize}

\item {} 
\sphinxhref{https://github.com/rkern/line\_profiler}{line\_profiler and kernprof - Line-by-line profiling for
Python}
\begin{itemize}
\item {} 
\sphinxstylestrong{line\_profiler} is a module for doing line-by-line profiling of
functions. \sphinxstylestrong{kernprof} is a convenient script for running either
line\_profiler or the Python standard library’s cProfile or profile
modules, depending on what is available.

\item {} 
Can use \sphinxstylestrong{cProfile} to identify “hotspot” (function that is the
“bottleneck”), then use \sphinxstylestrong{line\_profiler} to exame the issue
carefully.

\end{itemize}

\end{itemize}


\paragraph{Visualization}
\label{\detokenize{resource/programing/python_write_your_project:visualization}}\begin{itemize}
\item {} 
\sphinxhref{https://github.com/jrfonseca/gprof2dot}{gprof2dot - Converts profiling output to a dot
graph}
\begin{itemize}
\item {} 
A general tool to convert different profiling software output to a
dot graph.

\end{itemize}

\item {} 
\sphinxhref{https://jiffyclub.github.io/snakeviz}{SnakeViz - An in-browser Python profile
viewer}
\begin{itemize}
\item {} 
\sphinxstylestrong{SnakeViz} is a viewer for Python profiling data that runs as a
web application in your browser.

\end{itemize}

\item {} 
\sphinxhref{https://github.com/gak/pycallgraph}{pycallgraph - Python module that creates call graphs for Python
programs}
\begin{itemize}
\item {} 
No longer maintained by the original author, but still available
through a fork:
\sphinxhref{https://github.com/daneads/pycallgraph2}{pycallgraph2}

\end{itemize}

\end{itemize}


\section{C Programming Language}
\label{\detokenize{resource/programing/clang_basic:c-programming-language}}\label{\detokenize{resource/programing/clang_basic::doc}}

\subsection{Learning C}
\label{\detokenize{resource/programing/clang_basic:learning-c}}\begin{itemize}
\item {} 
\sphinxstylestrong{The best way to learn programming is, always, writing your own
code!}

\item {} 
\sphinxhref{http://www.drdobbs.com/cpp/why-code-in-c-anymore/240149452}{Why Code in C Anymore?….Instead of
C++}
\begin{itemize}
\item {} 
C still has a little advantage on the performance and portability.

\end{itemize}

\item {} 
\sphinxhref{http://www.cs.cornell.edu/courses/cs2022/2011sp/}{CS2022: Introduction to C at
Cornell}
\begin{itemize}
\item {} 
Lecture slides are available.

\end{itemize}

\item {} 
\sphinxhref{https://www.learn-c.org/}{learn-c.org free interactive C
tutorial}

\item {} 
\sphinxhref{https://github.com/fredsiika/30-seconds-of-c}{30-seconds-of-c}
\begin{itemize}
\item {} 
Curated collection of useful C Programming tutorials, snippets,
and projects that you can understand in 30 seconds or less

\end{itemize}

\end{itemize}


\subsection{Useful Libraries}
\label{\detokenize{resource/programing/clang_basic:useful-libraries}}

\subsubsection{Performance}
\label{\detokenize{resource/programing/clang_basic:performance}}\begin{itemize}
\item {} 
\sphinxhref{http://icps.u-strasbg.fr/~bastoul/local\_copies/lee.html}{Optimization of Computer Programs in
C}
\begin{itemize}
\item {} 
By Michael Lee. “It focuses on minimizing time spent by the CPU
and gives sample source code transformations that often yield
improvements. Memory and I/O speed improvements are also
discussed.”

\end{itemize}

\item {} 
\sphinxhref{https://people.cs.clemson.edu/~dhouse/courses/405/papers/optimize.pdf}{Tips for Optimizing C/C++
Code}
\begin{itemize}
\item {} 
Very practical and useful guides for optimizing C/C++.

\end{itemize}

\item {} 
\sphinxhref{https://github.com/microsoft/mimalloc}{mimalloc - mimalloc is a compact general purpose allocator with
excellent performance}
\begin{itemize}
\item {} 
By Microsoft. mimalloc (pronounced “me-malloc”) is a general
purpose allocator with excellent performance characteristics. It
is a drop-in replacement for malloc and can be used in other
programs without code changes.

\end{itemize}

\end{itemize}


\subsubsection{Numerical}
\label{\detokenize{resource/programing/clang_basic:numerical}}\begin{itemize}
\item {} 
\sphinxhref{https://www.gnu.org/software/gsl/}{GSL - GNU Scientific Library}
\begin{itemize}
\item {} 
The library provides a wide range of mathematical routines such as
random number generators, special functions and least-squares
fitting. There are over 1000 functions in total with an extensive
test suite.

\item {} 
The \sphinxhref{https://www.gnu.org/software/gsl/doc/html/index.html}{online reference manual can be found
here}.

\end{itemize}

\item {} 
\sphinxhref{http://www.fftw.org/}{FFTW}
\begin{itemize}
\item {} 
\sphinxstylestrong{FFTW} is a C subroutine library for computing the discrete
Fourier transform (DFT) in one or more dimensions, of arbitrary
input size, and of both real and complex data (as well as of
even/odd data, i.e. the discrete cosine/sine transforms or
DCT/DST).

\item {} 
The \sphinxhref{http://fftw.org/fftw3\_doc/}{online manual can be found
here}

\end{itemize}

\end{itemize}


\subsubsection{Astronomy Related}
\label{\detokenize{resource/programing/clang_basic:astronomy-related}}\begin{itemize}
\item {} 
\sphinxhref{https://github.com/healpy/cfitsio}{cfitsio - ANSI C routines for reading and writing FITS format data
files}

\end{itemize}


\subsection{Specific Topics}
\label{\detokenize{resource/programing/clang_basic:specific-topics}}

\subsubsection{Multiprocessing}
\label{\detokenize{resource/programing/clang_basic:multiprocessing}}

\subsubsection{GPU Enhancement}
\label{\detokenize{resource/programing/clang_basic:gpu-enhancement}}

\subsection{Code to Study}
\label{\detokenize{resource/programing/clang_basic:code-to-study}}\begin{itemize}
\item {} 
\sphinxhref{https://github.com/astromatic/sextractor}{sextractor - Extract catalogs of sources from astronomical
images}

\item {} 
\sphinxhref{https://github.com/astromatic/psfex}{psfex - Generate PSF super-tabulated
models}

\item {} 
\sphinxhref{https://github.com/LSSTDESC/CCL}{CCL - DESC Core Cosmology
Library}
\begin{itemize}
\item {} 
Also teaches you how to interact with Python.

\end{itemize}

\item {} 
\sphinxhref{https://github.com/manodeep/Corrfunc}{Corrfunc - fast correlation functions on the
CPU}
\begin{itemize}
\item {} 
\sphinxstylestrong{utils} are written in C; and wrapped in Python.

\end{itemize}

\end{itemize}


\subsubsection{On Interacting with Python}
\label{\detokenize{resource/programing/clang_basic:on-interacting-with-python}}\begin{itemize}
\item {} 
\sphinxhref{https://github.com/kbarbary/sep}{sep - Python and C library for source extraction and
photometry}
\begin{itemize}
\item {} 
Using \sphinxstylestrong{Cython}.

\end{itemize}

\item {} 
\sphinxhref{https://github.com/esheldon/cosmology}{cosmology - Some code for calculating cosmological
distances}
\begin{itemize}
\item {} 
By Erin Sheldon. Using \sphinxstylestrong{CPython}

\end{itemize}

\item {} 
\sphinxhref{https://github.com/esheldon/smatch}{smatch - Code to match points on the sphere using the healpix
scheme}
\begin{itemize}
\item {} 
By Erin Sheldon. Using \sphinxstylestrong{CPython}

\end{itemize}

\end{itemize}


\section{C++ Programming Language}
\label{\detokenize{resource/programing/cpp_basic:c-programming-language}}\label{\detokenize{resource/programing/cpp_basic::doc}}

\subsection{Learning C++}
\label{\detokenize{resource/programing/cpp_basic:learning-c}}\begin{itemize}
\item {} 
\sphinxstylestrong{The best way to learn programming is, always, writing your own
code!}

\item {} 
\sphinxhref{https://github.com/isocpp/CppCoreGuidelines}{CppCoreGuidelines - The C++ Core Guidelines are a set of
tried-and-true guidelines, rules, and best practices about coding in
C++}
\begin{itemize}
\item {} 
The \sphinxhref{http://isocpp.github.io/CppCoreGuidelines/CppCoreGuidelines\#main}{website version is
here}

\item {} 
The C++ Core Guidelines are a collaborative effort led by Bjarne
Stroustrup, much like the C++ language itself. They are the result
of many person-years of discussion and design across a number of
organizations.

\item {} 
\sphinxhref{https://github.com/lynnboy/CppCoreGuidelines-zh-CN}{A Chinese translation can be found
here}

\end{itemize}

\item {} 
\sphinxhref{https://www.learncpp.com/}{LearnCpp.com - Tutorials to help you master C++ and object-oriented
programming}
\begin{itemize}
\item {} 
LearnCpp.com is a free website devoted to teaching you how to
program in C++. Whether you’ve had any prior programming
experience or not, the tutorials on this site will walk you
through all the steps to write, compile, and debug your C++
programs, all with plenty of examples.

\end{itemize}

\item {} 
\sphinxhref{https://github.com/rigtorp/awesome-modern-cpp}{Awesome Modern C++ - A collection of resources on modern
C++}
\begin{itemize}
\item {} 
The goal is to collect a list of resources to help people learn
about and leverage modern C++11 and beyond. The \sphinxhref{https://awesomecpp.com/}{website is
here}

\end{itemize}

\item {} 
\sphinxhref{https://github.com/changkun/modern-cpp-tutorial}{Modern CPP Tutorial -
C++11/14/17}
\begin{itemize}
\item {} 
This is a \sphinxstylestrong{Chinese} on-line book: \textless{}高速上手 C++11/14/17\textgreater{}

\end{itemize}

\item {} 
\sphinxhref{https://github.com/AnthonyCalandra/modern-cpp-features}{modern-cpp-features - A cheatsheet of modern C++ language and
library
features}
\begin{itemize}
\item {} 
Descriptions and examples of new features in C++

\end{itemize}

\end{itemize}


\subsubsection{For Python Coders}
\label{\detokenize{resource/programing/cpp_basic:for-python-coders}}\begin{itemize}
\item {} 
\sphinxhref{https://cs.slu.edu/~goldwasser/publications/python2cpp.pdf}{A Transition Guide from Python 2.x to
C++}

\end{itemize}


\subsection{Useful Libraries}
\label{\detokenize{resource/programing/cpp_basic:useful-libraries}}

\subsubsection{Performance}
\label{\detokenize{resource/programing/cpp_basic:performance}}\begin{itemize}
\item {} 
\sphinxhref{https://people.cs.clemson.edu/~dhouse/courses/405/papers/optimize.pdf}{Tips for Optimizing C/C++
Code}
\begin{itemize}
\item {} 
Very practical and useful guides for optimizing C/C++.

\end{itemize}

\item {} 
\sphinxhref{https://github.com/fenbf/AwesomePerfCpp}{AwesomePerfCpp - A curated list of awesome C/C++ performance
optimization resources}
\begin{itemize}
\item {} 
Including talks, articles, libraries, and tools

\end{itemize}

\item {} 
\sphinxhref{https://github.com/cpp-taskflow/cpp-taskflow}{cpp-taskflow - Modern C++ Parallel Task Programming
Library}
\begin{itemize}
\item {} 
A fast C++ header-only library to help you quickly write parallel
programs with complex task dependencies

\end{itemize}

\item {} 
\sphinxhref{https://github.com/microsoft/mimalloc}{mimalloc - mimalloc is a compact general purpose allocator with
excellent performance}
\begin{itemize}
\item {} 
By Microsoft. mimalloc (pronounced “me-malloc”) is a general
purpose allocator with excellent performance characteristics. It
is a drop-in replacement for malloc and can be used in other
programs without code changes.

\end{itemize}

\end{itemize}


\subsubsection{Numerical}
\label{\detokenize{resource/programing/cpp_basic:numerical}}\begin{itemize}
\item {} 
\sphinxhref{https://www.gnu.org/software/gsl/}{GSL - GNU Scientific Library}
\begin{itemize}
\item {} 
The library provides a wide range of mathematical routines such as
random number generators, special functions and least-squares
fitting. There are over 1000 functions in total with an extensive
test suite.

\item {} 
The \sphinxhref{https://www.gnu.org/software/gsl/doc/html/index.html}{online reference manual can be found
here}.

\end{itemize}

\item {} 
\sphinxhref{http://eigen.tuxfamily.org/index.php?title=Main\_Page}{Eigen - a C++ template library for linear algebra: matrices,
vectors, numerical solvers, and related
algorithms}

\item {} 
\sphinxhref{http://arma.sourceforge.net/}{Armadillo - C++ library for linear algebra \& scientific
computing}
\begin{itemize}
\item {} 
\sphinxstylestrong{Armadillo} is a high quality linear algebra library (matrix
maths) for the C++ language, aiming towards a good balance between
speed and ease of use

\end{itemize}

\item {} 
\sphinxhref{https://github.com/QuantStack/xtensor}{xtensor - Multi-dimensional arrays with broadcasting and lazy
computing}
\begin{itemize}
\item {} 
\sphinxstylestrong{xtensor} is a C++ library meant for numerical analysis with
multi-dimensional array expressions. \sphinxstylestrong{xtensor} can be used to
process NumPy data structures inplace using Python’s buffer
protocol.

\end{itemize}

\end{itemize}


\subsubsection{Optimization}
\label{\detokenize{resource/programing/cpp_basic:optimization}}\begin{itemize}
\item {} 
\sphinxhref{https://github.com/kthohr/optim}{OptimLib - a lightweight C++ library of numerical optimization
methods for nonlinear functions}
\begin{itemize}
\item {} 
A C++11 library of local and global optimization algorithms, as
well as root finding techniques, supporting a large number of
algorithms.

\end{itemize}

\item {} 
\sphinxhref{https://github.com/stevengj/nlopt}{nlopt - library for nonlinear
optimization}
\begin{itemize}
\item {} 
\sphinxstylestrong{NLopt} is a library for nonlinear local and global
optimization, for functions with and without gradient information.

\end{itemize}

\item {} 
\sphinxhref{https://github.com/ceres-solver/ceres-solver}{ceres-solver - open source C++ library for modeling and solving
large, complicated optimization
problems}
\begin{itemize}
\item {} 
It can be used to solve Non-linear Least Squares problems with
bounds constraints and general unconstrained optimization
problems. \sphinxhref{http://ceres-solver.org/features.html}{Used by Google for good
reasons}.

\end{itemize}

\item {} 
\sphinxhref{https://github.com/PatWie/CppNumericalSolvers}{CppNumericalSolvers - L-BFGS-B for TensorFlow or pure C++11 and
other optimization
methods}
\begin{itemize}
\item {} 
A \sphinxstylestrong{header-only} library with bindings to C++, \sphinxstylestrong{TensorFlow} and
\sphinxstylestrong{Matlab}.

\item {} 
Easy to use, and support a list of algorithms.

\end{itemize}

\end{itemize}


\subsubsection{Astronomy Related}
\label{\detokenize{resource/programing/cpp_basic:astronomy-related}}\begin{itemize}
\item {} 
\sphinxhref{https://heasarc.gsfc.nasa.gov/fitsio/CCfits/}{CCfits - object oriented interface to the cfitsio
library}
\begin{itemize}
\item {} 
It is designed to make the capabilities of cfitsio available to
programmers working in C++. It is written in ANSI C++ and
implemented using the C++ Standard Library with namespaces,
exception handling, and member template functions.

\end{itemize}

\end{itemize}


\subsection{Code to Study}
\label{\detokenize{resource/programing/cpp_basic:code-to-study}}\begin{itemize}
\item {} 
\sphinxhref{https://github.com/GalSim-developers/GalSim}{GalSim - The modular galaxy image simulation
toolkit}
\begin{itemize}
\item {} 
\sphinxstylestrong{GalSim} is a very sophisticated system that pretty much
includes everything you need to know about optics, observational
astronomy, photometry, models of galaxies, and weak lensing.

\item {} 
It also teachs you how to communicate between C++ and Python.

\end{itemize}

\item {} 
\sphinxhref{https://github.com/perwin/imfit}{imfit - an open-source astronomical image-fitting
program}
\begin{itemize}
\item {} 
Everything on modeling galaxies using least-chi-square or MCMC
methods.

\end{itemize}

\item {} 
\sphinxhref{https://github.com/ICRAR/libprofit}{libprofit - low-level C++ library that produces images based on
different luminosity profiles}
\begin{itemize}
\item {} 
Light-weight library to learn about modeling galaxies and image
convolution.

\end{itemize}

\item {} 
\sphinxhref{https://github.com/rmjarvis/TreeCorr}{TreeCorr - Code for efficiently computing 2-point and 3-point
correlation functions}

\end{itemize}


\section{The Julia Programming Language}
\label{\detokenize{resource/programing/julia_basic:the-julia-programming-language}}\label{\detokenize{resource/programing/julia_basic::doc}}\begin{itemize}
\item {} 
\sphinxstylestrong{Still highly imcomplete}

\item {} 
High-level programming language for future astronomer….maybe?

\end{itemize}


\bigskip\hrule\bigskip



\subsection{Basic Information}
\label{\detokenize{resource/programing/julia_basic:basic-information}}\begin{itemize}
\item {} 
\sphinxhref{https://julia.mit.edu/}{The JuliaLab@MIT}

\item {} 
\sphinxhref{https://julialang.org/}{The Julia Programming Language}
\begin{itemize}
\item {} 
Julia was designed from the beginning for high performance.

\item {} 
The source code can be found on
\sphinxhref{https://github.com/JuliaLang/julia}{Github}

\end{itemize}

\item {} 
\sphinxhref{https://docs.julialang.org/en/v1/}{Julia 1.1 Documentation}

\item {} 
\sphinxhref{https://juliaobserver.com/packages}{Julia Observer}
\begin{itemize}
\item {} 
Good place to search for useful tools.

\end{itemize}

\item {} 
\sphinxhref{https://github.com/svaksha/Julia.jl}{Julia.jl - Curated decibans of Julia programming
language}
\begin{itemize}
\item {} 
Check out the
\sphinxhref{https://github.com/svaksha/Julia.jl/blob/master/Space-Science.md}{Space-Science.md}
for astro-related packages.

\end{itemize}

\end{itemize}


\subsubsection{Distributions}
\label{\detokenize{resource/programing/julia_basic:distributions}}\begin{itemize}
\item {} 
\sphinxhref{https://juliacomputing.com/products/juliapro.html}{JuliaPro - A Julia distribution crafted for your
success}

\end{itemize}


\subsubsection{Learning Materials}
\label{\detokenize{resource/programing/julia_basic:learning-materials}}\begin{itemize}
\item {} 
\sphinxhref{https://github.com/chrisvoncsefalvay/learn-julia-the-hard-way}{Learn Julia the hard
way!}
\begin{itemize}
\item {} 
GitBook for learning Julia language.

\end{itemize}

\item {} 
\sphinxhref{https://github.com/JuliaDocs/Julia-Cheat-Sheet}{Julia Cheat
Sheet}
\begin{itemize}
\item {} 
The \sphinxhref{https://juliadocs.github.io/Julia-Cheat-Sheet/}{one page
version}

\end{itemize}

\item {} 
\sphinxhref{https://github.com/JuliaCN/JuliaZH.jl}{JuliaZH.jl -
Julia语言中文文档}

\end{itemize}


\subsection{Useful Packages}
\label{\detokenize{resource/programing/julia_basic:useful-packages}}

\subsubsection{Basic}
\label{\detokenize{resource/programing/julia_basic:basic}}\begin{itemize}
\item {} 
\sphinxhref{https://github.com/JuliaData/DataFrames.jl}{DataFrames.jl - In-memory tabular data in
Julia}
\begin{itemize}
\item {} 
Tools for working with tabular data in Julia.

\end{itemize}

\item {} 
\sphinxhref{https://github.com/JuliaData/CSV.jl}{CSV.jl - Utility library for working with CSV and other delimited
files in the Julia programming
language}
\begin{itemize}
\item {} 
A fast, flexible delimited file reader/writer for Julia.

\end{itemize}

\item {} 
\sphinxhref{https://github.com/GenieFramework/Genie.jl}{Genie.jl - The highly productive Julia web
framework}
\begin{itemize}
\item {} 
Genie is a full-stack MVC web framework that provides a
streamlined and efficient workflow for developing modern web
applications.

\end{itemize}

\end{itemize}


\subsubsection{Visualization}
\label{\detokenize{resource/programing/julia_basic:visualization}}\begin{itemize}
\item {} 
\sphinxhref{https://github.com/JuliaPlots/Plots.jl}{Plots.jl - Powerful convenience for Julia visualizations and data
analysis}
\begin{itemize}
\item {} 
Documents can be found
\sphinxhref{http://docs.juliaplots.org/latest/}{here}

\end{itemize}

\item {} 
\sphinxhref{https://github.com/GiovineItalia/Gadfly.jl}{Gadfly.jl - Crafty statistical graphics for
Julia}
\begin{itemize}
\item {} 
Gadfly is a plotting and data visualization system written in
Julia.

\end{itemize}

\end{itemize}


\subsubsection{Optimization}
\label{\detokenize{resource/programing/julia_basic:optimization}}\begin{itemize}
\item {} 
\sphinxhref{https://github.com/JuliaNLSolvers/Optim.jl}{Optim.jl - Optimization functions for
Julia}
\begin{itemize}
\item {} 
Optim.jl is a package for univariate and multivariate optimization
of functions.

\end{itemize}

\item {} 
\sphinxhref{https://github.com/JuliaNLSolvers/LsqFit.jl}{LsqFit.jl - Simple curve fitting in
Julia}
\begin{itemize}
\item {} 
The LsqFit package is a small library that provides basic
least-squares fitting in pure Julia under an MIT license.

\end{itemize}

\end{itemize}


\subsubsection{Statistics}
\label{\detokenize{resource/programing/julia_basic:statistics}}\begin{itemize}
\item {} 
\sphinxhref{https://github.com/TuringLang/Turing.jl}{Turing.jl - The Turing language for probabilistic
programming}
\begin{itemize}
\item {} 
Turing allows the user to write models in standard Julia syntax,
and provide a wide range of sampling-based inference methods for
solving problems across probabilistic machine learning, Bayesian
statistics and data science etc.

\end{itemize}

\end{itemize}


\subsubsection{Machine Learning}
\label{\detokenize{resource/programing/julia_basic:machine-learning}}\begin{itemize}
\item {} 
\sphinxhref{https://github.com/FluxML/Flux.jl}{Flux.jl - the ML library that doesn’t make you
tensor}
\begin{itemize}
\item {} 
Flux is an elegant approach to machine learning. It’s a 100\%
pure-Julia stack, and provides lightweight abstractions on top of
Julia’s native GPU and AD support.

\end{itemize}

\item {} 
\sphinxhref{https://github.com/denizyuret/Knet.jl}{knet.jl - Koç University deep learning
framework}
\begin{itemize}
\item {} 
Knet (pronounced “kay-net”) is the Koç University deep learning
framework implemented in Julia by Deniz Yuret and collaborators

\end{itemize}

\end{itemize}


\section{Resources and Tools for Numerical Method and Scientific Computing}
\label{\detokenize{resource/programing/numerical_method:resources-and-tools-for-numerical-method-and-scientific-computing}}\label{\detokenize{resource/programing/numerical_method::doc}}

\subsection{Resources}
\label{\detokenize{resource/programing/numerical_method:resources}}\begin{itemize}
\item {} 
\sphinxhref{http://numerical.recipes/}{Numerical Recipes - The Art of Scientific
Computing}
\begin{itemize}
\item {} 
The famous book. Algorithms available in Fortran, C, and C++.

\item {} 
A \sphinxhref{https://www.cec.uchile.cl/cinetica/pcordero/MC\_libros/NumericalRecipesinC.pdf}{Second Edition of the C
version}
is available here.

\item {} 
\sphinxhref{https://www.astro.umd.edu/~ricotti/NEWWEB/teaching/ASTR415/InClassExamples/NR3/legacy/nr2/CPP\_211/index.htm}{Numerical Recipes 2nd ed. ANSI C++
Files}

\end{itemize}

\item {} 
\sphinxhref{https://github.com/nschloe/awesome-scientific-computing}{Awesome Scientific Computing - Curated list of awesome software for
numerical
analysis}
\begin{itemize}
\item {} 
Scientific computing and numerical analysis are research fields
that aim to provide methods for solving large-scale problems from
various areas of science with the help of computers.

\end{itemize}

\end{itemize}


\subsection{Tools}
\label{\detokenize{resource/programing/numerical_method:tools}}\begin{itemize}
\item {} 
\sphinxhref{https://github.com/nschloe/meshio}{meshio - I/O for various mesh
formats}
\begin{itemize}
\item {} 
By Nico Schlömer. There are various mesh formats available for
representing unstructured meshes. meshio can read and write a lot
of them, and can convert among them.

\end{itemize}

\item {} 
\sphinxhref{https://github.com/nschloe/quadpy}{quadpy - Numerical integration (quadrature, cubature) in
Python}
\begin{itemize}
\item {} 
By Nico Schlömer. More than 1500 numerical integration schemes for
line segments, circles, disks, triangles, quadrilaterals, spheres,
balls, tetrahedra, hexahedra, wedges, pyramids, n-spheres,
n-balls, n-cubes, n-simplices, and the 1D/2D/3D/nD spaces with
weight functions exp(-r) and exp(-r2) for fast integration of
real-, complex-, and vector-valued functions.

\end{itemize}

\item {} 
\sphinxhref{https://github.com/nschloe/optimesh}{optimesh - Mesh optimization, mesh
smoothing}
\begin{itemize}
\item {} 
By Nico Schlömer. Several mesh smoothing/optimization methods with
one simple interface. Only works for triangular meshes

\end{itemize}

\end{itemize}


\chapter{中文内容 (Material in Chinese)}
\label{\detokenize{index:material-in-chinese}}

\section{如何正确地踏上天文科研的不归路}
\label{\detokenize{resource/research/getting_started_cn:id1}}\label{\detokenize{resource/research/getting_started_cn::doc}}\begin{itemize}
\item {} 
\sphinxstylestrong{作者}:黄崧 (UCSC)

\item {} 
\sphinxstylestrong{最近更新}: 2019年7月

\end{itemize}

\begin{figure}[htbp]
\centering
\capstart

\noindent\sphinxincludegraphics{{/Users/songhuang/Dropbox/work/project/taotie/doc/_build/doctrees/images/507d4a74bba11789c49c00786c01746085935377/phd051017s}.gif}
\caption{Ah..research life}\label{\detokenize{resource/research/getting_started_cn:id23}}\end{figure}
\begin{itemize}
\item {} 
这份“指南”是\sphinxhref{https://dr-guangtou.github.io/taotie/}{饕餮天文科研公共书签}的一部分。它虽不能保证你成为优秀的天文学家,也不能帮你解决具体的科学问题,但它列出了一些有助于你开启科研生涯的,具有\sphinxstylestrong{可操作性}的建议。虽然每个人的经历经验大不相同,但大体上,这份列表的作用是总结那些让多数科学工作者在某个阶段感到“要是我当初知道这个就好了!”的经验和教训。

\item {} 
虽然这份指南是中文的,但是其中的所有链接内容均是英语的。英语依然是天文学界和科学界的“lingua
franca” (通用语言),不畏惧英语阅读是迈入科研的重要一步。

\item {} 
我们希望这份指南可以得到社区中不同背景和领域的科学工作者的共同维护,帮助它超越个人经验的局限。我们也希望这份指南可以被经常更新和纠错。

\item {} 
除了这些大多不局限于天文领域的内容,我们还整理了\sphinxhref{https://github.com/dr-guangtou/taotie/blob/master/astro/astro\_readme.md}{在线免费天文学教材以及对天文科研有实际帮助的各种指南},和\sphinxhref{https://github.com/dr-guangtou/taotie/blob/master/astro/astro\_research\_basic.md}{日常天文科研中常用工具和服务的书签}。这两份文档也在不断更新中,但我们也很推荐刚刚开始天文科研的朋友作为参考。

\end{itemize}


\subsection{审视你的动机与使命}
\label{\detokenize{resource/research/getting_started_cn:id2}}\begin{itemize}
\item {} 
成为一名科学工作者意味着你将代表人类勇敢地踏入未知。这样的旅程难免艰难险阻,但沿途风景却从不让人失望。在开始你的旅程前,希望你可以诚挚地考察自己的动机与决心,质问自己的每一个职业与个人选择,认真地思考自己的“使命”
(Mission)。

\item {} 
\sphinxhref{https://arxiv.org/abs/1805.09963}{选择你自己的冒险 - 用价值观作为导向来规划你的职业 作者:Lucianne
Walkowicz}
\begin{itemize}
\item {} 
这份来自天文学家Lucianne
Walkowics的职业建议值得推荐给每一个立志从事天文科研的人。在读完后,认真地思考一下自己的“\sphinxstylestrong{职业使命}”,并将其整理进入个人简历,并经常审视,更新是不错的选择。

\end{itemize}

\item {} 
同时,无论你是刚走出校园的学生,亦或已经是经验丰富的科学工作者,虽然求职道路都注定不会轻松,但请记得你有着丰富和多样的职业选择。学习天文或者在一段时间内从事天文科研,\sphinxstylestrong{并不意味着你必须要成为职业天文学家或者有义务将科研进行到底},没有必要将各种科研外的职业道路视为“退路”,它们可以是属于你的新的冒险和同样精彩的人生。
\begin{itemize}
\item {} 
即便在科研生涯的早期,也推荐你开始了解关于学术求职和其他职业规划的基本信息和经验。\sphinxstylestrong{饕餮}也试图提供\sphinxhref{https://github.com/dr-guangtou/taotie/blob/master/research/job\_and\_career.md}{一些基本的资料供你参考}

\item {} 
从一开始就为自己建立一份符合规范的\sphinxhref{https://gradschool.cornell.edu/academic-progress/pathways-to-success/prepare-for-your-career/take-action/resumes-and-cvs/}{个人简历}
并阶段性地审视和更新是非常好的习惯。

\end{itemize}

\end{itemize}


\subsection{维护社区的包容与开放}
\label{\detokenize{resource/research/getting_started_cn:id3}}\begin{itemize}
\item {} 
成为一名科学工作者也意味着你进入了一个繁荣的社区。旅程中,你理应得到社区公平的对待,也会分享到社区的资源与经验。与此同时,你也有义务用规范的职业道德约束自己的言行,并为构建一个\sphinxstylestrong{包容与开放}的社区贡献自己的一点力量。

\item {} 
职业道德是一个公平开放社区的基础。请抽空阅读\sphinxhref{https://aas.org/ethics}{美国天文学会职业道德规范}
或者
\sphinxhref{http://asa.astronomy.org.au/code\_of\_ethics.php}{澳大利亚天文学会职业道德规范}
作为参考。
\begin{itemize}
\item {} 
同时也请考虑阅读\sphinxhref{https://aas.org/policies/anti-harassment-policy-aas-division-meetings-activities}{美国天文学会反骚扰的政策规范}
并不断审视和约束自己在学术活动中的一举一动。

\end{itemize}

\item {} 
宇宙包容万物,但遗憾的是,研究宇宙的人依然不能免于一些人类社会里常见的先入为主与认知偏差。你应当有勇气去了解自己所处社区中的各种问题,包括任何人都不能避免的偏见,并试图减少它们对于科研环境的影响。\sphinxstylestrong{AstroBetter}网站收藏整理了一系列\sphinxhref{http://www.astrobetter.com/wiki/Diversity}{帮助天文学社区能分享平等机会的文章与资源}可供你参考。
\begin{itemize}
\item {} 
在具体问题上,\sphinxhref{http://www.astronomyallies.com/Astronomy\_Allies/Welcome.html}{由天文学家组成的反对性骚扰和性别歧视的联盟 -
AstronomyAllies},以及\sphinxhref{https://astro-outlist.github.io/}{由天文学和天体物理学从业者组成的“Outlist”}
也许能对你有所帮助。

\end{itemize}

\item {} 
\sphinxstylestrong{AstroOutlist}网站上的这段话,值得每一个科研工作者都时刻记在心里:
\textgreater{}
“作为从事天文与天体物理科研的职业人士,无论我们是学生,教职人员,普通员工,资料管理员,还是身处任何其他职位,我们都坚信我们的工作环境应该仅仅由我们作为学生或者科研工作者的职业能力来塑造,并且不受任何个人偏见的干扰。只有所有成员能够在不受歧视和骚扰的情况下,在开放与包容的环境中沟通,我们才有可能培养出高效而健康的职业氛围。“

\end{itemize}


\subsection{ORCID个人标识以及Google学术}
\label{\detokenize{resource/research/getting_started_cn:orcidgoogle}}\begin{itemize}
\item {} 
\sphinxhref{https://orcid.org/}{ORCID - 开放式科研人员与投稿身份识别码}
提供了一个永久性的数字标识。这个ID系统可以方便地收集你的科研成果,并将你和其他研究人员区分开。在广泛只使用姓标记科学成果的英语世界,这一点对频繁重姓的中国科学工作者尤其重要。\sphinxstylestrong{ORCID}还有助于将你的科研记录与提交论文或者申请基金这样的科研需求整合起来,确保你的成果得到正确的认可。
\begin{itemize}
\item {} 
\sphinxstylestrong{从一开始就为自己建立一个}ORCID\sphinxstylestrong{并保持更新是一个不错的主意}。

\end{itemize}

\item {} 
\sphinxhref{https://scholar.google.com}{谷歌学术}
是另一个常用的展示和整理自己科学成果的平台。它还可以协助你追踪自己的引用记录。

\end{itemize}


\subsection{选择趁手的编程语言}
\label{\detokenize{resource/research/getting_started_cn:id4}}\begin{itemize}
\item {} 
熟练掌握科研相关的计算机技能,熟练使用至少一门编程语言早已是成为一名科研工作者的最低要求。对天文学工作者来说,熟练使用命令行环境,熟悉像\sphinxstylestrong{Linux}或者\sphinxstylestrong{MacOSX}这样基于\sphinxstylestrong{Unix}的操作系统是很有必要的。\sphinxstylestrong{饕餮}中也提供了\sphinxhref{https://github.com/dr-guangtou/taotie/blob/master/research/computer\_basics.md}{相关资料}

\item {} 
目前,\sphinxhref{https://www.python.org/}{Python}
已是天文,天体物理,和宇宙学研究的“标准”语言。如果你刚开始科研训练,\sphinxstylestrong{把Python作为第一门认真学习的语言也是最好的选择},不仅因为Python不错的性能,交互的特性,更因为社区里已经有一大批成熟的工具供你参考使用
(比如社区共同维护的{}`Astropy工具库 \textless{}\sphinxurl{https://www.astropy.org/}\textgreater{}{}`\_\_
以及其旗下的一系列工具包)。

\item {} 
\sphinxstylestrong{饕餮}中也为你提供了一系列关于Python的参考资料,比如关于\sphinxhref{https://github.com/dr-guangtou/taotie/blob/master/programing/python\_basic.md}{Python的基础学习资料}
,\sphinxhref{https://github.com/dr-guangtou/taotie/blob/master/programing/python\_performance.md}{关于如何提升Python程序性能},
\sphinxhref{https://github.com/dr-guangtou/taotie/blob/master/programing/python\_optimazaton.md}{关于利用Python进行模型拟合和优化},
\sphinxhref{https://github.com/dr-guangtou/taotie/blob/master/programing/python\_statistics.md}{关于利用Python进行统计分析和建模},
\sphinxhref{https://github.com/dr-guangtou/taotie/blob/master/programing/python\_visualization.md}{关于Python中的数据可视化},
\sphinxhref{https://github.com/dr-guangtou/taotie/blob/master/programing/python\_write\_yourown\_project.md}{以及如何开始属于你自己的Python项目}的资料。

\item {} 
在对计算性能和效率有更高要求的应用场景,\sphinxhref{https://en.wikipedia.org/wiki/C\_(programming\_language)}{C}
以及 \sphinxhref{https://en.wikipedia.org/wiki/C\%2B\%2B}{C++}
这样需要编译的编程语言依然有很大的需求。比如在数值模拟和大数据处理等领域里,你经常可以发现由\sphinxstylestrong{C}和\sphinxstylestrong{C++}写就的底层和核心程序。当然,现在很多这些工具都提供了便于和用户进行交互的\sphinxstylestrong{Python}外层程序。学习基础的\sphinxstylestrong{C}和\sphinxstylestrong{C++}编程,至少可以读懂程序会对科研很有帮助。\sphinxstylestrong{饕餮}中也提供了关于\sphinxhref{https://github.com/dr-guangtou/taotie/blob/master/programing/clang\_basic.md}{C语言}和\sphinxhref{https://github.com/dr-guangtou/taotie/blob/master/programing/cpp\_basic.md}{C++语言}
的基础资料。\sphinxhref{http://fortranwiki.org/fortran/show/HomePage}{Fortran}是另一门历史悠久且主要用于数值计算的语言。目前\sphinxstylestrong{Fortran}在天文中的影响日渐降低,但依然可以在一些依赖高效计算的场景下遇到它。

\item {} 
\sphinxhref{https://julialang.org/}{Julia语言}
是另一种近期兴起的高级编程语言。\sphinxstylestrong{Julia}继承了\sphinxstylestrong{Python}的交互与易用特性,同时又试图解决\sphinxstylestrong{Python}语言的一些基本难题。虽然\sphinxstylestrong{Julia}还是一门非常年轻的语言,但是在很多\sphinxhref{https://discourse.julialang.org/t/julia-motivation-why-werent-numpy-scipy-numba-good-enough/2236}{关键问题上,已经展现出了对Python的优势和潜力}。如果你已经有了\sphinxstylestrong{Python}的编程基础,学习\sphinxstylestrong{Julia}应该不难。当然,目前专门针对\sphinxstylestrong{Julia}的天文工具包还不算健全,可能意味着你要更多依赖自己的探索。\sphinxstylestrong{饕餮}中也提供了\sphinxhref{https://github.com/dr-guangtou/taotie/blob/master/programing/julia\_basic.md}{关于Julia的基础资料以及目前和天文相关的各种工具}

\item {} 
走进大数据时代的天文学自然离不开优秀的统计工具和模型。而\sphinxhref{https://www.r-project.org/about.html}{R语言}是一种在统计研究界通用的高级交互式语言。尽管\sphinxstylestrong{Python}在统计方面的功能日趋健全,但很多前沿的统计模型依然会首选\sphinxstylestrong{R}语言。

\item {} 
\sphinxhref{https://en.wikipedia.org/wiki/IDL\_(programming\_language)}{IDL-交互式数据语言}曾经是天文数据分析的主力工具,但随着\sphinxstylestrong{Python}的崛起已经迅速失宠。不过,由于历史原因,很多仪器的数据处理流程以及很多项目的默认数据分析平台依然是基于\sphinxstylestrong{IDL}的。在这里,我们\sphinxstylestrong{不再建议刚进入天文学的人学习}IDL\sphinxstylestrong{语言}。但如果在工作中不可避免的要使用它,理解\sphinxstylestrong{IDL}工具并上手\sphinxhref{http://mathesaurus.sourceforge.net/idl-numpy.html}{并不是很难的事情}。值得提醒的是,\sphinxstylestrong{IDL}不是免费工具,且授权价格不菲,请保证自己或者所在机构能够合法使用\sphinxstylestrong{IDL}工具。

\item {} 
尽管在天文界使用有限,\sphinxhref{https://www.mathworks.com/products/matlab.html}{MATLAB}在数据处理和数值模拟方面,\sphinxhref{http://www.wolfram.com/mathematica/}{Mathematica}
在符号计算和理论研究方面都有着广泛的应用。同样,这两种工具均不是免费获取的,请确保自己拥有正确的使用授权。如果你感兴趣他们的功能但没有授权,开源工具\sphinxhref{http://www.gnu.org/software/octave/}{Octave},
\sphinxhref{https://www.scilab.org/}{Scilab},
以及\sphinxstylestrong{Python}下的\sphinxhref{https://www.sympy.org/en/index.html}{符号计算系统sympy}
都可供考虑。

\item {} 
在这个展示和沟通科研都离不开网络的时代,掌握一点点面向网络的编程技能是值得推荐的。\sphinxhref{https://www.w3schools.com/html/}{HTML},
\sphinxhref{https://www.w3schools.com/css/}{CSS},
和\sphinxhref{https://www.javascript.com/}{Javascript}
是这方面最基本的编程语言。由于应用广泛,围绕着他们的资料和工具层出不穷,在这里就不赘述了。

\item {} 
另外,也希望你可以
\sphinxstylestrong{永远不因为别人使用的编程语言和工具去带有偏见地评价他人的工作}。目前依然有很多科学工作者使用\sphinxstylestrong{IDL},
\sphinxhref{https://en.wikipedia.org/wiki/IRAF}{IRAF},
\sphinxstylestrong{Fortran}做出优秀的科学工作,用\sphinxhref{https://www.astro.princeton.edu/~rhl/sm/}{SuperMongo},
\sphinxhref{http://www.gnuplot.info/}{gnuplot}
这样的工具进行数据可视化。继续使用这些工具有着多种多样的现实考虑,而且它们依然可以为杰出的科研成果做贡献。工具只是工具,请努力让自己带着积极和建设性的视角来进行科研工作。

\item {} 
\sphinxhref{https://jupyter.org/}{Jupyter项目}是一个开源的交互式科研平台。通过其提供的\sphinxhref{https://jupyter.org/try}{Jypyter笔记本},科学工作者可以在本地或者远程计算机上,\sphinxhref{https://jupyter.org/try}{在Python,
Julia, R, C++,
Ruby等多种语言选择下进行交互式的工作}。笔记本可以方便地被保存,移植,和分享。

\end{itemize}


\subsubsection{杂七杂八}
\label{\detokenize{resource/research/getting_started_cn:id5}}\begin{itemize}
\item {} 
\sphinxhref{http://people.duke.edu/~ccc14/sta-663-2019/}{Duke大学统计学课程STA663的在线文档收集了大量关于科研所需的计算机技能的资料}

\item {} 
\sphinxhref{https://stackoverflow.com/}{StackOverflow}
是一个社区维护的,关于编程和计算机的知识库与问答平台。在未来无数个被程序中的bug困扰的不眠夜晚,你都会意识到\sphinxstylestrong{StackOverflow}也许是你最知心的朋友。你可以大方地在这里寻求帮助,也请考虑在这里帮助别人。

\end{itemize}


\subsection{整理你的科研项目}
\label{\detokenize{resource/research/getting_started_cn:id6}}\begin{itemize}
\item {} 
每一个科研项目都会积累大量的笔记,程序,图表,文献,以及草稿等资料。应该从科研初期就不断思考和探索如何整理好自己的科研项目。这样不仅有助于保持工作效率,也方便你分享科研成果给社区。

\item {} 
当前,业界流行的在线版本控制代码托管平台,如\sphinxhref{https://github.com/}{GitHub},
或 \sphinxhref{https://about.gitlab.com/}{GitLab},
\sphinxhref{https://bitbucket.org/}{bitbucket},
以及\sphinxhref{https://coding.net/git}{coding})
都可以帮助你整理科研项目。它们通过\sphinxhref{https://git-scm.com/}{git}
或者\sphinxhref{https://www.mercurial-scm.org/}{mercurial}
进行版本控制,也实现了本地关键文件的在线备份,还提供了分享科学的平台。(\sphinxstylestrong{git}并不难上手,网上有大量教学资料,也有像\sphinxhref{https://github.github.com/training-kit/downloads/github-git-cheat-sheet.pdf}{这样的命令速查文档})
\begin{itemize}
\item {} 
\sphinxhref{https://github.com/github/hub}{hub}
可以帮助你在命令行和\sphinxstylestrong{Github}进行交互。

\item {} 
\sphinxhref{https://moox.io/blog/keep-in-sync-git-repos-on-github-gitlab-bitbucket/}{将你的项目或者程序同步到几个不同的托管平台也是很容易的}.
(注意目前**gitlab**使用的是\sphinxstylestrong{v4} API接口).
基本上你要做的就是保持你的项目库在不同平台上命名一致,然后给本地库添加多个\sphinxstylestrong{remote}目标即可.

\end{itemize}

\item {} 
目前,像\sphinxstylestrong{GitHub}这样的平台都\sphinxhref{https://github.blog/2019-06-06-generate-new-repositories-with-repository-templates}{允许你从一个模板开始你的项目}:
你可以自己设计一个项目再保存成私有模板,也可以使用像\sphinxhref{https://github.com/uwescience/shablona}{shablona}这样为科研设计的现成模板。

\item {} 
如果你项目的基础是一个软件包,可以考虑使用
\sphinxhref{https://github.com/audreyr/cookiecutter}{cookiecutter}
工具,从命令行创建基于不同编程语言的模板。目前\sphinxstylestrong{cookiecutter}已经支持了包括\sphinxstylestrong{Python},
\sphinxstylestrong{Javascript}, \sphinxstylestrong{Ruby}, \sphinxstylestrong{Markdown}, \sphinxstylestrong{HTML}在内的主流语言。

\item {} 
此外,\sphinxstylestrong{astropy}社区也专门准备了\sphinxhref{https://github.com/astropy/package-template}{astropy工具包模板}。如果你的项目需要一个规范的\sphinxstylestrong{Python}程序包作为核心,这是一个很好的选择。

\end{itemize}


\subsection{构建自己的编程环境与习惯}
\label{\detokenize{resource/research/getting_started_cn:id7}}\begin{itemize}
\item {} 
在兴奋地开始你自己的科研项目之前,希望你可以耐心地了解一下优秀的编程习惯,并结合自己的喜好建立一个高效的计算机工作环境。刚开始走上“学习曲线”总会显得有些艰难,但请相信,越早熟悉各种编程工具并培养起好的工作习惯会在不远的未来起到事半功倍的效果。另外,借助网络,体贴的经验和优秀的榜样无处不在。
\begin{itemize}
\item {} 
如果你的主要编程语言是\sphinxstylestrong{Python}, \sphinxstylestrong{C/C++},
我们推荐你参考\sphinxhref{https://developer.lsst.io/}{大视场全景巡天项目的开发者指南}。其中包括了很多关于代码编写,测试的可以实际参考的规范。

\item {} 
关于\sphinxstylestrong{Python},也可以参考更为详细的\sphinxhref{https://devguide.python.org/}{Python开发者指南}。

\item {} 
\sphinxhref{C++\%20Core\%20Guidelines}{C++核心指南}
提供了\sphinxstylestrong{C++}编程的最权威的规范指南。

\end{itemize}

\item {} 
与此同时,\sphinxstylestrong{请不要浪费时间在挑选和比较代码编辑器或者编程环境
(IDE)}
上:挑一个让你第一感觉良好的环境,上手使用,探索各种高级功能;如果趁手就坚持用下去,如果觉得别扭就试试下一个。工具只是工具,\sphinxhref{https://www.vim.org/}{vim}
和 \sphinxhref{https://www.gnu.org/software/emacs/}{emacs}
都是非常优秀的编辑器; \sphinxhref{https://atom.io/}{atom},
\sphinxhref{https://code.visualstudio.com/}{VScode}, 和
\sphinxhref{https://www.sublimetext.com/}{sublime}
都是非常高端的整合式编程环境.
它们都有优异的功能和强大的扩展能力,可以在科研的道路上助你一臂之力。

\item {} 
而且,你不需要从零开始,已经有很多优秀的资源可以帮助你搭建一个优秀的工作环境:
\begin{itemize}
\item {} 
\sphinxhref{https://github.com/syl20bnr/spacemacs}{spacemacs}:
社区维护的\sphinxstylestrong{emacs}发行版本,包含了很多扩展。

\item {} 
\sphinxhref{https://github.com/SpaceVim/SpaceVim}{spacevim}:
同样由社区维护,更为现代的\sphinxstylestrong{vim}编程环境。

\item {} 
\sphinxhref{https://github.com/emacs-tw/awesome-emacs}{Awesome Emacs}
整理了关于\sphinxstylestrong{emacs}你需要了解的一切。而\sphinxhref{https://github.com/akrawchyk/awesome-vim}{Awesome
Vim}
列出了大量对你有帮助的\sphinxstylestrong{vim}编辑器扩展。

\item {} 
\sphinxhref{https://github.com/amix/vimrc}{究极vimrc}
提供了一个非常优秀的\sphinxstylestrong{vim}编程环境的设置文件,可以拿来就用。

\item {} 
\sphinxhref{https://github.com/neovim/neovim}{neovim}:
一个注重于提升扩展性和易用性的现代\sphinxstylestrong{vim}版本。

\item {} 
\sphinxhref{https://github.com/viatsko/awesome-vscode}{awesome-vscode}
收集了大量关于微软\sphinxstylestrong{VSCode}的资源。

\item {} 
\sphinxhref{https://github.com/mehcode/awesome-atom}{awesome-atom}
则整理了大量\sphinxstylestrong{atom}编辑器相关的资料。

\end{itemize}

\end{itemize}


\subsection{备份,备份,备份}
\label{\detokenize{resource/research/getting_started_cn:id8}}\begin{itemize}
\item {} 
\sphinxstylestrong{你永远都不会高估备份对你科研项目的重要性!}

\item {} 
线下备份:
\begin{itemize}
\item {} 
你应该经常地利用移动硬盘等设备对你科研用电脑的系统和重要文件进行备份。各种操作系统上都有相关工具可以帮助你简化这一步骤,比如\sphinxstylestrong{MacOSX}下的
\sphinxhref{https://support.apple.com/en-us/HT201250}{TimeMachine},以及\sphinxstylestrong{Linux}下的
\sphinxhref{https://wiki.ubuntu.com/TimeVault}{TimeVault} 和
\sphinxhref{http://duplicity.nongnu.org/}{Duplicity}。

\item {} 
命令行一键备份关键文件其实是很容易的,你只需要稍微学习一下\sphinxhref{https://linux.die.net/man/1/rsync}{rsync}
这个命令行工具的使用: 基本上, \sphinxstylestrong{rsync -av \textendash{}delete /Directory1/
/Directory2/} 这个命令就够了。
\begin{itemize}
\item {} 
在\sphinxstylestrong{Linux}或者\sphinxstylestrong{MacOSX}上,你还可以使用\sphinxhref{https://opensource.com/article/17/11/how-use-cron-linux}{Cron}命令行工具让电脑再指定时间自动备份。可以参考\sphinxhref{https://nickjanetakis.com/blog/automatic-offline-file-backups-with-bash-and-rsync}{如下这个例子}。

\end{itemize}

\end{itemize}

\item {} 
在线备份:
\begin{itemize}
\item {} 
如果条件允许,也可以考虑使用 \sphinxhref{https://www.dropbox.com}{Dropbox}
或者 \sphinxhref{https://www.jianguoyun.com/}{jianguoyun (坚果云)}
这样的服务时刻保持关键科研文件的同步和备份。这些服务的免费部分往往容量有限,但对备份最为关键的草稿,笔记,代码等应该足足有余。

\end{itemize}

\end{itemize}


\subsection{让你的科研“有据可查”}
\label{\detokenize{resource/research/getting_started_cn:id9}}\begin{itemize}
\item {} 
科研过程当中往往会产生大量的笔记或者其他文字资料。这些笔记包括了想法的整理,数据处理细节,理论推导,等等等等。建立良好的整理习惯是很有帮助的。

\item {} 
无论你用什么方法整理笔记,都应该努力让资料做到\sphinxstylestrong{随时备份以及可以被搜索}
(从一个电脑文档中搜索一个关键词比从一个厚厚的笔记本中要容易得多)。现在已经有很多跨平台的软件或者在线服务可以帮助你整理笔记,比如微软的{}`OneNote \textless{}\sphinxurl{https://www.onenote.com/signin?wdorigin=ondc}\textgreater{}{}`\_\_,在线笔记和合作平台\sphinxhref{https://evernote.com}{evernote}
(国内叫做象印笔记),以及**Dropbox**旗下的\sphinxhref{https://paper.dropbox.com}{Paper服务}
都是不错的选择.
如果你还是习惯使用纸质笔记,这些服务的移动应用都可以帮助你扫描整理笔记。

\item {} 
如果你已经在使用\sphinxstylestrong{GitHub}这样的托管平台来整理你的科学项目,\sphinxhref{https://guides.github.com/features/wikis/}{使用其提供的维基页面服务}
同样是非常方便的选择。

\item {} 
\sphinxhref{https://en.wikipedia.org/wiki/Markdown}{Markdown}
是一种轻量级的普通文本标记语言,有着简洁易学的语法和很好的可移植性,是非常适合用来规范整理科研记录的工具。\sphinxstylestrong{Markdown}文件可以直接被渲染成优雅的在线文本,也可以很容易地通过特定工具转换成其他格式
(\sphinxhref{https://pandoc.org/index.html}{如PDF甚至是LaTeX格式})。
\begin{itemize}
\item {} 
网上提供\sphinxstylestrong{Markdown}语法教学的地方很多。\sphinxhref{https://guides.github.com/features/mastering-markdown/}{GitHub的这份}
是个不错的开始。而更完整的语法说明可以在
\sphinxhref{https://www.markdownguide.org/}{Markdown指南} 中找到.

\item {} 
不同的平台上都有免费的\sphinxstylestrong{Markdown}查看和编辑软件
(比如{}`Typora \textless{}\sphinxurl{https://typora.io/}\textgreater{}{}`\_\_)。也有很多方便的在线\sphinxstylestrong{Markdown}编辑器可以选用
(如{}`StackEdit \textless{}\sphinxurl{https://stackedit.io/}\textgreater{}{}`\_\_)。大部分现有的代码编辑器都能通过扩展支持对\sphinxstylestrong{Markdown}格式的语法检查
(\sphinxstylestrong{.md}或者\sphinxstylestrong{.markdown}格式文件)。

\end{itemize}

\end{itemize}


\subsection{发表你的科学成果}
\label{\detokenize{resource/research/getting_started_cn:id10}}\begin{itemize}
\item {} 
科学论文写作往往让人痛不欲生,但却又是科研生活中最为重要的一个环节。为此,我们\sphinxhref{https://github.com/dr-guangtou/taotie/blob/master/research/writing\_paper.md}{专门准备了一个单独的文档整理和论文写作的相关资源与工具}。
\begin{itemize}
\item {} 
目前,\sphinxstylestrong{英语作为科研界通用语言的地位依然不可动摇,应当在日常科研中就坚持使用英语}。

\item {} 
同时,\sphinxhref{https://www.latex-project.org/}{LaTeX}
依然是天文界最为常用的科学文献编辑工具。应当让自己尽快熟悉这个工具。目前在各个平台上都有相应的\sphinxstylestrong{LaTeX}图形界面软件
(如{}`LyX \textless{}\sphinxurl{https://www.lyx.org/}\textgreater{}{}`\_\_,
\sphinxhref{http://www.winedt.com/}{WinEdit},
\sphinxhref{https://pages.uoregon.edu/koch/texshop/}{TexShop},
和\sphinxhref{https://www.texstudio.org/}{TexStudio}),大大降低了使用\sphinxstylestrong{LaTeX}写作的难度。在线编辑平台
\sphinxhref{https://www.overleaf.com/}{Overleaf} 和
\sphinxhref{https://www.authorea.com/}{Authorea}
的流行也进一步降低了合作写作和论文投稿的难度。

\item {} 
多阅读,多练习是唯一可靠的提高科研写作的手段。在实际写作中,也有一些\sphinxhref{http://www.cws.illinois.edu/workshop/writers/tips/writersblock/}{有用的经验可以帮助你有效地克服每个从事写作的人都会遇到的“脑闭塞”障碍}

\end{itemize}

\item {} 
投稿前请参考\sphinxhref{https://www.scimagojr.com/journalrank.php?category=3103}{天文学和天体物理学常见的期刊列表}
\begin{itemize}
\item {} 
请不要过分在意列出的期刊的影响因子和H-指数,这些数字参考价值有限。

\end{itemize}

\end{itemize}


\subsection{分享和传播你的科学成果}
\label{\detokenize{resource/research/getting_started_cn:id11}}

\subsubsection{让你的科研结果更加透明}
\label{\detokenize{resource/research/getting_started_cn:id12}}\begin{itemize}
\item {} 
\sphinxhref{https://en.wikipedia.org/wiki/Open\_science}{开放的,可重复的科学}
百利无一害!
近些年来,在天文学和宇宙领域里也看到了整个学术社区进一步开放的可喜迹象。你也应该努力借助各种工具和数据分享平台,努力使自己的科学结果\sphinxstylestrong{可以被重现和检查},应该在力所能及的范围内努力做到数据分享。

\item {} 
如上所述,\sphinxstylestrong{Github}是非常好的用于整理和分享你的科研成果的平台。你可以将科研所用的代码,数据,交互式\sphinxstylestrong{Jupyter}笔记本,以及论文草稿一并在这里分享。目前已经有很多优秀的例子展示了\sphinxstylestrong{Github}在科学项目分享中的作用。但是,\sphinxstylestrong{Github}不是非常适合分享体积比较庞大的数据。

\item {} 
\sphinxhref{https://zenodo.org/}{zenodo}
是由欧洲核子研究中心(CERN)支持的,隶属于欧洲\sphinxstylestrong{OpenAIRE} (Open
Access Infrastructure Research for Europe;
欧洲开放获取基础设施研究项目)
项目下的面向全科学界的通用科学结果公开获取平台。
\begin{itemize}
\item {} 
\sphinxstylestrong{zenodo}可以获取你\sphinxstylestrong{Github}仓库中的资料,并创建数字对象标识符
(DOI),使得你的项目可以在各种学术文献中被正确认可。

\end{itemize}

\item {} 
\sphinxhref{https://dataverse.org/}{Dataverse} 类似zenodo,提供DOI。

\item {} 
\sphinxhref{https://ascl.net/}{The Astrophysics Source Code Library (ASCL)}
用于发布代码,可被ADS收录。

\item {} 
\sphinxhref{https://figshare.com/}{figshare}
是另外一个可以协助你上传,整理,分享,以及发表科学结果的在线系统。\sphinxstylestrong{figshare}同样可以为你分享的数据创建永久的DOI标识,方便别人引用和参考。

\item {} 
\sphinxhref{https://osf.io/}{公开科学框架 (OSF)}
也是一个协助公开科研合作的开源平台,有着非常丰富的供能。

\end{itemize}


\subsubsection{用报告介绍你的科研成果}
\label{\detokenize{resource/research/getting_started_cn:id13}}\begin{itemize}
\item {} 
通过各种学术报告介绍、推广自己的科学成果也是学术生活中非常重要的一个环节。优秀的科学报告可以让同行对你和你的工作留下深刻的印象,不仅有助于开展合作,也对未来求职非常有帮助。

\item {} 
做好一个科学报告并不是一件容易的事情,需要反复的学习和练习。虽然不会有一份简单的指南可以将你变成天文界的乔布斯,但总还是有一些经验可以参考的。下面这些文章虽然听上去有些“标题党”,但是其中的意见非常中肯:
\begin{itemize}
\item {} 
\sphinxhref{http://www.planetary.org/blogs/emily-lakdawalla/2018/0206-speak-your-science.html}{讲出你的科学:如何在科学会议上做更出色的报告 - Emily
Lakdawalla}
\begin{itemize}
\item {} 
这是一篇非常值得参考的文章。作者是行星学会的科学家,也非常善于介绍科学以及与公众沟通。

\end{itemize}

\item {} 
\sphinxhref{https://www.nature.com/articles/d41586-018-07780-5}{《自然》杂志给出的关于做好科学报告的建议}

\item {} 
\sphinxhref{https://www.sciencemag.org/careers/2019/04/three-tips-giving-great-research-talk}{《科学》杂志给出的关于做好科学报告的三个小技巧}

\item {} 
\sphinxhref{http://www.cgd.ucar.edu/cms/agu/scientific\_talk.html}{来自美国UCAR整理的关于做好科学报告的十个秘诀}

\item {} 
\sphinxhref{https://astrobites.org/2018/02/10/speak-your-science-part-1/}{AstroBites网站曾经有过三期关于如何做好报告的文章},也值得参考

\item {} 
\sphinxhref{https://arxiv.org/abs/1712.08088}{Chat Hull的文章 How to Give a Great
Talk}

\end{itemize}

\item {} 
绝大多数科学报告场合都离不开使用“幻灯片”,无论你使用什么软件,一份清晰而美观的报告文件对你的报告总是加分的。虽然报告文件的制作取决于个人审美喜好,但同样有一些基本的标准可以参考:
\begin{itemize}
\item {} 
\sphinxhref{http://blogs.nature.com/naturejobs/2017/01/11/scientific-presentations-a-cheat-sheet/}{《自然》杂志介绍的展示科学用的幻灯片的“快速入门”}

\item {} 
\sphinxhref{http://blogs.nature.com/naturejobs/2016/02/10/a-david-letterman-like-countdown-to-the-10-biggest-pitfalls-in-scientific-presentations/}{《自然》杂志整理的用幻灯片展示科学结果时最容易犯的十大错误}

\item {} 
\sphinxhref{https://www.slideshare.net/}{SlideShare}是一个报告文件在线分享与展示平台。上面有很多来自不同领域的优秀报告文件供你参考。你也可以考虑把你的报告上传到这里分享。

\item {} 
\sphinxhref{https://speakerdeck.com/}{SpeakerDeck}
与\sphinxstylestrong{SlideShare}功能和形式都类似,同样有很多优秀的报告文件可供参考。

\end{itemize}

\end{itemize}


\subsubsection{制作一张醒目的科学海报}
\label{\detokenize{resource/research/getting_started_cn:id14}}\begin{itemize}
\item {} 
科学海报的作用随着科学会议的组织形式的变化在逐渐下降,但在很多场合依然不失为一种展示你的科学成果,增进科学合作的机会。任何类似的机会,无论是报告还是海报都值得认真对待。\sphinxhref{https://www.makesigns.com/tutorials/}{这里也有一些关于制作优秀科学海报的技巧可以参考}。

\item {} 
同时,传统的科学海报设计有一些不是很好避免的问题。为此,科学家 Mike
Morrison设计了一个叫做\sphinxhref{https://osf.io/ef53g/}{“更好的科学海报” (Better Scientific
Poster)}
的模板来更快速的设计让人印象深刻的海报。这个模板背后的设计逻辑就是用海报清晰的传递一个最关键的科学成果,同时给出方便获取的链接让人可以进一步了解你的成果。
\begin{itemize}
\item {} 
\sphinxhref{https://www.youtube.com/watch?v=1RwJbhkCA58\&feature=youtu.be}{这个Youtube视频也详细地解释了这个海报模板的背后动机和设计理念}

\item {} 
除了常见的\sphinxstylestrong{PowerPoint}和\sphinxstylestrong{Keynote}格式外,目前也有为\sphinxhref{https://github.com/rafaelbailo/betterposter-latex-template}{LaTeX}
和\sphinxhref{https://github.com/GerkeLab/betterposter}{R
Markdown}准备的模板。

\end{itemize}

\end{itemize}


\subsection{文献阅读和整理}
\label{\detokenize{resource/research/getting_started_cn:id15}}\begin{itemize}
\item {} 
刚开始科学文献的阅读可能会显得很艰难,但请相信,\sphinxhref{https://web.stanford.edu/class/ee384m/Handouts/HowtoReadPaper.pdf}{阅读科学论文是有章可循的}。\sphinxhref{https://www.sciencemag.org/careers/2016/03/how-seriously-read-scientific-paper}{《科学》杂志总结的这份来自不同研究者的经验}
也值得参考。

\end{itemize}


\subsubsection{论文预印本文库 arXiv}
\label{\detokenize{resource/research/getting_started_cn:arxiv}}\begin{itemize}
\item {} 
论文预印本文库\sphinxstylestrong{arXiv}是及时获取最新论文的最好手段。在日常的科研中,通过\sphinxstylestrong{arXiv}跟踪最新的科学动向和掌握领域内最新的科研进展是非常好的习惯。
\begin{itemize}
\item {} 
\sphinxhref{https://www.voxcharta.org}{voxCharta}
是一个在线组织\sphinxstylestrong{arXiv}讨论的平台。如果你所在研究机构已经在\sphinxstylestrong{voxCharta}上注册,你可以在这里看到你的同行都在读什么论文。在这里你还可以看到来自全世界的天文学者最关心以及最近最“流行”的工作是哪些。\sphinxstylestrong{voxCharta}还有按照你的个人关注话题定期推荐文章给你的功能。

\item {} 
\sphinxhref{https://www.arxivsorter.org/}{arXivSorter}是一个由数据科学家和天文学家共同开发的,基于机器学习算法的\sphinxstylestrong{arXiv}论文推荐工具。在这里,你阅读的工作越多,\sphinxstylestrong{arXivSorter}背后的算法就能更准确地分析你的喜好,为你推荐工作。

\item {} 
按照一定的规范来记录自己感兴趣的工作也是一个很好的习惯。\sphinxhref{https://github.com/dr-guangtou/daily\_astroph}{这里是一个利用Markdown笔记和GitHub平台进行arXiv笔记整理的例子}

\item {} 
\sphinxhref{https://github.com/lukasschwab/arxiv.py}{arxiv.py是一个Python工具包},可以帮助你通过\sphinxstylestrong{Python}搜索和获取\sphinxstylestrong{arXiv}论文。

\end{itemize}

\item {} 
\sphinxhref{https://astrobites.org}{Astrobites}是一个通过简短的文章来介绍最新\sphinxstylestrong{arXiv}文章的网站。这个项目由一群天文学专业学生发起,目前依然由来自世界各地的学生供稿和维护。在介绍文章的同时,\sphinxstylestrong{Astrobites}也经常提供各种对学习和科研有帮助的参考文章。比如这个分成三部分的介绍如何阅读科学文献的文章:\sphinxhref{https://astrobites.org/2017/12/19/tools-for-reading-papers-part-1/}{Part
I},
\sphinxhref{https://astrobites.org/2018/03/09/tools-for-reading-papers-part-2/}{Part
II},
\sphinxhref{https://astrobites.org/2018/09/06/tools-for-reading-papers-part-3/}{Part
III}

\end{itemize}


\paragraph{如何向\sphinxstylestrong{arXiv}“投稿”:}
\label{\detokenize{resource/research/getting_started_cn:id16}}\begin{itemize}
\item {} 
当你有了新的论文即将发表,及时将工作上传\sphinxstylestrong{arXiv}是很有必要的。\sphinxstylestrong{arXiv}支持上传\sphinxstylestrong{LaTeX}源文件,但会在其服务器上进行重新的编译。这个过程有时会比你想象的繁琐。所以,在上传之前,请务必阅读\sphinxhref{https://arxiv.org/help/submit}{arXiv官方提供的关于上传预印本的指南}

\item {} 
\sphinxstylestrong{arXiv}在每个工作日都有固定的截止时间来判断下一天会上线那些预印本文章。你可以在\sphinxhref{https://arxiv.org/localtime}{这里查看arXiv系统的“本地时间”以及距离下个截止时间还有多久}。
\begin{itemize}
\item {} 
\sphinxstylestrong{arXiv}系统会按照上传的顺序来在网页上展示预印本。有证据显示这个决定会在论文关注度上\sphinxhref{https://arxiv.org/pdf/0712.1037.pdf}{展示出偏差}让先上传的文章获得更高的关注度,甚至引用。一反面,利用这个系统设置,尽早上传并不是错误的;不过,另一方面,也请了解这个偏差并能用更全面的眼光来审视每天\sphinxstylestrong{arXiv}上的工作。目前有很多科学家在争取改变\sphinxstylestrong{arXiv}系统,用随机的顺序展示预印本以消除这个偏差。

\end{itemize}

\item {} 
\sphinxhref{https://github.com/google-research/arxiv-latex-cleaner}{arXiv LaTeX
整理器}
是一个很有用的\sphinxstylestrong{Python}工具。它可以帮助你“过滤”你论文的\sphinxstylestrong{LaTeX}源文件,并作出修改以适应\sphinxstylestrong{arXiv}的要求。

\end{itemize}


\paragraph{SAO/NASA天文文献资料库\sphinxstylestrong{ADS}}
\label{\detokenize{resource/research/getting_started_cn:sao-nasaads}}\begin{itemize}
\item {} 
由美国哈佛-史密松森天体物理中心和美国航空航天局共同资助的“天体物理数据系统”
(\sphinxstylestrong{ADS})目前一共收藏整理了超过八百万篇来自于天文学和物理学领域的科学工作。这些工作包括了同行评议后发表的期刊文章,也有未经评议的会议记录和望远镜提案等信息。\sphinxstylestrong{ADS}是每一个进行天文科研的人都必须熟练使用的系统。近日,\sphinxstylestrong{ADS}系统进行了彻底的升级,拥有了一个\sphinxhref{https://github.com/adsabs}{更为现代化和先进的系统和界面}。如果你刚开始使用ADS,请直接学习使用新版的系统。\sphinxhref{http://adsabs.github.io/help/search/}{ADS提供了非常好的入门材料}
\begin{itemize}
\item {} 
\sphinxhref{https://github.com/adsabs/adsabs-dev-api}{新版的ADS还提供了完整的API方便你进行交互}。

\item {} 
\sphinxhref{https://github.com/andycasey/ads}{ads} 是天文学家Andy
Casey开发的\sphinxstylestrong{Python}工具。它可以帮助你和\sphinxstylestrong{ADS}系统进行交互,搜索资料。

\item {} 
新版的\sphinxstylestrong{ADS}系统允许你通过你的\sphinxstylestrong{ORCID}进行登录,并将你的工作和\sphinxstylestrong{ADS}记录联系起来。

\item {} 
\sphinxhref{https://adsabs.github.io/help/libraries/creating-libraries}{ADS的个人图书馆系统}来分门别类的整理你自己感兴趣的工作,并批量输出论文引用信息。利用\sphinxstylestrong{ADS}的引用信息和作者网络,你也可以方便的通过一篇经典文章迅速学习一个领域的最新工作。

\end{itemize}

\end{itemize}


\paragraph{其他文献阅读和整理信息}
\label{\detokenize{resource/research/getting_started_cn:id17}}\begin{itemize}
\item {} 
几乎所用的主流天文期刊都提供了RSS和邮件提醒系统。每当新一期的期刊发表,这些服务都可以把最新的论文通过邮件列表推送给你。这是另外一个及时了解最新\sphinxstylestrong{发表}论文的方式。

\item {} 
天文论文中普遍使用\sphinxstylestrong{BibTex}格式整理论文引用信息。\sphinxhref{https://bibdesk.sourceforge.io/}{BibDesk}
和 \sphinxhref{http://www.jabref.org/}{JabRef}
都是很好的免费工具。将你阅读过的或者对你的领域有用的文献整理到一个\sphinxstylestrong{.bib}文件中对未来的论文和申请写作会很有帮助的。

\item {} 
关于整理大量文献,\sphinxhref{https://www.mendeley.com/?interaction\_required=true}{Mendeley}
和 \sphinxhref{https://www.zotero.org/}{Zotero}
都是很好的跨平台,免费的服务。都能够帮助你从网页上获取论文信息,并将你的文献资料进行备份的工具。\sphinxstylestrong{MacOSX}和\sphinxstylestrong{iOS}上的\sphinxhref{https://www.papersapp.com/}{Papers}也是非常好的文献整理工具,不过可惜售价不菲。

\end{itemize}


\subsection{沟通与合作}
\label{\detokenize{resource/research/getting_started_cn:id18}}\begin{itemize}
\item {} 
不分国界和时区的频繁沟通是当前科学合作的基础。目前,电子邮件仍然是最重要的沟通和学术交流工具。整体来说,天文社区规模并不算特别大,而且天文学家中性格放松的人居多,把日常工作邮件当做短信对待并不为过,一般不需要过分紧张。但如果是比较正式的场合,或者是和刚接触的合作者沟通,一封得体的职业邮件也是应该的。关于写邮件,也有一些简单的建议可以参考。如\sphinxhref{https://www.grammarly.com/blog/professional-email-in-english/}{Grammarly的这篇短文}
或者\sphinxhref{https://www.amanet.org/articles/how-to-write-the-perfect-email/}{美国管理学协会的这篇指导}
\begin{itemize}
\item {} 
如果你在联系人的工作时间外发邮件,注明“请不必马上回复”或者“请按照您的工作时间安排回复邮件”是一种很礼貌的行为。

\end{itemize}

\item {} 
\sphinxhref{https://slack.com/}{Slack}作为从硅谷兴起的一种在线合作工具目前也得到了学术界的广泛喜爱。包括LSST或者DESI在内的一系列大科学合作均已开始使用\sphinxstylestrong{Slack}组织学术活动和进行日常科学交流。和邮件比,\sphinxstylestrong{Slack}既有聊天软件的放松,又有更好的享文件和扩展功能。根据自己合作的科学需求编写相应的\sphinxstylestrong{Slack}“机器人”
(bot)也并不是很难的事情。\sphinxstylestrong{Slack}的缺点是只有付费版本才有完整的交流历史记录,但其实免费版本的功能已足以应付大多数使用场景。

\item {} 
随着跨机构,跨国界的学术交流增多,电话或者电视会议(“telecon”)也变得越来越重要和普遍。目前常用的远程会议软件包括了\sphinxhref{https://www.skype.com/en/}{Skype},
\sphinxhref{https://zoom.us/}{zoom}, 和
\sphinxhref{https://www.gotomeeting.com/}{GoToMeetings}。这些软件的基本功能都不要付费,很容易上手,而且提供了远程分享桌面或者报告文件的功能,可以帮助你展示你的工作。
\begin{itemize}
\item {} 
如果你的合作者来自五湖四海,确定一个对大家都合适的会议时间有时并不容易。\sphinxhref{https://doodle.com/make-a-poll}{Doodle}服务是最常用的在线投票系统,可以帮助你组织各种远程交流活动。

\end{itemize}

\end{itemize}


\subsection{个人主页}
\label{\detokenize{resource/research/getting_started_cn:id19}}\begin{itemize}
\item {} 
简洁,专业的个人主页可以帮助别人迅速认识你和你的工作,对于学术交流和求职的帮助都很大。对于来自中国研究机构的学者,能够通过个人主页提供准确的个人信息在国际学术交流中尤其重要。毕竟很多国外学者并不熟悉国内的天文科研机构,而又有太多的中国学者重姓甚至重名。确保别人能够在搜索引擎中迅速找到你的个人简历和联系信息,把你和其他学者区分开,有的时候是非常重要的。

\item {} 
如果你不熟悉个人网站的制作,或者你所在的机构并不提供个人主页服务
\sphinxhref{https://pages.github.com/}{GitHub页面}
服务不失为一个很好的选择。它可以帮助你利用各种现有模板建立一个简洁的个人主页,并通过相应的\sphinxstylestrong{GitHub}软件仓库维护管理你的主页。
\begin{itemize}
\item {} 
\sphinxhref{https://marisacarlos.com/pages/create-simple-academic-website}{这里有一个关于如何用GitHub页面功能建立简单学术个人主页的指南}

\item {} 
\sphinxstylestrong{GitHub}页面支持\sphinxstylestrong{Jekyll}格式的模板。\sphinxstylestrong{Jekyll}基于\sphinxstylestrong{Ruby}语言,可以很容易将普通文本转换成一个个人主页或者个人博客。而使用\sphinxstylestrong{GitHub}页面基本不需要你有任何\sphinxstylestrong{Ruby}或者\sphinxstylestrong{Jekyll}的知识,只需要使用\sphinxstylestrong{GitHub}提供的\sphinxhref{https://pages.github.com/themes/}{一系列简洁模板},修改相应文件就可以了。

\item {} 
虽然\sphinxstylestrong{GitHub}的模板已经足够用,但如果你还是希望可以进一步丰富你的个人主页的话,\sphinxhref{https://jekyllthemes.io/github-pages-templates}{还有更多的Jekyll模板可以供你参考}。需要注意的是,一般认为学术个人主页不需要过于“花哨”,绝大多数场景下,访问者可能只需要你的个人简历和联系信息。确保访问者能够迅速找到这些信息就可以了。

\item {} 
\sphinxhref{https://github.com/alshedivat/al-folio}{al-folio}
就是一个专门为学术场景设计的简洁美观的\sphinxstylestrong{Jekyll}模板,可以很方便的在\sphinxstylestrong{GitHub}页面上使用。

\item {} 
如果你不喜欢\sphinxstylestrong{GitHub}页面默认的:
\sphinxstylestrong{https://user-name.github.io}
的域名,也可以\sphinxhref{https://help.github.com/en/articles/using-a-custom-domain-with-github-pages}{自己注册一个更容易记住的域名,并链接到你的GitHub页面}。

\item {} 
\sphinxhref{https://help.github.com/en/articles/search-engine-optimization-for-github-pages}{GitHub页面还通过Jekyll的插件支持搜索引擎优化功能}。确保你的主页能在搜索引擎的第一页也是很有帮助的。

\end{itemize}

\item {} 
已经有很多科学家使用\sphinxstylestrong{GitHub}页面作为个人主页服务,并且有不少优秀的例子可供参考。如果以简洁和清晰作为标准的话,下面两位科学家的主页是很好的参考:
\begin{itemize}
\item {} 
\sphinxhref{http://adrian.pw/}{Adrian
Price-Whelan}。\sphinxhref{https://github.com/adrn/adrn.github.io}{模板和网页代码在这里}

\item {} 
\sphinxhref{https://dfm.io/}{Dan Foreman-Mackey}。
\sphinxhref{https://github.com/dfm/dfm.io}{模板和网页代码在这里}

\end{itemize}

\end{itemize}


\subsection{学术会议和报告}
\label{\detokenize{resource/research/getting_started_cn:id20}}

\subsubsection{参加学术会议}
\label{\detokenize{resource/research/getting_started_cn:id21}}\begin{itemize}
\item {} 
参加不同级别和形式的学会会议是学术生涯中非常重要,也非常有意思的一部分。通过会议不仅可以让你推广自己的工作,也可以帮助你结识志同道合的合作者,以及快速地了解一个领域的最新进展。无论会议在何处举行,是什么级别,都请记住学术会议整个过程中务必要保证自己的行为\sphinxhref{https://confcodeofconduct.com/}{符合应用的学术行为规范}。

\item {} 
\sphinxhref{http://www.cadc-ccda.hia-iha.nrc-cnrc.gc.ca/en/meetings/}{加拿大天文数据中心维护着一个很好的国际天文会议档案}。在这里你可以搜索你关心的领域里会有哪些有意思的学术会议即将举行;你也可以通过这里找到之前举行的会议的主页,查看有哪些有趣的报告。页面提供了RSS订阅服务和苹果系统下\sphinxstylestrong{iCal}日历软件的订阅服务。

\item {} 
{[}推特上的
@astromeetings账号{]}(\sphinxurl{https://twitter.com/astromeetings?lang=en})
是另外一个了解近期天文学术会议动向的途径。值得一提的是,越来越多的学术会议会制定一个推特话题(“hashtag”),并利用推特宣传会议和展示有趣的报告。鉴于我们不可能参加我们关心的所有会议
(这样做{}`对环境也是有不友好 \textless{}\sphinxurl{https://onlinelibrary.wiley.com/doi/pdf/10.1111/1746-692X.12106}\textgreater{}{}`\_\_),通过推特这样的媒介了解会议进展也是不错的选择。如果你参加国际会议,也可以选择利用这个手段增加自己工作的可见度。

\item {} 
国际天文学联合会 (IAU) 每年都会组织一批面向不同领域的高规格会议。
\sphinxhref{https://www.iau.org/science/meetings/future/}{在这里你可以查看未来IAU会议的安排}

\item {} 
此外,近年来有越来越多的科学家开始审视传统会议的组织形式,毕竟大量集中的报告并不一定是最高效的学术交流形式,而有研究现实\sphinxhref{http://blog.gigabase.org/en/contents/132}{通风不良的会议甚至会不利于人的健康}。目前,有一些新的会议组织形式更多的着眼于与会者的交流和合作,也非常值得参考:
\begin{itemize}
\item {} 
\sphinxhref{https://www.dotastronomy.com/}{.Astronomy}
会议面向大数据时代的天文学,努力构建一个活跃的会议氛围,让科学家,编程人员,和教育者可以开展有创造性的合作。目前已经进行到了第11次。

\item {} 
\sphinxhref{http://astrohackweek.org/2019/}{Astro Hack Week}
从2015年开始,旨在通过一系列的讲座,活跃的交流,和围绕一定项目的集体编程来组织氛围活跃的学术会议。会议的话题也经常围绕天文学中的统计和机器学习应用,并强调与会者的动手参与。往年会议的讲座大多可以在网上找到。

\item {} 
\sphinxhref{http://gaia.lol/}{Gaia Sprint} 是围绕着 \sphinxstyleemphasis{Gaia}
卫星数据和科学开展短期“冲刺周”。同样也是非常强调交流互动,并着眼于在短期内推动实际科学项目开展的新颖学术活动。

\end{itemize}

\end{itemize}


\subsubsection{在线天文学报告}
\label{\detokenize{resource/research/getting_started_cn:id22}}\begin{itemize}
\item {} 
除了学术会议,不同研究机构自己的学术报告活动也是非常重要的学习和交流机会。然而,任何一个研究机构的报告频率和覆盖范围都比较有限,尤其是对于规模不大的科研机构来说。不过,在网络时代,像\sphinxstylestrong{Youtube}这样的视频分享平台也给我们提供很多“免费”学习的机会。在这里你可以听到来自世界不同机构不同话题的报告。例如:
\begin{itemize}
\item {} 
\sphinxhref{https://www.youtube.com/channel/UCApHNlZLkxmiV95A0ChueYg}{美国哈佛大学和CfA的每周正式学术报告}
and
\sphinxhref{https://www.youtube.com/channel/UCTuACIrLKPTlp6XMZbeipig/featured}{理论与计算天体物理中心ITC的学术报告}

\item {} 
\sphinxhref{https://www.youtube.com/channel/UC7D7uBI6-47leDWg1sbuJLg}{美国华盛顿卡耐基研究院天文台的学术报告}

\item {} 
\sphinxhref{http://www.stsci.edu/events}{美国空间望远镜科学中心的各种学术报告和会议视频}

\item {} 
\sphinxhref{https://www.youtube.com/user/AstronomyHeidelberg}{德国海德堡大学的天文学系列报告}

\item {} 
\sphinxhref{https://www.youtube.com/user/SimonsFoundation/playlists}{美国纽约的计算天体物理中心CCA}
也会把很多讲座和报告上传到网上。其中有不少是关于天体物理和宇宙学的。

\item {} 
\sphinxhref{https://www.youtube.com/user/UofUPhysAstro/featured}{美国犹他大学天文和物理系的每周学术报告}

\item {} 
\sphinxhref{https://www.youtube.com/user/UCHiPACCVideos}{加州大学高性能计算天体物理中心}
也会把组织的各种学术会议的报告上传到网上。其中覆盖的话题很广,不只是关于计算天体物理的内容。

\item {} 
\sphinxhref{http://online.kitp.ucsb.edu/}{加州大学圣芭芭拉分校的科维理理论物理研究所}
会把每年组织的多次学术活动的报告视频整理上传。其中有很多是和天体物理与宇宙学有关的。

\end{itemize}

\end{itemize}


\chapter{Motivation and Goals}
\label{\detokenize{index:motivation-and-goals}}\begin{itemize}
\item {} 
Over the past \textasciitilde{}10 years, I have gradually accumulated some useful resources for
astrophysical research and I believe the best way to make use of these information is
to share them.

\item {} 
The name \sphinxcode{\sphinxupquote{taotie}} (饕餮) refers to a greedy and gluttonous monster in ancient Chinese myth.
It is indeed very greedy and over-ambitious to collect all useful resources for astrophysics
and cosmology.

\item {} 
\sphinxstylestrong{The goal of {}`{}`taotie{}`{}`} is to become a curated list of resources for astronomy and
astrophysics maintained by the community. Just like many other \sphinxcode{\sphinxupquote{awesome}} lists available
on \sphinxcode{\sphinxupquote{GitHub}}. We hope \sphinxcode{\sphinxupquote{taotie}} can become the first stop for new student or researcher in
astronomy to seek for practical guidance and experience.
And \sphinxcode{\sphinxupquote{taotie}} can also be a handy and organized “bookmark” for researchers in different fields.

\end{itemize}


\chapter{Progress and Plans}
\label{\detokenize{index:progress-and-plans}}\begin{itemize}
\item {} 
\sphinxstylestrong{Under Construction}: This repo is still under heavy and frequent development.
I am currently working on this as a weekend project.

\item {} 
Currently working on improving the collections of \sphinxcode{\sphinxupquote{taotie}}, especially for research fields
outside my personal comfort zone. \sphinxstylestrong{Any help will be highly appreciated!}

\item {} 
Also working on improving the presentation by making a user-friendly website based on a list
of markdown files.

\end{itemize}


\chapter{On Using \sphinxstyleliteralintitle{\sphinxupquote{taotie}} (饕餮)}
\label{\detokenize{index:on-using-taotie}}\begin{itemize}
\item {} 
Now \sphinxcode{\sphinxupquote{taotie}} is available on both \sphinxhref{https://github.com/dr-guangtou/taotie}{GitHub}
and \sphinxhref{https://gitlab.com/dr-guangtou/taotie}{GitLab}.
And you can also find \sphinxcode{\sphinxupquote{taotie}} using \sphinxhref{https://dr-guangtou.github.io/taotie/}{this address}

\item {} 
For users in mainland China: in case the \sphinxcode{\sphinxupquote{Github}} link below is not available,
please replace \sphinxcode{\sphinxupquote{github}} with \sphinxcode{\sphinxupquote{gitlab}} and try again.
\sphinxstylestrong{Please let me know if you have trouble accessing both sites}.

\item {} 
The main language for \sphinxcode{\sphinxupquote{taotie}} is English. Some of the important documents will be slowly
translated into Chinese.

\item {} 
\sphinxstylestrong{Please let me know if you have any suggestion or recommendation}.

\end{itemize}


\section{Contribution}
\label{\detokenize{index:contribution}}\begin{itemize}
\item {} 
Contribution from the community is highly welcomed! Please feel free to fork the repo and
make you own change.
If you want your modifications be included in this repo, please submit a pull request
(and make sure to describe the changes you made).

\item {} 
And if you notice anything wrong with the current content (wrong or unavailable link for example),
please raise an issue.

\item {} 
Also, if your repo or project is included here and you are not comfortable with that, just
let me know.

\end{itemize}


\section{Indices and tables}
\label{\detokenize{index:indices-and-tables}}\begin{itemize}
\item {} 
\DUrole{xref,std,std-ref}{genindex}

\item {} 
\DUrole{xref,std,std-ref}{modindex}

\item {} 
\DUrole{xref,std,std-ref}{search}

\end{itemize}



\renewcommand{\indexname}{Index}
\printindex
\end{document}